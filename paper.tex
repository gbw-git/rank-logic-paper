\documentclass[12pt]{article}

% This first part of the file is called the PREAMBLE. It includes customizations
% and command definitions. The preamble is everything between \documentclass
% and \begin{document}.asd

\usepackage[margin=1in]{geometry} % set the margins to 1in on all sides
\usepackage{graphicx} % to include figures
\usepackage{amsmath} % great math stuff
\usepackage{amsfonts} % for blackboard bold, etc
\usepackage{amsthm} % better theorem environments
\usepackage{xspace}
\usepackage{subfiles}%allow for subfiles
\usepackage{blindtext}
\usepackage{mymacros}
%\usepackage{anujmacros}
\usepackage{amssymb}
\usepackage{bm}
\usepackage{etoolbox}
\usepackage{marginnote}

\newenvironment{myitemize}
{ \begin{itemize}
    \setlength{\itemsep}{0pt}
    \setlength{\parskip}{0pt}
    \setlength{\parsep}{0pt}     }
{ \end{itemize}                  } 

\newenvironment{myenum}
{ \begin{enumerate}
    \setlength{\itemsep}{0pt}
    \setlength{\parskip}{0pt}
    \setlength{\parsep}{0pt}     }
{ \end{enumerate}                  } 


\newenvironment{drem}
{ \begin{remark}[DEBUG]}
  { \end{remark}} 

\newtheorem{thm}{Theorem}
\newtheorem{claim}[thm]{Claim}
\newtheorem{remark}[thm]{Remark}
\newtheorem{definition}[thm]{Definition}
\newtheorem{lem}[thm]{Lemma}
\newtheorem{prop}[thm]{Proposition}
\newtheorem{cor}[thm]{Corollary}
\newtheorem{conj}[thm]{Conjecture}

\DeclareMathOperator{\id}{id}

\begin{document}

\title{Symmetric Circuits for Rank Gates}

\author{Anuj Dawar and Gregory Wilsenach \\ 
Computer Laboratory \\
University of Cambridge}

\maketitle

\tableofcontents
\newpage

\section{Introduction}
\subfile{sections/introduction}
\newpage

\section{Background}\label{sec:background}
\subfile{sections/background}
\newpage

\section{Symmetric Circuits}\label{sec:symm-circ}
\subfile{sections/symmetric-circuits}
\newpage

\section{Symmetry and Support}\label{sec:symm-support}
\subfile{sections/symmetry-and-support}
\newpage

\section{Transparent Circuits}\label{sec:transparent}
\subfile{sections/transparency-and-unique-extensions.tex}

\section{Translating Formulas into Circuits}\label{sec:formulas-to-circuits}
\subfile{sections/translating-formulas-to-cicuits.tex}
\newpage

\section{Translating Circuits into Formulas}\label{sec:circuits-to-formulas}
\subfile{sections/translating-circuits-into-formulas}
\newpage

\section{Concluding Remarks}
\subfile{sections/conclusion}
\newpage

\bibliographystyle{plain}
\bibliography{references.bib}
\end{document}