\documentclass[12pt]{report}

% This first part of the file is called the PREAMBLE. It includes
% customizations and command definitions. The preamble is everything
% between \documentclass and \begin{document}.

\usepackage[margin=1in]{geometry} % set the margins to 1in on all sides
\usepackage{graphicx} % to include figures
\usepackage{amsmath} % great math stuff
\usepackage{amsfonts} % for blackboard bold, etc
\usepackage{amsthm} % better theorem environments


% various theorems, numbered by section

\newtheorem{thm}{Theorem} \newtheorem{claim}{Claim}
\newtheorem{remark}{Remark} \newtheorem{definition}{Definition}
\newtheorem{lem}[thm]{Lemma} \newtheorem{prop}[thm]{Proposition}
\newtheorem{cor}[thm]{Corollary} \newtheorem{conj}[thm]{Conjecture}

\DeclareMathOperator{\id}{id}

\newcommand{\bd}[1]{\mathbf{#1}} % for bolding symbols
\newcommand{\RR}{\mathbb{R}} % for Real numbers
\newcommand{\ZZ}{\mathbb{Z}} % for Integers
\newcommand{\MB}{\mathbb{B}_{\matsym}} % for Integers
\newcommand{\SB}{\mathbb{B}_{\sym}} % for Integers
\newcommand{\col}[1]{\begin{matrix} #1 \end{matrix} \right]}
\newcommand{\comb}[2]{\binom{#1^2 + #2^2}{#1+#2}}
\newcommand{\stab}{\text{\textbf{Stab}}}
\newcommand{\setstab}{\text{\textbf{SetStab}}}
\newcommand{\matstab}{\text{\textbf{MatStab}}}
\newcommand{\inv}{\text{\textbf{inv}}}
\newcommand{\aut}{\text{\textbf{Aut}}}
\newcommand{\kernal}{\text{\textbf{ker}}}
\newcommand{\alt}{\text{\textbf{Alt}}}
\newcommand{\sym}{\text{\textbf{Sym}}}
\newcommand{\rank}{\text{\textbf{rk}}}
\newcommand{\orb}{\text{\textbf{Orb}}}
\newcommand{\SP}{\text{\textbf{SP}}}
\newcommand{\maj}{\text{\textbf{Maj}}}
\newcommand{\dom}{\text{\textbf{Dom}}}
\newcommand{\child}{\text{child}}
\newcommand{\countgate}{\text{count}}
\newcommand{\agree}{\text{AGREE}}
\newcommand{\supp}{\text{SUPP}}
\newcommand{\rankparam}{\text{PARAM}}

\newcommand{\Alpha}{A}
\newcommand{\matsym}{\text{\textbf{MatSym}}}
\begin{document}


\nocite{*}

\title{Circuits}

\author{Gregory Wilsenach \\ 
Computer Laboratory \\
University of Cambridge}

\maketitle

\tableofcontents

\chapter{Introduction}

\chapter{Background}

\chapter{Symmetric Circuits}

\chapter{Extending Symmetric Circuits}

\chapter{Matrix-Symmetric Circuits}

The symmetric circuits of Anderson and Dawar \cite{} have been shown to express
$FPC$.

In this section we develop a notion of a matrix-symmetric circuit, an extension
of the symmetric circuit defined by Anderson and Dawar \cite{}. This is my
latest version of the support and circuit capturing result.

\section{Circuits and Symmetry}
\begin{definition}
  Let $A$ and $B$ be sets, say that a function $f: \{ 0,1 \}^{A \times B}
  \rightarrow \{ 0,1 \}$ is matrix-symmetric if for any $\omega: A \times B
  \rightarrow \{0, 1\}$ and $(\alpha, \beta) \in \sym_A \times \sym_B$ we have
  $f((\alpha, \beta)\cdot \omega) = f (\omega)$.
  \\
  A function $f: \{0,1\}^* \rightarrow \{0,1\}$ is matrix-symmetric iff for any
  two finite sets $A$, $B$ we have that that $f|_{A \times B}$ is
  matrix-symmetric (where $f|_{A \times B}$ is the restriction of the function
  $f$ to inputs sequences of size $\vert A \times B \vert$ and where elements of
  the input sequence are labelled by elements of $A \times B$).
\end{definition}


\section{The Support Theorem}

Without a loss of generality we assume that $A$ and $B$ are always initial
segments of the natural numbers.

% Let $C$ be a circuit over some Boolean basis. For each gate $g \in C$ we
% associate a relation $\omega_g$ (or just $\omega$ when the context makes it
% clear). We say a gate is symmetric if that relation

% So far we have defined a circuit over a Boolean basis where the circuit
% elements all compute symmetric functions. The definition of an induced
% automorphism only makes. As such, we begin by augmenting the definition of a
% circuit and automorphism.

\begin{definition}[Circuits on Structures]
  For $\mathbb{B}_{\sym}$ a basis of Boolean symmetric functions,
  $\mathbb{B}_{\matsym}$ a basis of Boolean matrix-symmetric functions and
  $\tau$ a set of relation symbols, we define a $(\mathbb{B}_\sym,
  \mathbb{B}_\sym, \tau)$-circuit $C_n$ computing a $q$-ary query $Q$ is a
  structure $\langle G, W, \Omega, \Sigma, \Lambda, L\rangle$.
  \begin{itemize}
  \item $G$ is called the set of gates of $C_n$ and $\vert C_n \vert := \vert G
    \vert$.
  \item $W \subseteq G \times G$, where $W$ is called the wires of the circuit.
    $(G,W)$ must be a directed acyclic graph. For $g \in G$ we $H_g := \{ h \in
    C_n : W(h,g)\}$ be the set of children of $g$.
  \item $\Omega$ is an injective function from $[n]^q$ to $G$. The gates in the
    image of $\Omega$ are called the output gates. When $q = 0$, $\Omega$ is a
    constant function mapping to a single output gate.
  \item $\Sigma$ is a function from $G$ to $\mathbb{B}_\sym \uplus
    \mathbb{B}_\matstab \uplus \tau \uplus \{0,1\} $ which maps input gates to
    $\tau \uplus \{0,1\}$ and where $\Sigma^{-1} (0) \leq 1$ $\Sigma^{-1} (1)
    \leq 1$ and the internal gates get mapped into $\mathbb{B}_\sym \uplus
    \mathbb{B}_\matstab$. Gates mapped to $\tau$ are called relational gates and
    gates mapped to 1 or 0 are called constant gates.
  \item $\Lambda$ is a sequence of injective functions $(\Lambda_R)_{R \in
      \tau}$ where for each $R \in \tau$, $\Lambda_R$ maps each relational gate
    $g$ with $R = \Sigma (g)$ to the tuple $\Lambda_R (g) \in [n]^r$, where $r$
    is the arity of the symbol $R$. When no ambiguity arises we write $\Lambda
    (g)$ for $\Lambda_R (g)$.
  \item $L$ is the set of labelings of $C_n$. Let $g \in G$ be an internal gate
    and $H$ be the children of $g$. Then let $S$ be a set where $\{0,1\}^S =
    \dom(\Sigma (g))$. Then $L$ is a function that assigns to $g$ an onto
    function $\omega_g:S \rightarrow H$. We call this the matrix labelling of
    $g$.
  \end{itemize}
\end{definition}

% \begin{definition}
%   let $C = \langle G, W, \Omega, \Sigma, \Lambda, L\rangle$ be a
%   $(\mathbb{B}_\sym, \mathbb{B}_\sym, \tau)$-circuit, then take $g_1,g_2 \in
%   G$ such that $\Sigma (g_1) \Sigma(g_2) \in \mathbb{B}_\matsym$.

\begin{definition}
  Let $\omega_1: A \times B \rightarrow H$ and $\omega_2: A' \times B'
  \rightarrow H'$ define the equivalence relation $\sim$ as $\omega_1 \sim
  \omega_2$ iff $\omega_1$ and $\omega_2$ have the same domain and co-domain and
  there exists $(\alpha, \beta) \in \sym_A \times \sym_B$ and for all $(a,b) \in
  A \times B$ we have $\omega_1 = \omega_2 \circ (\alpha, \beta)$.
\end{definition}

% \begin{definition}
%   Let $\omega_1: A \rightarrow H$ and $omega_2:A' \rightarrow H'$. Define the
%   equivalence relation $\sim$ as $\omega_1 \sim \omega_2$ iff $\omega_1$ and
%   $\omega_2$ have the same domain and co-domain, and
% \end{definition}

\begin{definition}[Automorphism]
  let $C = \langle G, W, \Omega, \Sigma, \Lambda, L\rangle$ be a
  $(\mathbb{B}_\sym, \mathbb{B}_\sym, \tau)$-circuit computing at $q$-ary query
  on structures of size $n$. Let $\sigma \in \sym_n$ and $\pi: G \rightarrow G$
  be a bijection such that
  \begin{itemize}
  \item for all gates $g, h \in G$, $W(g,h)$ iff $W(\pi g, \pi h)$,
  \item for all output tuples $x \in [n]^q$, $\pi \Omega (x) = \Omega (\sigma
    x)$,
  \item for all gates $g \in G$, $\Sigma (g) = \Sigma (\pi g)$, and
  \item for each relational gate $g \in G$, $\sigma \Lambda (g) = \Lambda (\pi
    g)$
  \item for each internal gate $g$ if $\Sigma (g) \in \mathbb(B)_\matsym$. Then
    we have that $L(\pi g) \sim \pi \circ L(g)$.
  \end{itemize}
\end{definition}

\begin{remark}
  I also need a new notion of robustness here in order to collapse gates that
  have the same children and equivalent labelings are properly regarded as being
  the same gate.
\end{remark}
% \begin{definition}
%   Let $C$ be a circuit, $g$ be a gate in $C$, $H$ be $g$'s set of children and
%   let $\omega: A \times B \rightarrow H$ be a surjection, where $A$ and $B$
%   are sets and $g$ takes $\vert A \vert \cdot \vert B \vert$ inputs. Call such
%   an $\omega$ a matrix labeling for $g$.
% \end{definition}

% \begin{definition}
%   Let $C$ be a circuit and $g_1$ and $g_2$ be gates in $C$. Let $\sigma \in
%   \sym_n$ be a permutation such
% \end{definition}

\begin{definition}
  Let $\omega$ be a matrix labeling for $g$. Define the matrix stabilizer for
  $\omega$, denoted by $\matstab(\omega)$, to be the set of all $\sigma \in
  \sym_n$ such that $\sigma H = H$ and there exists $(\alpha, \beta) \in \sym_A
  \times \sym_B$ such that for all $(i,j) \in A \times B$ we have that $\omega
  (\alpha i, \beta j) = \sigma \omega (i,j)$.
\end{definition}


\begin{definition}
  Let $g$ be a gate with matrix labeling $\omega$, $h,h' \in H$ and $\sigma \in
  \sym_n$. We say that a pair $(h, h')$ is compatible with $(\sigma, \omega)$ if
  $\sigma h, \sigma h' \in H$ and there exists $(\gamma_1, \gamma_2),
  (\gamma_1', \gamma_2') \in \sym_A \times \sym_B$ s.t.
  \begin{align*}
    &(\gamma_1, \gamma_2) \cdot \omega^{-1} (h) = \omega^{-1}(\sigma h), \text{ and} \\ 
    &(\gamma_1', \gamma_2') \cdot \omega^{-1} (h') = \omega^{-1}(\sigma h'),
  \end{align*}
  and for all $(i,j) \in \omega^{-1}(h)$, $(i',j') \in \omega^{-1}(h')$ we have
  that
  \begin{align*}
    &i =i' \implies \gamma_1(i) = \gamma_1'(i'), \text{ and} \\
    &j =j' \implies \gamma_2(j) = \gamma_2'(j').
  \end{align*}

\end{definition}

\begin{lem}
  Let $g$ be a gate, $\omega$ be a matrix labeling of $g$, $\sigma \in \sym_n$.
  Then $\sigma \in \matstab(\omega)$ iff for all $h,h' \in H$ we have that
  $(h,h')$ is compatible with $(\sigma, \omega)$.
\end{lem}

\begin{proof}
  $`\Rightarrow'$: We have that $\sigma \in \matstab(\omega)$ and so there
  exists $(\alpha, \beta) \in \sym_A \times \sym_B$ such that for all $(i,j) \in
  A \times B$ we have $\sigma \omega (i,j) = \omega (\alpha i, \beta j)$. From
  compatibility we have that $h,h' \in H$ Let $h, h' \in H$ . Let $(\gamma_1,
  \gamma_2) := (\gamma_1', \gamma_2') := (\alpha, \beta)$. This assignment is
  sufficient to prove the direction.

  $`\Leftarrow'$: Suppose for all $h,h' \in H$ we have $(\gamma_1, \gamma_2),
  (\gamma_1, \gamma_2') \in \sym_A \times \sym_B$ satisfying the above
  requirements. Notice that for a given $i \in A$ and any $j, j' \in B$, let $h
  = \omega(i,j)$ and $h' = \omega(i,j')$, then we have that $\gamma_1 (i) =
  \gamma_1'(i)$. It follows that we can define a $\alpha \in \sym_A$ by
  $\alpha(i) = \gamma_1 (i)$. Similarly we can define $\beta \in \sym_B$ by
  $\beta (j) = \gamma_2 (j)$.

\end{proof}

\begin{definition}
  Let $g$ be a gate with matrix labeling $\omega$. Let $(\sigma, h, h') \in
  \sym([n]) \times H^2$. Say that $(\sigma, h, h')$ is useful if $(h,h')$ is
  incompatible with $(\sigma, \omega)$.

  Say that two distinct pairs $(\sigma_1, h_1, h_1'), (\sigma_2, h_2, h_2') \in
  \sym([n]) \times H^2$ are mutually independent if
  \begin{itemize}
  \item $\sigma_2 h_1 = h_1$,
  \item $\sigma_2 \sigma_1 h_1 = \sigma_1 h_1$,
  \item $\sigma_2 h_1' = h_1'$,
  \item $\sigma_2 \sigma_1 h_1' = \sigma h_1'$,
  \end{itemize}
  We say that a set $S \subseteq \sym([n]) \times H^2$ is useful if each pair in
  it is useful. We say that S is independent if each pair of distinct pairs in
  $S$ are mutually independent.
\end{definition}

We denote the usual equivalence relation on $\sym_n$ from the (right) cosets of
$\matstab(\omega)$ by $\sim_\omega$.

\begin{lem}
  Let $\sigma_1, \sigma_1 \in \sym_n$ and suppose $\sigma_1 \sim_\omega
  \sigma_2$ then for any $(h,h') \in H^2$ we have that $(h,h')$ is compatible
  with $(\sigma_1, \omega)$ iff $(h,h')$ is compatible with $(\sigma_2,
  \omega)$.
\end{lem}

\begin{proof}
  Suppose we have that $\sigma_1 \sim_\omega \sigma_2$. And additionally
  supposed $(h, h')$ compatible with $(\sigma_1, \omega)$.

  It follows there exists $(\alpha, \beta) \in \sym_A \times \sym_B$ such that
  $\sigma_1 \omega (i,j) = \sigma_2\omega (\alpha i, \beta j)$, and we have
  $(\gamma_1, \gamma_2), (\gamma_1', \gamma_2') \in \sym_A \times \sym_B$
  satisfying the requirements of compatibility. We note that $(\alpha, \beta)
  \cdot \omega^{-1}(\sigma_1 h) = \omega^{-1}(\sigma_2 h)$. Thus by composition
  we have that $(\alpha \gamma_1, \beta \gamma_2) \cdot \omega^{-1}(h) =
  \omega^{-1}(\sigma_2 h)$ and $(\alpha \gamma_1', \beta \gamma_2') \cdot
  \omega^{-1}(h') = \omega^{-1}(\sigma_2 h')$, and clearly the remaining
  requirement for compatibility follows from the bijectivity of $\alpha$ and
  $\beta$. The result follows. The other direction of the implication follows
  from symmetry.
\end{proof}

\begin{claim}
  \label{claim:useful-independant-set}
  Let $g$ be a rank gate with labeling $\omega$ and child set $H$. Let $S$ be a
  useful and independent. We then have that $\vert \sym_n: \matstab(\omega)
  \vert \leq 2^{\vert S \vert}$.
\end{claim}

\begin{proof}
  Let $R \subseteq S$ and define $\sigma_R = \Pi_{(\sigma, h, h') \in R} \sigma$
  (with some arbitrary order assumed on $S$). Let $R$ and $Q$ be distinct
  subsets of $S$ and WLOG let $\vert R \vert \geq \vert Q \vert$. We want to
  show that we don't have $\sigma_R \omega \sim_\omega \sigma_Q \omega$. Pick
  any $(\sigma, h, h') \in R/Q \neq \emptyset$. Given that $\sigma_R h = \sigma
  h$ and $\sigma_R h' = \sigma h'$ it is easy to see that the usefulness of
  $(\sigma, h,h')$ implies the incompatibility of $(h,h')$ with with $(\omega,
  \sigma_R)$. Moreover, the fact that $\sigma_Q h = h$ and $\sigma_Q h' = h'$
  makes it easy to see $(h,h')$ is compatible with $(\omega, \sigma_Q)$. From
  the above lemma we may conclude that that we do not have $\sigma_R \sim_\omega
  \sigma_Q$, and the result follows.
\end{proof}

The following two lemmas proved by Anderson and Dawar \cite{} are both of use in
proving the following theorem.

\begin{lem}
  \label{lem:big-or-small}
  For any $\epsilon$ and $n$ such that $0 < \epsilon < 1$ and $\log n \geq
  \frac{4}{\epsilon}$, if $\mathcal{P}$ is a partition of $[n]$ with $k$ parts,
  $s = [\sym_n : \setstab (\mathcal{P})]$ and $n \leq s leq 2^{n^{1-\epsilon}}$,
  then $\min \{k, n-k\} \leq \frac{8}{\epsilon} \frac{\log s}{\log n}$.
\end{lem}

\begin{lem}
  \label{lem:small-means-support}
  For any $\epsilon$ and $n$ such that $0 < \epsilon < 1$ and $\log n \geq
  \frac{8}{\epsilon^2}$, if $\mathcal{P}$ is a partition of $[n]$ with $\vert
  \mathcal{P} \vert \leq \frac{n}{2}$, $s:= [\sym_n : \setstab (\mathcal{P})]$
  and $n \leq s \leq 2^{n^{1-\epsilon}}$, then $\mathcal{P}$ contains a part $P$
  with at least $n - \frac{33}{\epsilon} \cdot \frac{\log s} {\log n}$.
\end{lem}

If $g$ is a symmetric gate (i.e. the usual gates in a circuit) we note that
$\orb(g) = [\sym_n : \stab (g)]$ by the orbit-stabilizer theorem.

If $g$ is a matrix-symmetric gate then $\orb(g) = [\sym_n:\matstab (\omega)]$,
where $\omega$ is the matrix labelling associated with $g$.

\begin{thm}
  For any $\epsilon$ and $n$ such that $\frac{2}{3} \leq \epsilon \leq 1$ and $n
  \geq \frac{128}{\epsilon^2}$, if $C$ is a symmetric, rigid circuit on
  structures of size $n$ and $s := \max_{g \in C} \vert \orb (g)\vert \leq
  2^{n^{1-\epsilon}}$, then, $SP(C) \leq \frac{33}{\epsilon}\frac{log s}{log
    n}$.
\end{thm}

\begin{proof}
  It is easy to see that if $g$ is a gate in $C$ then $\stab (g) \subseteq
  \setstab(\SP(g))$, and so $s \geq \orb(g) = [\sym_n : \stab(g)] \geq [\sym_n :
  setstab(\SP(g))]$. Thus if $\vert \SP(g) \vert \leq frac{n}{2}$, then from
  Lemma \ref{lem:small-means-support}, we have $\| SP (g) \| \leq
  \frac{33}{\epsilon} \cdot \frac{\log s} {\log n}$. The result thus follows
  from showing that for each $g$ in $C$ we have that $\vert \SP (g) \vert \leq
  \frac{n}{2}$.
  
  % Include some detail here for constant and relational gates
  The cases where $g$ is a constant or relational gate are easy to handle.

  We now consider the case for internal gates. Let $g$ be the topologically
  first internal gate with $\vert \SP(g) \vert > \frac{n}{2}$. If $g$ is not a
  matrix-symmetric gate then the result follows from the argument presented by
  Anderson and Dawar \cite{}. Suppose $g$ is a matrix-symmetric gate, and
  suppose $g$ has a labelling $\omega$. We now argue that this leads to a
  contradiction.

  Let $k' := \lceil \frac{8 \log s}{\epsilon \log n} \rceil$. From the
  assumptions on $s, n$ and $\epsilon$ we have that $k' \leq
  \frac{1}{4}n^{1-\epsilon} < \frac{n}{2}$. Lemma \ref{lem:big-or-small} implies
  that $n - \vert \SP(g) \vert \leq k'$
  
  From Claim \ref{claim:useful-independant-set} it remains to show that we can
  construct a sufficiently large useful and independent set of gate-automorphism
  pairs $S$. Divide $[n]$ into $\lfloor \frac{n}{k' + 2} \rfloor$ disjoint sets
  $S_i$ of size $k' + 2$ and ignore the elements left over. It follows that for
  each $i$ there is a permutation $\sigma_i$ which fixes $[n] / S_i$ pointwise
  but moves $g$. Suppose there was no such $\sigma_i$ it follows that every
  every permutation that fixes $[n]/S_i$ pointwise fixes $g$. Thus the partition
  of all the singletons in $[n]/S_i$ and $S_i$ is a supporting partition. As
  $\SP(g)$ is the coarsest partition it follows that $\vert \SP(g) \vert \leq n
  - (k'+2) + 1 = n - k' - 1$, which contradicts the inequality $n - \vert \SP(g)
  \vert \leq k'$.

  Since $g$ is moved by each $\sigma_i$ and $C$ is rigid it follows that we
  don't have $\sigma_i \notin \matstab(\omega)$. Thus there exists $(h_i, h_i')$
  that is inconsistent with $(\sigma_i, \omega)$, and so the triple $(\sigma_i,
  h_i, h_i')$ is useful.

  Let $\SP (h)^*$ be the union of all parts of $\SP(h)$ except for the largest
  part. Let $Q_i = \SP(h_i)^* \cup \SP(\sigma_i h_i)^* \cup \SP (h_i')^* \cup
  \SP (\sigma_i h_i')^*$. Then note that if $\sigma_j$ fixes $Q_i$ then by
  construction, we have that $\sigma_j \in \stab_n(\SP(h_i)) \cap
  \stab_n(\SP(\sigma_i h_i)) \cap \stab_n(\SP(h_i')) \cap \stab_n(\SP(\sigma_i
  h_i'))$

  Define a graph $K$ with vertices given by the sets $S_i$ and an edge from
  $S_i$ to $S_j$ (with $i \neq j$) if $Q_i \cap S_j \neq \emptyset$. It follows
  then that if there is no edge between $S_i$ and $S_j$ then $(\sigma_i, h_i,
  h_i')$ and $(\sigma_j, h_j, h_j')$ are mutually independent. It remains to
  argue that $K$ has a large independent set. This is possible as the out-degree
  of $S_i$ in $K$ is bounded by
  \begin{align*}
    \vert Q_i \vert \leq \|SP(h_i) \| + \|SP(\sigma_i h_i) \| + \|SP(h_i') \| + \|SP(\sigma_i h_i') \leq 4 \cdot \frac{33\log s}{\epsilon \log n}
  \end{align*}. 

  This follows as the sets $S_i$ are disjoint and we may apply Lemma
  \ref{lem:small-means-support} to each of the child gates. It follows that the
  average total degree (in + out degree) of $K$ is at most $2 \cdot \vert Q_i
  \vert \leq 34 \cdot k'$. Now greedily select a maximal independent set in $K$
  by repeatedly selecting $S_i$ with the lowest total degree and eliminating it
  and its neighbours. This action does not affect the bound on the average total
  degree of $K$ and hence determines an independent set $I$ in $K$ of size at
  least
  \begin{align*}
    \frac{\lfloor \frac{n}{k' + 2} \rfloor}{34k' + 1} \geq \frac{n - (k'+2)}{34k'+1k'+2} \geq \frac{n\frac{7}{16}}{34k'^2 + 69k' +2} \geq \frac{n}{(16k')^2}.
  \end{align*}

  Take $S = \{(\sigma, h, h') : S_i \in I \}$. Then from the above argument we
  have that $S$ is useful and independent.
  
  Moreover, from Claim \ref{claim:useful-independant-set}, we have that $s \geq
  \vert \orb(g) \vert \geq 2^{\vert S \vert} \geq 2^{\frac{n}{(16k')^2}}$ then
  $n^{1-\epsilon} \geq \log s \geq n \cdot (\frac{128}{\epsilon}\frac{\log
    s}{\log n})^{-2} > n \cdot (n^{1-\epsilon})^{-2} = n^{2\epsilon -1} \geq
  n^{1-\epsilon}$. This is a contradiction.
\end{proof}


\section{Rigid Circuits}

\begin{definition}
  We say that a circuit $C$ has bijective labels if for each gate $g$ in $C$,
  $L(g)$ is a bijection.
\end{definition}

\begin{lem}
  \label{lem:bij_labels}
  There is an algorithm that runs in polynomial time that takes in a circuit $C$
  and outputs a circuit with unique gates $C'$. Moreover, if $C$ was symmetric
  then $C'$ is symmetric. If $C$ is rigid then $C'$ is rigid.
\end{lem}

\begin{proof}
  Let $S = 2*\vert C \vert$. Recurse through the gates of $C$ topologically and
  let $h$ be the next gate topologically. If for all $g \in W(h, \cdot)$ we have
  that $L\vert (g)^{-1}(h) \vert = 1$ continue on to the next gate. If not add
  in a tower of $S$ $\and$ gates such that $h \rightarrow \and^h_1 \rightarrow
  \ldots \rightarrow \and^h_S$ (i.e. we have a tower of $\and$ gates with $h$ as
  input to $\and^h_1$ and the output of each $\and^h_i$ connected to the input
  of each $\and^h_{i+1}$ for each $1 \leq i < S$). Now for each $g \in W(h,
  \cdot)$, if $L(g)^{-1}(h) = \{ s_0, \ldots, s_{r}\}$, for each $1 \leq i \leq
  r$ add in the wires $W(\and^h_i, g)$ and set $L(g)(s_{i}) = \and^h_i$. Now
  continue on to the next gate topologically and run the above algorithm.

  % First, notice that for a given gate $h$ and $k \in \mathbb{N}$, we can
  % define
  % a sub-circuit $h^k = 1 \and \cdots, 1 \and h$ (or rather a $k$-height tower
  % of
  % $k$ binary $\and$ gates with $h$ at the top and all other inputs set to
  % $1$).
  % We call $h$ the top of the sub-circuit $h^k$ and the bottom $and$ gate (the
  % output of the sub-circuit) the bottom gate. We may think of this as a
  % $k$-height copy of the gate $h$, in the sense that it has the same output as
  % $h$ and similar orbits. We now use different copies to distinguish gates
  % that
  % otherwise have the same labelling.

  % Let $h$ be the next gate topologically. Then let $r$ be maximal such that
  % $W(h,g)$ and $\omega_g^{-1}(h) = \{ s^g_1, s^g_2, \ldots, s^g_r \}$. Then
  % let
  % $k$ be the height of the highest tower with $h$ at the top and with it's
  % bottom connected to a gate in $W(h, \cdot)$. Then create a single
  % $k+r-1$-height tower by adding in the appropriate number of binary $\and$
  % gates below $h$.

  % For each $g \in W(h, \cdot)$ and for each $\omega_g^{-1}(h) = \{s^g_1,
  % s^g_2,\ldots , s^g_{r_g}\}$, set $L'(g)(s^g_1) = h$ each child starting with
  % the $and$-gate child to $h$ add in a wire from from the $i$th gate in the
  % tower to $g$ and set $L'(g)(s^g_{i+1}) = \and_i$, where $\and_i$ is the
  % $i$th
  % $\and$ gate in the tower starting with the $\and$ gate in the tower child to
  % $h$. There are enough gates in the tower as $r_g \leq r$.

  We call this updated circuit $C'$.
  
  Firstly, note that after running the above algorithm for each $g$ the
  labelling of $g$ will be a bijection. Moreover, it's easy to see that the
  output of each gate remains unchanged, and as such the output of the circuit
  is unchanged.

  Secondly, notice that the size of $C'$ is at most $2*\vert C \vert^2$, and
  note that the above algorithm runs in polynomial time.

  Now suppose that $C$ is symmetric. Let $\sigma$ be a permutation on the input
  universe and $\pi_C$ the induced automorphism on $C$. We now define $\pi$, the
  induced automorphism on $C'$. For each gate $g$ in $C'$, if $g$ is in $C$ then
  set $\pi(g) := pi_C(g)$. If $g$ is not in $C$ then $g$ must be some
  $\and^h_i$, for some $h$ in $C$. Then set $\pi(\and^h_i) :=
  \and^{\pi_C(h)}_i$. It is easy to see that $\pi$ is an automorphism, and $\pi$
  extends $\sigma$.

  Suppose that $C$ is rigid. It is easy to see that $C'$ will be rigid as well.

\end{proof}

% If $h$ is a binary $\and$ gate, then we say that $h$ has height $k+1$ if it
% has a $k$-height tower of binary $and$ gates connected above it, with a
% non-$\and$ gate $p$ connected to the final binary $\and$ gate, and all other
% inputs connected to $1$. We then say that $h$ is the base of a tower of height
% $k$ and type $p$.

% Then for each $h \in W_g$, if $h$ is not a binary $\and$ gate, the let $k$ be
% the largest integer such that there a gate $h' \in W(h, \cdot)$ which is the
% base of a tower of height $k$ and type $h$. it's output directly connected to
% one of the gates in $W(h^k, \cdot)$ (with k = 0 corresponding to just $h$).
% Then each $i \in \{1, \ldots , \vert \omega^{-1}(h)\vert\}$ add into the
% circuit $h^{k+i}$ and wire it to $g$. Notice that this step requires adding
% $\vert \omega^{-1}(h) \vert -1$ gates to the circuit. For some assumed
% ordering $\dom(\omega)$, let $\{s_0, \cdot, s_m\} = \omega^{-1}(h)$. Define
% $L'(g)(s_i) = h^{k+i}$. Now continue on to the next gate topologically where
% $L(g)$ is not a bijection, and execute the same process

\begin{lem}
  There is an algorithm that runs in polynomial time that takes in a circuit $C$
  and outputs a circuit $C'$ such that $C'$ is rigid and has bijective labels.
  Moreover, if $C$ was symmetric it follows that $C'$ will be symmetric.
\end{lem}

\begin{proof}
  First run the algorithm from Lemma \ref{lem:bij_labels} on $C$, and call the
  output circuit $C$.
  
  Recurse through the gates of $C$ topologically. For each internal gate $g$,
  for all $g'$ in $C$ such that $g \neq g'$, $W(\cdot, g) = W(\cdot, g')$, $W(g,
  \cdot) = W(g', \cdot)$, $\Sigma(g) = \Sigma(g')$, $L(g) \sim L(g')$ and
  $\Omega^{-1}(g) = \Omega^{-1}(g')$, delete $g'$ and for all $s \in L^{-1}(g')$
  set $L(s) := g$.

  Now re-run the algorithm from Lemma \ref{lem:bij_labels} on $C$ and output the
  result.
\end{proof}

  \begin{lem}
    Let $C = \langle G, W, \Omega, \Sigma, \Lambda, L \rangle$ be a $(\SB, \MB,
    \tau)-circuit$ on structures of size $n$. There is a deterministic algorithm
    which runs in Poly($\vert C \vert$) and outputs a rigid $(\SB, \MB, \tau)
    circuit$ $C'$ such that $G' = G$ and for any $g \in G$, and any input
    $\tau$-structure $\mathcal{A}$ and any bijection $\gamma$ from $A$ to $[n]$,
    $C[\gamma \mathcal{A}](g) = C'[\gamma \mathcal{A}](g)$ and if $C$ is
    symmetric then so is $C'$.
  \end{lem}

  \begin{proof}
  
  \end{proof}

  \section{Computing Supports}
  \begin{lem}
    Let $C$ be a rigid $(\SB, MB, \tau)$-circuit on structures of size $n$ and
    $\sigma \in \sym_n$. There is a deterministic algorithm which runs in time
    Poly($\vert C \vert$) and outputs for each gate $g$ its image under the
    automorphism $\pi$ induced by $\sigma$, if it exists.
  \end{lem}
  \begin{proof}
    The proof proceeds by recursively going through the circuit and building the
    mapping $\pi$ induced by $\sigma$.

    Suppose $g$ is a constant gate, then $\pi g := g$. Suppose $g$ is a
    relational gate, then there is at most one gate $g'$ such that $\Sigma (g) =
    \Sigma (g')$ and $\sigma\Lambda (g') = \Lambda (g)$. If such a $g'$ exists
    assign $\pi g := g'$, else terminate with failure.

    If $g$ is an symmetric internal gate then (from rigidity) there is at most
    one gate $g'$ such that $\Sigma (g) = \Sigma(g')$ and $W_{g'} = \pi W_g$.
    Assign $\pi g := g'$ if such a gate exists, or else terminate with failure.

    If $g$ is a matrix-symmetric internal gate then consider the set of gates
    $g'$ such that $g'$ has children $\pi W_g$ and $\Sigma(g) = \Sigma(g')$, and
    let $A \times B = \dom (Sigma(g))$. If no such gate $g'$ exists, terminate
    with failure. Define $\sigma_{\pi, g'}:A \times B \rightarrow A \times B$ by
    $\sigma_{\pi, g'} = \omega^{-1}_{g'} \pi \omega_{g}$. Then clearly
    $\omega_{g'} \sigma_{\pi, g'} = \pi \omega_{g}$, and it's easy to show that
    $\pi \omega_g \sim \omega_{g'}$ iff $\sigma_{\pi,g'} \in \sym_A \times
    \sym_B$. But this just involves checking that $\sigma$ acts as a bijection
    on $A$ and $B$ separately and, given that $\vert A \vert$ and $\vert B
    \vert$ are both bounded by $\vert C \vert$, the algorithm which just
    iterates through $A$ and $B$ is sufficient. If for every $g'$ it is found
    that $\pi_{\sigma,g'}$ is not in $\sym_A \times \sym_B$ then terminate with
    failure. If there is a $g'$ for which $\pi_{\sigma, g'} \in \sym_A \times
    \sym_B$ then it is unique by rigidity and so set $\pi g := g'$.

    If $g$ is an output gate, then check that for all tuples in $[n]^{q}$ we
    have that $\pi \Omega (x) = \Omega (\sigma (x))$, and terminate with failure
    if the condition is not met.

    If the algorithm has not terminated with failure, output the automorphism.

    The algorithm clearly runs in Poly($\vert C \vert$)
  \end{proof}

  \section{Evaluating Symmetric Circuits}
  We need to extend two Lemmas in this section.

\begin{lem}
  Let $g$ be a matrix-symmetric gate in $C_n$ with children $H$. Let $\alpha \in
  U^{\underbar{\sp(g)}}$, and suppose $\gamma_1, \gamma_2: U \rightarrow [n]$
  with $\gamma^{-1}_1 \sim \alpha$ and $\alpha \sim \gamma^{-1}_2$. Let $A
  \times B = \dom (L(g))$.

  For an input structure $\mathcal{A}$, let $L^0_\gamma: A \times B \rightarrow
  \{0,1\}$, where $L^0_\gamma (a,b):= C[\gamma \mathcal{A}] (L(g)(a,b))$.

  Then $L^0_{\gamma_1} \sim L^0_{\gamma_2}$.
\end{lem}

\begin{proof}
  We have that there exists a unique $\pi \in \sym_n$ such that $\pi \gamma_1 =
  \gamma_2$. Moreover, since $\gamma^{-1}_1$ and $\gamma^{-1}_2$ are both
  consistent with $\alpha$, it follows that $\pi$ must fix $\sp(g)$. Thus $L(g)
  \sim \pi L(g)$, and so there exists $(\sigma, \lambda)$ such that $\pi L(g) =
  L(g) (\sigma, \lambda)$.

  We then have that,
  \begin{align*}
    L^0_{\gamma_1} (a,b) &= C_n[\gamma_1 \mathcal{A}](L(g)(a,b))\\
                         & = C_n[\pi gamma_1 \mathcal{A}][\pi L(g)(a,b)] \\
                         & = C_n[gamma_2 \mathcal{A}][L(g)((\sigma, \lambda)(a,b))]\\
                         & = L^0_{\gamma_2} ((\sigma, \lambda) (a,b)),
  \end{align*}
  and it follows that $L^0_{\gamma_1} \sim L^0_{\gamma_2}$.

\end{proof}

\begin{lem}
  Let $g$ be a matrix symmetric gate in $C_n$ with children $H$. Let $\alpha \in
  U^{\underbar{\sp(g)}}$ and $\gamma: U \rightrrow [n]$ such that
  $gamma^{-1}\sim \alpha$. Let $A \times B = \dom (L(g))$ and define
  $L^{EV}_\alpha: A \times B \rightarrow \{0,1\}$ by
  \begin{align*}
    L^{EV}_\alpha (a,b):=
    \begin{cases*}
      1 & if $A_h \cap \EV_h \neq \emptyset$ 0 & otherwise
    \end{cases*},
  \end{align*}
  where for $h \in W_g, A_h:= \{\beta \in U^{\underline{\sp(h)}} \vert \alpha
  \sim \beta\}$. Then $L^0_\gamma = L^{EV}_\alpha$ and $\alpha \in \EV(g)$ iff
  $\rk^u_p(L^0_\gamma)$
\end{lem}
\begin{proof}
  
\end{proof}

\begin{lem}
  Let $g$ be a matrix-symmetric gate in $C_n$ with children $W_g$. $Let \alpha
  \in U^{\underbar{\sp(g)}}$ and let $\gamma: U \rightarrow [n]$ such that
  $\gamma^{-1} \sim \alpha$. Let $A \times B = \dom (L(g))$, and let
  $f:\{0,1\}^{A \times B} \rightarrow \{0,1\}$ be the matrix-symmetric function
  associated with $g$. Define $L^{EV}_\alpha: A \times B \rightarrow \{0,1\}$ by
  \begin{align*}
    L^{EV}_\alpha (a,b):=
    \begin{cases*}
      1 & if $A_h \cap \EV_h \neq \emptyset$ 0 & otherwise
    \end{cases*},
  \end{align*}
  where for $h \in W_g, A_h:= \{\beta \in U^{\underline{\sp(h)}} \vert \alpha
  \sim \beta\}$. Then $\alpha \in \EV_g$ iff $f(L^\EV_\alpha)$.
\end{lem}
\begin{proof}
  First note that if $\alpha \in U^{\underbar{\sp{g}}}$ then $\alpha \in \EV_g$
  iff we have that $C_n[\gamma \mathcal{A}](g)= f (L^0_\gamma) = 1$. Notice that
  from Lemma \ref{} that this holds for any particular $\gamma$ iff it holds for
  all such $\gamma$.

  We now show that $L^\EV_\alpha = L^0_\gamma$, and so complete the proof.

  Suppose for $(a,b) \in A \times B$ we have that $L^\EV_\alpha(a,b) = 1$. Let
  $h:= L(g)(a,b)$. It follows that $A_h \cap \EV_h \neq \emptyset$, i.e there
  exits $\beta \in \sp(h)$ such that $\beta \sim \alpha$ and there exists
  $\gamma_h \in \Gamma(h)$ such that $\gamma^{-1}_h \sim \beta$. It follows from
  transitivity and Lemma \ref{} that $\gamma_h \sim \alpha$ and so
  $L^0_\alpha(a,b) = C_n[\gamma \mathcal{A}](h) = C_n[\gamma_h \mathcal{A}](h) =
  1$.


  Now suppose that $L^0_\gamma (a,b) = 1$
\end{proof}


\section{Translating to Formulas of FPC}
Let $\mathcal{C} = (C_n)_{n \in \mathbb{N}}$ be a P-uniform family of
polynomial-size symmetric $(\SB, \MB, \tau)$ circuits, and where $\MB$ is the
rank basis. It remains to show that there is a formula $Q$ in the vocabulary
$\tau \uplus \{\leq\}$ such that for any $n$ and $\tau$-structure $\mathcal{A}$
with a universe $U$ of carnality $n$, the $q$-ary query defined by $C_n$ on the
input $\mathcal{A}$ is defined by the formula $Q$ when interpreted in the
structure $\mathcal{A}^\leq = \mathcal{A} \uplus \langle [n], \leq \rangle$.

Since $\mathcal{C}$ is $P$-uniform, and from Lemmas \ref{} and \ref{} and the
Immerman-Vardi theorem, we have an FP interpretation defining a rigid symmetric
circuit with bijective labels equivalent to $C_n$ (which we also call $C_n$)
over the the number sort of $\mathcal{A}^\leq$, where $\Phi = (\phi_G,\phi_W,
\phi_\Omega, (\phi_s)_{s \in \SB \uplus \{rank\} \uplus \tau \uplus \{0,1\}},
(\phi_{\and_R})_{R \in \tau}, \phi_L)$. We note that $Phi$ is a $t$-width
interpretation over the universe $[n]$. So if $\mu, \nu, \nu_1, \ldots \in
[n]^t$, the formulas are defined such that:

\begin{itemize}
\item $\phi_G(\mu)$ holds iff $\mu$ encodes a gate
\item $\phi_W(\nu,\mu)$ holds iff $\nu$ and $\mu$ encode gates and if $g_{\nu}$
  and $g_{\mu}$ are these two encoded gates respectively then $W(g_{\nu},
  g_{\mu})$.
\item $\phi_{\Omega}(\nu_1, \ldots, \nu_q, \mu)$ holds iff $\nu_1, \ldots, nu_q$
  each encode numbers less than or equal to $[n]$ and $\Omega(\nu_1, \ldots,
  \nu_q) = g_\mu$.
\item For all $s \in \SB \uplus \{rank\} \uplus \tau \uplus \{0,1\}$ we have
  that $\phi_s (\mu)$ holds iff $\mu$ encodes a gate such that $\Sigma(g_\mu) =
  s$ (or, in the case that $s = \rank$, we require that $\Sigma(g_\mu)$ is a
  rank symbol.)
\item For all $R \in \tau$ with arity $r$ we have that $\phi_{\and_R}(\nu_1,
  \ldots, \nu_r, \mu)$ holds iff $\nu_1 , \ldots , \nu_r$ encodes numbers less
  than or equal to $[n]$, $\mu$ encodes a gate such that $\Sigma(g_\mu) = R$ and
  $\Lambda_R(g_\mu) = (\nu_1, \ldots, \nu_r)$.
\item $\phi_L(\nu, \mu, \nu_1, nu_2)$ holds iff $\nu$ and $\mu$ encode gates and
  $\nu_1, \nu_2$ encode numbers, and $L(g_{\mu})(\nu_1,\nu_2) = g_\nu$. If
  $\dom(L (g_\mu)) = [a] \times [b]$, then the encoding of $[a]$ and $[b]$ are
  initial segments under the lithographically induced order. Moreover, the
  encodings of $[a]$ and $[b]$ preserve the order relation.
\end{itemize}

From now on we use $\mu$ and $\nu$ to stand for $t$-length sequences encoding
gates, and use $\nu_1, \ldots$ for sequences that encode numbers. We use $g_\mu$
and $g_\nu$ for the gates encoded by these sequence, while simply identifying
the sequence $\nu_i$ with the number it encodes.

Using the Immerman-Vardi Theorem, we can define an FP($\leq$) formula
$\rankpara$ such that $\langle [n], \leq\rangle \models \rankpara[g,p,u]$ iff
$\langle [n], \leq\rangle \models \phi_G[g] \land \phi_{\rank}[g]$ and the rank
gate $g$ (as per the interpretation) computes the rank over characteristic $p$
and has threshold $u$.

We can define FP($\leq$) formulas $\max_A$ and $\max_B$ such that $\langle [n],
\leq \rangle \models \max_A[g, \nu_1]$ iff $\nu_1$ encodes $a$, where
$\dom(L(g)) = [a]\times [b]$. Similarly, $\langle [n], \leq \rangle \models
\max_B[g, \nu_1]$ iff $\nu_1$ encodes $b$ where $\dom(L(g)) = [a] \times [b]$.

Again using Lemma \ref{} and the Immerman-Vardi theorem, we can construct a
formula $\supp$ such $\langle [n], \leq \rangle \models \supp[g,u]$ iff $\langle
[n], \leq \rangle \models \phi_G[g]$ and $u$ is in $sp(g)$. This formula can be
used as in \cite{} to inductively define $\supp_i(g,u)$ for each $i \in [n]$
which holds iff $u$ is the $i$th element of the support of $g$.

Define the $\agree$ and $\theta_s$ formulas for all $s \in \SB \uplus \tau
\uplus \{0,1\}$ as in \cite{}.

Now we define the formula
\begin{align*}
  \theta_{\rank}:= (\mu, \bar{x}) := \bigwedge_{1 \leq i < j \leq k} x_i
  \neq x_j \wedge \forall \bar{y} ( [ \rank (x \leq \phi_{mr}, y \leq
  \phi_{mc}, \pi \leq r). \theta^{'} ] ),
\end{align*}
\begin{align*}
  \theta^{'}(a,b):= \exists \nu (W(\nu, \mu) \wedge \agree (\mu,\nu, \bar{x}, \bar{y}) \wedge \phi_L (\mu, \nu, a, b) \wedge V(\nu, y))
\end{align*}


%%%%%%%%%%%%%%%%%%%%%%%%%%%%%%%%%%%%%%%%%%%%%%%%%%%%%%%%%%%%%%%%%%%%%%%%%%%%%%%%%%%%%%%%%%
%%%%%%%%%%%%%%%%%%%%%%%%%%%%%%%%%%%%%%%%%%%%%%%%%%%%%%%%%%%%%%%%%%%%%%%%%%%%%%%%%%%%%%%%%%
%%%%%%%%%%%%%%%%%%%%%%%%%%%%%%%%%%%%%%%%%%%%%%%%%%%%%%%%%%%%%%%%%%%%%%%%%%%%%%%%%%%%%%%%%%
%%%%%%%%%%%%%%%%%%%%%%%%%%%%%%%%%%%%%%%%%%%%%%%%%%%%%%%%%%%%%%%%%%%%%%%%%%%%%%%%%%%%%%%%%%

\chapter{October 27th: Counting Gates Simulate Symmetric Gates}

In this chapter we prove that symmetric circuits with majority gates are at
least as powerful as symmetric circuits over any other Boolean basis consisting
of symmetric functions.

Recall the following that $\mathbb{B} = \{ \neg , \wedge , \lor \}$ and
$\mathbb{B}_\maj = \{ \maj \} \cup \mathbb{B}$.

Let $F: \{0,1\}^* \rightarrow \{0,1\}$ be a symmetric Boolean function. Since
$F$ is symmetric we note that for a fixed size input the output of $F$ is
entirely determined by the number of 1's in its input. Then let
$c_{F}:\mathbb{N} \rightarrow 2^{\mathbb{N}}$ define a function where $c_{F}(n)$
is the set of all $m \leq n$ such that for all $\vec{x} \in \{ 0,1 \}^n$ with
$m$ 1's we have $F (\vec{x}) = 1$. Clearly a symmetric Boolean function $F$ is
entirely determined by $c_{F}$.
 
\begin{prop}
  \label{prop:fuctions-maj}
  There is a polynomial $p(k)$ such that for any symmetric function $F$ and a
  given $k \in \mathbb{N}$ there is a circuit $C_k$ on $k$ inputs over the basis
  $\mathbb{B}_\maj$ which is symmetric, constant depth and with width bounded by
  $p(k)$.
\end{prop}

\begin{proof}
  We define the circuit $C_k$ for inputs $\vec{x} = ( x_1, \ldots, x_k )$.

  For $a \in \mathbb{N}$ we define a get $\countgate_a$ by
  \begin{align*}
    &\countgate_a = \maj (x_1, \ldots, x_k, \underbrace {0, \ldots, 0}_{2a -
      k}) \land \neg \maj (x_1, \ldots , x_k, \underbrace{0, \ldots,  0}_{2a - k + 2})\text{ if $a \geq \frac{k}{2}$,} \\
    &\countgate_a = \maj (x_1, \ldots, x_k, \underbrace {1, \ldots, 1}_{x -
      2a}) \land \neg \maj (x_1, \ldots , x_k, \underbrace{1, \ldots,
      1}_{k - 2a -2}) \text{ if $a < \frac{k}{2}$. }
  \end{align*}

  Then let $g = \bigvee_{a \in c_{F_i}(k)}\countgate_a$ and let $C_{k}$ be the
  circuit with input gates labeled by $\vec{x}$ and output gate $g$.

  t is easy to see that $C_k$ is constant depth and it's width is a polynomial
  in $k$. We have that in each layer of $C_{g'}$ each gate is connected to all
  gates in the previous layer, and as such the circuit is symmetric.
\end{proof}

The above proposition has a straight forward application to circuit
characterizations.

\begin{prop}
  Let $F = \{F_i : i \in I \}$ be a family of symmetric Boolean functions where
  $F_i: \{0,1\}^* \rightarrow \{ 0,1 \}$.

  Let $(C_n)_{n \in \mathbb{N}}$ be family of symmetric circuits over the
  Boolean basis $\mathbb{B} \cup F$, where $C_n$ is a circuit on structures of
  size $n$, and the size of each circuit in the family is bounded by some
  function $f(n)$. Then there exists a polynomial $q(n)$ and a family of
  symmetric circuits $(C_n')_{n \in \mathbb{N}}$ over $\mathbb{B}_\maj$, where
  $C_n'$ is a circuit on structures of size $n$ and $\vert C_n' \vert \leq
  q(f(n))$.
\end{prop}

\begin{proof}
  From $C_n$ we construct $C_n'$ in the obvious way. For each gate $g \in C_n$
  of type $F_i$ we have a symmetric circuit $C_g$ from Proposition
  \ref{prop:function-maj} that computes the same function as $g$. Then let
  $C_n'$ be $C_n$ but with each gate $g$ replaced by $C_g$. It is easy to see
  that $C_n'$ is symmetric. We also have that each gate $g$ must have at most
  $f(n)$ inputs, and the size of $C_g$ is bounded by $p(f(n))$. Thus the size of
  $C_n'$ is bounded by $f(n)p(f(n))$.
\end{proof}

\chapter{October 29th: Symmetric Circuits and CPT}

We already have that that the set of queries decidable by symmetric
circuits (respectively with counting) is exactly the set of queries
expressible in Fixed-Point Logic. We know that the CFI query is not
expressible in FPC, and so there is no symmetric circuit that can
compute the query. 

In our attempts to find a circuit characterisation of CPT, we might try and weaken the requirement of symmetry in our notion of symmetric circuits in order to express the CFI query (and perhaps the whole of CPT). One approach would be to weaken the requirement of having small orbits, to rather only having small orbits with respect to our particular input structure. We formalise this notion below.


\begin{definition}
Let $C$ be a circuit on structures of size $n$. Let $g$ be a gate in $C$ and let $X$ be a set of input strings closed under the action of the symmetric group. The $inv_X = \{ \sigma \in \sym_n : \forall x \in X g \circ H (x) = g \circ H (\sigma x)$.
\end{definition}
\begin{definition}
  Let $(C_n)_{n \in \mathbb{N}}$ be a family of circuits. For any
  finite structure $\mathcal{A}$ let $X_{\mathcal{A}}$ be the set of
  all string encodings of $\mathcal{A}$. We say a circuit family has
  small isomorphism orbits if there is a polynomial $p(n)$ such that
  for any sufficiently large $n$, finite structure $\mathcal{A}$ of
  cardinality $n$ and any $g \in C_n$ we have
  $\vert S_n : \inv_{X_{\mathcal{A}}}(g) \vert \leq p(n)$.
\end{definition}

\begin{thm}
  \label{thm:simple_group}
  Let $n \in \mathbb{N}$, $n \geq 5$. Let $r \in \mathbb{N}$ such that
  $r \leq n/2$. Suppose that $G$ is a subgroup of $\sym(n)$, then
  there is a constant $k \in mathbb{N}$ such that for all $n \geq k$
  if $\vert \sym(n):G \vert \leq {{n}\choose {r}}$ then there exists
  $X \subseteq [n]$ with $\vert X \vert \leq r$ and such that
  $\alt([n] \setminus X) \leq G$.
\end{thm}

The following result shows that even this weakened definition fails to
capture CPT.
\begin{thm}
  There is no family of circuits with small isomorphism orbits that
  expresses the CFI query.
\end{thm}
\begin{proof}
  Let $(G_n)_{n \in \mathbb{N}}$ be a family of ordered graphs such that each $G_n$ has
  odd and even CFI graphs $\mathfrak{G}$ and $\tilde{\mathfrak{G}}$ respectively, each of
  cardinality $n$. Furthermore, suppose the CFI query for $(G_n)$ is not expressible in
  FPC (the existence of such a family is guaranteed by
  \cite{}). Suppose there exists a family of circuits $(C_n)_{n \in \mathbb{N}}$ with
  small isomorphism orbits that decides the CFI query for $(G_n)$.
  Then we have that $C_n(\mathfrak{G}) = 1$ and
  $C_n(\tilde{\mathfrak{G}}) = 0$, and for any $g \in C_n$

  \begin{align*}
    \vert S_n : \inv_{X_{\mathcal{G}}}(g) \vert &\leq p(n),\\
    \vert S_n : \inv_{X_{\tilde{\mathcal{G}}}}(g) \vert &\leq p(n)
  \end{align*}.
  Let $k$ be such that $p(n) \leq n^k$ for all sufficiently large $n$. Then we have that $p(n) \leq n^k \leq {{n}\choose{k+1}}$ for all sufficiently large $n$, and (again, taking $n$ large enough) we have that $k \leq n/2$. Taking $n$ larger than the constant in Theorem \ref{thm:simple_group} it follows that there exists an $X_1, X_2 \subseteq [n]$ such that $\vert X_1 \vert, \vert X_2 \vert \leq k$ and $\alt ([n] \setminus X_1) \subseteq  \inv_{X_{\mathfrak{G}}}(g)$, $\alt ([n] \setminus X_2) \subseteq  \inv_{X_{\tilde{\mathfrak{G}}}}(g)$. It follows that
  \begin{align*}
    \alt ([n] \setminus (X_1 \cup X_2)) &\subseteq \alt ([n] \setminus X_1) \cap \alt ([n] \setminus X_2)\\
                                        &\subseteq \inv_{X_{\mathfrak{G}}}(g) \cap \inv_{X_{\tilde{\mathfrak{G}}}}(g) \\
                                        &\subseteq \inv_{X_{\mathfrak{G} \cup \tilde{\mathfrak{G}}}}(g) =  \inv_{CFI(G)}(g),
  \end{align*}
  where $CFI(G)$ is set of all encodings of all CFI graphs of $G_n$.

  We then have that
  \begin{align*}
    \vert S_n : \inv_{CFI(G)}(g) \vert \leq \vert S_n : \alt ([n] \setminus (X_1 \cup X_2)) \leq  \frac{n!}{(n-2k)!/2} \leq n^{2k+1},
  \end{align*}
  with the last inequality following for large enough $n$. We thus
  have from Theorem \ref{} that there is a symmetric circuit $D_n$
  that decides the CFI problem for $G_n$, and hence a family $(D_n)$
  that decides the CFI problem for the family of graphs $(G_n)$. From
  the theorem of Dawar et al. \cite{}, it follows that there is as
  formula of FPC that decides the CFI query for $(G_n)$ contradicting
  our assumption.
\end{proof}
\end{document}