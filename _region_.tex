\message{ !name(paper.tex)}\documentclass[12pt]{report}

% This first part of the file is called the PREAMBLE. It includes
% customizations and command definitions. The preamble is everything
% between \documentclass and \begin{document}.

\usepackage[margin=1in]{geometry} % set the margins to 1in on all sides
\usepackage{graphicx} % to include figures
\usepackage{amsmath} % great math stuff
\usepackage{amsfonts} % for blackboard bold, etc
\usepackage{amsthm} % better theorem environments


% various theorems, numbered by section

\newtheorem{thm}{Theorem} \newtheorem{claim}{Claim}
\newtheorem{remark}{Remark} \newtheorem{definition}{Definition}
\newtheorem{lem}[thm]{Lemma} \newtheorem{prop}[thm]{Proposition}
\newtheorem{cor}[thm]{Corollary} \newtheorem{conj}[thm]{Conjecture}

\DeclareMathOperator{\id}{id}

\newcommand{\bd}[1]{\mathbf{#1}} % for bolding symbols
\newcommand{\RR}{\mathbb{R}} % for Real numbers
\newcommand{\ZZ}{\mathbb{Z}} % for Integers
\newcommand{\MB}{\mathbb{B}_{\matsym}} % for Integers
\newcommand{\SB}{\mathbb{B}_{\sym}} % for Integers
\newcommand{\col}[1]{\begin{matrix} #1 \end{matrix} \right]}
\newcommand{\comb}[2]{\binom{#1^2 + #2^2}{#1+#2}}
\newcommand{\stab}{\text{\textbf{Stab}}}
\newcommand{\setstab}{\text{\textbf{SetStab}}}
\newcommand{\matstab}{\text{\textbf{MatStab}}}
\newcommand{\inv}{\text{\textbf{inv}}}
\newcommand{\aut}{\text{\textbf{Aut}}}
\newcommand{\kernal}{\text{\textbf{ker}}}
\newcommand{\alt}{\text{\textbf{Alt}}}
\newcommand{\sym}{\text{\textbf{Sym}}}
\newcommand{\rank}{\text{\textbf{rk}}}
\newcommand{\orb}{\text{\textbf{Orb}}}
\newcommand{\SP}{\text{\textbf{SP}}}
\newcommand{\maj}{\text{\textbf{Maj}}}
\newcommand{\dom}{\text{\textbf{Dom}}}
\newcommand{\child}{\text{child}}
\newcommand{\countgate}{\text{count}}

\newcommand{\Alpha}{A}
\newcommand{\matsym}{\text{\textbf{MatSym}}}
\begin{document}

\message{ !name(paper.tex) !offset(540) }
\section{Translating to Formulas of FPC}
Let $C = (C_n)_{n \in \mathbb{N}}$ be a P-uniform family of polynomial-size
symmetric $(\SB, \MB, \tau)$ circuits, and where $\MB$ is the rank basis.

It remains to show that there is a formula $Q$ when interpreted in the structure
$\mathcal{A}^\leq \uplus \langle [n], \leq \rangle$. Since $C$ is p-uniform, by
the Immerman-Vardi theorem and Lemma \ref{}, we have an FP interpretation
defining a rigid symmetric circuit equivalent to $C_n$ (also called $C_n$) over
the number sort of $\mathcal{A}^\leq$, where $\Phi = (\phi_G,\phi_W,
\phi_\Omega, (\phi_s)_{s \in \SB \uplus \{rank\} \uplus \tau \uplus \{0,1\}},
(\phi_{\and_R})_{R \in \tau}, \phi_L)$, where:
\begin{itemize}
\item $\phi_G(\bar{x})$ holds iff $x$ encodes a gate
\item $\phi_W(\bar{x},\bar{y})$ holds iff $\bar{x}$ and $\bar{y}$ encode gates
  and if $g_{\bar{x}}$ and $g_{\bar{y}}$ are these two encoded gates
  respectively then $W(g_{\bar{x}}, g_{\bar{y}})$.
\item $\phi_{\Omega}(\bar{x},\bar{y})$
\item For all $s \in \SB \uplus \{rank\} \uplus \tau \uplus \{0,1\}$ we have
  that $\phi_s (\bar{x})$ holds iff $\bar{x}$ encodes a gate with a labelling
  corresponding to $s$.
\item For all $R \in \tau$ we have $\phi_{\and_R}(\bar{x}, \bar{y})$ holds iff
\item $\phi_L(\bar{x}, \bar{y}, u, v)$ holds iff $L(g_{\bar{x}})(u,v) =
  g_{\bar{y}}$.
\end{itemize}

Moreover, note that for any such circuit we could check if a given gate $g$ is a
rank gate, and check $\Sigma(g)$ in polynomial time. Thus, using the
Immerman-Vardi Theorem, we can define a formula $\text{CHAR}$ such that $\langle
[n], \leq\rangle \models \text{CHAR}[g,p,u]$ iff $\langle [n], \leq\rangle
\models \phi_G[g] \land \phi_{rank}[g]$ and the rank gate $g$ (as per the
interpretation) is over characteristic $p$ and has threshold $u$.

Similarly, using Lemma \ref{} and using the Immerman-Vardi theorem, we can
construct a formula $\text{SUPP}$ such $\langle [n], \leq \rangle \models
\text{SUPP}[g,u]$ iff $\langle [n], \leq \rangle \models \phi_G[g]$ and $u$ is
in $sp(g)$. This formula can be used as in \cite{} to inductively define
$\text{SUPP}_i(g,u)$ for each $i \in [n]$ which holds iff $u$ is the $i$th
element of the support of $g$.

Define the $\text{AGREE}$ and $\theta_s$ formulas for all $s \in \SB \uplus \tau
\uplus \{0,1\}$ as in \cite{}.

Now we define the formula
\begin{align*}
  theta_{rank}(\mu, \bar{x}) := \bigwedge_{1 \leq i < j \leq k} x_i
  \neq x_j \wedge \forall \bar{y} \( \[\rk (x \leq \phi_{mr}, y \leq
  \phi_{mc}, \pi \leq r) \theta^' \]\),
  \end{\align*}

      \begin{align*}
        \theta(a,b):= \exists \nu \(W(\nu, \mu) \and \agree (\mu,\nu, \bar{x}, \bar{y}) \and \phi_L (\mu, \nu, a, b) \and V(\nu, y)\) 
      \end{align*}


%%%%%%%%%%%%%%%%%%%%%%%%%%%%%%%%%%%%%%%%%%%%%%%%%%%%%%%%%%%%%%%%%%%%%%%%%%%%%%%%%%%%%%%%%%
%%%%%%%%%%%%%%%%%%%%%%%%%%%%%%%%%%%%%%%%%%%%%%%%%%%%%%%%%%%%%%%%%%%%%%%%%%%%%%%%%%%%%%%%%%
%%%%%%%%%%%%%%%%%%%%%%%%%%%%%%%%%%%%%%%%%%%%%%%%%%%%%%%%%%%%%%%%%%%%%%%%%%%%%%%%%%%%%%%%%%
%%%%%%%%%%%%%%%%%%%%%%%%%%%%%%%%%%%%%%%%%%%%%%%%%%%%%%%%%%%%%%%%%%%%%%%%%%%%%%%%%%%%%%%%%%

\message{ !name(paper.tex) !offset(683) }

\end{document}