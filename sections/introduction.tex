\documentclass[../paper.tex]{subfiles}
\begin{document}
When we study algorithmic complexity we often study complexity classes defined
in terms of some restricted class of machines. Circuit complexity, where one
makes use of similarly restricted families of Boolean circuits rather than
machines, has long served as an alternative approach. These circuits models have
a natural attraction for researchers, as their simpler structure seems to make
them more amendable to analysis and their connections with machine models
provides an alternative avenue for better understanding fundamental problems in
complexity theory, including the $\PT$ vs $\NP$ problem.

Descriptive complexity instead concerns the characterisation of complexity
classes in terms of logics. The central open question in the field -- whether
there is a logical characterisation of the polynomial-time decidable properties
of finite relational structures -- essentially asks if there is a logic the
sentences of which describe all (and only) polynomial-time decidable
isomorphism-closed classes of graphs (i.e. every \emph{graph property} in
$\PT$). As in the algorithmic complexity case, the use of circuits in
descriptive complexity has proved useful, with Otto \cite{Otto1997} and
Denenberg et al.~\cite{DENENBERG1986216} developing circuit-based
characterisations for logics. These characterisation are given in terms of
\emph{symmetric circuits}, circuits that take in finite structures over some
universe and where permutations of the universe of the input structure are
reflected by automorphisms of the circuit.

% In this way symmetric circuits provide a natural class from which to develop
% circuit characterisations for logics, taking in abstract structures and
% ensuring that isomorphism (permutations) on the input structure are reflected
% in the circuit structure, which consequently implies that the function
% computed by the circuit is \emph{invariant} (i.e. computes a graph property)
% makes symmetric circuits a natural class from which to develop circuit
% characterisations of logics.

Anderson and Dawar \cite{AndersonD17} developed a similar
circuit-characterisation for $\FPC$, proving that families of P-uniform families
of symmetric circuits defined over a complete Boolean basis with majority gates
completely characterises \emph{fixed-point logic with counting} ($\FPC$).

While it is known that $\FPC$ does not capture $\PT$ \cite{Cau1992}, $\FPC$ does
capture $\PT$ over many classes of structures \cite{Grohe2010} and many natural
$\PT$ problems are expressible in $\FPC$ (e.g.\ Maximum Matching
\cite{Anderson2013}). While $\FPC$ is thus an important logic of interest for
its own sake, it also plays a role as a logic from which other important logics
of study are defined. The particular logic of interest in this paper is
\emph{fixed-point logic with rank} ($\FPR$), introduced by Dawar et
al.~\cite{Dawar09logicswith}, and is defined as an extension of $\FPC$ that
includes operators that can define the rank of a definable matrix over a fixed
finite field. This extension was prompted by the work of Atserias et
al.~\cite{Atserias20091666}, who had shown that that some of the most difficult
to express problems in $\PT$ (that are not expressible in $\FPC$) were reducible
to the general problem of solving systems of linear equations over prime fields.
Indeed, problems of this sort have remained the primary source of
difficult-to-express problems in descriptive complexity, motivating the modern
study of $\FPR$.

It is worth noting the logic defined by Dawar et al.\cite{} defines for each
prime a separate operator that computes the rank of the matrix over that prime.
It was shown by Gr\"{a}del and Pakusa \cite{GradelP15a} that this logic does not
capture $\PT$. Instead, they present a strictly more expressive alternative rank
logic which includes only a single rank operator that takes in both the matrix
and a specific prime over which to compute the rank as input. In this paper when
we refer to $\FPR$ we are referring to this more expressive version of
Gra\"{a}del and Pakusa.

In this paper we develop a circuit characterisation of $\FPR$ in terms of
symmetric circuits that have gates that compute rank. Concretely, a circuit
$C_n$ for structures of size $n$ is a directed acyclic graph (DAG) whose
internal gates are marked by elements of some basis. This basis consists of
operations or, more concretely, Boolean functions such that a gate $g$ in $C_n$
with $k$ input gates is labelled by a function $f: \{0,1\}^A \rightarrow
\{0,1\}$, where $\vert A \vert = k$. When evaluating the circuit, $g$ is
computed by evaluating $f$ with the input given by the binary string formed by
evaluating the gates input to $g$, implicitly mapping $A$ to the input gates of
$g$. However, the output of $f$ in general depends on the order (or structure)
of the binary string taken as input, and as such the evaluation of $g$ depends
on which gates are assigned to each element of $A$.

This point is often ignored in circuit complexity by ensuring that functions in
the Boolean basis are \emph{symmetric}, i.e. depend only on the number of 1's in
it's input, rendering the particular assignment from $A$ to the input gates
irrelevant. This is the approach taken by Anderson and Dawar \cite{AndersonD17}.
However, functions that compute an operation like rank, i.e. a function that
takes in a matrix with Boolean entries and outputs 1 iff that matrix, when
interpreted in $\mathbb{F}_p$, has rank at least $r$, certainly depend on how
the input is formed into a matrix, and as such is not a symmetric function.

We introduce the notion of a \emph{structured function}, which allows us to both
index inputs appropriately and introduce appropriate generalisations of symmetry
depending on the structure of the input. Each such function $F$ is uses a many-sorted
structure, with some signature $\tau_F$, in order to index the input. This
definition allows us to generalise the notion of symmetry to a particular
labelling, and so introduce the notion of a \emph{$\tau$-symmetric functions}.

We also introduce a generalised symmetric circuit allows, which allows for
$\tau$-symmetric functions in the Boolean basis, not just symmetric ones. The
circuit also provides a function $L$ that maps to each internal gate $g$ to a
labelling on its input corresponding the labelling on the input of the Boolean
function that labels this gate.

\begin{remark}
  I'm not sure how much detail to go into here, this could be a major
  digression.
\end{remark}



% That is, eacFor example, if we should like The circuit $C_n$ associates with
% each internal gate $g$ a structure $\dom (g)$, and each gate input a sequence of
% elements from each sort $\dom(g)$ in accordance with the sorts of the

% of definition includes an additional function which assosiates for each gate $g$
% a


% require that the Basis over which the circuit is defined are symemtric.. We
% begin by generalising the notion of a symmetric circuit. A symmetric circuit, as
% presented in \cite{AndersonD17}, and in many parts of complexity theory,
% supposes that the in Boolean function that labels in the gate is
% \emph{symmetric} that labels a particular gate We first generalise the notion of
% a symmetric circuit so as to allow for gate


% In this general notion of a symmetric circuit, a Boolean basis consists not just
% of a set of Boolean functions, but a set of \emph{structured functions} -- a
% function indexed A symmetric $\tau$-circuits is defined for some universe size
% $n$ and Boolean basis $\mathbb{B}$ of Boolean functions. Each Boolean function
% in $\mathbb{B}$ is a \emph{structured Boolean function} in the sense that

% In this abstract we review the work of Anderson and Dawar on symmetric
% circuits, and extend their analysis by broadening the basis of functions they
% consider to include so-called `matrix-symmetric' functions. This class of
% functions includes important non-symmetric functions, e.g.\ the rank of a
% matrix, and this allows us to study symmetric circuits with gates that compute
% rank. Our main result is a circuit characterisation for FPR. In particular, we
% show that P-uniform families of symmetric circuits with rank gates
% characterise FPR. This requires a significant advance on the methods
% of~\cite{AndersonD17}.
\end{document}
