\documentclass[../paper.tex]{subfiles}
\begin{document}
A finite relational structure, such as a graph, can be represented as a binary
string in a standard way (for instance, by enumerating its adjacency matrix).
When we speak of the computational complexity of properties of graphs, we assume
such an encoding in order to present the structure as input to a machine or
circuit or other appropriate computational model. The properties of structures
we are interested in studying are invariant under isomorphisms. This requirement
translates into an invariance condition on sets of strings---say that a language
$L \subseteq \{0,1\}^\star$ is a \emph{graph property} if, whenever $x \in L$
for a string $x$ representing a graph $G$, we have $y\in L$ for any string $y$
obtained from $x$ by re-ordering the vertices of $G$. The central open question
in descriptive complexity---whether there is a logical characterisation of the
polynomial-time decidable properties of finite relational structures--- can then
be understood as the question of characterizing those languages $L$ that are in
$\PT$ and are graph properties.

Anderson and Dawar~\cite{AndersonD17} study this question through the lens of
circuit complexity, by considering \emph{symmetric circuits}. These are circuits
that necessarily accept graph properties, with this fact explicitly witnessed by
automorphisms of the circuit. To be precise, we can think of a property of
(directed) graphs on $n$ vertices as being given by a Boolean function $f:
\{0,1\}^{n^2} \rightarrow \{0,1\}$ which takes as input the $n^2$ potential
edges and outputs $1$ if the property is satisfied by the graph given on the
input. The function $f$ is \emph{graph-symmetric} in the sense that any
permutation of its inputs induced by a permutation of the $n$ vertices leaves
$f$ unchanged (note this is not the same thing as requiring that $f$ is
invariant under all permutations of its inputs). A symmetric circuit is a
circuit computing a Boolean function of $n^2$ inputs where each permutation of
$[n]$ can be extended to an automorphism of the circuit. Such a circuit
necessarily computes an invariant function.

The results of~\cite{AndersonD17} show a tight relationship between the
decdability of a graph property by means of a polynomial-size family of
symmetric circuits and its definability in extensions of fixed-point logic. In
particular, they show that a property is definable in fixed-point logic with
counting ($\FPC$) if, and only if, it is decidable by a polynomially-uniform
family of symmetric circuits using Boolean and majority gates (or, equivalently,
threshold gates). Without the symmetry restriction, polynomial-size families of
circuits have the same computational power whether we restrict ourselves to
Boolean gates only or allow richer bases such as threshold gates. One
consequence of the results in~\cite{AD14} is that, with the requirement of
symmetry, these different bases do yield different expressive power, suggesting
some lack of robustness to the model. One of the results in the present paper
shows that the model with majority circuits is robust in the sense that it is
able to simulate, by means of symmetric circuits, any symmetric circuit composed
of gates computing Boolean functions that are themselves symmetric (in the sense
that they are invariant under \emph{all} permutations of their inputs. Thus, any
property decidable by a polynomially-uniform family of such circuits is
definable in $\FPC$.

This motivates the study of symmetric circuit families that involve gates which
themselves may compute non-symmetric functions. In particular, motivated by the
study of rank logic~\cite{DGHL09,GP15}, we consider gates that are
\emph{matrix-symmetric}. These are gates computing a Boolean function $f:
\{0,1\}^{A \times B} \rightarrow \{0,1\}$ which is invariant under arbitrary
permutations of $A$ and $B$. A typical example is the function computing the
rank of an $A \times B$ $0$-$1$ matrix. We show that the support theorem of
Anderson and Dawar can be shown to hold, even in the presence of
matrix-symmetric gates. This allows us to show that definability in the logic
fixed-point with rank ($\FPR$) is the same as decidability by
polynomially-uniform families of symmetric circuits with rank gates.
% Another extension of $\FPC$ that has received much attention in the
% descriptive complexity literature is the class $\CPTC$ of \emph{choiceless
% polynomial time with counting}. The work of~\cite{AD14} leaves open the
% question of obtaining a circuit characterization of this class and it was
% suggested there that this would require a relaxation of the notion of symmetry
% used in defining symmetric circuits. In particular, while $\CPTC$ is based on
% a machine model designed to preserve symmetry by forbidding choices, the
% polynomial time and space restrictions apply int he context of the symmetries
% fo a \emph{single input structure}. In this paper, we consider a relaxation of
% the notion of symmetric circuit intended to capture exactly this intuition.
% However, we are able to show that this does not yield a characterization of
% $\CPTC$. In particular, we show that no family in this class of relaxed
% symmetric circuits can distinguish odd from even Cai-F\"urer-Immerman graphs,
% something we know is do-able in $\CPTC$~\cite{DRR08}.

\end{document}
