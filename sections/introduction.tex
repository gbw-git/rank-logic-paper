\documentclass[../paper.tex]{subfiles}
\begin{document}

The study of extensions of fixed-point logics plays an important role
in the field of descriptive complexity theory.  In particular,
fixed-point logic with counting ($\FPC$) has become a reference logic
in the search for a logic for polynomial-time
(see~\cite{Dawar-siglog}).  In this context, Anderson and
Dawar~\cite{AndersonD17} provide an interesting characterization of
the expressive power of $\FPC$ in terms of circuit complexity.  They
show that the properties expressible in this logic are exactly those
that can be decided by polynomially-uniform families of circuits (with
threshold gates) satisfying a natural \emph{symmetry} condition.  Not
only does this illustrate the robustness of $\FPC$ as a complexity
class within $\PT$ by giving a distinct and natural characterization
of it, it also demonstrates that the techniques for proving
inexpressibility in the field of finite model theory can be understood
as lower-bound methods against a natural circuit complexity class.
This raises an obvious question (explicitly posed in the concluding
section of~\cite{AndersonD17}) of how to obtain circuit
characterizations of logics more expressive than $\FPC$, such as
choiceless polynomial time (CPT) and fixed-point logic with rank
($\FPR$).  It is this last question that we address in this paper.

Fixed-point logic with rank extends the expressive power of $\FPC$ by
means of operators that allow us to define the rank of a matrix over a
finite field.  Such operators are natural extensions of
counting---counting the dimension of a definable vector space
rather than just the size of a definable set.  At the same time they
make the logic rich enough to express many of the known examples that
separate $\FPC$ from $\PT$.  Rank logics were first introduced
in~\cite{Dawar09logicswith}.  The version $\FPR$ we consider here is that
defined by Gr\"adel and Pakusa~\cite{GradelP15a} where the prime
characteristic is a parameter to the rank operator, and we do not have
a distinct operator for each prime number.  Formal definitions of
these logics are given in Section~\ref{sec:background}.  We give a circuit
characterization, in terms of symmetric circuits, of $\FPR$.  One
might think, at first sight, that this is a simple matter of extending
the circuit model with gates for computing the rank of a matrix.
It turns out, however, that the matter is not so simple as the
symmetry requirement interacts in surprising ways with such rank
gates.  It requires a new framework for defining classes of such
circuits, which yields remarkable new insights. 

It is important to clarify at the outset what is meant by symmetric
circuits.  Indeed, there are different senses in which the word
\emph{symmetry} is used in the context of circuits (and also in this
paper).  We say that a Boolean function $f:\{0,1\}^n \ra \{0,1\}$ is
symmetric if the value of the function on a string $s$ is determined
by the number of $1$s in $s$.  In other words, $f$ is invariant under
\emph{all} permutations of its input.  In contrast, when we consider
the input to a Boolean function to be the adjacency matrix of an
$n$-vertex graph, for example, and
$f : \{0,1\}^{n \choose 2} \ra \{0,1\}$ decides a graph property, then
$f$ is invariant under all permutations of its input induced by
permutations of the $n$ vertices of the graph.  We call such a
function \emph{graph-invariant}.  More generally, for a relational
vocabulary $\tau$ and a standard encoding of $n$-element
$\tau$-structures as strings over $\{0,1\}$, we can say that function
taking such strings as input is $\tau$-invariant if it is invariant
under permutations induced by the $n$ elements.
A circuit $C$ computing such an invariant function is said to be \emph{symmetric} if every
permutation of the $n$ elements extends to an automorphism of $C$.  It
is families of symmetric circuits in this sense that characterize
$\FPC$ in~\cite{AndersonD17}.  The restriction to symmetric circuits
arises naturally in the study of logics and has appeared previously
under the names of generic circuits in the work
of~\cite{DENENBERG1986216} and explicitly order-invariant circuits in
the work of Otto~\cite{Otto1997}.  

The main result of~\cite{AndersonD17} says that the properties of
$\tau$-structures definable in $\FPC$ are exactly those that can be
decided by $\PT$-uniform families of symmetric circuits using AND, OR,
NOT and majority gates.  Note that each of these gates itself computes
a Boolean function that is symmetric in the strong sense identified
above.  On the other hand, a gate for computing a rank threshold
function, e.g.\ one that takes as input a $n \times n$ matrix and
outputs $1$ if the rank of the matrix is greater than a threshold $t$,
is \emph{not} symmetric.  In our circuit characterization of $\FPR$ we
necessarily have to consider such non-symmetric gates.  Indeed, we
show in Section~\ref{sec:symm-circ} that  $\PT$-uniform families of
symmetric circuits using gates for \emph{any} symmetric functions do
not take us beyond the power of $\FPC$.  This is a further
illustration of the robustness of $\FPC$.  In order to go beyond it,
we need to introduce gates for Boolean functions that are not
symmetric.  We construct a systematic framework for including
functions computing $\tau$-invariant functions for arbitrary
multi-sorted relational vocabularies $\tau$ in
Section~\ref{sec:symm-circ}.  We also explore what it means for such
circuits to be symmetric.

The proof of the circuit characterization of $\FPC$ relies on the
\emph{support theorem} proved in~\cite{AndersonD17}.  This establishes
that for any $\PT$-uniform family of circuits using AND, OR, NOT and
majority gates there is a constant $k$ such that every gate has a
support of size at most $k$.  That is to say that we can associate
with every gate $g$ in the circuit $C_n$ (the circuit in the family
that works on $n$-element structures) a subset $X$ of $[n]$ of size at
most $k$ such that any permutation of $[n]$ fixing $X$ pointwise
extends to an automorphism of $C_n$ that fixes $g$.  This theorem is
crucial to the translation of the family of circuits into a formula of
$\FPC$, which is the difficult (and novel) direction of the
equivalence.  In attempting to do the same with circuits that now use
rank-threshold gates we are faced with the difficulty that the proof
of the support theorem in~\cite{AndersonD17} relies in an essential
way on the fact that the Boolean function computed at each gate is
symmetric.  We are able to overcome this difficulty and prove a
support theorem for circuits with rank gates but this requires
substantial, novel technical machinery, which we develop in
Section~\ref{sec:symm-support}. 

Another crucial ingredient in the proof of Anderson and Dawar is that
we can eliminate redundancy in the circuit $C_n$ by making it
\emph{rigid}.  That is, we can ensure that the \emph{only}
automorphisms of $C_n$ are those that are induced by permutations of
$[n]$.  Here we face the difficulty that identifying the symmetries
and eliminating redundancy in a circuit that involves gates computing
$\tau$-invariant functions requires us to solve the isomorphism
problem for $\tau$-structures.  This is a hard problem (or, at least,
one that we do not know how to solve efficiently) even when the
$\tau$-structures are $0$-$1$-matrices.  We overcome this difficulty
by placing a further restriction on circuits that we call
\emph{transparency}.  Circuits satisfying this condition have the
property that their lack of redundancy is transparent.  We explore
this condition and its consequences in Section~\ref{sec:transparent}.

In the characterization of $\FPC$, the translation from formulas into
families of circuits is easy and, indeed, standard.  In our case, we
have to show that formulas of $\FPR$ translate into uniform families of
circuits using rank-threshold gates that are symmetric and
transparent.  This is somewhat more involved technically and the
details are presented in Section~\ref{sec:formulas-to-circuits}.
Finally, with all the tools in place, the translation of such
$\PT$-uniform families of circuits into formulas of $\FPR$ given in Section~\ref{sec:circuits-to-formulas} completes
the characterization.

\end{document}




% When we study algorithmic complexity we often study complexity classes defined
% in terms of some restricted class of machines. Circuit complexity, where one
% makes use of similarly restricted families of Boolean circuits rather than
% machines, has long served as an alternative approach. These circuits models have
% a natural attraction for researchers, as their simpler structure seems to make
% them more amendable to analysis and their connections with machine models
% provides an alternative avenue for better understanding fundamental problems in
% complexity theory, including the $\PT$ vs $\NP$ problem.

% Descriptive complexity instead concerns the characterisation of complexity
% classes in terms of logics. The central open question in the field -- whether
% there is a logical characterisation of the polynomial-time decidable properties
% of finite relational structures -- essentially asks if there is a logic the
% sentences of which describe all (and only) polynomial-time decidable
% isomorphism-closed classes of graphs (i.e. every \emph{graph property} in
% $\PT$). As in the algorithmic complexity case, the use of circuits in
% descriptive complexity has proved useful, with Otto \cite{Otto1997} and
% Denenberg et al.~\cite{DENENBERG1986216} developing circuit-based
% characterisations for logics. These characterisation are given in terms of
% \emph{symmetric circuits}, circuits that take in finite structures over some
% universe and where permutations of the universe of the input structure are
% reflected by automorphisms of the circuit.

% % In this way symmetric circuits provide a natural class from which to develop
% % circuit characterisations for logics, taking in abstract structures and
% % ensuring that isomorphism (permutations) on the input structure are reflected
% % in the circuit structure, which consequently implies that the function
% % computed by the circuit is \emph{invariant} (i.e. computes a graph property)
% % makes symmetric circuits a natural class from which to develop circuit
% % characterisations of logics.

% Anderson and Dawar \cite{AndersonD17} developed a similar
% circuit-characterisation for $\FPC$, proving that families of P-uniform families
% of symmetric circuits defined over a complete Boolean basis with majority gates
% completely characterises \emph{fixed-point logic with counting} ($\FPC$).

% While it is known that $\FPC$ does not capture $\PT$ \cite{Cau1992}, $\FPC$ does
% capture $\PT$ over many classes of structures \cite{Grohe2010} and many natural
% $\PT$ problems are expressible in $\FPC$ (e.g.\ Maximum Matching
% \cite{Anderson2013}). While $\FPC$ is thus an important logic of interest for
% its own sake, it also plays a role as a logic from which other important logics
% of study are defined. The particular logic of interest in this paper is
% \emph{fixed-point logic with rank} ($\FPR$), introduced by Dawar et
% al.~\cite{Dawar09logicswith}, and is defined as an extension of $\FPC$ that
% includes operators that can define the rank of a definable matrix over a fixed
% finite field. This extension was prompted by the work of Atserias et
% al.~\cite{Atserias20091666}, who had shown that that some of the most difficult
% to express problems in $\PT$ (that are not expressible in $\FPC$) were reducible
% to the general problem of solving systems of linear equations over prime fields.
% Indeed, problems of this sort have remained the primary source of
% difficult-to-express problems in descriptive complexity, motivating the modern
% study of $\FPR$.

% It is worth noting the logic defined by Dawar et al.\cite{} defines for each
% prime a separate operator that computes the rank of the matrix over that prime.
% It was shown by Gr\"{a}del and Pakusa \cite{GradelP15a} that this logic does not
% capture $\PT$. Instead, they present a strictly more expressive alternative rank
% logic which includes only a single rank operator that takes in both the matrix
% and a specific prime over which to compute the rank as input. In this paper when
% we refer to $\FPR$ we are referring to this more expressive version of
% Gra\"{a}del and Pakusa.

% In this paper we develop a circuit characterisation of $\FPR$ in terms of
% symmetric circuits that have gates that compute rank. Concretely, a circuit
% $C_n$ for structures of size $n$ is a directed acyclic graph (DAG) whose
% internal gates are marked by elements of some basis. This basis consists of
% operations or, more concretely, Boolean functions such that a gate $g$ in $C_n$
% with $k$ input gates is labelled by a function $f: \{0,1\}^A \rightarrow
% \{0,1\}$, where $\vert A \vert = k$. When evaluating the circuit, $g$ is
% computed by evaluating $f$ with the input given by the binary string formed by
% evaluating the gates input to $g$, implicitly mapping $A$ to the input gates of
% $g$. However, the output of $f$ in general depends on the order (or structure)
% of the binary string taken as input, and as such the evaluation of $g$ depends
% on which gates are assigned to each element of $A$.

% This point is often ignored in circuit complexity by ensuring that functions in
% the Boolean basis are \emph{symmetric}, i.e. depend only on the number of 1's in
% it's input, rendering the particular assignment from $A$ to the input gates
% irrelevant. This is the approach taken by Anderson and Dawar \cite{AndersonD17}.
% However, functions that compute an operation like rank, i.e. a function that
% takes in a matrix with Boolean entries and outputs 1 iff that matrix, when
% interpreted in $\mathbb{F}_p$, has rank at least $r$, certainly depend on how
% the input is formed into a matrix, and as such is not a symmetric function.

% We introduce the notion of a \emph{structured function}, which allows us to both
% index inputs appropriately and introduce appropriate generalisations of symmetry
% depending on the structure of the input. Each such function $F$ is uses a many-sorted
% structure, with some signature $\tau_F$, in order to index the input. This
% definition allows us to generalise the notion of symmetry to a particular
% labelling, and so introduce the notion of a \emph{$\tau$-symmetric functions}.

% We also introduce a generalised symmetric circuit allows, which allows for
% $\tau$-symmetric functions in the Boolean basis, not just symmetric ones. The
% circuit also provides a function $L$ that maps to each internal gate $g$ to a
% labelling on its input corresponding the labelling on the input of the Boolean
% function that labels this gate.

% \begin{remark}
%   I'm not sure how much detail to go into here, this could be a major
%   digression.
% \end{remark}



% % That is, eacFor example, if we should like The circuit $C_n$ associates with
% % each internal gate $g$ a structure $\dom (g)$, and each gate input a sequence of
% % elements from each sort $\dom(g)$ in accordance with the sorts of the

% % of definition includes an additional function which assosiates for each gate $g$
% % a


% % require that the Basis over which the circuit is defined are symemtric.. We
% % begin by generalising the notion of a symmetric circuit. A symmetric circuit, as
% % presented in \cite{AndersonD17}, and in many parts of complexity theory,
% % supposes that the in Boolean function that labels in the gate is
% % \emph{symmetric} that labels a particular gate We first generalise the notion of
% % a symmetric circuit so as to allow for gate


% % In this general notion of a symmetric circuit, a Boolean basis consists not just
% % of a set of Boolean functions, but a set of \emph{structured functions} -- a
% % function indexed A symmetric $\tau$-circuits is defined for some universe size
% % $n$ and Boolean basis $\mathbb{B}$ of Boolean functions. Each Boolean function
% % in $\mathbb{B}$ is a \emph{structured Boolean function} in the sense that

% % In this abstract we review the work of Anderson and Dawar on symmetric
% % circuits, and extend their analysis by broadening the basis of functions they
% % consider to include so-called `matrix-symmetric' functions. This class of
% % functions includes important non-symmetric functions, e.g.\ the rank of a
% % matrix, and this allows us to study symmetric circuits with gates that compute
% % rank. Our main result is a circuit characterisation for FPR. In particular, we
% % show that P-uniform families of symmetric circuits with rank gates
% % characterise FPR. This requires a significant advance on the methods
% % of~\cite{AndersonD17}.
% \end{document}
