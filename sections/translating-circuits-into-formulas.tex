\documentclass[../paper.tex]{subfiles}
\begin{document}

In this section we develop the theory for defining a formula of FPR from a
family of $P$-uniform family of symmetric circuits with rank gates. We first
prove results on the generality of rigid circuits and circuits with
bijective labelings. Second, we prove that there are polynomial time algorithms
for determining canonical supports of gates. Third, we develop a FPR definable
way of evaluating a rank gate in the circuit. Finally, we complete the result by
explicitly defining an FPR formula corresponding to a given family of symmetric
circuits with rank gates.

We note that most of these results are for the more general case of symmetric
circuits with matrix-symmetric gates.

\subsection{Rigid Circuits and Labellings}

In this section we prove important simplifying lemmas which allow for
assumptions of rigidity and bijective labels.

\begin{definition}
  We say that a circuit $C$ has bijective labels if for each gate $g$ in $C$,
  $L(g)$ is a bijection.
\end{definition}

\begin{lem}
  \label{lem:bij_labels}
  There is an algorithm that runs in polynomial time that takes in a circuit $C$
  and outputs a circuit with unique gates $C'$. Moreover, if $C$ was symmetric
  then $C'$ is symmetric. If $C$ is rigid then $C'$ is rigid.
\end{lem}

\begin{proof}
  Let $S = 2*\vert C \vert$. Recurse through the gates of $C$ topologically and
  let $h$ be the next gate topologically. If for all $g \in W(h, \cdot)$ we have
  that $L\vert (g)^{-1}(h) \vert = 1$ continue on to the next gate. If not add
  in a tower of $S$ $\and$ gates such that $h \rightarrow \and^h_1 \rightarrow
  \ldots \rightarrow \and^h_S$ (i.e. we have a tower of $\and$ gates with $h$ as
  input to $\and^h_1$ and the output of each $\and^h_i$ connected to the input
  of each $\and^h_{i+1}$ for each $1 \leq i < S$). Now for each $g \in W(h,
  \cdot)$, if $L(g)^{-1}(h) = \{ s_0, \ldots, s_{r}\}$, for each $1 \leq i \leq
  r$ add in the wires $W(\and^h_i, g)$ and set $L(g)(s_{i}) = \and^h_i$. Now
  continue on to the next gate topologically and run the above algorithm.

  % First, notice that for a given gate $h$ and $k \in \mathbb{N}$, we can
  % define
  % a sub-circuit $h^k = 1 \and \cdots, 1 \and h$ (or rather a $k$-height tower
  % of
  % $k$ binary $\and$ gates with $h$ at the top and all other inputs set to
  % $1$).
  % We call $h$ the top of the sub-circuit $h^k$ and the bottom $and$ gate (the
  % output of the sub-circuit) the bottom gate. We may think of this as a
  % $k$-height copy of the gate $h$, in the sense that it has the same output as
  % $h$ and similar orbits. We now use different copies to distinguish gates
  % that
  % otherwise have the same labelling.

  % Let $h$ be the next gate topologically. Then let $r$ be maximal such that
  % $W(h,g)$ and $\omega_g^{-1}(h) = \{ s^g_1, s^g_2, \ldots, s^g_r \}$. Then
  % let
  % $k$ be the height of the highest tower with $h$ at the top and with it's
  % bottom connected to a gate in $W(h, \cdot)$. Then create a single
  % $k+r-1$-height tower by adding in the appropriate number of binary $\and$
  % gates below $h$.

  % For each $g \in W(h, \cdot)$ and for each $\omega_g^{-1}(h) = \{s^g_1,
  % s^g_2,\ldots , s^g_{r_g}\}$, set $L'(g)(s^g_1) = h$ each child starting with
  % the $and$-gate child to $h$ add in a wire from from the $i$th gate in the
  % tower to $g$ and set $L'(g)(s^g_{i+1}) = \and_i$, where $\and_i$ is the
  % $i$th
  % $\and$ gate in the tower starting with the $\and$ gate in the tower child to
  % $h$. There are enough gates in the tower as $r_g \leq r$.

  We call this updated circuit $C'$.
  
  Firstly, note that after running the above algorithm for each $g$ the
  labelling of $g$ will be a bijection. Moreover, it's easy to see that the
  output of each gate remains unchanged, and as such the output of the circuit
  is unchanged.

  Secondly, notice that the size of $C'$ is at most $2*\vert C \vert^2$, and
  note that the above algorithm runs in polynomial time.

  Now suppose that $C$ is symmetric. Let $\sigma$ be a permutation on the input
  universe and $\pi_C$ the induced automorphism on $C$. We now define $\pi$, the
  induced automorphism on $C'$. For each gate $g$ in $C'$, if $g$ is in $C$ then
  set $\pi(g) := pi_C(g)$. If $g$ is not in $C$ then $g$ must be some
  $\and^h_i$, for some $h$ in $C$. Then set $\pi(\and^h_i) :=
  \and^{\pi_C(h)}_i$. It is easy to see that $\pi$ is an automorphism, and $\pi$
  extends $\sigma$.

  Suppose that $C$ is rigid. It is easy to see that $C'$ will be rigid as well.

\end{proof}


\begin{lem}
  There is an algorithm that runs in polynomial time that takes in a circuit $C$
  and outputs a circuit $C'$ such that $C'$ is rigid and has bijective labels.
  Moreover, if $C$ was symmetric it follows that $C'$ will be symmetric.
\end{lem}

\begin{proof}
  First run the algorithm from Lemma \ref{lem:bij_labels} on $C$, and call the
  output circuit $C$.
  
  Recurse through the gates of $C$ topologically. For each internal gate $g$,
  for all $g'$ in $C$ such that $g \neq g'$, $W(\cdot, g) = W(\cdot, g')$, $W(g,
  \cdot) = W(g', \cdot)$, $\Sigma(g) = \Sigma(g')$, $L(g) \sim L(g')$ and
  $\Omega^{-1}(g) = \Omega^{-1}(g')$, delete $g'$ and for all $s \in L^{-1}(g')$
  set $L(s) := g$.

  Now re-run the algorithm from Lemma \ref{lem:bij_labels} on $C$ and output the
  result.
\end{proof}

  \begin{lem}
    Let $C = \langle G, W, \Omega, \Sigma, \Lambda, L \rangle$ be a $(\SB, \MB,
    \tau)-circuit$ on structures of size $n$. There is a deterministic algorithm
    which runs in Poly($\vert C \vert$) and outputs a rigid $(\SB, \MB, \tau)
    circuit$ $C'$ such that $G' = G$ and for any $g \in G$, and any input
    $\tau$-structure $\mathcal{A}$ and any bijection $\gamma$ from $A$ to $[n]$,
    $C[\gamma \mathcal{A}](g) = C'[\gamma \mathcal{A}](g)$ and if $C$ is
    symmetric then so is $C'$.
  \end{lem}

  \begin{proof}
  
  \end{proof}

  \subsection{Computing Supports}
  
 \begin{lem}
   Let $C$ be a rigid $(\SB, MB, \tau)$-circuit on structures of size $n$ and
   $\sigma \in \sym_n$. There is a deterministic algorithm which runs in time
   Poly($\vert C \vert$) and outputs for each gate $g$ its image under the
   automorphism $\pi$ induced by $\sigma$, if it exists.
 \end{lem}
 \begin{proof}
   The proof proceeds by recursively going through the circuit and building the
   mapping $\pi$ induced by $\sigma$.

   Suppose $g$ is a constant gate, then $\pi g := g$. Suppose $g$ is a
   relational gate, then there is at most one gate $g'$ such that $\Sigma (g) =
   \Sigma (g')$ and $\sigma\Lambda (g') = \Lambda (g)$. If such a $g'$ exists
   assign $\pi g := g'$, else terminate with failure.

   If $g$ is an symmetric internal gate then (from rigidity) there is at most
   one gate $g'$ such that $\Sigma (g) = \Sigma(g')$ and $W_{g'} = \pi W_g$.
   Assign $\pi g := g'$ if such a gate exists, or else terminate with failure.

   If $g$ is a matrix-symmetric internal gate then consider the set of gates
   $g'$ such that $g'$ has children $\pi W_g$ and $\Sigma(g) = \Sigma(g')$, and
   let $A \times B = \dom (Sigma(g))$. If no such gate $g'$ exists, terminate
   with failure. Define $\sigma_{\pi, g'}:A \times B \rightarrow A \times B$ by
   $\sigma_{\pi, g'} = \omega^{-1}_{g'} \pi \omega_{g}$. Then clearly
   $\omega_{g'} \sigma_{\pi, g'} = \pi \omega_{g}$, and it's easy to show that
   $\pi \omega_g \sim \omega_{g'}$ iff $\sigma_{\pi,g'} \in \sym_A \times
   \sym_B$. But this just involves checking that $\sigma$ acts as a bijection on
   $A$ and $B$ separately and, given that $\vert A \vert$ and $\vert B \vert$
   are both bounded by $\vert C \vert$, the algorithm which just iterates
   through $A$ and $B$ is sufficient. If for every $g'$ it is found that
   $\pi_{\sigma,g'}$ is not in $\sym_A \times \sym_B$ then terminate with
   failure. If there is a $g'$ for which $\pi_{\sigma, g'} \in \sym_A \times
   \sym_B$ then it is unique by rigidity and so set $\pi g := g'$.

   If $g$ is an output gate, then check that for all tuples in $[n]^{q}$ we have
   that $\pi \Omega (x) = \Omega (\sigma (x))$, and terminate with failure if
   the condition is not met.

   If the algorithm has not terminated with failure, output the automorphism.

   The algorithm clearly runs in Poly($\vert C \vert$)
 \end{proof}

\subsection {Evaluating Circuits}
Let $\mathcal{C} = (C_n)_{n \in \mathbb{N}}$ be a family of polynomial-size
rigid symmetric circuits with matrix-symmetric gates. that compute a $q$-ary
query. Let $n_0$ the constant in the hypothesis of the Support Theorem.

Let $\mathcal{A}$ be a structure of size $n$ over the universe $U$, and let $g$
be a matrix-symmetric gate in $C_n$. In this section we develop the theory which
will allow us to evaluate this gate in FPR and (using results from Anderson and
Dawar \cite{}) recursively evaluate the circuit.

We recall that in order to evaluate the gate $g$ we need to consider a bijection
$\gamma: U \rightarrow [n]$, with the evaluation of $g$ given by $C_n[\gamma
\mathcal{A}](g)$. We first show that the evaluation of $g$ depends only on which
elements of $U$ are mapped to $\consp(g)$.

\begin{definition}
  Let $f: X \rightarrow Y$ and $g : X' \rightarrow Y'$. We say that $f$ is
  \emph{compatible} with $g$, or $f \sim g$, if for all $a \in S_1 \cap S_2$,
  $f(a) = g(a)$ and for all $a \in S_1 \setminus S_2$ and $b \in S_2 \setminus
  S_1$, $f(a) \neq g(b)$.
\end{definition}

\begin{definition}
  Let $f: A \times B \rightarrow H$ and $p: A' \times B' \rightarrow H$. We say
  that $f$ and $p$ are \emph{row-column equivalent} if there exist bijections
  $\alpha: A \rightarrow A'$ and $\beta: B \rightarrow B'$ such that for all
  $(a, b) \in A \times B$, $f(a,b)g(\alpha(a), \beta(b))$. In this case we write
  $f \sim p$.
\end{definition}

For a given bijection $\gamma: U \rightarrow [n]$ we can define a matrix
$L^{\gamma} : \dom (g) \rightarrow \{0,1\}$ by $L^{\gamma} (a,b) := C[\gamma
\mathcal{A}](L(g)(a,b))$. In the following Lemma we show that the the matrix
symmetry of the gate $g$ ensures that for any two bijections $\gamma_1,
\gamma_2$, if they agree on the support of $g$ then $L^{\gamma_1}_g$ is
row-column equivalent to $L^{\gamma_2}_g$. This implies that the evaluation of
$g$ depends only on the assignment to its support.

\begin{lem}
  Let $g$ be a matrix-symmetric gate in $C_n$ with children $H$. Let $\eta:
  \consp(g) \rightarrow U$ be an injection, and suppose $\gamma_1, \gamma_2: U
  \rightarrow [n]$ with $\gamma^{-1}_1 \sim \eta$ and $\eta \sim \gamma^{-1}_2$.
  Let $A \times B = \dom (L(g))$. Then $L^{\gamma_1}$ and $L^{\gamma_2}$ are
  row-column equivalent.
\end{lem}

\begin{proof}
  We have that there exists a unique $\pi \in \sym_n$ such that $\pi \gamma_1 =
  \gamma_2$. Moreover, since $\gamma^{-1}_1$ and $\gamma^{-1}_2$ are both
  consistent with $\alpha$, it follows that $\pi$ must fix $\consp(g)$. Thus
  $L(g) \sim \pi \cdot L(g)$, and so there exists $(\sigma, \lambda)$ such that
  $\pi L(g) = L(g) (\sigma, \lambda)$.

  We then have that,
  \begin{align*}
    L^{\gamma_1} (a,b) &= C_n[\gamma_1 \mathcal{A}](L(g)(a,b))\\
                       & = C_n[\pi \gamma_1 \mathcal{A}][\pi L(g)(a,b)] \\
                       & = C_n[\gamma_2 \mathcal{A}][L(g)((\sigma, \lambda)(a,b))]\\
                       & = L^{\gamma_2} ((\sigma, \lambda) (a,b)),
  \end{align*}
  and it follows that $L^{\gamma_1} \sim L^{\gamma_2}$.
\end{proof}

For each gate we define the set of all assignments to support of that gate which
cause the evaluate to true: $\EV_g := \{ \alpha: U^{\underline{\consp(g)}} :
\exists \gamma \in U^{\underline{[n]}} C_n [\gamma \mathcal{A}](g) = 1 \wedge
\alpha \sim \gamma^{-1}\}$. We now construct $\EV_g$ using $\EV_h$ for each $h
\in H_g$. In doing so we entirely characterise exactly which bijections $\gamma:
U \rightarrow [n]$ cause $g$ to evaluate to true.

% \begin{remark}
%   Let $\sigma \in \spstab{g}$, then we know that there exists $(\sigma_r,
%   \sigma_c) \in \sym_A \times \sym_B$ and $\sigma_r (i) = \row (\sigma h)$,
%   for any $h \in H$ such that $\row (h) = i$ and similarly $\sigma_c (j) =
%   \column (\sigma h)$, for any $h \in H$ such that $\column (h) = j$.
% \end{remark}

We introduce here some useful notation.

\begin{definition}
  Let $S^r_h = \{ \sigma \in \stab (\consp(g)) : \sigma \vec{r_h} = \vec{r_h}
  \}$ and $G^r_h = \{ \sigma \in \stab(\consp(g)) : \row(\sigma h) = \row(h)\}$.
  Let $S^c_h = \{ \sigma \in \stab (\consp(g)) : \sigma \vec{c_h} = \vec{c_h}
  \}$ and $G^c_h = \{ \sigma \in \stab(\consp(g)) : \column(\sigma h) =
  \column(h)\}$.
\end{definition}

Clearly, by definition of a row and column support, $S^r_h \subseteq G^r_h$ and
$S^c_h \subseteq G^c_h$.

% \begin{remark}
%   In this section I assume for the moment that:
%   \begin{itemize}
%   \item $\consp (g) = \{\}$,
%   \item $S^r_h = G^r_h$ and $S^c_h = G^c_h$,
%   \item $\forall h,h' \in H$ we have $\vert r_h \vert = \vert r_{h'} \vert$
%     and $\vert c_h \vert = \vert c_{h'} \vert$,
%   \item For all $i,i' \in A$ $\exists \sigma \in \stab(\consp (g))$ such that
%     $\sigma_r i = i'$. Similarly for all $j, j' \in B$ $\sigma \exists \in
%     \spstab{g}$ such that $\sigma_c j = j'$.
%   \item $\forall h \in H$, $\orb (h) = H$
%   \end{itemize}
% \end{remark}

% Let $k_r = \vert r_h \vert$ and $k_c = \vert c_h \vert$. Let $\mathcal{G}_r =
% \{\sigma_r : \sigma \in \spstab{g}\}$ and $\mathcal{G}_c = \{sigma_c : \sigma
% \in \spstab{g}\}$.

% \begin{definition}
%   Let $\mathcal{G} \leq \sym_n$ and $x$ be an element of a set on which an
%   action of $\sym_n$ is defined. Then $\orb_{\mathcal{G}}(x) = \{x^\pi: \pi
%   \in \mathcal{G}\}$.

%   For $i \in [a]$ let $\orb_r(i) = \orb_{\mathcal{G}_r}(i)$ and for $j \in
%   [b]$ let $\orb_c (j) = \orb_{\mathcal{G}_c}(j)$.
% \end{definition}

% \begin{definition}
%   Let $A, B \subseteq [n]$ and let $t = (t_1, t_2, t_3) \in \mathbb{N}^3$. We
%   say that $(A,B)$ has \emph{type} $t$ if $\vert A \vert = t_1$, $\vert B
%   \vert = t_2$ and $\vert A \cap B \vert = t_{3}$. We say that $\type (A,B) =
%   t$.
% \end{definition}

For the sake of brevity, for sets $Z,X,Y$, where $f : Z \rightarrow X$ and $p :
Z \rightarrow Y$ are injections, let $f_p = f \cdot p^{-1}$.

\begin{definition}
  Let $\vec{x}, \vec{y} : Z \rightarrow X$ and $\vec{r}, \vec{c}: Z \rightarrow
  Y$ be injections. If $\vec{x} \cdot \vec{r}^{-1} \sim \vec{y} \cdot
  \vec{c}^{-1}$ we say that $(\vec{x}, \vec{y})$ and $(\vec{r} \vec{c})$ have
  the same \emph{type}.
\end{definition}

We note that if we have $f : X \rightarrow Y$ and $p: X' \rightarrow Y'$ and $f
\sim p$ then define $(f \vert p): X \cup X' \rightarrow Y \cup Y'$ by

\begin{align*}
  (f \vert p) (x) =
  \begin{cases}
    f (x) & x \in X \\
    p (x) & x \in Y.
  \end{cases}
\end{align*}

The compatibility of these two functions ensures that this function is well
defined.

% \begin{claim}
%   Let $A_1,B_1, A_2, B_2 \subseteq [n]$. Then $(A_1, B_1)$ and $(A_2, B_2)$
%   have the same type iff $\exists \pi \in \sym_n$ such that $A_1 = A^\pi_2$
%   and $B_1 = B^\pi_2$.
% \end{claim}

% \begin{definition}
%   Let $f: A \rightarrow S$ and $g: B \rightarrow S$ be injections, with $A$
%   and $B$ being finite sets. Then $(f,g)$ has \emph{type} $t= (t_1 , t_2, t_3)
%   \in \mathbb{N}^3$ if $\vert A \vert = t_1$ and $\vert B \vert = t_2$ $\vert
%   \{i \in A \cap B : f(i) = g(i)\} \vert = t_3$. We say that $\type (f,g) =
%   t$.
% \end{definition}

% \begin{definition}
%   Let $h \in H$ the \emph{type} of $h$ is the type of $(r_h, c_h)$. We denote
%   the type of $h$ by $\type (h)$.
% \end{definition}

We now fix an injection $\eta : \consp(g) \rightarrow U$ and describe an
approach for determining if $\eta \in EV_g$.

Let $X$ be a set on which the left group action on $\sym_n$ is defined and $s
\in X$. We denote $\vec{x}$ then $A_s = \{\vec{x} \in U^{\underline{[\vert
    \sp(s) \vert ]}} : \eta \sim \vec{x} \cdot \vec{\consp}(s)^{-1} \}$, where
$\vec{\consp}(s): [\vert \consp(s) \vert] \rightarrow \consp(s)$ maps $i \in
[\vert \consp(s) \vert]$ to the $i$th element of $\consp(s)$. For $\vec{x} \in
A_s$, we use $\vec{x}_s$ from here forward to denote $\vec{x} \cdot
\vec{\consp(s)^{-1}}$.

% Let $L(g)(i,j) = h \in H$, $\vec{x} \in A_i$ and $\vec{y} \in A_j$. Let
% $\vec{r} \in ^{\underline{\consp(i)}}$ and $\vec{c}
% \in\consp{j}^{\underline(\consp(j))}$. Suppose $(\vec{r}, \vec{c})$ has the
% type of $(\vec{x}, \vec{y})$. Then we can define $(\vec{x}\vert \vec{y}):
% \consp(i) \cup \consp(j) \rightarrow U$ by
% \begin{align*}
%   (\vec{x}_{\vec{r}} \vert \vec{y}_{\vec{c}}) (z) =
%   \begin{cases}
%     \vec{x} (\vec{r}^{-1}(z)) \text{ if } z \in \consp(i) \\
%     \vec{y} (\vec{r}^{-1}(z)) \text{ if } z \in \consp(j). \\
%   \end{cases}
% \end{align*}

This mapping allows us to combine two assignments into a single assignment. This
will be useful when combining the row and column supports of a gate.

% \begin{claim}
%   Let $f: A \rightarrow S$ and $g B \rightarrow S$ be injections. Then $(f,g)$
%   has the same type as $(A,B)$. It follows that if $h \in H$, and $(i,j) =
%   L^{-1}(h)$,with $f \in A^r_i$ and $ g \in A^c_j$ then $h$ has the same type
%   as $(f,g)$.
% \end{claim}

% For $(i,j) \in [a] \times [b]$ let $H_{i,j} = \{L(p,q): (p,q) \in \orb_r(i)
% \times \orb_c(j)]\}$.

% \begin{claim}
%   For all $(i', j') \in \orb_r(i) \times \orb_c(j)$ and $\sigma \in
%   \spstab{g}$, $(\sigma_r i', \sigma_cj') \in \orb_r(i) \times \orb_c(j)$.
% \end{claim}

% \begin{claim}
%   For all $(i,j) \in [a] \times [b]$, $H_{i,j}$ is a union of orbits.
% \end{claim}

% \begin{lem}
%   Let $(i,j) \in [a] \times [b]$. If $h, h' \in H_{i,j}$ and $h, h'$ are the
%   same type then $h' \in \orb(h)$.
% \end{lem}

We now define a matrix that we later be shown to be definable in FPR. We then
show that this matrix is row-column equivalent to $L^{\gamma}$, for any
bijection $\gamma: U \rightarrow [n]$, and hence allowing us to determine the
rank of the matrix input to $g$ in FPR.

Let $R^{\min} = \{\min (\orb(\row(h))) : h \in H\}$ and $C^{\min} = \{ \min
(\orb (\column(h))) : h \in H\}$ and let
\begin{align*}
  I = \{(i, \vec{x}): i \in R^{\min}, \vec{x} \in A_i\},
\end{align*}
and
\begin{align*}
  J = \{(j, \vec{y}): j \in C^{\min}, \vec{y} \in A_j \}.
\end{align*}

% \begin{align*}
%   M ((i, \vec{x}), (j, \vec{y})) := \bigvee_{t \in \types} ((\vec{x},
%   \vec{y}) \text{ has type } t) \land (\vec{x} | \vec{y}) \in \EV_{\mu_{i,j}(t)}.
% \end{align*}

Let $(i, \vec{x}) \in I$ and $(j, \vec{y}) \in J$.

Let $u_1 , \ldots , u_{2k}$ be the first $2k$ elements of $[n] \setminus
\consp(g)$ in the natural order on the structure. Then let
\[r = \eta^{-1} (\img(\vec{x}) \cap \eta (\consp(g))) \cup
  \{u_{\vec{x}^{-1}(a)}: a \in \img(x) \setminus \eta (\consp (g))\} \] and
\[s = \eta^{-1} (y \cap \eta (\consp(g))) \cup (x \cap y) \cup \{ u_{k +
    \vec{x}^{-1}(a)} : a \in y \setminus (x \cup \eta (\consp (g))) \}). \]
Define
\[
  \vec{r} (a) =
  \begin{cases}
    \eta^{-1} (\vec{x} (a)) & a \in \vec{x}^{-1} (x \cap \eta (\consp(g))) \\
    u_{a} & a \in \vec{x}^{-1} (x \setminus \eta(\consp(g)),
  \end{cases}
\]
and
\[
  \vec{c} (a) =
  \begin{cases}
    \eta^{-1} (\vec{y} (a)) & a \in \vec{y}^{-1}(y \cap \eta (\consp (g))) \\
    \vec{r} (a) & a \in \vec{y}^{-1} (x \cap y \setminus \eta (\consp (g))) \\
    u_{k+a} & \text{otherwise}.
  \end{cases}
\]

\begin{lem}
  \label{lem:permutation_row-column}
  There exists $\sigma_1, \sigma_2 \in \spstab{g}$ such that $\sigma_1 \cdot
  \vec{\consp(i)} = \vec{r}$ and $\sigma_2 \cdot \vec{\consp(j)} = \vec{c}$.
\end{lem}
\begin{proof}
  To be added from the book
\end{proof}

Let $h = L(g)(\sigma_1(i), \sigma_2 (j))$. Then we can define the matrix $M : I
\times J \rightarrow \{0,1\}$ by

\begin{align*}
  M((i , \vec{x}), (j, \vec{y})) := (\vec{x}_{\vec{r}} \vert \vec{y}_{\vec{c}}) \in EV_h.
\end{align*}

It remains to show that $M \sim L^{\gamma}$ for some (and so all) bijections
$\gamma: U \rightarrow [n]$.
% \begin{definition}
%   Let $M_1$ and $M_2$ be matrices over some field, with row and column indexes
%   given by $(A_1, B_1)$ and $(A_2, B_2)$, respectively. We say that $M_1$ and
%   $M_2$ are row-column equivalent iff there exists bijections $\alpha: A_1
%   \rightarrow A_2$ and $\beta: B_1 \rightarrow B_2$ such that for all $(a,b)
%   \in A_1 \times B_1$ we have that $M_1(a,b) = M_2 (\alpha (a), \beta (b))$.
%   If $M_1$ and $M_2$ are row-column equivalent we say that $M_1 \sim M_2$.
% \end{definition}

% Fix $\gamma: U \rightarrow [n]$ such that $\gamma^{-1} \sim \eta$. We define
% $L^\gamma: A \times B \rightarrow \{0,1\}$ by $L^\gamma(a,b) = C[\gamma
% \mathcal{A}](L (a,b))$.
% \\~\\
% We note that, in fact, it is the choice of the assignment to the support of
% $g$, i.e. $\eta$, that really matters in the sense that for any two global
% assignments that agree on the support of $g$ will produce row-column
% equivalent matrices. The following lemma formalises this observation.

% \begin{lem}
%   Let $g$ be a matrix-symmetric gate in $C_n$ with children $H$ and matrix
%   labelling $L(g)$. Let $\eta \in U^{\consp(g)}$, and suppose $\gamma_1,
%   \gamma_2: U \rightarrow [n]$ with $\gamma^{-1}_1 \sim \alpha$ and $
%   \gamma^{-1}_2 \sim \alpha$. Let $A \times B = \dom (L(g))$.

%   Then for an input structure $\mathcal{A}$, $L^{\gamma_1} \sim L^{\gamma_2}$.
% \end{lem}
% \begin{proof}
%   We have that there exists a unique $\pi \in \sym_n$ such that $\pi \gamma_1
%   = \gamma_2$. Moreover, since $\gamma^{-1}_1$ and $\gamma^{-1}_2$ are both
%   consistent with $\eta$, it follows that $\pi$ must fix $\consp(g)$. Thus
%   $L(g) \sim \pi L(g)$, and so there exists $(\alpha, \beta)$ such that $\pi
%   L(g) = L(g) \cdot (\alpha, \beta)$.

%   We then have that,
%   \begin{align*}
%     L^{\gamma_1} (a,b) &= C_n[\gamma_1 \mathcal{A}](L(g)(a,b))\\
%                        & = C_n[\pi \gamma_1 \mathcal{A}][\pi L(g)(a,b)] \\
%                        & = C_n[\gamma_2 \mathcal{A}][L(g)((\alpha, \beta)(a,b))]\\
%                        & = L^{\gamma_2} ((\alpha, \beta) (a,b)),
%   \end{align*}
%   and it follows that $L^{\gamma_1} \sim L^{\gamma_2}$.
% \end{proof}

\begin{claim}
  Let $\sigma \in \spstab{g}$, $\eta \in U^{\underline{\consp(g)}}$, $\gamma: U
  \rightarrow [n]$ a bijection such that $\gamma^{-1} \sim \eta$. Then
  $\gamma^{-1} \cdot \sigma \sim \eta$.
\end{claim}
\begin{proof}
  Suppose $a \in \consp(g)$, then $\sigma (a) = a$ and so $\gamma^{-1} (\sigma
  (a)) = \gamma^{-1} (a) = \eta (a)$.
\end{proof}

% \begin{claim}
%   For $h \in H$ and $\sigma \in \stab(\consp (g))$, we have that
%   $\vec{r}_{\sigma h_1} = \sigma \vec{r}_{h_1}$.
% \end{claim}
% \begin{proof}
%   Proof in book
% \end{proof}

The following result shows that the action on the support of an object
determines the action on that object.

\begin{lem}
  \label{lem:support_determine_action}
  Let $\sigma, \sigma' \in \spstab{g}$, $a$ be an object on which on the action
  of these two permutations is defined, and $\consp(a) \subset [n]$ be a support
  of $a$. Then if $\sigma (\vec{\consp}(a)) = \sigma' (\vec{\consp}(a))$ then
  $\sigma (a) = \sigma' (a)$.
\end{lem}
\begin{proof}
  From $\sigma (\vec{\consp(a)}) = \sigma' (\vec{\consp(a)})$, it follows that
  $\pi = (\sigma')^{-1} \sigma$ fixes $\vec{\consp}(a)$. Thus $\sigma (a) =
  \sigma' (\pi (a)) = \sigma' (a)$.
\end{proof}

% \begin{lem}
%   \label{lem:map_same_support_same_row}
%   Let $\sigma, \sigma' \in \spstab{g}$ and suppose for some $i \in [a]$ we
%   have that $\sigma (\vec{r}_i) = \sigma (\vec{r}_i)$. It follows that
%   $\sigma_r (i) = \sigma'_r (i)$.
% \end{lem}
% \begin{proof}
%   Let $pi = (\sigma^{-1} \sigma')_r \in G^r_i$. Then $ (\sigma^{-1} \sigma')_r
%   (i) = \sigma^{-1}_r \sigma'_r (i) = i$, and so $\sigma_r (i) = \sigma'_r
%   (i)$.
% \end{proof}

% Let $X$ be a set on which the left group action on which the left group action
% of $\sym_n$ is defined and let $a \in X$. Let $\vec{x} \in A_i$ and $f \in
% U^{\underline{[\vert \consp{a} \vert]}}$. Then $\Pi^{\gamma}_{\vec{x}_f} (a)$
% is the action on $a$ defined by $\Pi^{\gamma}_{\vec{x}_f} (z) =
% \gamma(\vec{x_f}(z))$, for all $z \in \consp (a)$. Lemma
% \ref{lem:support_determine_action} tells us that this action is well defined.


% Let $(i, j) \in A \times B$, let $\vec{x} \in A^r_i$ and $\vec{y} \in A^c_j$.
% Define permutations $\Pi^{\gamma}_{\vec{x}}$ and $\Pi^{{\gamma},c}_{\vec{y}}$
% such that $\Pi^{\gamma,r}_{\vec{x}}(\vec{r}_i) = \gamma (\vec{x})$ and
% $\Pi^{\gamma,c}_{\vec{y}}(\vec{c}_j) = \gamma (\vec{y})$. From Lemma
% \ref{lem:support_determine_action} we have that the choice of $\vec{x}$ and
% $\vec{y}$ uniquely determine the mappings $\Pi^{\gamma,r}_{\vec{x}}(i)$ and
% $\Pi^{\gamma,c}_{\vec{y}}(j)$

Let $\alpha^{\gamma}: I \rightarrow A$ and $\beta^{\gamma}: J \rightarrow B$ be
defined by $\alpha^{\gamma} (i, \vec{x}) = \Pi^{\gamma}_{\vec{x}_{i}}(i)$ and
$\beta^{\gamma} (j, \vec{y}) = \Pi^{\gamma}_{\vec{y}_{j}}(j)$.

We now show that $\alpha^{\gamma}$ and $\beta^{\gamma}$ act as witnesses to the
row-column of equivalence of $M$ and $L^{\gamma}$. The following lemma proves
surjectivity.

\begin{lem} 
  For any bijection $\gamma : U \rightarrow [n]$ both $\alpha^{\delta}$ and
  $\beta^{\delta}$ are surjective.
\end{lem}
\begin{proof}
  We show that $\alpha$ is surjective, with the same result for $\beta$
  following similarly. Let $q \in A$ and let $i = \min (\orb (q))$. Then there
  exists $\sigma \in \spstab{g}$ such that $\sigma i = q$. Let $\vec{x} =
  \gamma^{-1} (\sigma(\vec{\consp}(i)))$. Notice that for $a \in \consp(i)$ we
  have that $\vec{x}_i(a) = \gamma^{-1} (\sigma (a))$, and since $\gamma^{-1}
  \cdot \sigma \sim \eta$, it follows that $\vec{x} \in A_i$.

  For $a \in \consp(i)$ we have $\Pi^{\gamma}_{\vec{x}_i} (a) = \gamma
  (\vec{x}_i(a)) = \gamma \cdot \gamma^{-1} \sigma (a) = \sigma (a)$. From Lemma
  \ref{lem:support_determine_action} it follows that $\alpha(i, \vec{x}) = q$.
\end{proof}

The following lemma allows us to factor through permutations.
\begin{lem}
  \label{lem:alpha_and_gamma}
  Let $(i,\vec{x}) \in I$ and $(j, \vec{y}) \in J$. Let $\gamma: U \rightarrow
  [n]$ be a bijection such that $\gamma^{-1} \sim \eta$ and $\pi \in spstab{g}$.
  Then $\pi \alpha^{\gamma}(i, \vec{x}) = \alpha^{\pi \gamma}(i, \vec{x})$ and
  $\pi \beta^{\gamma}(j, \vec{y}) = \beta^{\pi \gamma}(j, \vec{y})$.
\end{lem}
\begin{proof}
  We have that $\pi \alpha^{\gamma}(i, \vec{x}) = \pi
  \Pi^{\gamma}_{\vec{x}_i}(i)$ and $(\pi
  \Pi^{\gamma}_{\vec{x}_{i}}(\vec{\consp}(i)) = \pi \cdot \gamma (\vec{x}) =
  \Pi^{\pi \gamma}_{\vec{x}_i}(\vec{\consp}(i))$. Since
  $\Pi^{\gamma}_{\vec{x}_i}$ and $\Pi^{\pi \gamma}_{\vec{x}_i}$ are in
  $\spstab{g}$, it follows from Lemma \ref{lem:support_determine_action}, that
  $\pi \alpha^{\gamma}(i, \vec{x}) = \pi \Pi^{\gamma}_{\vec{x}_i} (i) = \Pi^{\pi
    \gamma}_{\vec{x}_i}(i) = \alpha^{\pi \gamma}(i, \vec{x})$. Similarly, $\pi
  \beta^{\gamma}(j, \vec{y}) = \beta^{\pi \gamma} (j, \vec{y})$.
\end{proof}

\begin{lem}
  \label{lem:alpha_ind_gamma}
  Let $(i,\vec{x}) \in I$ and $(j, \vec{y}) \in J$. Let $\gamma_1, \gamma_2: U
  \rightarrow [n]$ be bijections such that $\gamma^{-1}_1 \sim \eta$ and
  $\gamma^{-1}_2 \sim \eta$. Let $\mathcal{A}$ be a structure. Then
  $C_n[\gamma_1 \mathcal{A}] (L(\alpha^{\gamma_1}(i, \vec{x}),
  \beta^{\gamma_1}(j, \vec{y}))) = C_n[\gamma_2 \mathcal{A}]
  (L(\alpha^{\gamma_2}(i, \vec{x}), \beta^{\gamma_2}(j, \vec{y})))$.
\end{lem}
\begin{proof}
  We note that there exists $\pi \in \sym_n$ such that $\gamma_1 = \pi
  \gamma_2$. Moreover, since $\gamma^{-1}_1$ and $\gamma^{-1}_2$ are both
  consistent with $\eta$, it follows that $\pi \in \spstab{g}$. We then have
  that
  \begin{align*}
    C_n[\gamma_1 \mathcal{A}](L(\alpha^{\gamma_1}(i, \vec{x}), \beta^{\gamma_1}(j,
    \vec{y})) &= C_n[\pi \gamma_1 \mathcal{A}](\pi L(\alpha^{\gamma_1}(i, \vec{x}),
                \beta^{\gamma_1}(j, \vec{y})) \\
              &= C_n[\pi \gamma_1 \mathcal{A}](L(\pi
                \alpha^{\gamma_1}(i, \vec{x}), \pi \beta^{\gamma_1}(j, \vec{y}))\\
              &= C_n[\pi
                \gamma_1 \mathcal{A}](L(\alpha^{\pi \gamma_1}(i, \vec{x}), \pi \beta^{\pi
                \gamma_1}(j, \vec{y})\\
              &= C_n[\gamma_2 \mathcal] (L(\alpha^{\gamma_2}(i,
                \vec{x}), \beta^{\gamma_2}(j, \vec{y})))\\
  \end{align*}The third equality follows from Lemma \ref{lem:alpha_and_gamma}.
\end{proof}

\begin{lem}
  \label{lem:defining_h_from_IJ}
  Let $(i, \vec{x}) \in I$ and $j, \vec{y} \in J$. Let $\vec{r}$ and $\vec{c}$
  be described as above. Let $\sigma_1$ and $\sigma_2$ be as from Lemma
  \ref{lem:permutation_row-colum}. Let $h = L(g) (\sigma_1 (i), \sigma_2 (j))$.
  Then $\alpha^{\gamma'} (i, \vec{x}) = \row (h)$ and $\beta^{\gamma'}
  (i,\vec{y}) = \column(h)$.
\end{lem}
\begin{proof}
  $\alpha^{\gamma'}(i, \vec{x}) = \Pi^{\gamma'}_{\vec{x}_i} (i)$. It is
  sufficient to show that for all $a \in sp(g)$, $\Pi^{\gamma}_{\vec{x}_i} (a) =
  \sigma_1 (a)$. Note that $\Pi^{\gamma'}_{\vec{x}_i} (a) =
  (\Pi^{\gamma}_{(\vec{x}_{\vec{r}} \vert \vec{y}_{\vec{c}})})^{-1} \gamma
  (\vec{x}_i (a))$, and if $b = \sigma_1 (a)$ we have that $\gamma (\vec{x}_i
  (a)) = \gamma (\vec{x} (\vec{\consp}(i)^{-1} (a))) = \gamma (\vec{x}
  (\vec{\consp}(i)^{-1} (\sigma^{-1}_1 b))) = \gamma (\vec{x} ((\sigma_1 \cdot
  \vec{\consp}(i))^{-1} (b))) = \gamma (\vec{x} (\vec{r}^{-1}(b))) =
  \Pi^{\gamma}_{(\vec{x}_{\vec{r}} \vert \vec{y}_{\vec{c}})}(b)$. So
  $\Pi^{\gamma'}_{\vec{x}} (a) = b = \sigma_1 (a)$. Similarly for
  $\beta^{\gamma'}$.
\end{proof}

\begin{thm}
  Let $\delta: U \rightarrow [n]$, $M$ is row-column equivalent to $L^{\gamma}$.
  This equivalence is witnessed by $\alpha^{\gamma}$ and $\beta^{\gamma}$.
\end{thm}
\begin{proof}
  We have that $\alpha^{\gamma}$ and $\beta^{\gamma}$ are surjective. Let
  $\gamma' = (\Pi^{\gamma}_{(\vec{x}_{\vec{r}} \vert \vec{y}_{\vec{c}})})^{-1}
  \gamma$.

  \begin{align*}
    M((i, \vec{x}), (j, \vec{y}))
    &= (\vec{x}_{\vec{r}} \vert \vec{y}_{\vec{c}}) \in \EV_h \\
    &= C_n[\gamma \mathcal{A}] (\Pi^{\gamma}_{(\vec{x}_{\vec{r}} \vert \vec{y}_{\vec{c}})} (h)) \\
    &= C_n[\gamma' \mathcal{A}] (h) \\
    &= C_n[\gamma' \mathcal{A}](L(\row(h), \column (h)))\\
    &= C_n[\gamma' \mathcal{A}](L(\alpha^{\gamma'}(i, \vec{x}), \beta^{\gamma'}(j, \vec{y})))\\
    &= C_n[\gamma \mathcal{A}](L(\alpha^{\gamma}(i, \vec{x}), \beta^{\gamma}(j, \vec{y}))y)\\
    &= L^{\gamma}(\alpha^{\gamma}(i, \vec{x}), \beta^{\gamma}(j, \vec{y}))
  \end{align*}
  The second equality follows from Lemma \ref{}. The fifth equality follows from
  Lemma \ref{lem:defining_h_from_IJ}. The sixth equality follows from Lemma
  \ref{lem:alpha_ind_gamma}.

  
\end{proof}

\subsection{FPR Formulas}

% \begin{lem}
%   Let $(i, \vec{x}) \in I$ and $(j, \vec{y}) \in J$. Let $h \in H$ such that
%   $\row (h) \in \orb_r (i)$, $\column (h) \in \orb_c(j)$, and $\type(h) =
%   \type(\vec{x}, \vec{y})$. Let $\gamma: U \rightarrow [n]$ be a bijection and
%   $\mathcal{A}$ a structure. Then $C_n[\gamma
%   \mathcal{A}](\Pi^{\gamma}_{\vec{x} \vert \vec{y}} (h)) = C_n[\gamma
%   \mathcal{A}](L(\alpha^{\gamma}(i, \vec{x}), \beta^{\gamma}(j, \vec{y})))$.
% \end{lem}
% \begin{proof}
%   Let $\gamma' = (\Pi^{\gamma}_{\vec{x} \vert \vec{y}})^{-1} \gamma$. First we
%   show that $h = L(\alpha^{\gamma'}(i, \vec{x}), \beta^{\gamma'}(j,
%   \vec{y}))$. Notice that $\Pi^{\gamma'}_{\vec{x}}(\vec{r}_i) =
%   \gamma'(\vec{x}) = (\Pi^{\gamma}_{\vec{x} \vert \vec{y}})^{-1} \gamma
%   (\vec{x}) = \vec{r}_h$. But $\row(h) \in \orb_r(i)$ and so there exists
%   $\vec{x}' \in A^r_i$ such that $\Pi^{\gamma'}_{vec{x}'}(i) = \row{h}$ From
%   Lemma \ref{lem:support_determine_action} it follows that $\row (h) =
%   \Pi^{\gamma'}_{\vec{x}}(i) = \alpha^{\gamma'}(i, \vec{x})$. Similarly, we
%   can show that $\column(h) = \beta^{\gamma'}(j, \vec{y})$. It follows that $h
%   = L(\alpha^{\gamma'}(i, \vec{x}), \beta^{\gamma'}(j, \vec{y}))$, and so
%   \begin{align*}
%     C_n[\gamma \mathcal{A}](\Pi^{\gamma}_{\vec{x}
%     \vert \vec{y}} (h))
%     &= C_n[\gamma' \mathcal{A}](h)\\
%     &= C_n[\gamma' \mathcal{A}](L(\alpha^{\gamma'}(i, \vec{x}), \beta^{\gamma'}(j, \vec{y})))\\
%     &= C_n[\gamma \mathcal{A}](L(\alpha^{\gamma}(i, \vec{x}), \beta^{\gamma}(j, \vec{y}))).
%   \end{align*}
%   The final equality follows from Lemma \ref{lem:alpha_ind_gamma}.
% \end{proof}


% \chapter{New Stuff}
% Let $x,y \subset U$ such that $\vert x \vert = \vert y \vert = k \in
% \mathbb{N}$. Let $\vec{x}: [k] \rightarrow x$ and $\vec{y}: [k] \rightarrow y$
% be bijections.

% We need to define $r, c \subset [n]$ such that $\vert r \vert = \vert c \vert
% = k$ and there exists bijections $\vec {r}: [k] \rightarrow r$ and $\vec{c}:
% [k] \rightarrow c$.

% Let $u_1 , \ldots , u_{2k}$ be the first $2k$ elements of $[n] \setminus
% \SP(g)$ in order. Then let
% \[r = \eta^{-1} (x \cap \eta (\SP(g))) \cup \{u_{\vec{x}^{-1}(a)}: a \in x
%   \setminus \alpha (\SP (g))\} \] and
% \[s = \eta^{-1} (y \cap \eta (\SP(g))) \cup (x \cap y) \cup \{ u_{k +
%   \vec{x}^{-1}(a)} : a \in y \setminus (x \cup \alpha (\SP (g))) \}). \]
% Define
% \[
%   \vec{r} (a) =
%   \begin{cases}
%     \eta^{-1} (\vec{x} (a)) & a \in \vec{x}^{-1} (x \cap \eta (\SP(g))) \\
%     u_{a} & a \in \vec{x}^{-1} (x \setminus \eta(\SP(g)),
%   \end{cases}
% \]
% and

% \[
%   \vec{c} (a) =
%   \begin{cases}
%     \eta^{-1} (\vec{y} (a)) & a \in \vec{y}^{-1}(y \cap \eta (\SP (g))) \\
%     \vec{r} (a) & a \in \vec{y}^{-1} (x \cap y \setminus \eta (\SP (g))) \\
%     u_{k+a} & \text{otherwise}.
%   \end{cases}
% \]

% \begin{lem}
%   $x_r \sim \eta$ and $x_c \sim \eta$
% \end{lem}

% \begin{lem}
%   $\SP(g) \cap \SP(i) = \SP(g) \cap r$ and $\SP (g) \cap SP (j) = \SP (g) \cap
%   c$.
% \end{lem}

% Let $(\vec{x}, \vec{y})_{\vec{r}, \vec{c}}: r \cup c \rightarrow U$ be defined
% by
% \[
%   (\vec{x}, \vec{y})_{\vec{r}, \vec{c}}(a) =
%   \begin{cases}
%     \vec{x}(\vec{r}^{-1}(a)) & a \in r \\
%     \vec{y}(\vec{c}^{-1}(a)) & a \in c
%   \end{cases}
% \]

% This function is well defined.

% % \begin{lem}
% %   Let $r, c \subset U$ and let $r_1, r_2 : [k_1] \rightarrow r$ and $c_1, c_2 :
% %   [k_2] \rightarrow c$ then $\type(\vec{r}_1, \vec{c}_1) = \type (\vec{r}_2 ,
% %   \vec{c}_2)$.
% % \end{lem}

% \begin{lem}
%   $(\vec{x}_{\vec{r}} \vert \vec{y}_{\vec{y}}) \sim \eta$
% \end{lem}

% \begin{lem}
%   There exists $\sigma_1, \sigma_2 \in \spstab{g}$ such that $\sigma_1 \cdot
%   \vec{\consp(i)} = \vec{r}$ and $\sigma_2 \cdot \vec{\consp(j)} = \vec{c}$.
% \end{lem}

% \begin{lem}
%   Let $(i, \vec{x}) \in I$ and $j, \vec{y} \in J$. Let $\vec{r}$ and $\vec{c}$
%   be described as above. Let $\sigma_1$ and $\sigma_2$ be as from Lemma
%   \ref{}. Let $h = L(g) (\sigma_1 (i), \sigma_2 (j))$. Then $\alpha^{\gamma'}
%   (i, \vec{x}) = \row (h)$ and $\beta^{\gamma'} (i,\vec{y}) = \column(h)$.
% \end{lem}
% \begin{proof}
%   $\alpha^{\gamma'}(i, \vec{x}) = \Pi^{\gamma'}_{\vec{x}} (i)$. It is
%   sufficient to show that for all $a \in sp(g)$, $\Pi^{\gamma}_{\vec{x}} (a) =
%   \sigma_1 (a)$. Note that $\Pi^{\gamma'}_{\vec{x}_i} (a) =
%   (\Pi^{\gamma}_{(\vec{x}, \vec{y})_{\vec{r}, \vec{c}}})^{-1} \gamma
%   (\vec{x}_i (a))$, and if $b = \sigma_1 (a)$ we have that $\gamma (\vec{x}_i
%   (a)) = \gamma (\vec{x} (\vec{\consp}(i)^{-1} (a))) = \gamma (\vec{x}
%   (\vec{\consp}(i)^{-1} (\sigma^{-1}_1 b))) = \gamma (\vec{x} ((\sigma_1 \cdot
%   \vec{\consp}(i))^{-1} (b))) = \gamma (\vec{x} (\vec{r}^{-1}(b))) =
%   \Pi^{\gamma}_{(\vec{x}, \vec{y})_{\vec{r}, \vec{c}}}(b)$. So
%   $\Pi^{\gamma'}_{\vec{x}} (a) = b = \sigma_1 (a)$. Similarly for
%   $\beta^{\gamma'}$.
% \end{proof}

% \begin{lem}
%   Let $(i, \vec{x}) \in I$ and $(j, \vec{y}) \in J$. There exists $h \in H$
%   such that $h$ has type $(\vec{x}, \vec{y})$ and for any bijection $\gamma: U
%   \rightarrow [n]$ and structure $\mathcal{A}$ then $C_n[\gamma \mathcal{A}]
%   (\Pi^{\gamma}_{(\vec{x} \vert \vec{y})_{h}}) = C_n [\gamma
%   \mathcal{A}](L(\alpha^{\gamma}(i, \vec{x}), \beta^{\gamma}(j, \vec{y})))$.
% \end{lem}

% \begin{lem}
%   Let $h \in H$ and $\vec{z} \in A_h$ and $\Pi^{\delta}_h$ be a permutation
%   such that $\Pi^{\delta}_{\vec{z}}(a) = \delta(\vec{z}(a))$ for all $a \in
%   \sp(h)$. Then $C_n[\gamma \mathcal{A}](\Pi^{delta}_{\vec{z}} (h))$ iff
%   $\vec{z} \in \EV_h$.
% \end{lem}

% \begin{thm}
%   The matrix $M$ is equivalent to $L$.
% \end{thm}

\end{document}
