% !TEX root = ../paper.tex
\documentclass[../paper.tex]{subfiles}
\begin{document}
In this section we construct for every query computable by a $P$-uniform family
of transparent symmetric rank-circuits a corresponding $\FPR$ formula. In the
first subsection we show that for a rank gate in the circuit, the rank of that
gate depends only on which elements of the input structure are assigned to the
support of $g$. We then construct for $g$ and an assignment to the support of
$g$ a matrix $M$ which has the same rank as the matrix defined at that gate. In
the second subsection we implement the construction of this matrix in $\FPC$,
and hence produce a formula for evaluating a rank gate in $\FPR$. This formula,
together with the Immerman-Vardi theorem and the work of Anderson and Dawar
\cite{AndersonD17}, allows us to both define and evaluate a $P$-uniform family
of circuits in $\FPR$, completing the result.

\subsection {Evaluating Circuits}
In this subsection we fix $\mathcal{C} = (C_n)_{n \in \mathbb{N}}$, a family of
polynomial-size symmetric matrix-circuits with unique labels that compute a
$q$-ary query. We let $n_0$ be the constant in the hypothesis of the Support
Theorem. We fix a structure $\mathcal{A}$ of size $n > n_0$ over the universe
$U$ and an internal gate $g$ in the circuit $C_n$. Recall that in order to
evaluate the gate $g$ we need to consider a bijection $\gamma \in
[n]^{\underline{U}}$, with the evaluation of $g$ given by $C_n[\gamma
\mathcal{A}](g)$.

In this subsection we first show that the evaluation of $g$ depends only what
$\gamma$ maps to $\consp(g)$. We use this fact to limit our discussion to
characterising those assignments to the support of $g$ for which $g$ evaluates
to true. We then show that the set of assignments to the support of $g$ that
make it evaluate to true can be determined from the set of assignments to the
supports of its child gates that make them evaluate to true. In the next
subsection we show that this recursive step may be defined in $\FPR$, and hence
that the circuit (and family of circuits) can be evaluated in $\FPR$. We finally
construct the a f


We note that for a symmetric gate $g$ in $C_n$ the results of Anderson and
Dawar~\cite{AndersonD17} will suffice for evaluating $g$. As such, we assume
that $g$ is a matrix-symmetric gate. In order to simplify notation we let $A
\times B := \ind(g)$, and for $h \in H_g$ we let $\row(h) = L(g)^{-1}(h)(1)$ and
$\column(h) = L(g)^{-1}(h)(2)$.

Recall that in order to evaluate the gate $g$ we need to consider a bijection
$\gamma \in [n]^{\underline{U}}$, with the evaluation of $g$ given by
$C_n[\gamma \mathcal{A}](g)$. In this subsection we show that the evaluation of
$g$ depends only what $\gamma$ maps to $\consp(g)$. This result allows us to
characterise all those bijections for which $g$ evaluates to true using only
elements of $U^{\consp(g)}$, i.e.\ injections with a constant size domain. In the
next subsection we use this succinct encoding, along with the fixed-point
operator, to evaluate the entire circuit.
 
It will often be important that two assignments to a support be
\emph{compatible} with one another in the sense that there is an injection over
the union of their domains which agrees with each assignment on their respective
domains. We formalise this in the following definition.

\begin{definition}
	Let $f \in Y^{\underline{X}}$ and $g : Z^{\underline{W}}$. We say that $f$ is
	\emph{compatible} with $g$, and we write $f \sim g$, if for all $a \in X \cap
	W$, $f(a) = g(a)$ and for all $a \in X \setminus W$ and $b \in W \setminus X$,
	$f(a) \neq g(b)$.
\end{definition}

It is also useful to have some notation for combining two compatible functions.
Let $f : X \rightarrow Y$ and $p: X' \rightarrow Y'$ be compatible injections.
Define the combination $(f | p): X \cup X' \rightarrow Y \cup Y'$ by
\begin{align*}
	(f \vert p) (x) =
	\begin{cases}
	f (x) & x \in X  \\
	p (x) & x \in Y. 
	\end{cases}
\end{align*}

We also introduce some notation for the case where $f: X \rightarrow Y$ is an
injection and $g: X \rightarrow Z$ be a function. Then we let $g_f : \range(f)
\rightarrow Z$ be defined by for all $a \in \range(Y)$, $g_f(a) := g \circ
f^{-1}(a)$.

Given an assignment $\gamma \in [n]^{\underline{U}}$, we can evaluate the child
gates of $g$, forming the matrix $L^{\gamma} : A \times B \rightarrow \{0,1\}$
defined by $L^{\gamma} (a,b) := C[\gamma \mathcal{A}](L(g)(a,b))$ for $(a,b) \in
A \times B$. We may then evaluate $g$ by taking $L^{\gamma}$ as the input to the
Boolean function labelling $g$. In the following lemma we show that for any
$\gamma_1, \gamma_2 \in [n]^{\underline{U}}$ that agree on the support of $g$,
$L^{\gamma_1}$ is isomorphism-equivalent to $L^{\gamma_2}$. Since $g$ is matrix
symmetric, we may conclude that the evaluation of $g$ for $\gamma$ depends only
on the assignment to the support of $g$ (i.e. on what $\gamma$ maps to
$\consp(g)$).

\begin{lem}
	Let $g$ be a matrix-symmetric gate in $C_n$. Let $\eta \in
	U^{\underline{\consp(g)}}$ and $\gamma_1, \gamma_2 \in [n]^{\underline{U}}$
	such that $\gamma^{-1}_1 \sim \eta$ and $\eta \sim \gamma^{-1}_2$. Then
	$L^{\gamma_1}$ and $L^{\gamma_2}$ are isomorphism-equivalent.
	\label{lem:support-determines-evaluation}
\end{lem}

\begin{proof}
	We have that there exists a unique $\pi \in \sym_n$ such that $\pi \gamma_1 =
	\gamma_2$. Moreover, since $\gamma^{-1}_1$ and $\gamma^{-1}_2$ are both
	consistent with $\eta$, it follows that $\pi$ must fix $\consp(g)$ point-wise.
	Thus $L(g)$ is isomorphism-equivalent to $\pi L(g)$, and so there exists
	$(\sigma, \lambda) \in \sym_A \times \sym_B$ such that $\pi L(g) = L(g)
	(\sigma, \lambda)$.
		
	We then have that,
	\begin{align*}
		L^{\gamma_1} (a,b) & = C_n[\gamma_1 \mathcal{A}](L(g)(a,b))                    \\
		                   & = C_n[\pi \gamma_1 \mathcal{A}][\pi L(g)(a,b)]            \\
		                   & = C_n[\gamma_2 \mathcal{A}][L(g)((\sigma, \lambda)(a,b))] \\
		                   & = L^{\gamma_2} ((\sigma, \lambda) (a,b)),                 
	\end{align*}
	and it follows that $L^{\gamma_1} \sim L^{\gamma_2}$.
\end{proof}

For each gate $h \in C_n$ we associate with it a set $\Gamma_h$ consisting of
all those bijections which cause $h$ to evaluate to true, i.e. $\Gamma_h:=
\{\gamma \in [n]^{\underline{U}} : C[\gamma \mathcal{A}](h) = 1 \}$. Lemma
\ref{lem:supports-determine-evaluation} gives us that the membership of $\gamma$
in $\Gamma_h$ is entirely determined by what $\gamma$ maps to $\consp(h)$. As
such, we also associate with $h$ a set $\EV_h \subseteq
U^{\underline{\consp(g)}}$ consisting of all assignments to the support of $h$
for which $h$ evaluates to true, i.e. $\EV_h := \{ \eta:
U^{\underline{\consp(g)}} : \exists \gamma \in \Gamma_h \wedge \eta \sim
\gamma^{-1})\}$, and note that $\Gamma_h$ is entirely determined by $\EV_h$. The
important point to note is that each $\eta \in \EV_h$ is defined on a
constant-size domain, and as such $\EV_h$ gives us a succinct way of encoding
$\Gamma_h$.

We aim to show that we can recursively construct $\EV_g$. In particular, we show
that for any $\eta \in U^{\underline{sp(g)}}$, there is an $\FPR$-definable
matrix $M$ such that for any $\gamma \in [n]^{\underline{U}}$ with $\gamma^{-1}
\sim \eta$, $M$ is isomorphism-equivalent to $L^{\gamma}$. This result allows us
to decide the membership of $\eta$ in $\EV_g$, where $g$ is a $\rank^r_p$ gate,
by defining $M$ for $\eta$, computing the rank of $M$ over the field of
characteristic $p$, and then checking if this result is less than or equal to
the threshold value $r$.

For a gate $h \in H_g$, let $A_h := \{\vec{x} \in U^{\underline{\consp(h)}} :
\eta \sim \vec{x}\}$ be the set of assignment to the support of $h$ that are
compatible with $\eta$, our chosen assignment to the support of $g$. We should
also like to consider similar sets of assignments for other objects in the
circuit which may be permuted and which are relevant to $g$. More generally, let
$X$ be a set on which the left group action of $\sym_n$ is defined and let $s
\in X$. We let $A_s = \{\vec{x} \in U^{\underline{\consp(s)}} : \eta \sim
\vec{x}\}$. We will be particularly interested in the case where $X$ is the set
of `rows' or `columns' (i.e. $A$ or $B$) of the matrix indexing the inputs of
$g$.


We now define a matrix. We begin by defining the index sets for the matrix.

Let $R^{\min} = \{\min (\orb_g(\row(h))) : h \in H_g\}$ and $C^{\min} = \{ \min
(\orb_g (\column(h))) : h \in H_g\}$, and let
\begin{align*}
	I = \{(i, \vec{x}): i \in R^{\min}, \vec{x} \in A_i\}, 
\end{align*}
and
\begin{align*}
	J = \{(j, \vec{y}): j \in C^{\min}, \vec{y} \in A_j\}. 
\end{align*}

We should like to associate with each index $((i, \vec{x}), (j, \vec{y})$ a gate
$h$ and then assign $\vec{x}$ and $\vec{y}$ to the support $h$. The matrix at
that point is then set to 1 if, and only if, the assignment causes $h$ to
evaluate to true. It remains to show how to select an appropriate gate $h$ and
ensure that the assignments to its row and column supports are consistent. The
following lemma is used to make this selection.

\begin{lem}
	\label{lem:permutation-row-column}
	For any $(i, \vec{x}) \in I$ and $(j, \vec{y}) \in J$ there exists $\sigma_r,
	\sigma_c \in \spstab{g}$ such that if $\vec{r} := \sigma_r \vec{\consp}_g(i)$
	and $\vec{c} := \sigma_c \vec{\consp}_g(j)$ then $\vec{x}_{\vec{r}} \sim
	\vec{y}_{\vec{c}}$.
\end{lem}
\begin{proof}
	Let $\sigma_r$ be the identity. Then $\vec{r} = \vec{\consp}(i)$ and
	$\vec{x}_{\vec{r}} = \vec{x}$. Let $u_1, \ldots , u_k$ be the first $k$
	elements of $[n] \setminus (\consp(g) \cup \consp(i))$. Let $T = \{a \in
	\consp(j) \setminus \consp(g) : \vec{y}(a) \in \vec{x}(\consp(i) \setminus
	\consp(g))\}$, let $\vec{c}' \in [n]^{\underline{\consp(j)}}$ be defined by
	\[
		\vec{c}' (a) =
		\begin{cases}
			a                        & a \in \consp(g) \cap \consp(j)                                     \\
			\vec{x}^{-1}(\vec{y}(a)) & a \in T                                                            \\
			u_{i}                    & \text{$a$ is the $i$th element of $(\consp(j) \setminus (\consp(g) 
			\cup T)))$}.
		\end{cases}
	\]
	Let $\sigma_c$ be any permutation such that $\sigma_c \vec{\consp}(j)) =
	\vec{c}'$. Then $\vec{c} := \sigma_c \vec{\consp}(j) = \vec{c}'$. Let $c$ be
	the image of $\vec{c}$.
		
	It remains to show that $\vec{x} \sim \vec{y}_{\vec{c}}$. Let $z \in \consp(i)
	\cap c$ and $z' = \vec{c}^{-1}(z)$. Suppose $z' \in \consp(g)\cap \consp(j)
	\cap \consp(i)$. Then we have $\vec{x}(z) = \vec{x}(z') \eta (z') =
	\vec{y}(z') = \vec{y}_{\vec{c}}(z)$. By a similar argument, if $z' \consp(g)
	\cap \consp(j) \setminus \consp(i)$ we have $\vec{x}(z) \neq
	\vec{y}_{\vec{c}}(z)$. Suppose $z' \in T$. Then $\vec{c}(z') =
	\vec{x}^{-1}(\vec{y}(z'))$ which gives us that $\vec{y}_{\vec{c}}(z) =
	\vec{y}(z') = \vec{x}(\vec{c}(z')) = \vec{x}(z)$. We finally note that $z'
	\notin (\consp(j) \setminus (\consp(g) \cup T))$ as $\{u_1 , \ldots , u_k\}
	\cap \consp(i) = \emptyset$. This completes the intersection component of
	compatibility.
		
	Let $z \in \consp(i) \setminus c$ and $w \in c \setminus \consp(i)$. Let $w' =
	\vec{c}^{-1}(w)$. Suppose $w' \in \consp(g) \cap \consp(j)$. Then from the
	fact that $w \notin \consp(i)$ we have that $\vec{y}_{\vec{c}}(w) \neq
	\vec{x}(z)$ (using a similar argument as in the above case). Suppose $w' \in
	T$. Then there exists $b \in \consp(i) \setminus \consp(g)$ such that
	$\vec{y}(w') = \vec{x}(b)$. It follows that $w = \vec{c}(w') =
	\vec{x}^{-1}\vec{y} (w') = b$, which is a contradiction as $w \notin
	\consp(i)$ by assumption, and we conclude $w' \notin T$. Suppose finally that
	$w' \in \consp(j) \setminus (\consp(g) \cup T)$. It follows that for all $b
	\in \consp(i) \setminus \consp(g)$ we have that $\vec{x}(b) \neq
	\vec{y}_{\vec{c}}(w)$. It remains to check the result for $b \in (\consp(i)
	\setminus c) \setminus (\consp(i) \setminus \consp(g)) = (\consp(g) \cap
	\consp(i)) \setminus c$. But then $b \notin \consp(j)$, and so $\vec{x}(b) =
	\eta (b)$. But $w' \notin \consp(g)$, and so $\vec{x}(b) = eta(b) \neq
	\vec{y}(w') = \vec{y}(w)$. The result follows.
\end{proof}

From Lemma~\ref{lem:permutation-row-column} we have for any $(i, \vec{x}) \in I$
and $(j, \vec{y}) \in J$, corresponding vectors and permutations $\vec{r}$,
$\vec{c}$, $\sigma_r$ and $\sigma_c$. Let $h := L(g)(\sigma_r(i), \sigma_c
(j))$. Let $\vec{b}$ for $(i, \vec{x})$ and $(j, \vec{y})$ be the restriction of
$(\vec{x}_{\vec{r}} \vert \vec{y}_{\vec{c}})$ to the support of $h$. We define
the matrix $M : I \times J \rightarrow \{0,1\}$ by

\begin{align*}
	M((i , \vec{x}), (j, \vec{y})) := \vec{b} \in EV_h. 
\end{align*}

% Let $I /{\sim} = \{(i, [\vec{x}]): i \in R^{\min}, [\vec{x}] \in
% A_i/{\sim_i}\},$ and $J/{\sim} = \{(j, [\vec{y}]): j \in C^{\min}, [\vec{y}]
% \in A_j/{\sim_j}\}$. Let the matrix $M_{~} : I /{\sim} \times J / {\sim}
% rightarrow \{0,1\}$ be the function defined by $M_{~} ((i, [\vec{x}]) (j,
% [\vec{y}])) = M(((i, \vec{x}), (j, \vec{y})))$. Lemma
% \ref{lem:matrix-quot-well-defined} gives us that this function is
% well-defined.

% It remains to show that $M_{~} \sim L^{\gamma}$ for some $\gamma: U
% \rightarrow [n]$.

Let $X$ be a set on which the left action of $\sym_n$ is defined, and let $x \in
X$. Let $f \in U^{\underline{\consp(x)}}$ and $\gamma\in [n]^{\underline{U}}$,
then we let $\Pi^{\gamma}_{f} \in \spstab{g}$ be any permutation in $\sym_n$
such that $\Pi^{\gamma}_f (a) = \gamma (f(a))$ for all $a \in \consp(x)$. Note
that from Lemma~\ref{lem:support-determine-action}, $\Pi^{\gamma}_f(x)$ is
well-defined independently of the particular choice of permutation.

Let $\alpha^{\gamma}: I \rightarrow A$ and $\beta^{\gamma}: J \rightarrow B$ be
defined by $\alpha^{\gamma} (i, \vec{x}) := \Pi^{\gamma}_{\vec{x}}(i)$ and
$\beta^{\gamma} (j, \vec{y}) := \Pi^{\gamma}_{\vec{y}}(j)$, respectively.

The following result shows that we may check if an assignment to a gate $h$
makes $h$ evaluate to true by analysing certain gates in the orbit of $h$.

\begin{lem}
	Let $\gamma\in [n]^{\underline{U}}$ and $h \in C_n$. Then $\vec{z} \in \EV_h$
	if, and only if, $C_n[\gamma \mathcal{A}](\Pi^{\gamma}_{\vec{z}} (h)) = 1$.
\label{lem:translate-EV-circuits}
\end{lem}
\begin{proof}
	We have that $C_n[\gamma \mathcal{A}](\Pi^{\gamma}_\nu(h))$ if, and only if,
	$C_n[(\Pi^{\gamma}_{\nu})^{-1}\gamma \mathcal{A}] (h)$. Then, by the
	definition of $\EV_h$, $\nu \in \EV_h$ if, and only if, there exists $\gamma'
	\in [n]^{\underline{U}}$ such that $C_n[\gamma' \mathcal{A}](h) = 1$ and $\nu
	= \gamma\restriction{\consp(h)}$.
\end{proof}

We now show that $\alpha^{\gamma}$ and $\beta^{\gamma}$ are surjective.

\begin{lem} 
	For any bijection $\gamma \in [n]^{\underline{U}}$ such that $\gamma^{-1} \sim
	\eta$ both $\alpha^{\gamma}$ and $\beta^{\gamma}$ are surjective.
\label{lem:alpha-beta-surjective}
\end{lem}
\begin{proof}
	We show that $\alpha^{\gamma}$ is surjective, with the same result for
	$\beta^{\gamma}$ following similarly. Let $q \in A$ and let $i = \min
	(\orb_{g} (q))$. Then there exists $\sigma \in \spstab{g}$ such that $\sigma
	(i) = q$. Let $\vec{x} := \gamma^{-1} \sigma \vec{\consp}_g(i)$. Notice that
	for $a \in \consp_g(i)$ we have that $\vec{x}(a) = \gamma^{-1} (\sigma (a))$,
	and since $\gamma^{-1} \sigma \sim \eta$, it follows that $\vec{x} \in A_i$.
		
	For $a \in \consp_g(i)$ we have $\Pi^{\gamma}_{\vec{x}} (a) = \gamma
	(\vec{x}(a)) = \gamma \gamma^{-1} \sigma (a) = \sigma (a)$. From Lemma~\ref{lem:support-determine-action} it follows that $\alpha(i, \vec{x}) =
	\sigma(i) = q$.
\end{proof}

% Note that $\alpha^{\gamma}$ and $\beta^{\gamma}$ can be lifted to functions on
% $I /{\sim}$ and $J /{\sim}$ respectively. Lemma
% \ref{lem:functions-well-defined} gives us that these liftings are
% well-defined.

% We now show that $\alpha^{\gamma}$ and $\beta^{\gamma}$ witness the fact that
% $M_{~}$ and $L^{\gamma}$ are isomorphism-equivalent. We first show that
% $\alpha^{\gamma}$ and $\beta^{\gamma}$ are surjective.

% \begin{remark}
%   I am still in the process of rewriting this section (and figuring out how to
%   restructure it) so as to include this quotienting operation. The remainder
%   of this section should still be looked at, but I have yet to integrate the
%   quotienting operation. I would also like to talk with you about this step.
% \end{remark}

We now prove a number of important technical lemmas that will ultimately allow
us to to prove that the matrix $M$ (quotiented by some appropriate equivalence
relation) is isomorphism-equivalent to $L^{\gamma}$. The following lemma
allows us to factor a permutation through $\alpha^{\gamma}$ and $\beta^{\beta}$.

\begin{lem}
\label{lem:alpha-and-gamma}
	Let $(i,\vec{x}) \in I$ and $(j, \vec{y}) \in J$. Let $\gamma: U \rightarrow
	[n]$ be a bijection such that $\gamma^{-1} \sim \eta$ and $\pi \in spstab{g}$.
	Then $\pi \alpha^{\gamma}(i, \vec{x}) = \alpha^{\pi \gamma}(i, \vec{x})$ and
	$\pi \beta^{\gamma}(j, \vec{y}) = \beta^{\pi \gamma}(j, \vec{y})$.
\end{lem}
\begin{proof}
	We have that $\pi \alpha^{\gamma}(i, \vec{x}) = \pi \Pi^{\gamma}_{\vec{x}}(i)$
	and $(\pi \Pi^{\gamma}_{\vec{x}}(\vec{\consp}(i)) = \pi \cdot \gamma (\vec{x})
	= \Pi^{\pi \gamma}_{\vec{x}}(\vec{\consp}(i))$. Since $\Pi^{\gamma}_{\vec{x}}$
	and $\Pi^{\pi \gamma}_{\vec{x}}$ are in $\spstab{g}$, it follows from Lemma
	~\ref{lem:support-determine-action}, that $\pi \alpha^{\gamma}(i, \vec{x}) =
	\pi \Pi^{\gamma}_{\vec{x}} (i) = \Pi^{\pi \gamma}_{\vec{x}}(i) = \alpha^{\pi
		\gamma}(i, \vec{x})$. Similarly, $\pi \beta^{\gamma}(j, \vec{y}) =
	\beta^{\pi \gamma} (j, \vec{y})$.
\end{proof}

The following result applies Lemma~\ref{lem:alpha-and-gamma} to the evaluation
of gates in the circuit.

\begin{lem}
	\label{lem:alpha-ind-gamma}
	Let $(i,\vec{x}) \in I$ and $(j, \vec{y}) \in J$. Let $\gamma_1, \gamma_2 \in
	[n]^{\underline{U}}$ be such that $\gamma^{-1}_1 \sim \eta$ and $\gamma^{-1}_2
	\sim \eta$. Let $\mathcal{A}$ be a structure. Then $C_n[\gamma_1 \mathcal{A}]
	(L(\alpha^{\gamma_1}(i, \vec{x}), \beta^{\gamma_1}(j, \vec{y}))) =
	C_n[\gamma_2 \mathcal{A}] (L(\alpha^{\gamma_2}(i, \vec{x}),
	\beta^{\gamma_2}(j, \vec{y})))$.
\end{lem}
\begin{proof}
	We note that there exists $\pi \in \sym_n$ such that $\gamma_1 = \pi
	\gamma_2$. Moreover, since $\gamma^{-1}_1$ and $\gamma^{-1}_2$ are both
	consistent with $\eta$, it follows that $\pi \in \spstab{g}$. We then have
	that
	\begin{align*}
		C_n[\gamma_1 \mathcal{A}](L(\alpha^{\gamma_1}(i, \vec{x}), \beta^{\gamma_1}(j,
		\vec{y})) & = C_n[\pi \gamma_1 \mathcal{A}](\pi L(\alpha^{\gamma_1}(i, \vec{x}), 
		\beta^{\gamma_1}(j, \vec{y})) \\
		          & = C_n[\pi \gamma_1 \mathcal{A}](L(\pi                                
		\alpha^{\gamma_1}(i, \vec{x}), \pi \beta^{\gamma_1}(j, \vec{y}))\\
		          & = C_n[\pi                                                            
		\gamma_1 \mathcal{A}](L(\alpha^{\pi \gamma_1}(i, \vec{x}), \pi \beta^{\pi
		\gamma_1}(j, \vec{y})\\
		          & = C_n[\gamma_2 \mathcal] (L(\alpha^{\gamma_2}(i,                     
		\vec{x}), \beta^{\gamma_2}(j, \vec{y})))\\
	\end{align*}The third equality follows from Lemma \ref{lem:alpha-and-gamma}.
\end{proof}

The following result shows that, for a given $(i, \vec{x}) \in I$ and $(j,
\vec{y}) \in J$ we may define a gate $h \in H_g$ and a bijection $\gamma' \in
[n]^{\underline{U}}$, such that $(\gamma')^{-1} \sim \eta$, and
$\alpha^{\gamma'}(i, \vec{x})$ and $\beta^{\gamma'}(j, \vec{y})$ define the row
and column of $h$.

\begin{lem}
	\label{lem:defining-h-from-IJ}
	Let $(i, \vec{x}) \in I$ and $(j, \vec{y}) \in J$. From Lemma
	\ref{lem:permutation-row-column} we have $\vec{r}$, $\vec{c}$, $\sigma_r$ and
	$\sigma_c$. Let $h := L(g) (\sigma_r (i), \sigma_c (j))$ and $\vec{z}$ be the
	restriction of $(\vec{x}_{\vec{r}} \vert \vec{y}_{\vec{c}})$ to the support of
	$h$. Let $\gamma' := (\Pi^{\gamma}_{\vec{z}})^{-1} \gamma$. Then
	$\alpha^{\gamma'} (i, \vec{x}) = \row (h)$ and $\beta^{\gamma'} (i,\vec{y}) =
	\column(h)$.
\end{lem}
\begin{proof}
	We prove the result for $\alpha^{\gamma'}$. The $\beta^{\gamma'}$ case follows
	similarly. We have from Lemma \ref{lem:support-determine-action} that it is
	sufficient to show that for all $a \in \consp_g(i)$, $\alpha^{\gamma'}(i,
	\vec{x}) = \Pi^{\gamma'}_{\vec{x}} (a) = \sigma_r (a)$. Let $a \in
	\consp_g(i)$. We have that $\Pi^{\gamma'}_{\vec{x}} (a) =
	(\Pi^{\gamma}_{\vec{z}})^{-1} \gamma (\vec{x} (a))$ and $\gamma (\vec{x}(a)) =
	\gamma (\vec{x} (\vec{r}^{-1}(\vec{r}(a)))) = \gamma (\vec{x}_{\vec{r}}
	(\sigma_r(a)))) = \Pi^{\gamma}_{\vec{z}}(\sigma_r(a))$. The final equality
	follows from the fact that $r \cup c = \consp(g) \cup \consp(h)\in $, and thus
	$\sigma_r(a) \in \consp(g)$ or $\sigma_r(a) \in \consp(h)$. If $\sigma_r(a)
	\in \consp(g)$ then $\sigma_r(a) = a = \Pi^{\gamma}_{\vec{z}}(\sigma_r(a))$ as
	both $\sigma_r$ and $\Pi^{\gamma}_{\vec{z}}$ are in $\spstab{g}$. If
	$\sigma_r(a) \in \consp(g)$ then, since $\sigma_r(a) \in r$,
	$\Pi^{\gamma}_{\vec{z}}(\sigma_r(a)) = \gamma(\vec{z}(\sigma_r(a))) =
	\gamma(\vec{x}_{\vec{r}}(\sigma_r(a)))$. We thus have that
	$\Pi^{\gamma'}_{\vec{x}}(a) =
	(\Pi^{\gamma}_{\vec{z}})^{-1}\Pi^{\gamma}_{\vec{z}} (\sigma_r(a)) =
	\sigma_r(a)$. The result follows.
\end{proof}

We combine the above lemmas in order to show that $\alpha^{\gamma}$ and
$\beta^{\gamma}$ together defines a surjective homomorphism from $M$ to
$L^{\gamma}$.

\begin{thm}
	Let $\gamma\in [n]^{\underline{U}}$ be such that $\eta \sim \gamma^{-1}$ and
	let $(i, \vec{x})\in I$ and $(j, \vec{y})\in J$. It follows that $M((i,
	\vec{x}), (j, \vec{y})) = L^{\gamma}(\alpha^{\gamma}(i, \vec{x}),
	\beta^{\gamma}(j, \vec{y})$.
	\label{lem:ML-equal-elements}
\end{thm}
\begin{proof}
	From Lemma \ref{lem:permutation-row-column} we have $\vec{r}$, $\vec{c}$,
	$\sigma_r$ and $\sigma_c$. Let $\gamma' = (\Pi^{\gamma}_{(\vec{x}_{\vec{r}}
		\vert \vec{y}_{\vec{c}})})^{-1} \gamma$. Let $h = L(g)(\sigma_r(i),
	\sigma_c(j))$. Then
	\begin{align*}
		M((i, \vec{x}), (j, \vec{y}))
		  & = (\vec{x}_{\vec{r}} \vert \vec{y}_{\vec{c}}) \in \EV_h                                    \\
		  & = C_n[\gamma \mathcal{A}] (\Pi^{\gamma}_{(\vec{x}_{\vec{r}} \vert \vec{y}_{\vec{c}})} (h)) \\
		  & = C_n[\gamma' \mathcal{A}] (h)                                                             \\
		  & = C_n[\gamma' \mathcal{A}](L(\row(h), \column (h)))                                        \\
		  & = C_n[\gamma' \mathcal{A}](L(\alpha^{\gamma'}(i, \vec{x}), \beta^{\gamma'}(j, \vec{y})))   \\
		  & = C_n[\gamma \mathcal{A}](L(\alpha^{\gamma}(i, \vec{x}), \beta^{\gamma}(j, \vec{y}))y)     \\
		  & = L^{\gamma}(\alpha^{\gamma}(i, \vec{x}), \beta^{\gamma}(j, \vec{y}))                      
	\end{align*}
	The second equality follows from Lemma~\ref{lem:translate-EV-circuits}. The
	fifth equality follows from Lemma~\ref{lem:defining-h-from-IJ}. The sixth
	equality follows from Lemma~\ref{lem:alpha-ind-gamma}.
\end{proof}

We have shown that, but for injectivety, $\alpha^{\gamma}$ and $\beta^{\gamma}$
witnesses an isomorphism-equivalence between $M$ and $L^{\gamma}$. It remains to
show that $\alpha^{\gamma}$ and $\beta^{\gamma}$ are also injective. However, it
can be shown that $\alpha^{\gamma}(i, \vec{x}) = \alpha^{\gamma}(i, \vec{x}')$
if, and only if, there exists $\pi \in \stab_g(i) \cap \spstab{g}$ such that
$\vec{x} = \vec{x}'\sigma pi$ (see Lemma~\ref{lem:functions-mutual-equivalence}). As such, $\alpha^{\gamma}$ and $\beta^{\gamma}$ are not in general injective and
we define a new matrix $M_{\equiv}$ by quotienting $M$ by an appropriate
equivalence relation. We also show that the liftings of $\alpha^{\gamma}$ and
$\beta^{\gamma}$ to the equivalence classes of $\equiv$ witness an
isomorphism-equivalence between $M_\equiv$ and $L^{\gamma}$. We now define the
equivalence relation.

Recall that $\tau$ and $D$ denote the vocabulary and universe of $g$
respectively. Let $s \in D$ and let $\vec{x}, \vec{x}' \in A_s$ and we say that
$\vec{x}$ and $\vec{x}'$ are \emph{mutually stable} on $s$ if there exists $\pi
\in \stab_g(s) \cap \spstab{g}$ such that $\vec{x} (\vec{\consp}_g(s)) =
\vec{x}'\pi (\vec{\consp}_g(s))$. We also say that two permutations $\sigma,
\sigma' \in \spstab{g}$ are \emph{mutually stable} on $s$ if there exists $\pi
\in \stab_g(s) \cap \spstab{g}$ such that $\sigma (\vec{\consp}_g(s)) = \sigma'
\pi (\vec{\consp}_g(s)) $. Note that mutual stability is an equivalence relation
on $A_s$ (and $\spstab{g}$), and we denote the equivalence of two vectors
$\vec{x}, \vec{x}' \in A_s$ by $\vec{x} \equiv_s \vec{x}'$ and the equivalence
of two permutations $\sigma, \sigma' \in \spstab{g}$ by $\sigma \equiv_s
sigma'$).

% We define a similar equivalence relation $\stab_n{\consp_g(g)} \cap
% \spstab{g}$

% We now show that any permutation

% \begin{lem}
%   Let $s \in D$ and let $\sigma, \sigma' \in \spstab{g} \cap \stab_g(s))$. We
%   have that $\sigma(s) = \sigma' (s)$ if, and only if, $\vec{x} \sigma
%   \equiv_s \vec{x} \sigma'$ for all $\vec{x} \in A_s$.
%   \label{lem:functions-mutual-equivalence}
% \end{lem}
% \begin{proof}
%   Suppose $\sigma(s) = \sigma'(s)$. Then let $\pi = (\sigma')^{-1} \sigma$.
%   Then clearly $\pi \in \stab_g(s) \cap \spstab{g} \subseteq
%   \setstab(\consp_g{s}) \cap \spstab{g}$ and for all $\vec{x} \in A_s$,
%   $\vec{x} \sigma = \vec{x} \sigma' \pi$.

%   Suppose $\vec{x} \sigma \equiv_s \vec{x} \sigma'$ for all $\vec{x} \in A_s$.
%   Thus there exists $\pi \in \spstab{g} \cap \setstab(\consp_g(i))$ such that
%   $\vec{x} \sigma = \vec{x} \sigma' \pi$ for all $\vec{x} \in A_s$. It can be
%   shown that $\pi = (\sigma')^{-1} \sigma$. From mutual stability $\pi \in
%   \stab_g (s)$ and so $(\sigma')^{-1}\cdot \sigma (s) = \pi (s) = s$. The
%   result follows.
% \end{proof}

% \begin{lem}
%   Let $\gamma \in [n]^{\underline{U}}$ such that $\gamma^{-1} \sim \eta$. Let
%   $s \in D$ and let $\sigma, \sigma' \in \spstab{g} \cap
%   \setstab(\consp_g(s))$. Then $\sigma(s) = \sigma' (s)$ if, and only if,
%   $\vec{x} \sigma \equiv_s \vec{x} \sigma'$ for all $\vec{x} \in A_s$.
%   \label{lem:functions-well-defined}
% \end{lem}

% \begin{proof}
%   Suppose $\sigma (s) = \sigma'(s)$. Then $s = \sigma^{-1} \sigma' (s)$. So
%   $\sigma^{-1}\sigma'$ (removed, to be included later)
%   %   Suppose $\sigma(s) = \sigma'(s)$. Take $x := \sigma(s)$. So then
%   %   $\gamma^{-1}(\sigma(s))$
% \end{proof}

We now define a new matrix $M_{\equiv}$ and show that $\alpha^{\gamma}$ and
$\beta^{\gamma}$, lifted to act on equivalence classes, define the required
isomorphism.

Let $I_{\equiv} := \{(i, [\vec{x})]_{\equiv_i}) : (i, \vec{x}) \in I\}$ and
$J_\equiv := \{(j, [\vec{y}]_{\equiv_j}) : (j, \vec{y})\}$. Let $M_{\equiv} :
I_{\equiv} \times J_{\equiv} \rightarrow \{0,1\}$ be defined by $M_\equiv ((i,
[\vec{x}]_\equiv), (j, [\vec{y}]_\equiv)) := M((i,\vec{x}), (j, \vec{y}))$, for
$(i, [\vec{x}]_{\equiv}) \in I_\equiv$ and $(j, [\vec{y}]_{\equiv}) \in
J_\equiv$.

We show in Lemma~\ref{lem:matrix-quot-well-defined} that this matrix is
well-defined. We show in Lemma~\ref{lem:alpha-beta-mutal-equivalence} that
$\alpha^{\gamma}$ and $\beta^{\gamma}$ are constant on mutual equivalence
classes, and as such may be lifted to act on equivalence classes. We combine
these results to prove the required isomorphism-equivalence.

We first prove the claim that those permutations that agree on $s \in D$ are
exactly those that are mutually stable on $s$.

\begin{lem}
	Let $s \in D$ and let $\sigma, \sigma' \in \stab(\consp(g))$. Then $\sigma(s)
	= \sigma' (s)$ if, and only if, $\sigma \equiv_s \sigma'$.
	% there exists $\pi \in \stab_g(s) \cap \spstab{g}$ such that $\sigma
	% (\vec{\consp}_g(i)) = \sigma' \pi (\vec{\consp}_g(i))$.
	\label{lem:functions-mutual-equivalence}
\end{lem}
\begin{proof}
	Suppose $\sigma(s) = \sigma'(s)$. Then $\pi := (\sigma')^{-1}\sigma \in
	stab_g(s)$ and $\sigma = \sigma' \pi$.
		
	Suppose there exists $\pi \in \stab_g(s)$ such that $\sigma
	(\vec{\consp}_g(i))= \sigma' \pi (\vec{\consp}_g(i))$. From Lemma~\ref{lem:support-determine-action} $\sigma (i) = \sigma' \pi (i)$, and so,
	since $\pi \in \stab_g(s)$, $\sigma(i) = \sigma' \pi (i) = \sigma' (i)$.
\end{proof}

% \begin{lem}
%   Let $X$ be a set on which the left group action of $\sym_n$ is defined, and
%   let $\sigma, \sigma' \in \stab(\consp(g)) \stab_g(i)$. We have that
%   $\sigma(s) = \sigma' (s)$ if, and only if, there exists $\sigma \equiv_s
%   \sigma'$.
%   \label{lem:functions-mutual-equivalence}
% \end{lem}
% \begin{proof}
%   Suppose $\sigma(s) = \sigma'(s)$. Then let $\pi = (\sigma')^{-1} \sigma$.
%   Then clearly $\pi \in \stab(s) \cap \spstab(g)$ and $\sigma = \sigma' \pi$.

%   Suppose $\sigma \equiv_s \sigma'$. Then let $\pi = (\sigma')^{-1} \cdot
%   \sigma$. From mutual stability $\pi \in \stab (s)$ and so
%   $(\sigma')^{-1}\cdot \sigma (s) = \pi (s) = s$. The result follows.
% \end{proof}

The following lemma gives us that $\alpha^{\gamma}$ and $\beta^{\gamma}$ are
constant on classes of mutually stable assignments.

\begin{lem}
	Let $((i, \vec{x}), (j, \vec{y})), ((i, \vec{x}'), (j, \vec{y}')) \in I \times
	J$ and let $\gamma \in [n]^{\underline{U}}$. If $\vec{x} \equiv_i \vec{x}'$
	then $\alpha^{\gamma}(i, \vec{x}) = \alpha^{\gamma}(i, \vec{x}')$ and if
	$\vec{y} \equiv_j \vec{y}'$ then $\beta^{\gamma}(j, \vec{y}) =
	\beta^{\gamma}(j, \vec{y}')$.
	\label{lem:alpha-beta-mutal-equivalence}
\end{lem}
\begin{proof}
	From mutual equivalence there exists $\sigma \in \stab_g(i) \cap \spstab{g}$
	such that $\vec{x}' = \vec{x} \sigma$. We have that $\alpha^{\gamma}(i,
	\vec{x})$ stabilises $\consp(g)$. Also, for all $a \in \consp_g(i)$,
	$\Pi^{\gamma}_{\vec{x}} (\sigma (a)) = \gamma (\vec{x}(\sigma (a))) = \gamma
	(\vec{x}'(a)) = \Pi^{\gamma}_{\vec{x}'}(a)$. It follows from Lemma
	\ref{lem:functions-mutual-equivalence} that $\alpha^{\gamma}(i,\vec{x}) =
	\Pi^{\gamma}_{\vec{x}} (i) = \Pi^{\gamma}_{\vec{x}'}(i) = \alpha^{\gamma}(i,
	\vec{x}')$. The result follows similarly for $\beta$.
\end{proof}

We define the liftings of $\alpha^{\gamma}$ and $\beta^{\gamma}$ as follows. Let
$\alpha^{\gamma}_{\equiv} : I_\equiv \rightarrow A$ and $\beta^{\gamma}_\equiv:
J_\equiv \rightarrow B$ by $\alpha^{\gamma}_{\equiv}(i, [\vec{x}]) :=
alpha^{\gamma} (i, \vec{x})$ and $\beta^{\gamma}_{\equiv}(j, [\vec{y}]) =
\beta^{\gamma} (j, \vec{y})$, for all $(i, \vec{x}) \in I_{\equiv}$ and $(j,
\vec{y}) \in J_\equiv$. Lemma~\ref{lem:alpha-beta-mutal-equivalence} gives us
that these liftings are well defined.

% \begin{proof}
%   From mutual equivalence there exists $\sigma_1 \in \stab(i) \cap \spstab{g}$
%   and $\sigma_2 \in \stab{j} \cap \spstab{g}$ such that $\vec{x}' = \vec{x}
%   \cdot \sigma_1\restriction{\consp(i)}$ and $\vec{y}' = \vec{y} \cdot
%   \sigma_2\restriction{\consp(j)}$. Notice that $\alpha^{\gamma}(i, \vec{x})$
%   and $\beta^{\gamma}(j, \vec{y})$ both stabilise $\consp(g)$. Then for $a \in
%   \consp(i)$, $\Pi^{\gamma}_{\vec{x}} (\sigma_1 (a)) = \gamma
%   (\vec{x}(\sigma_1 (a))) = \gamma (\vec{x}'(a)) =
%   \Pi^{\gamma}_{\vec{x}'}(a)$. It follows from Lemma
%   \ref{lem:functions-mutual-equivalence} that $\alpha^{\gamma}(i,\vec{x}) =
%   \Pi^{\gamma}_{\vec{x}} (i) = \Pi^{\gamma}_{\vec{x}'}(i)$. The result follows
%   similarly for $\beta$.
% \end{proof}

The following lemma gives us that$M_\equiv$ is well defined.

\begin{lem}
	Let $((i, \vec{x}), (j, \vec{y})), ((i, \vec{x}'), (j, \vec{y}')) \in I \times
	J$ be such that $\vec{x} \equiv_i \vec{x}'$ and $\vec{y} \equiv_j \vec{y}'$,
	then $M(((i, \vec{x}), (j, \vec{y}))) = M((i, \vec{x}'), (j, \vec{y}'))$.
	\label{lem:matrix-quot-well-defined}
\end{lem}
\begin{proof}
	\begin{align*}
		M((i, \vec{x}),(j, \vec{y})) & = L^{\gamma}(\alpha^{\gamma}(i, \vec{x}), \beta^{\gamma}(j, \vec{y})    \\
		                             & = L^{\gamma}(\alpha^{\gamma}(i, \vec{x}'), \beta^{\gamma}(j, \vec{y}')) \\
		                             & = M((i, \vec{x}'), (j, \vec{y}'))                                       
	\end{align*}
	The second equality follows from Lemma \ref{lem:alpha-beta-mutal-equivalence}.
\end{proof}

We now prove the required isomorphism-equivalence.

\begin{thm}
	Let $\gamma \in [n]^{\underline{U}}$ such that $\gamma^{-1} \sim \eta$. Then
	$L^{\gamma}$ is isomorphism-equivalent to $M_{\equiv}$.
	\label{thm:LM-equivalence}
\end{thm}
\begin{proof}
	Let $(i, [\vec{x}]) \in I_\equiv$ and $(j, [\vec{y}]) \in J_\equiv$. Then,
	from Lemmas \ref{lem:matrix-quot-well-defined} and
	\ref{lem:ML-equal-elements}, we have that $M_\equiv ((i, [\vec{x}]), (j,
	[\vec{y}])) = M ((i, \vec{x}), (j, \vec{y})) = L^{\gamma}(\alpha^{\gamma}(i,
	\vec{x}), \beta^{\gamma}(j, \vec{y}))$.
		
	Moreover, from Lemma \ref{lem:alpha-beta-mutal-equivalence} we can lift
	$\alpha^\gamma$ to $I_\equiv$ and $\beta^{\gamma}$ to $J_\equiv$. It remains
	to show that $\alpha^\gamma_{equiv}$ and $\beta^{\gamma}_{\equiv}$ are
	bijections. We prove the result for $\alpha^{\gamma}_{\equiv}$, with the proof
	for $\beta^\gamma_\equiv$ following similarly.
		
	We first note that the lifting of $\alpha^{\gamma}_{\equiv}$ is surjective as
	both $\alpha^{\gamma}$ (Lemma \ref{lem:alpha-beta-surjective}) and the lifting
	function (i.e. the quotient map from $I$ to $I_\equiv$) are surjective.
		
	Suppose $\alpha^{\gamma}_\equiv((i, [\vec{x}])) = \alpha^{\gamma}_\equiv((i',
	[\vec{x}']))$, and so $\Pi^{\gamma}_{\vec{x}}(i) =
	\Pi^{\gamma}_{\vec{x}'}(i')$. But then $i$ and $i'$ are in the same orbit and
	thus, from the definition of $I_{\equiv}$ they are both the minimum element in
	that orbit, and so $i = i'$. Thus $\Pi^{\gamma}_{\vec{x}}(i) =
	\Pi^{\gamma}_{\vec{x}'}(i)$, and so from Lemma
	\ref{lem:functions-mutual-equivalence} there exists $\sigma \in \stab_g(i)
	\cap \spstab{g}$ such that for all $a \in \consp_g(i)$ we have that
	$\Pi^{\gamma}_{\vec{x}}(a) = \Pi^{\gamma}_{\vec{x}'} (\sigma (a))$. But then
	$\Pi^{\gamma}_{\vec{x}}(a) = \gamma (\vec{x}(a)) =
	\Pi^{\gamma}_{\vec{x}'}(\sigma (a)) = \gamma (\vec{x}' (\sigma (a))$ and,
	since $\gamma$ is a bijection, it follows that $\vec{x}(a) = \vec{x}' \sigma
	(a)$. Thus $\vec{x} \equiv_i \vec{x}'$, and so $[\vec{x}] = [\vec{x}']$. We
	thus have that $\alpha^{\gamma}_\equiv$ and $\beta^{\gamma}_\equiv$ are
	injections, and the result follows.
\end{proof}

We show in the next lemma that, while $M$ is not in general
isomorphism-equivalent to $L^\gamma$, $M$ can be shown to have the same rank as
$L^{\gamma}$.

\begin{lem}
	Let $\gamma \in U^{\underline{n}}$ be such that $\gamma^{-1} \sim \eta$ and
	let $p \in \nats$ be prime. Then $\rank_p (M) = \rank_p (M_\equiv) = \rank_p
	(L^{\gamma})$.
\end{lem}
\begin{proof}
	Let $f : \{0,1\}^{J_\equiv} \rightarrow \{0,1\}^{J}$ be defined such that for
	all $\vec{b} \in {0,1}^{J_\equiv}$, $f(\vec{b}) (j, \vec{y}) := \vec{b}(j,
	[\vec{y}])$ for all $(j, \vec{y}) \in J$. We have that $f$ is injective as,
	for $\vec{b}_1, \vec{b}_2 \in \{0,1\}^{J_\equiv}$, if $f(\vec{b}_1) =
	f(\vec{b}_2)$ then $\vec{b}_1 (j, [\vec{y}]) = f(\vec{b}_1)(j, \vec{y}) =
	f(\vec{b}_2)(j, \vec{y}) = \vec{b}_2(j, [\vec{y}])$, for all $(j, \vec{y}) \in
	J$. Also $f$ is linear as, for $\lambda \in \ff_p$ and $\vec{b}_1, \vec{b}_2
	\in \{0,1\}^{J_\equiv}$, $f(\lambda \vec{b}_1) (j, \vec{y}) = (\lambda
	\vec{b}_1)(j, [\vec{y}]) = \lambda (\vec{b}_1 (j, [\vec{y}])) = \lambda (f(
	\vec{b}_1) (j, \vec{y}))$, and $ f(\vec{b}_1 + \vec{b}_2) (j, \vec{y}) =
	(\vec{b}_1 + \vec{b}_2) (j, [\vec{y}]) = \vec{b}_1(j, [\vec{y}]) +
	\vec{b}_2(j, [\vec{y}]) = f(\vec{b}_1)(j, \vec{y}) + f(\vec{b}_2)(j,
	\vec{y})$, for all $(j, \vec{y}) \in J$.
		
	From the construction of $M_\equiv$ that $f$ maps rows vectors in $M_\equiv$
	to row vectors in $M$. Let $\vec{b}'$ be a row vector in $M$. Then there
	exists $\vec{b} \in \{0,1\}^{J_\equiv}$ defined by $\vec{b} (j, [\vec{y}]) :=
	\vec{b}' (j, \vec{y})$ (Lemma~\ref{lem:matrix-quot-well-defined} gives us that
	this function is well defined). It follows that $f(\vec{b}) = \vec{b}'$, and
	so $f$ is a surjective mapping from the row vectors of $M_\equiv$ to the row
	vectors of $M$.
		
	It follows from the fact that $f$ is an injective linear map and fact that $f$
	maps rows in $M_\equiv$ to rows in $M$ that if $B$ is a maximal independent
	set of row vectors in $M_\equiv$ then $f(B)$ is an independent set of row
	vectors in $M$. Moreover, suppose $f(B)$ is not a maximal independent set of
	row vectors in $M$, then there exists a row vector $\vec{b}'$ in $M$ such that
	$\vec{b}' \notin f(B)$ and $f(B) \cup \{b'\}$ is independent. But since $f$
	maps rows in $M_\equiv$ to rows in $M$ surjectively, it follows that $B \cup
	\{f^{-1}(b') \}$ must be an independent set of rows in $M_\equiv$. But since
	$B$ is maximal, it follows that $f^{-1}(b') \in B$ and so $b' \in f(B)$, a
	contradiction. We thus have that $f(B)$ is a maximal independent set of rows
	in $M$ and, since $f$ is injective, we have that $\rank_p (M) = \vert f(B) \vert 	= \vert B \vert = \rank_p (M_\equiv) = \rank_p(L^{\gamma})$. (The final equality
	follows from Theorem \ref{thm:LM-equivalence}).
\end{proof}


% Let $(i,\vec{x}), (i, \vec{x}') \in I$ be such that $\vec{x} \equiv \vec{x}'$.
% Let $B_\equiv \subseteq \{0,1\}^{J_{\equiv}}$ be a maximal independent set of
% row vectors in $M_\equiv$. Let $f: B_\equiv \rightarrow \{0,1\}^{J}$ be such
% that, for any $\vec{b}_\equiv \in B_\equiv$, $f(\vec{b}_\equiv) = \vec{b}$ if,
% and only if, $\vec{b}(j, \vec{y}) = \vec{b}_\equiv (j, [\vec{y}])$ for all
% $(j, \vec{y}) \in J$. Let $B := \image (f)$. Then from the definition of
% $M_\equiv$ we have that $B$ is a set of row vectors in $M$.

% From Lemma \ref{lem:matrix-quot-well-defined}, $f$ is injective

% Suppose $B$ is not independent. Then there exists $\vec{b}_1, \ldots,
% \vec{b}_p$

% From Lemma \ref{lem:matrix-quot-well-defined} it follows that for all $(j,
% \vec{y}) \in J$, $M((i, \vec{x}), (j, \vec{y})) = M((i, \vec{x}'), (j,
% \vec{y}))$. In other words any two rows indexed by vectors in the same
% equivalence class are equal. It follows that the set of rows vectors in $M$
% equals the set of rows vectors in $M_\equiv$, and so (from Theorem
% \ref{thm:LM-equivalence}) $\rank (M) = \rank (M_{\equiv}) = \rank
% (L^{\gamma})$.

% \begin{lem}
%   Let $\gamma \in [n]^{\underline{U}}$ such that $\gamma^{-1} \sim \eta$. Then
%   $\rk_p (M) = \rk_p (L^{\gamma})$.
% \end{lem}
% \begin{proof}
%   We show that $\rk(M) = \rk (M_\equiv)$, and the result will follow from
%   Theorem \ref{thm:LM-equivalence}.
% \end{proof}

% Let $((i, \vec{x}), (j, \vec{y})), ((i, \vec{x}'), (j, \vec{y}')) \in I \times
% J$ be such that $\vec{x} \equiv_i \vec{x}'$.
% \begin{lem}

% \end{lem}


% \begin{thm}
%   Let $\gamma\in [n]^{\underline{U}}$, $M$ is row-column equivalent to
%   $L^{\gamma}$. This equivalence is witnessed by $\alpha^{\gamma}$ and
%   $\beta^{\gamma}$.
% \end{thm}
% \begin{proof}
%   We have that $\alpha^{\gamma}$ and $\beta^{\gamma}$ are surjective. Let
%   $\gamma' = (\Pi^{\gamma}_{(\vec{x}_{\vec{r}} \vert \vec{y}_{\vec{c}})})^{-1}
%   \gamma$.

%   \begin{align*}
%     M((i, \vec{x}), (j, \vec{y}))
%     &= (\vec{x}_{\vec{r}} \vert \vec{y}_{\vec{c}}) \in \EV_h \\
%     &= C_n[\gamma \mathcal{A}] (\Pi^{\gamma}_{(\vec{x}_{\vec{r}} \vert
%       %     \vec{y}_{\vec{c}})} (h)) \\
%       %     &= C_n[\gamma' \mathcal{A}] (h) \\
%       %     &= C_n[\gamma' \mathcal{A}](L(\row(h), \column (h)))\\
%       %     &= C_n[\gamma' \mathcal{A}](L(\alpha^{\gamma'}(i, \vec{x}),
%       %     \beta^{\gamma'}(j, \vec{y})))\\
%       %     &= C_n[\gamma \mathcal{A}](L(\alpha^{\gamma}(i, \vec{x}),
%       %     \beta^{\gamma}(j, \vec{y}))y)\\
%       %     &= L^{\gamma}(\alpha^{\gamma}(i, \vec{x}), \beta^{\gamma}(j,
%       %     \vec{y}))
%   \end{align*}
%   The second equality follows from Lemma \ref{lem:translate_EV_circuits}. The
%   fifth equality follows from Lemma \ref{lem:defining_h_from_IJ}. The sixth
%   equality follows from Lemma \ref{lem:alpha_ind_gamma}.
% \end{proof}

% \begin{remark}
%   In fact, the above still requires injectivity. This still needs to be
%   integrated.
% \end{remark}


% \subsection{FPR Formulas}

% \begin{lem}
%   Let $(i, \vec{x}) \in I$ and $(j, \vec{y}) \in J$. Let $h \in H$ such that
%   $\row (h) \in \orb_r (i)$, $\column (h) \in \orb_c(j)$, and $\type(h) =
%   \type(\vec{x}, \vec{y})$. Let $\gamma: U \rightarrow [n]$ be a bijection and
%   $\mathcal{A}$ a structure. Then $C_n[\gamma
%   \mathcal{A}](\Pi^{\gamma}_{\vec{x} \vert \vec{y}} (h)) = C_n[\gamma
%   \mathcal{A}](L(\alpha^{\gamma}(i, \vec{x}), \beta^{\gamma}(j, \vec{y})))$.
% \end{lem}
% \begin{proof}
%   Let $\gamma' = (\Pi^{\gamma}_{\vec{x} \vert \vec{y}})^{-1} \gamma$. First we
%   show that $h = L(\alpha^{\gamma'}(i, \vec{x}), \beta^{\gamma'}(j,
%   \vec{y}))$. Notice that $\Pi^{\gamma'}_{\vec{x}}(\vec{r}_i) =
%   \gamma'(\vec{x}) = (\Pi^{\gamma}_{\vec{x} \vert \vec{y}})^{-1} \gamma
%   (\vec{x}) = \vec{r}_h$. But $\row(h) \in \orb_r(i)$ and so there exists
%   $\vec{x}' \in A^r_i$ such that $\Pi^{\gamma'}_{vec{x}'}(i) = \row{h}$ From
%   Lemma \ref{lem:support_determine_action} it follows that $\row (h) =
%   \Pi^{\gamma'}_{\vec{x}}(i) = \alpha^{\gamma'}(i, \vec{x})$. Similarly, we
%   can show that $\column(h) = \beta^{\gamma'}(j, \vec{y})$. It follows that $h
%   = L(\alpha^{\gamma'}(i, \vec{x}), \beta^{\gamma'}(j, \vec{y}))$, and so
%   \begin{align*}
%     C_n[\gamma \mathcal{A}](\Pi^{\gamma}_{\vec{x} \vert \vec{y}} (h))
%     &= C_n[\gamma' \mathcal{A}](h)\\
%     &= C_n[\gamma' \mathcal{A}](L(\alpha^{\gamma'}(i, \vec{x}),
%       %     \beta^{\gamma'}(j, \vec{y})))\\
%       %     &= C_n[\gamma \mathcal{A}](L(\alpha^{\gamma}(i, \vec{x}),
%     %     \beta^{\gamma}(j, \vec{y}))).
%   \end{align*}
%   The final equality follows from Lemma \ref{lem:alpha_ind_gamma}.
% \end{proof}


% \chapter{New Stuff}
% Let $x,y \subset U$ such that $\vert x \vert = \vert y \vert = k \in
% \mathbb{N}$. Let $\vec{x}: [k] \rightarrow x$ and $\vec{y}: [k] \rightarrow y$
% be bijections.

% We need to define $r, c \subset [n]$ such that $\vert r \vert = \vert c \vert
% = k$ and there exists bijections $\vec {r}: [k] \rightarrow r$ and $\vec{c}:
% [k] \rightarrow c$.

% Let $u_1 , \ldots , u_{2k}$ be the first $2k$ elements of $[n] \setminus
% \SP(g)$ in order. Then let
% \[r = \eta^{-1} (x \cap \eta (\SP(g))) \cup \{u_{\vec{x}^{-1}(a)}: a \in x
%   \setminus \alpha (\SP (g))\} \] and
% \[s = \eta^{-1} (y \cap \eta (\SP(g))) \cup (x \cap y) \cup \{ u_{k +
%   \vec{x}^{-1}(a)} : a \in y \setminus (x \cup \alpha (\SP (g))) \}). \]
% Define
% \[
%   \vec{r} (a) =
%   \begin{cases}
%     \eta^{-1} (\vec{x} (a)) & a \in \vec{x}^{-1} (x \cap \eta (\SP(g))) \\
%     u_{a} & a \in \vec{x}^{-1} (x \setminus \eta(\SP(g)),
%   \end{cases}
% \]
% and

% \[
%   \vec{c} (a) =
%   \begin{cases}
%     \eta^{-1} (\vec{y} (a)) & a \in \vec{y}^{-1}(y \cap \eta (\SP (g))) \\
%     \vec{r} (a) & a \in \vec{y}^{-1} (x \cap y \setminus \eta (\SP (g))) \\
%     u_{k+a} & \text{otherwise}.
%   \end{cases}
% \]

% \begin{lem}
%   $x_r \sim \eta$ and $x_c \sim \eta$
% \end{lem}

% \begin{lem}
%   $\SP(g) \cap \SP(i) = \SP(g) \cap r$ and $\SP (g) \cap SP (j) = \SP (g) \cap
%   c$.
% \end{lem}

% Let $(\vec{x}, \vec{y})_{\vec{r}, \vec{c}}: r \cup c \rightarrow U$ be defined
% by
% \[
%   (\vec{x}, \vec{y})_{\vec{r}, \vec{c}}(a) =
%   \begin{cases}
%     \vec{x}(\vec{r}^{-1}(a)) & a \in r \\
%     \vec{y}(\vec{c}^{-1}(a)) & a \in c
%   \end{cases}
% \]

% This function is well defined.

% % \begin{lem}
% %   Let $r, c \subset U$ and let $r_1, r_2 : [k_1] \rightarrow r$ and $c_1, c_2 :
% %   [k_2] \rightarrow c$ then $\type(\vec{r}_1, \vec{c}_1) = \type (\vec{r}_2 ,
% %   \vec{c}_2)$.
% % \end{lem}

% \begin{lem}
%   $(\vec{x}_{\vec{r}} \vert \vec{y}_{\vec{y}}) \sim \eta$
% \end{lem}

% \begin{lem}
%   There exists $\sigma_1, \sigma_2 \in \spstab{g}$ such that $\sigma_1 \cdot
%   \vec{\consp(i)} = \vec{r}$ and $\sigma_2 \cdot \vec{\consp(j)} = \vec{c}$.
% \end{lem}

% \begin{lem}
%   Let $(i, \vec{x}) \in I$ and $j, \vec{y} \in J$. Let $\vec{r}$ and $\vec{c}$
%   be described as above. Let $\sigma_1$ and $\sigma_2$ be as from Lemma
%   \ref{}. Let $h = L(g) (\sigma_1 (i), \sigma_2 (j))$. Then $\alpha^{\gamma'}
%   (i, \vec{x}) = \row (h)$ and $\beta^{\gamma'} (i,\vec{y}) = \column(h)$.
% \end{lem}
% \begin{proof}
%   $\alpha^{\gamma'}(i, \vec{x}) = \Pi^{\gamma'}_{\vec{x}} (i)$. It is
%   sufficient to show that for all $a \in sp(g)$, $\Pi^{\gamma}_{\vec{x}} (a) =
%   \sigma_1 (a)$. Note that $\Pi^{\gamma'}_{\vec{x}_i} (a) =
%   (\Pi^{\gamma}_{(\vec{x}, \vec{y})_{\vec{r}, \vec{c}}})^{-1} \gamma
%   (\vec{x}_i (a))$, and if $b = \sigma_1 (a)$ we have that $\gamma (\vec{x}_i
%   (a)) = \gamma (\vec{x} (\vec{\consp}(i)^{-1} (a))) = \gamma (\vec{x}
%   (\vec{\consp}(i)^{-1} (\sigma^{-1}_1 b))) = \gamma (\vec{x} ((\sigma_1 \cdot
%   \vec{\consp}(i))^{-1} (b))) = \gamma (\vec{x} (\vec{r}^{-1}(b))) =
%   \Pi^{\gamma}_{(\vec{x}, \vec{y})_{\vec{r}, \vec{c}}}(b)$. So
%   $\Pi^{\gamma'}_{\vec{x}} (a) = b = \sigma_1 (a)$. Similarly for
%   $\beta^{\gamma'}$.
% \end{proof}

% \begin{lem}
%   Let $(i, \vec{x}) \in I$ and $(j, \vec{y}) \in J$. There exists $h \in H$
%   such that $h$ has type $(\vec{x}, \vec{y})$ and for any bijection $\gamma: U
%   \rightarrow [n]$ and structure $\mathcal{A}$ then $C_n[\gamma \mathcal{A}]
%   (\Pi^{\gamma}_{(\vec{x} \vert \vec{y})_{h}}) = C_n [\gamma
%   \mathcal{A}](L(\alpha^{\gamma}(i, \vec{x}), \beta^{\gamma}(j, \vec{y})))$.
% \end{lem}

% \begin{lem}
%   Let $h \in H$ and $\vec{z} \in A_h$ and $\Pi^{\delta}_h$ be a permutation
%   such that $\Pi^{\delta}_{\vec{z}}(a) = \delta(\vec{z}(a))$ for all $a \in
%   \sp(h)$. Then $C_n[\gamma \mathcal{A}](\Pi^{delta}_{\vec{z}} (h))$ iff
%   $\vec{z} \in \EV_h$.
% \end{lem}

% \begin{thm}
%   The matrix $M$ is equivalent to $L$.
% \end{thm}

\subsection{Translating to Formulas of FPR}
\label{sec:translating-formulas-to-FPR}
Let $\mathcal{C}:= (C_n)_{n \in \nats}$ be a $P$-uniform family of transparent
symmetric $(\RB, \rho)$-circuits. In this section we define a formula $Q$ in
$\FPR$ such that for any $\rho$-structure $\mathcal{A}$ over the universe $U$
with $\vert U \vert = n$, the $q$-ary query defined by $C_n$ on input
$\mathcal{A}$ is defined by $Q$ when interpreted in $\mathcal{A}$.

We note that the Immerman-Vardi theorem~\cite{} give us that a polynomial time
algorithms (of a particular form) can be converted into equivalent $\FP(\leq)$
interpretations. AS such, from $P$-uniformity of $\mathcal{C}$ and Lemma \ref{},
there is a $\FP(\leq)$ interpretation $\Phi'$ that defines $n$ a symmetric
rank-circuit with unique labels equivalent to $C_n$ when interpreted in $\langle
[n], \leq \rangle$. We abuse notation and refer to this equivalent circuit as
$C_n$.

It is easy to see that we can define $\Phi$ in $\FPC$ from $\Phi'$ such that
$\Phi$ defines $C_n$ using a sequence of number-terms rather than formulas of
$\FP(\leq)$. To see this, let $t$ be the width of $\Phi'$ then the domain of
$\Phi'$ is a subset of $[n]^t$. Let $\phi_f$ be an number-term that defines a
bijection from $[n^t]$ to $[n]^t$ for each $n$. Then for each $\phi'
(\vec{\nu}_1, \ldots, \vec{\nu}_{p}) \in \Phi$, where $\vec{\nu}_i$ is an
$t$-length vector of variables, we can define a number-term $\phi''$ from
$\phi'$ by settings all variables to number-sort variables and bounding each
quantifier by the size of the universe of the input structure such that for
$\vec{a}_1, \ldots, \vec{a}_p \in [n]^t$, $\langle [n], \leq \rangle \models
\phi' [\vec{a}_1, \ldots, \vec{a}_p]$ if, and only if, $\mathcal{A} \models
\phi'' [\vec{a}_1, \ldots, \vec{a}_p]$. We then let $\phi(\nu_1, \ldots, \nu_p)
:= \phi''(\phi_f (\nu_1), \ldots, \phi_f(\nu_p))$, and let $\Phi$ be the
sequence of formulas $\phi$ defined from $\phi'$ in $\Phi'$.

We thus have an interpretation $\Phi := (\phi_G, \phi_\Omega, (\phi_s)_{ s \in
	\RB \cup \rho \cup \{0,1\}}, (\phi_{\Lambda_R})_{R \in \rho}, \phi_L)$ in
$\FPC$ defining the circuit $C_n$ when interpreted in $\mathcal{A}$. The
formulas in this interpretation are defined as follows.

\begin{itemize}
	\setlength\itemsep{0mm}
	\item The number term $\phi_G(\mu)$, when interpreted in $\mathcal{A}$, defines
	      a subset of $[n^t]$, which we identify with $G$ (the set of gates in the
	      circuit), and hence write $G \subseteq [n^t]$.
	\item The number term $\phi_{\Omega}(\nu_1, \ldots , \nu_q, \mu)$ is defined
	      such that $\mathcal{A} \models \phi_\Omega[a_1, \ldots, a_q, g]$ if, and only
	      if, $(a_1, \ldots, a_q) \in [n]^q$ and $\Omega(a_1, \ldots, a_q) = g$.
	\item The number term $\phi_s (\mu)$ is defined for $s \in \RB \uplus \rho
	      \uplus \{0,1\}$ such that $\mathcal{A} \models \phi_s [g]$ if, and only if,
	      $\Sigma$ maps $g$ to assigned to the symbol $s$.
	      	      
	\item For $R \in \rho$ of arity $r$ the number term $\phi_{\Lambda_R}(\nu_1,
	      \ldots, \nu_r, \mu)$ is defined such that $\mathcal{A} \models
	      \phi_{\Lambda_R} [a_1, \ldots, a_r, g]$ if, and only if, $(a_1, \ldots,
	      a_r)\in [n]^r$ and $g$ is a relational gate such that $\Sigma (g) = R$ and
	      $\Lambda_R (g) = (a_1, \ldots, a_r)$.
	\item The number term $\phi_L(\mu, \nu, \delta, \epsilon)$ is defined such that $\mathcal{A} \models \phi_L[g,h,i,j]$ if, and only if, $g$ and $h$ are gates and $L(g)(i,j) = h$.
\end{itemize}
% The Immerman-Vardi Therorem can be used to show that there exist number terms $\FA{row}(\mu, \delta)$ and $\FA{col} (\mu, \delta)$ such that $\mathcal{A} \models \FA{row}[g, i]$ (resp. $\mathcal{A} \models \FA {col} [g, j]$) if, and only if, $g$ is a gate and $i \in A$ (resp. $g$ is a gate and $j \in B$). 

Since there is an obvious polynomial-time computable function that takes in an $n \in \nats$ and a gate $g$ in $C_n$ and outputs the characteristic and threshold of a rank gate, we have from the Immerman-Verdi theorem that there exists a formula $\FA{rank-type}(\mu, \kappa , \pi)$ such that $\mathcal{A} \models \FA{rank-type}[g, r, p]$ if, and only if, $g$ is a rank
gate with characteristic $p$ and threshold $r$. We note that the bound on the
domain is definable as a closed number term $\FA{m} := (x\# (x = x))*t$.

We have from the Corollary \ref{} that there are constants $n_0$ and $k$ such
that for all $n \geq n_0$ the support of each gate has size at most $k$. Note
that for each $n \leq n_0$, $C_n$ can be evaluated by an $\FPC$ formula with
simply quantifies over all of the possible bijections from the universe of
$\mathcal{A}$ to $[n]$ (since there are at most constantly many such
bijections). As such we suppose $n > n_0$ in the rest of this subsection.

We will recursively construct $\EV_g$ for each gate $g$ in the circuit. While
we have that the canonical support of $g$ has size at most $k$, it may not be
equal to $k$. As such, if $\vert \consp(g) \vert = \ell$, we define

\begin{align*}
	\overline{\EV}_g = \{ (a_1, \ldots , a_k) \in [n]^k : (a_1 , \ldots , a_\ell ) \in \EV_g \text{ and } i \neq j \implies a_i \neq a_j \}. 
\end{align*}

In this subsection we use $\mu$ and $\nu$ to denote number variables that should be assigned to gates and use $\epsilon$ and $\delta$ to denote variables that should be assigned to elements of the universe of a gate. We use $\kappa$ and $\pi$ to denote other number variables. We use $x, y, z, \ldots$ to denote vertex variables and $U, V, \ldots$ to denote relational variables. We use $a, b, c , \ldots$ to denote elements of the vertex sort. When a vector of values or variables is used without reference to size, it is taken to be a $k$-tuple. 

Let $s \subseteq [n]$, $X$ be a set, and $f \in X^{\underline{s}}$. Then, since $[n]$, and hence $s$, is an ordered set, we let $\vec{s}$ denote the $\vert s \vert$-tuple given by listing the elements of $s$ in order. We let $\vec{f}$ denote a $\vert s \vert$-tuple formed by applying $f$ to $s$ in order (i.e. $\vec{f} := f \circ \vec{s}$). 

We seek to define a relation $V \subseteq [n^t] \times U^k$ by $V(g, \vec{a})$
if, and only if, $\vec{a} \in \overline {\EV}_g$. Our aim is to define a set of
formulas $\theta_s (\mu, \vec{x})$, where $s \in \mathbb{B} \cup \tau \cup
\{0,1\}$, which evaluate the gate indexed by $\mu$ for the assignment to its
support given by $\vec{x}$ if it is a gate labelled by the symbol $s$. Each of
these formulas are written in terms of $V$, the relation variable we are
inductively defining. 

We will define the $\FPR$ formula $\theta_\rank (\mu, \vec{x})$. The other
formulas have already been defined by Anderson and Dawar~\cite{AndersonD17} and
so this will suffice. In order to do this we first define a formula $\psi_M$
that defines the matrix $M$ from the previous subsection. We then use the rank
operator to compute the rank of the given matrix. In order to define $\psi_M$
and $\theta_\rank$ succinctly we first define a number of useful $\FPC$
formulas.

We define the closed number term $\FA{s} := x\# (x = x)$, which defines the size
of the structure. The following two formulas take in a gate $g$ and an element
of the universe of $g$ and check if that element is in the first or second sort
(i.e. of it is a row or a column).

\begin{align*}
	\FA{row} (\mu, \delta) :=   & \exists \nu, \epsilon \leq \FA{m} \, (\Phi_L (\mu, \delta, \epsilon, \nu)) \\
	\FA{col} (\mu, \epsilon) := & \exists \nu, \delta \leq \FA{m} \, (\Phi_L (\mu, \nu, \delta , \epsilon))  
\end{align*}

From Lemma~\ref{}, and invoking Immerman-Vardi theorem, we have an $\FPC$
formula $\FA{orb}(\mu, \delta, \epsilon)$ such that $\mathcal{A} \models
\FA{orb}[g,i, i']$ if, and only if, $i$ and $i'$ are in the universe of $g$ and
$i \in \orb_g(i')$. The following formula allows us to define the minimal
element of an orbit

\begin{align*}
	\FA{min-orb} (\mu, \delta) := \forall \epsilon \leq \FA{m} \, (\FA{orb} (\mu, \delta, \epsilon) \implies \delta \leq \epsilon). 
\end{align*}

% We define an $\FPC$ formula $\FA{agree}_s (\vec{s}_1, \vec{s}_1, \vec{x},
% \vec{y})$ such that for $\mathcal{A}$, $\mathcal{A}^{\leq} \models
% \FA{agree}_s[\vec{r}, \vec{c}, \vec{a}, \vec{b}]$ if, and only if,
% $\vec{a}_{\vec{r}} \sim \vec{b}_{\vec{c}}$.

From Anderson and Dawar~\cite{AndersonD17} there is an $\FPC$ formula $\FA{supp}$ such that $\mathcal{A} \models \FA{supp} [g, u]$ if, and only if, $\mathcal{A} \models \phi_G [g]$ and $u$ is in $\consp(g)$. Anderson and Dawar~\cite{AndersonD17} use this formula to inductively define a set of formulas $\FA{supp}_i$ for each $i \in \nats$ such that $\mathcal{A} \models \FA{supp}_i[g, u]$ if, and only if, $u$ is the $i$th element of $\consp(g)$.

Similarly, we can define $\FA{supp}^r$ such that $\mathcal{A} \models
\FA{supp}^r[g, a, u]$ if, and only if, $\mathcal{A} \models \phi_G [g] \land
\FA{row}[g, a]$ and $u$ is in $\consp_g(a)$, and $\FA{supp}^c$ such that
$\mathcal{A} \models \FA{supp}^c[g, a, u]$ if, and only if, $\mathcal{A} \models
\phi_G [g] \land \FA{col}[g, a]$ and $u$ is in $\consp_g(u)$. Again we can
define the formulas $\FA{supp}^r_i$ and $\FA{supp}^c_i$.

From Anderson and Dawar~\cite{AndersonD17} we have an $\FPC$ formula $\FA{agree}(\mu, \nu, \vec{x}, \vec{y})$ such that $\mathcal{A} \models \FA{agree}[g, h, \vec{a}, \vec{b}]$ if, and only if, $\vec{a}_{\vec{\consp}(g)}$ is compatible with $\vec{b}_{\vec{\consp}(h)}$.  

% A similar argument allows us to define $\FA{agree}_L (\mu, \nu, \delta , \vec{x}, \vec{y}, \vec{z})$ (where $\vec{z}$ is a tuple of size $2k$) such that $\mathcal{A} \models \FA{agree}_L [g, h, i, \vec{a}, \vec{b}, \vec{c}]$ if, and only if, $\vec{a}_{\vec{\consp}(g)}$, $\vec{b}_{\vec{\consp}(h)}$, and $\vec{c}_{\vec{\consp}_g(i)}$ are all pairwise compatible.

A similar argument allows us to define $\FA{agree}_L (\mu, \delta , \vec{x}, \vec{z})$ (where $\vec{z}$ is a tuple of size $2k$) such that $\mathcal{A} \models \FA{agree}_L [g, i, \vec{a}, \vec{r}]$ if, and only if, $\vec{a}_{\vec{\consp}(g)}$ and $\vec{r}_{\vec{\consp}_g(i)}$ are compatible.

It can be shown that given a number $n$, a gate $g$ in $C_n$ and two pairs $(i, \vec{x}) \in I$ and $(j, \vec{y}) \in J$ the function $\vec{c}$ in Lemma~\ref{lem:permutation-row-column} can be computed in time polynomial in $n$. Thus, from the Immerman-Vardi Theorem, it can be defined by a number term. Thus we assert that there exists an $\FPC$ formula $\FA{move} (\mu, \delta, \epsilon, \vec{x}, \vec{y}, \vec{z}, \vec{w})$ (where $\vec{y}$ and $\vec{z}$ are $2k$-tuples) such that $\mathcal{A} \models \FA{move} [g, i, j, \vec{a}, \vec{b}, \vec{d}]$ if, and only if, $\mathcal{A} \models \phi_G [g] \land \FA{row}_g[g,i] \land \FA{col}[g, j]$ and $\mathcal{A} \models \FA{agree}_L [g, i, \vec{a}, \vec{b}] \land \FA{agree}_L [g, j, \vec{a}, \vec{d}]$ and $\vec{d}_{\vec{\consp}_g(j)} = \vec{c}$, where $\vec{c}$ is the vector given in Lemma~\ref{lem:permutation-row-column} defined from the pairs $(i, \vec{a}_{\vec{\consp_g(i)}})$ and $(j, \vec{b}_{\vec{\consp}_g(j)})$.

From Lemma \ref{lem:permutation-row-column} and invoking the Immerman-Vardi theorem, there is an $\FPC$ formula $\FA{map}$ such that $\mathcal{A} \models FA{map}[g, j, j', \vec{c}]$ if, and only if, $\mathcal{A} \models \phi_G [g] \land \FA{col}[g, j] \land \FA{col}[g, j']$ and such that there exists $\sigma \in \spstab{g}$ such that $\sigma (\vec{\consp}_g(j)) = \vec{c}$ then $\sigma (j) = j'$. 
				
% Similarly we can define $\FA{agree}_1 (\mu, \delta , \vec{x}, \vec{y})$ such
% that $\mathcal{A}^leq \models \FA{agree}_1 [g, i, \vec{a}, \vec{b}]$ if, and
% only if, $\alpha \sim \beta$, where $\alpha \in U^{\underline{\consp(g)}}$ and
% $\beta \in U^{\underline {\consp_g(i)}}$ are the restrictions of $\vec{a}$ and
% $\vec{b}$ to the length of $\consp(g)$ and $\consp_g(i)$ respectively. Lastly
% we can define $\FA{agree}_2 (\mu, \delta, \epsilon , \vec{x}, \vec{y})$ such
% that $\mathcal{A}^\leq \models \FA{agree}_2 [g, i,j ,\vec{a}, \vec{b},
% \vec{c}]$ if, and only if, $\mathcal{A}^{\leq} \models \FA{agree}_1[g, i,
% \vec{a}, \vec{b}] \land \FA{agree}_1[g, j, \vec{a}, \vec{c}]$ and $\alpha \sim
% \beta$, where $\alpha \in U^{\underline{\consp_g(i)}}$ and $\beta \in
% U^{\underline {\consp_g(j)}}$ are the restrictions of $\vec{b}$ and $\vec{c}$
% to the length of $\consp_g(i)$ and $\consp_g(j)$ respectively.
				
We define the $\FPC$ formula $\FA{merge}$ that takes in two gates $g$ and $h$
(with $h \in H_g$) as well as assignments to the supports of $g$ and $h$ and the
row and column supports of $h$. It then checks if the assignment to the support
of $h$ is compatible with the given row and column supports of $h$.
				
% \begin{align*}
%     \FA{merge} (\mu, \nu, \vec{x}, \vec{y}, \vec{z}) := \exists u_1, u_2 \leq \FA{MAX} ( \phi_L (\mu, u_1, u_2, \nu) \land ( \bigwedge{1 \leq i < j \leq 2k}  \exists v \leq \FA{SIZE} ((\FA{supp}^r_j (\mu, u_1, v) \land \FA{supp}_i(\nu, v) \implies z_i = x_j) \land (\FA{supp}^c_j (\mu, u_2, v) \land \FA{supp}_i(\nu, v) \implies z_i = y_j))))
% \end{align*}
				
\begin{align*}
	\FA{merge} (\mu, \nu, \vec{x}, \vec{y}, \vec{w}_1, \vec{w}_2) := & \exists \delta, \epsilon \leq \FA{m} \, ( \phi_L (\mu, \delta, \epsilon, \nu) \\ & \land \FA{agree}_1 (\mu, \nu , \delta, \vec{x}, \vec{y}, \vec{w}_1) \land \\ & \FA{agree}_1 (\mu, \nu , \epsilon, \vec{x}, \vec{y}, \vec{w}_2))
\end{align*}
				
We are almost ready to define the matrix $M$ from the previous subsection. We
break this up into two formulas. The first checks that the input is within the
domain of the matrix and the second defines $M$ for the given row and column
indexes.
				
\begin{align*}
	\FA{check-dom}(\mu, \vec{x}, \delta, \vec{y}, \epsilon, \vec{z})  := & \FA{dom}_{\Phi_L} (\mu, \delta, \epsilon)                                                             
\land \FA{min-orb}(\delta, \vec{y}) \land \FA{min-orb} (\epsilon, \vec{z}) \\
& \land \FA{agree}_1 (g, \delta , \vec{x}, \vec{y}) \land \FA{agree}_1 (g, \epsilon , \vec{x}, \vec{z}) 
\end{align*}

\begin{align*}
	\FA{find-eval} (\mu, \vec{x}, \delta, \vec{y}, \epsilon, \vec{z}) := & \exists \vec{s} \leq \FA{s} \, (\FA{move}(\mu, \delta, \epsilon, \vec{x}, \vec{y}, \vec{z}, \vec{s}) \\ & \land (\exists \epsilon' \leq \FA{m} \, (\FA{map}^c (\mu, \epsilon, \epsilon', \vec{s}) \land (\exists \nu \leq \FA{m} \, (\phi_L (\mu, \delta, \epsilon', \nu) \\ & \land (\exists \vec{m} \, (\FA{merge} (\mu, \nu, \vec{x}, \vec{m}, \vec{y}, \vec{z}) \land V(\nu, \vec{m}))))))))
\end{align*}
				
% There is an $\FP$ formula $\FA{merge}(\nu , \vec{s}, \vec{s}_2, \vec{x},
% \vec{y}, \vec{w})$ such that $\mathcal{A} \models \FA{merge} [\vec{r},
% \vec{c}, \vec{a}, \vec{b}, \vec{d}]$ if, and only if, $\mathcal{A}^{\leq}
% \models \FA{agree}_s[\vec{r}, \vec{c}, \vec{a}, \vec{b}]$ and $\vec{d}_{\vec{r
% \cup c}} = \vec{a}_{\vec{r}} | \vec{b}_{\vec{c}}$.
				
% We define the $\FPC$ formula $\FA{FIND}$ that takes in gates $g$ and $h \in
% H_g$ and
% \begin{align*}
% \FA{FIND}(\mu, \nu, \vec{s}, \vec{t}) := \exists a, b \leq \FA{MAX} (\phi_L(\mu, a, b, \nu) \land \bigwedge_{1 \leq i \leq 2k} \left( (\exists u \leq \FA{SIZE} (\FA{supp}_i (\mu, a, u) \implies s_i = u)) \and (\exists u \leq \FA{SIZE} (\FA{supp}_i (\mu, b, u) \implies t_i = u)) \right)
% \end{align*}
% From Lemma \ref{} and the Immerman-Vardi theorem there is a formula
% $\FA{FIND}$ such that $\mathcal{A} \models \FA{FIND} [g, h, \vec{r}, \vec{c}]$
% if, and only if, $h \in H_g$ and $\mathcal{A} \models \bigwedge_\FA{supp}^_r_i
% [h, \vec{r}, g] \land \FA{supp}^c [h, \vec{c}, g]$. We can then define
% $\FA{MIN-FIND} (\vec{s}_1, \vec{s}_2, \mu, \vec{x}, \nu) := \FA{FIND}
% (\vec{s}_1, \vec{s}_2, \mu, \vec{x}, \nu) \land (\forall \nu' (\FA{FIND}
% (\vec{s}_1, \vec{s}_2, \mu, \vec{x}, \nu') \implies \nu \leq \nu'))$.
				
% We now define a formula for checking if the input is in the domain of the
% matrix, i.e. are elements of the sets $I$ and $J$ defined in the previous
% section.
% \begin{align*}
%     \FA{check-dom}(\mu, \vec{x}, \delta, \vec{y}, \epsilon, \vec{z})  := & \FA{dom}_{\Phi_L} (\mu, \delta, \epsilon) 
%     \land \FA{min-orb}(\delta, \vec{y}) \land \FA{min-orb} (\epsilon, \vec{z}) \\
%     &\land \FA{agree}_1 (g, \delta , \vec{x}, \vec{y}) \land \FA{agree}_1 (g, \epsilon , \vec{x}, \vec{z}).
% %     \FA{merge} (\mu, \vec{x}, \nu, \vec{w}, \delta, \vec{y}, \epsilon,
% %     \vec{z}, \vec{w}) := & \FA{agree} (\mu, \nu, \vec{x}, \vec{w}) \land
% %     \FA{agree}_2 (\mu , \delta, \epsilon, \vec{x}, \vec{y}, \vec{z}) \\ &
% %     \land \FA{agree}_2 (\nu , \delta, \epsilon, \vec{w}, \vec{y}, \vec{z}),
% %     and \\ \\
% %     \FA{MIN-MERGE} (\mu, \vec{x}, \nu, \vec{w}, \delta, \vec{y}, \epsilon,
% %     \vec{z}, \vec{w}) := & \FA{merge} (\mu, \vec{x}, \nu, \vec{w}, \delta,
% %     \vec{y}, \epsilon, \vec{z}, \vec{w}) \\ & \land (\forall \nu'
% %     (\FA{merge} (\mu, \vec{x}, \nu', \vec{w}, \delta, \vec{y}, \epsilon,
% %     \vec{z}, \vec{w}) \implies \nu \leq nu'))
% \end{align*}.
				
% We now define a formula that takes in a gate $g$ and an assignment to its
% support as well as an encoding of $(i, \vec{b}) \in I$ and $(j, \vec{d})\in
% J$, and constructs the vector $\vec{c}$, as defined in the previous section,
% such that $\vec{b}_{\consp_g(i)} \sim \vec{d}_{\vec{c}}$. It then defines a
% gate $h \in H_g$ that has row support $\consp_g(i)$ and column support
% $\vec{c}$, before evaluating for $h$ for the assignment to the support
% $(\vec{b}_{\consp_g(i)} | \vec{d}_{\vec{c}})$.
% \begin{align*}
%     \FA{find-eval} (\mu, \vec{x}, \delta, \vec{y}, \epsilon, \vec{z}) := & \exists \vec{v} (\FA{move}(\delta, \vec{y}, \epsilon, \vec{z}, \mu, \vec{x}, \vec{v}) \\ & \land (\exists \vec{u} (\FA{supp}_r(\delta, \vec{u}, \mu) \land (\exists \nu (\FA{MIN-FIND}(\vec{u}, \vec{v}, \nu, \mu) \\ & \land (\exists \vec{w} (\FA{merge}(\nu, \vec{y}, \vec{z}, \vec{w}) \FA{agree}(\mu, \nu, \vec{x}, \vec{w}) \land V(\nu, \vec{w}))))))))
% \end{align*}
				
We are now ready to define the formula that defines the matrix $M$.
				
\begin{align*}
	\psi_M (\mu, \vec{x}, \delta, \vec{x}, \epsilon, \vec{y}) :=  \FA{check-dom}(\mu, \vec{x}, \delta, \vec{y}, \epsilon, \vec{z}) \land \FA{find-eval} (\mu, \vec{x}, \delta, \vec{y}, \epsilon, \vec{z}) 
\end{align*}
				
We now define the formula that evaluates a rank gate $g$.
				
\begin{align*}
	\theta_\rank (\mu, \vec{x}) := & \bigwedge_{1 \leq i < j \leq k} x_i \neq x_j \land (\exists p, k \leq \FA{m} \, (\FA{rank-type}(\mu. p, k) \\ &\land [\rank (\vec{y}\delta\leq \FA{m}, \vec{z}\epsilon\leq \FA{m}, p \leq \FA{M}). \psi_M] \leq k )))
\end{align*}
				
As mentioned above we have the formulas $\theta_0$, $\theta_1$, $\theta_\land$,
$\theta_\lor$, $\theta_\nand$, $\theta_\maj$ and $(\theta_R)_{R \in \tau}$ from
Anderson and Dawar \cite{AndersonD17}. The relation $V$ is then given as a fixed
point by the following formula.
\begin{align*}
	\theta (\mu, \vec{x}) := [\ifp_{V,\nu \vec{y}} \bigvee_{s \in \mathbb{B}' \uplus \tau \uplus \{0,1\}} (\phi_s(\mu) \land \theta_s (\nu, \vec{y} ] (\mu, \vec{x}) 
\end{align*}
				
The following $\FPR$ formula defines the $q$-ary query computed by the circuit
family $\mathcal{C}$ \cite{AndersonD17}.
				
\begin{align*}
	Q (z_1, \ldots z_q) := & \exists \vec{x} \exists \mu, \nu_1 , \ldots  \nu_q \eta_1 , \ldots , \nu_k \leq \FA{m} [\phi_\omega (\nu_1, \ldots \nu_q, \mu) \land \\
	& \bigwedge_{1 \leq i \leq k} \FA{supp}_i (\mu, \eta_i) \wedge \forall \eta (\neg \FA{supp}_i (\mu, \eta))) \land \\
		& \bigwedge_{1 \leq i \leq k} \bigwedge_{1 \leq j \leq q}((\FA{supp}_i (\mu, \eta_i) \land (x_i = z_j) \implies \nu_j = \eta_i) \land \\ &
	\bigwedge_{1 \leq j \leq q} \bigvee_{1 \leq i \leq k} (x_i = z_j \land \FA{supp}_i (\mu, \eta_i)]
\end{align*}
				
This completes the proof of our main theorem.
				
				
				
				
				
% We define the following formulas:
% \begin{align*}
%     \FA{check-dom}(\mu, \vec{x}, \delta, \vec{y}, \epsilon, \vec{z})  := & \FA{dom}_{\Phi_L} (\mu, \delta, \epsilon) 
%     \land \FA{min-orb}(\delta, \vec{y}) \land \FA{min-orb} (\epsilon, \vec{z}) \\
%     &\land \FA{agree}_1 (g, \delta , \vec{x}, \vec{y}) \land \FA{agree}_1 (g, \epsilon , \vec{x}, \vec{z}), \\ \\
%     %     \FA{merge} (\mu, \vec{x}, \nu, \vec{w}, \delta, \vec{y}, \epsilon,
%     %     \vec{z}, \vec{w}) := & \FA{agree} (\mu, \nu, \vec{x}, \vec{w}) \land
%     %     \FA{agree}_2 (\mu , \delta, \epsilon, \vec{x}, \vec{y}, \vec{z}) \\
%     %     & \land \FA{agree}_2 (\nu , \delta, \epsilon, \vec{w}, \vec{y},
%     %     \vec{z}), and \\ \\
%     %     \FA{MIN-MERGE} (\mu, \vec{x}, \nu, \vec{w}, \delta, \vec{y},
%     %     \epsilon, \vec{z}, \vec{w}) := & \FA{merge} (\mu, \vec{x}, \nu,
%     %     \vec{w}, \delta, \vec{y}, \epsilon, \vec{z}, \vec{w}) \\ & \land
%     %     (\forall \nu' (\FA{merge} (\mu, \vec{x}, \nu', \vec{w}, \delta,
%     %     \vec{y}, \epsilon, \vec{z}, \vec{w}) \implies \nu \leq nu'))
% \end{align*}.
				
				
				
% \theta_M (\delta , \vec{y}, \epsilon, \vec{z}, \mu, \vec{x})
				
				
								
\end{document}
