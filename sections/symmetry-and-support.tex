\documentclass[../paper.tex]{subfiles}

\begin{document}
In order to analyse families of symmetric circuits we will need to develop some
theory for analysing symmetries. In this section we define the terms needed
(e.g. the notion of a support) and prove basic results. We then prove the first
important theorem in this paper, the support theorem, using these tools. It is
worth noting that we prove the support theorem for arbitrary symmetric circuits
with matrix-symmetric gates, not just for rank gates.

\subsection{Supports and Supporting Partitions}
In this section we formally define a support and the related notion of a
supporting partition.

\begin{definition}
  Let $G \leq \sym_n$ and let $S \subseteq [n]$. Then $S$ is a
  \emph{support} for $G$ if $\stab(S) \subset G$.
\end{definition}

\begin{definition}
  Let $G \leq \sym_n$ and $\mathcal{P}$ be a partition of $[n]$.
  Then $\mathcal{P}$ is a \emph{supporting partition} for $G$ if
  $\stab(\mathcal{P}) \subseteq [n]$.
\end{definition}

Notice that if $\mathcal{P}$ is a supporting partition for $G$ and $P \in
\mathcal{P}$ then $\stab([n] \setminus P) \subseteq \stab(\mathcal{P}) \subset
G$, i.e. $[n] \setminus P$ is a support for $G$.

These supporting partitions admits a natural partial order in terms of
coarseness. If $\mathcal{P}, \mathcal{P}'$ we say that $\mathcal{P}'$ is as
course as $\mathcal{P}$ (denoted by $\mathcal{P}' \supseteq \mathcal{P}$) if for
all $x \in \mathcal{P}$ there exists $y \in \mathcal{P}'$ such that $x \subseteq
y$ \cite{AndersonD17}.

Anderson and Dawar \cite{AndersonD17} show that for any permutation group $G
\subseteq \sym_n$ a unique coarsest partition can always be found. We call this
partition the \emph{canonical supporting partition}, and denote it by $\SP (G)$.
This canonical supporting partition, along with the above observation about
supports, will allow us to construct for each group a canonical support.

\begin{definition}
  Let $G \leq \sym_n$. Let $\| SP(G) \| = \min \{\vert [n]
  \setminus P \vert : P \in SP(G) \}$. We say that $G$ has small support if $\|
  SP(G) \| < \frac{n}{2}$.
\end{definition}

\begin{definition}
  Let $G \leq \sym_n$ such that $G$ has small support. Let
  $\consp(G) = [n] \setminus P$, where $P$ is the maximal element of $\SP(G)$.
  We call $\consp(G)$ the canonical support of $G$.
\end{definition}

\begin{remark}
  Include a remark about the importance of asserting the size consideration. Use
  the alternating group as an example.
\end{remark}

We state the following two results from Anderson and Dawar \cite{AndersonD17}.

\begin{lem}
  \label{lem:SP_conjugation}
  Let $G \leq \sym_n$ and $\sigma \in \sym_n$ then $\sigma \SP (G) = \SP(\sigma
  G \sigma^{-1})$.
\end{lem}

\begin{lem}
  For any $G \leq \sym_n$ we have that $\stab (\SP (G)) \subseteq G
  \setstab(\SP(G))$.
\end{lem}

\subsection{Group Action on Supports}
In this paper, distinct the case of Anderson and Dawar \cite{AndersonD17}, we
will need to develop theory and terminology for dealing with group actions (and
supports) on more than just gates in the circuit. In particular, we use this
theory for understanding the actions on the rows and columns of a labelling of a
gate.

\begin{definition}
  Let $G \leq \sym_n$ and $X$ be a set on which a left group
  action of $G$ on $X$ is defined. We denote the canonical supporting partition
  of $x\in X$ by $\SP (x) = \SP (\stab(x))$. We say that $x \in X$ has small
  support if $\stab(x)$ has small support. We say that $X$ has small supports if
  all $x \in X$, $x$ have small support.

  If $x \in X$ has small support, we denote the \emph{canonical support} of $x$
  by $\consp(x) = \consp (stab(x))$.
\end{definition}

\begin{definition}
  Let $G \leq X$ be a set on which a left group action of $G$ on $X$
  is defined. Then $\stab_G (x) = \{\pi \in G : \pi x = x\}$ and $\orb_G (x) =
  \{\pi x : \pi \in G\}$. In the event that the context makes the group obvious
  we omit the subscript.
\end{definition}

\begin{lem}
  \label{lem:stab_conjugation}
  Let $G \leq \sym_n$, $X$ be a set on which a left group action
  of $G$ on $X$ is defined and $\sigma \in G$. Then for any $x \in X$ it follows
  that $\sigma \stab (x) \sigma^{-1} = \stab(\sigma x)$.
\end{lem}

\begin{proof}
  Let $\pi \in \stab(x)$, then $\sigma \pi \sigma^{-1}(\sigma x) = \sigma \pi
  (x) = \sigma (x)$, and so $\sigma \pi \sigma^{-1} \in \stab(\sigma x)$. Let
  $\pi \in \stab(\sigma (x))$ then $\pi \sigma (x) = \sigma (x)$ and so
  $\sigma^{-1} \pi \sigma (x) = x$. It follows that $\sigma^{-1} \pi \sigma \in
  \stab(x)$ and so $\pi = \sigma (\sigma^{-1} \pi \sigma) \sigma ^{-1} \in
  \sigma \stab(x) \sigma^{-1}$.
\end{proof}

\begin{lem}
  \label{lem:support_mapping}
  Let $G \leq \sym_n$, $X$ be a set on which a left group action
  of $G$ on $X$ is defined and $\sigma \in G$. Then for any $x \in X$ it follows
  that $\sigma \SP (x) = \SP (\sigma x)$ and, if $x$ has small support, $\sigma
  \consp (x) = \consp (\sigma x)$.
\end{lem}
\begin{proof}
  From Lemma \ref{lem:SP_conjugation} and Lemma \ref{lem:stab_conjugation} have
  that $\sigma \SP (x) = \SP(\sigma \stab(h) \sigma^{-1}) = \SP (\stab(\sigma
  x)) = \SP (\sigma x)$. Thus proving the first part of the statement.

  From the fact that $\| \SP(x) \| < \frac{n}{2}$ it follows there exists a
  unique $P_1 \in \SP(x)$ such that $\vert P_1 \vert > \frac{n}{2}$ and
  $\consp(x) = [n] \setminus P_1$. But then $\sigma \consp (x) = \sigma([n]
  \setminus P_1) = [n] \setminus (\sigma P_1)$. We note that $\sigma P_1 \in
  \sigma \SP (x) = \SP(\sigma x)$ and $\vert \sigma P_1 \vert > \frac{n}{2}$.
  Thus $\sigma P_1$ is the unique largest partition in $\SP(\sigma x)$, and so
  $\consp(\sigma x) = [n]\setminus (\sigma P_1) = \sigma (consp(x))$. Thus
  proving the second part of the statement.
\end{proof}

\begin{lem}
  Let $G \leq \sym_n$, $X$ be a set on which a left group action
  of $G$ on $X$ is defined. Then for any $x \in X$ with small support, and
  $\sigma, \sigma' \in G$ such that $\sigma (vec{\consp}(x)) = \sigma'
  (\vec{\consp}((x))$ we have that $\sigma (x) = \sigma'(x)$.
\end{lem}
\begin{proof}
  We have that $(\sigma')^{-1}\sigma(\vec{\consp}(x)) = \vec{\consp}(x)$ and so
  $(\sigma')^{-1} \sigma (x) = x$ and thus $\sigma (x) = \sigma' (x)$.
\end{proof}

% \begin{lem}
%   Let $G \leq \sym_n$, $X$ be a set on which a left group action of $G$ on $X$
%   is defined and $\sigma \in G$. Then if $y \in \orb_{G} (x)$ and $\consp (x)
%   = \consp (y)$ then $x = y$.
% \end{lem}

% \begin{proof}
%   Let $\sigma \in G$ be such that $\sigma x = y$. Then $\consp(x) = \consp (y)
%   = \consp (\sigma x) = \sigma \consp (x)$. Thus $\sigma (\sp (x))$
% \end{proof}

% \begin{definition}
%   Let $h \in H$. Let $\stab_r(h)$ be a subgroup of $\stab(\consp(g))$
%   consisting of exactly those permutations $\sigma$ such that $\row (\sigma h)
%   = \row (h)$. Let $\stab_c(h)$ be a subgroup of $\stab(\consp(g))$ consisting
%   of exactly those permutations $\sigma$ such that $\column (\sigma h) =
%   \column(h)$.
% \end{definition}

% \begin{definition}
%   A set $S \subseteq [n]$ is called a row-support for $h$ if $S$ is a support
%   for $\stab_r(h)$. A set $S \subseteq [n]$ is called a column-support for $h$
%   iff $S$ is a support for $\stab_c(h)$.
% \end{definition}

% \begin{definition}
%   A partition $\mathcal{P}_r$ is called a row-supporting partition iff
%   $\mathcal{P}_r$ is a support for $\stab_r(h)$. A partition $\mathcal{P}_c$
%   is called a column-supporting partition iff $\mathcal{P}_c$ is a support for
%   $\stab_c(h)$.
% \end{definition}

% \begin{definition}
%   Let $\SP^r(h)$ denote the conical supporting partition and $r_h$ the
%   canonical support of $\stab_r(h)$. Similarly, let $\SP^c(h)$ denote the
%   conical supporting partition and $c_h$ the canonical support of
%   $\stab_c(h)$.
% \end{definition}

% \begin{definition}
%   Let $G \subseteq \sym_n$ be a subgroup. Then let $\SPs(G)$ be the set of all
%   supporting partitions of $G$ and $\consps(G)$ be the set of all supports of
%   $G$.
% \end{definition}

% We now prove a few important lemmas on supports.

% \begin{lem}
%   Let $A, B \leq \sym_n$ then $\SP(A \cap B) \subseteq \SP (A) $.
% \end{lem}
% \begin{proof}
%   Let $\mathcal{P}$ be a supporting partition of $A \cap B$. Then
%   $\mathcal{P}$ is a supporting partition of $A$. Since $\SP(A)$ is coarser
%   than all such partitions the result follows.
% \end{proof}

% \begin{lem}
%   Let $h \in H$, then $\SP(\stab(h)) \subseteq \SP (\stab (\row (h)))$ and
%   $\SP (\stab(h)) \subseteq \SP (\stab (\column(h)))$.
% \end{lem}
% \begin{proof}
%   It can be shown that $\stab (h)$
% \end{proof}

% The following lemma will be used to prove the relationship between the support
% of a gate and the supports of its row and columns.
% % Use this lemma + a support type theorem in order to get the result.
% \begin{lem}
%   Let $G \leq \sym_n$, $X_1$ and $X_2$ be sets on which left group actions of
%   $G$ on $X_1$ and $X_2$ are defined. Let $x_1 \in X_1$ and $x_2 \in X_2$ and
%   suppose $\vert \consp(x_1) \vert < \frac{n}{2}$ and $ \vert \consp(x_2)
%   \vert < \frac{n}{2}$. Then if $\SP(\stab(x_1)) \subseteq \SP (\stab(x_2))$
%   then $\consp (x_2) \subseteq \consp (x_1)$.
% \end{lem}
% \begin{proof}
%   Since $\vert \consp(x_1) \vert = \| \SP (\stab(x_1)) \| < \frac{n}{2}$ it
%   follows that there a unique maximal $P_1 \in \SP (\stab(x_1))$ such that
%   $\vert P_1 \vert > \frac{n}{2}$. Similarly there is a unique maximal $P_2
%   \in \SP (\stab(x_2))$. Moreover, from the fact that $\SP (\stab(x_2))$ is
%   coarser than $\SP (\stab(x_1))$, it follows that there exists $P \in
%   \SP(\stab(x_2))$ such that $P_1 \subseteq P$. From the fact that
%   $\frac{n}{2} < \vert P_1 \vert \leq \vert P \vert$ and $P_2$ is the only
%   part such that $\frac{n}{2} < \vert P_2 \vert$, it follows that $P_1
%   \subseteq P = P_2$ and so $\consp(x_2) = [n] - P_2 \subseteq [n] - P_1 =
%   \consp(x_1)$.
% \end{proof}

\subsection{Support Theorems}
In this section we develop an analogous theorem to the support theorem of
Anderson and Dawar \cite{AndersonD17}.

In this section let $\mathbb{B}$ be a basis of sort-symmetric functions and
$(C_n)_{n \in \nats}$ always denote a family of symmetric, rigid $(\mathbb{B},
\tau)$-circuits with bijective labels. For each $g \in C_n$ let $(\tau_g, A_g)
:= \type(g)$, with sorts $\tau_g = (R, [s], \nu)$ and $A_g = A_1 \sqcup \ldots
\sqcup A_s$.

We note that for each $g \in C_n$ there is a natural action of $\stab(g)$ on
$\ind(g)$ is given by $\sigma \cdot \vec{x} := L(g) \cdot \sigma \cdot L(g)^{-1}
(\vec{x})$.
%
%\begin{lem}[Support Lemma]
%  For any $\epsilon$ and $n$ such that $\frac{2}{3} \leq \epsilon \leq 1$ and
%  $n \geq \frac{128}{\epsilon^2}$, if $C$ is a symmetric, rigid circuit on
%  structures of size $n$ and $s := \max_{g \in C} \vert \orb (g)\vert \leq
%  2^{n^{1-\epsilon}}$, then, for all $g \in C$, $\vert \SP (g) \vert \leq
%  \frac{n}{2}$.
% \end{lem}

% \begin{thm}[Support Theorem]
%   For any $\epsilon$ and $n$ such that $\frac{2}{3} \leq \epsilon \leq 1$ and
%   $n \geq \frac{128}{\epsilon^2}$, if $C$ is a symmetric, rigid circuit on
%   structures of size $n$ and $s := \max_{g \in C} \vert \orb (g)\vert \leq
%   2^{n^{1-\epsilon}}$, then, $SP(C) \leq \frac{33}{\epsilon}\frac{\log s}{\log
%   n}$.
% \end{thm}

% \begin{lem}
%   For any $\epsilon$ and $n$ such that $\frac{2}{3} \leq \epsilon \leq 1$ and
%   $n \geq \frac{128}{\epsilon^2}$, if $C$ is a symmetric, rigid circuit on
%   structures of size $n$ and $s := \max_{g \in C} \vert \orb (g)\vert \leq
%   2^{n^{1-\epsilon}}$, then, for all $g \in C$, we have for all $(i,j) \in
%   \dom (L_g)$ that $\consp(i) \subseteq \consp (L_g(i,j))$ and $\consp(j)
%   \subseteq \consp(L_g (i,j)))$.
% \end{lem}



% \begin{lem}
%   Let $\mathcal{P}_1$ and $\mathcal{P}_2$ be partitions of $[n]$ such that
%   $\mathcal{P}_1 = A \cup S_1$ and $\mathcal{P}_2 = B \cup S_2$, and
%   \begin{enumerate}
%   \item for all $X \in A$ and $Y \in B$ we have that $X \cap Y = \emptyset$,
%     and
%   \item for all $x \in S_1 \cup S_2$, x is a singleton.
%   \end{enumerate}
%   It follows that $\varepsilon(\mathcal{P}_1, \mathcal{P}_2) = A \cup B \cup
%   S$, where $S$ is a set of singletons.
% \end{lem}
% \begin{proof}
% \end{proof}

% \begin{lem}
%   Let $G_1$ and $G_2$ be subgroups of $\sym_n$. Then $\SP^S(G_1) = \SP^S(G_2)$
%   iff $\consp^S(G_1) = \consp^S(G_2)$.
% \end{lem}
% \begin{proof}
  
% \end{proof}

% \begin{lem}
%   Let $G_1$ and $G_2$ be subgroups of $\sym_n$. Then $\consp^S(G_1) =
%   \consp^S(G_2)$ implies that $\consp (G_1) = \consp (G_2)$.
% \end{lem}
% \begin{proof}
% \end{proof}

% The following lemma asserts a few basic facts about the canonical row and
% column supports.

% \begin{lem}
%   Let $h \in H$, then $r_h \subset \consp(h)$, $c_h \subset \consp(h)$ and
%   $c_r \cup r_h = \consp(h)$.
% \end{lem}

% \begin{definition}
%   Let $\omega$ be a matrix labeling for $g$. Define the matrix stabilizer for
%   $\omega$, denoted by $\matstab(\omega)$, to be the set of all $\sigma \in
%   \sym_n$ such that $\sigma H = H$ and there exists $(\alpha, \beta) \in
%   \sym_A \times \sym_B$ such that for all $(i,j) \in A \times B$ we have that
%   $\omega (\alpha i, \beta j) = \sigma \omega (i,j)$.
% \end{definition}

% \begin{definition}
%   Let $\omega$ be a matrix labeling for $g$. Define the matrix stabilizer for
%   $\omega$, denoted by $\matstab(\omega)$, to be the set of all $\sigma \in
%   \sym_n$ such that $\sigma H = H$ and there exists $(\alpha, \beta) \in
%   \sym_A \times \sym_B$ such that for all $(i,j) \in A \times B$ we have that
%   $\omega (\alpha i, \beta j) = \sigma \omega (i,j)$.
% \end{definition}


% \begin{definition}
%   Let $g$ be a gate with matrix labeling $\omega$, $h,h' \in H$ and $\sigma
%   \in \sym_n$. We say that a pair $(h, h')$ is compatible with $(\sigma,
%   \omega)$ if $\sigma h, \sigma h' \in H$ and there exists $(\gamma_1,
%   \gamma_2), (\gamma_1', \gamma_2') \in \sym_A \times \sym_B$ s.t.
%   \begin{align*}
%     &(\gamma_1, \gamma_2) \cdot \omega^{-1} (h) = \omega^{-1}(\sigma h), \text{ and} \\ 
%     &(\gamma_1', \gamma_2') \cdot \omega^{-1} (h') = \omega^{-1}(\sigma h'),
%   \end{align*}
%   and for all $\vec{x} \in \L(g)^{-1}(h)$, $\vec{x}' \in \omega^{-1}(h')$ and
%   all $a \in $
%   \begin{align*}
%     &i =i' \implies \gamma_1(i) = \gamma_1'(i'), \text{ and} \\
%     &j =j' \implies \gamma_2(j) = \gamma_2'(j').
%   \end{align*}

% \end{definition}

% \begin{definition}
%   Let $g \in C_n$ and let $h, h' \in H_g$ and $\sigma \in \sym_n$. We say that
%   a pair $(h, h')$ is \emph{compatible with} $\sigma$ if $\sigma h, \sigma h'
%   \in H_g$ and there exists $\lambda, \lambda' \in \sym_{A_1} \times \ldots
%   \times \sym_{A_s}$ such that
%   \begin{align*}
%     & \lambda \cdot L(g)^{-1} (h)  = L(g)^{-1} (\sigma \cdot h), \text{ and} \\ 
%     & \lambda' \cdot L(g)^{-1} (h') = L(g)^{-1} (\sigma \cdot  h'),
%   \end{align*} 
%   and for all $\vec{x} \in L(g)^{-1}(h)$. $\vec{x}' \in L(g)^{-1}(h')$, and $i
%   \in \dom (\vec{x})$, $\vec{x}(i) = \vec{x}'(i)$ iff $\lambda (\vec{x})(i) =
%   \lambda' (\vec{x}')(i)$.
% \end{definition}

\begin{definition}
  Let $g \in C_n$, $\vec{x}, \vec{x}' \in \ind(g)$ and $\sigma \in \sym_n$. We
  say that a pair $(\vec{x}, \vec{x}')$ is \emph{compatible with} $\sigma$ if
  $(\sigma L(g)(\vec{x}), \sigma L(g)(\vec{x}') \in H_g$ and for all $i \in \dom
  (\vec{x})$, $\vec{x}(i) = \vec{x}'(i)$ iff $(\sigma \cdot \vec{x})(i) =
  (\sigma \cdot \vec{x}')(i)$. We say that a pair $(h, h')$ is \emph{compatible
    with} $\sigma$ if $(L(g)^{-1}(h), L(g)^{-1}(h'))$ is compatible with
  $\sigma$.
\end{definition}

For $g \in C_n$ let $\sortstab (g)$ denote the set of all $\sigma \in
\stab(H_g)$ such that $\sigma \cdot L(g)$ is sort-equivalent to $L(g)$.

\begin{lem}
  Let $g \in C_n$, and $\sigma \in \stab(H_g)$. Then $\sigma \in \sortstab(g)$
  iff for all $h, h' \in H_g$, $(h,h')$ is compatible with $\sigma$.
\end{lem}
\begin{proof}
  $`\Rightarrow'$: From the hypothesis there exists $\pi \in \sym_{A_1} \times
  \ldots \times \sym_{A_s}$ such that $L(g) \cdot \pi = \sigma \cdot L(g)$. Then
  let $h, h' \in H_g$ and let $\vec{x} = L(g)^{-1}(h)$ and $\vec{x}' =
  L(g)^{-1}(h')$. Let $i \in \dom (\vec{x})$. If $\vec{x}(i) = \vec{x}'(i)$ then
  $(\sigma \cdot \vec{x}) (i) = (L(g)^{-1} \cdot \sigma \cdot \L(g)
  (\vec{x}))(i) = (L(g)^{-1} \cdot \sigma \cdot \L(g) (\vec{x}'))(i) = (\sigma
  \cdot \vec{x}) (i)$. If $(\sigma \cdot \vec{x})(i) = (\sigma \cdot
  \vec{x}')(i)$ then $(L(g)^{-1} \cdot \sigma \cdot \L(g) (\vec{x}))(i) =
  (L(g)^{-1} \cdot \sigma \cdot \L(g) (\vec{x}'))(i)$. Cancelling bijections
  leaves $\vec{x}(i) = \vec{x}'(i)$.
  
  $`\Leftarrow'$: We now define an element $\pi \in \sym_{A_1} \times \ldots
  \sym_{A_s}$. Let $t \in [s]$. We now defined $\pi_t \in \sym_{A_t}$. If there
  is no tuple in $\ind (g)$ containing an element of sort $t$ then let $\pi_t$
  be the trivial permutation. Suppose there is an element of sort $t$. Let $a
  \in A_t$. Let $\vec{x}_a \in \ind(g)$ be such that for each $i \in \dom
  (\vec{x}_a)$, if $\vec{x}_a(i)$ has sort $t$ then $\vec{x}_a(i) = a$. Let
  $\pi_t (a) = (\sigma \cdot \vec{x}_a) (i)$. The compatibility condition
  guarantees that $\pi_t$ is well defined.

  We now show $\pi_t$ is indeed a bijection. Let $a, b \in A_t$ and suppose
  $\pi_t (a) = \pi_t(b)$, then $(\sigma \cdot \vec{x}_a)(i) = (\sigma \cdot
  \vec{x}_b)(i)$, and from compatibility $a = \vec{x}_a(i) = \vec{x}_b(i) = b$.
  Moreover, for $a \in A_t$, let $b = \sigma^{-1} \cdot \vec{x}_a$, then $\pi_t
  (b) = (\sigma \cdot \vec{x}_b)(i) = (\sigma \cdot (\sigma^{-1} \cdot
  \vec{x}_a))(i) = a$. So we have that $\pi_t$ is indeed a bijection.

  Let $\pi = \pi_1 \times \ldots \times \pi_s$. Let $\vec{x} \in \ind(g)$, let
  $i \in \dom (\vec{x})$, let $t$ be the sort of $\vec{x}(i)$ and let $a =
  \vec{x}(i)$. Then $(\pi \cdot \vec{x})(i) = \pi_t (\vec{x}(i)) = (\sigma \cdot
  \vec{x}_a)(i) = (L(g)^{-1}\cdot \sigma \cdot L(g)(\vec{x}_a))(i) =
  (L(g)^{-1}\cdot \sigma \cdot L(g)(\vec{x}))(i)$. It follows that $L(g) \cdot
  \pi = L(g)^{-1} \cdot \sigma \cdot L(g)$, and the result follows.
  % If there is no type in $\ind (g)$ containing an element of type $t$ then let
  % $\pi_t$ be the trivial permutation. Let $a \in A_t$. Let $\vec{x}_a,
  % \vec{x}_a' \in \ind(g)$ be such that for every $i \in \dom (\vec{x}_a)$ if
  % $\vec{x}_a(i)$ has sort $t$ then $\vec{x}_a(i) = a = \vec{x}_a'(i)$.

 
  % $`\Leftarrow'$: We now define an element $\pi \in \sym_{A_1} \times \ldots
  % \sym_{A_s}$. Let $t \in [s]$. We now defined $\pi_t \in \sym_{A_t}$. If
  % there
  % is no type in $\ind (g)$ containing an element of type $t$ then let $\pi_t$
  % be
  % the trivial permutation. Let $a \in A_t$. Let $\vec{x}_a, \vec{x}_a' \in
  % \ind(g)$ be such that for every $i \in \dom (\vec{x}_a)$ if $\vec{x}_a(i)$
  % has
  % sort $t$ then $\vec{x}_a(i) = a = \vec{x}_a'(i)$. From the hypothesis we
  % have
  % that $L(g)(\vec{x}_a)$ and $L(g)(\vec{x}_a')$ are compatible with $\sigma$,
  % and let this compatibility be witnessed by $\lambda_a$ and $\lambda_a'$. Let
  % $\pi_t (a) = \lambda_a (\vec{x}_a)(i)$. Note that $\lambda_a (\vec{x}_a)(i)
  % =
  % \lambda_a' (\vec{x}_a'(i)$ and so this function is well defined.

  % We have that $\pi_t$ is surjective, as for every $a \in A_t$,
  % $\pi_t(\lambda^{-1}_a(a)) = a$. Let $a, b \in A_t$, and suppose $a \neq b$.
  % Let $\vec{x}_a, \vec{x}_b$ be defined as above, but choose them such that
  % they
  % agree nowhere. We have that $(L(g)(\vec{x}_a), \L(g)(\vec{x}_b))$ are
  % compatible with $\sigma$, and it is easy to see that $\lambda_a$ and
  % $\lambda_b$ as defined above witness this compatibility. But then
  % $\lambda(\vec{x_b})$

  % $\pi_t (a) = \pi_t (b)$ implies $\lambda_a (a) = \lambda_b (b)$, but
  % $L(g)(\vec{x}_a)$ is compatible with $L(g)(\vec{x}_b)$
  
  % Let $\vec{x} \in \ind (g)$ and $i \in \dom(\vec{x})$ and let $t$ be the type
  % of $\vec{x}(i)$. Then $\pi_t (\vec{x}(i)) = \lambda (\vec{x}(i))$

  % $(\vec{x})$, then $\pi_t (a) = \lambda (\vec{x}_a)(i) = \lambda
  % (L(g)^{-1}(h_a))(i) = L(g)^{-1}(\sigma \cdot h_a') (i)$.

  % Let $\pi := \pi_1 \times \ldots \times \pi_s$. Let $\vec{x} \in \ind(g)$,
  % then
  % $L(g) (\pi \cdot \vec{x}) = $

  % Note that for any pair of $\vec{x}$, $\vec{x}'$ chosen with all entries of
  % type $t$ being $a$ it is easy to see that $\lambda (a)$
  

  % It remains to show that $\pi$ witnesses the sort-equivalence of $\sigma
  % \cdot
  % L(g)$ and $L(g)$. First note that $\sigma \cdot L(g)(\vec{x}) = \sigma
  % (h_{\vec{x}}) = $


  % Then there exists $\lambda_1, \lambda_2 \in \sym_{A_1} \times \ldots \times
  % \sym_{A_s}$ witnessing Let $\lambda (\vec{x}) = \$

  % we have $\lambda_1$ and $\lambda_2$ witnessing that $(h, h')$ is compatible
  % with $\sigma$. We now define $\lambda \in \sym_{A_1} \times \ldots \times
  % \sym_{A_s}$ by $\lambda (\vec{a}) = \lambda_1(\vec{a})$
\end{proof}

% \begin{lem}[not reviewed]
%   Let $g \in C_n$ be a gate, $\sigma \in \sym_n$. Then $\sigma \in
%   \sortstab(g)$ iff for all $h,h' \in H$ we have that $(h,h')$ is compatible
%   with $(\sigma, \omega)$.
% \end{lem}

% \begin{proof}
%   $`\Rightarrow'$: We have that $\sigma \in \matstab(\omega)$ and so there
%   exists $(\alpha, \beta) \in \sym_A \times \sym_B$ such that for all $(i,j)
%   \in A \times B$ we have $\sigma \omega (i,j) = \omega (\alpha i, \beta j)$.
%   From compatibility we have that $h,h' \in H$ Let $h, h' \in H$ . Let
%   $(\gamma_1, \gamma_2) := (\gamma_1', \gamma_2') := (\alpha, \beta)$. This
%   assignment is sufficient to prove the direction.

%   $`\Leftarrow'$: Suppose for all $h,h' \in H$ we have $(\gamma_1, \gamma_2),
%   (\gamma_1, \gamma_2') \in \sym_A \times \sym_B$ satisfying the above
%   requirements. Notice that for a given $i \in A$ and any $j, j' \in B$, let
%   $h = \omega(i,j)$ and $h' = \omega(i,j')$, then we have that $\gamma_1 (i) =
%   \gamma_1'(i)$. It follows that we can define a $\alpha \in \sym_A$ by
%   $\alpha(i) = \gamma_1 (i)$. Similarly we can define $\beta \in \sym_B$ by
%   $\beta (j) = \gamma_2 (j)$.

% \end{proof}

\begin{definition}
  Let $g \in C_n$. Say $(\sigma, h, h') \in \sym_n \times H^2_g$ is useful if
  $(h,h')$ is incompatible with $\sigma$.

  Say that two distinct pairs $(\sigma_1, h_1, h_1'), (\sigma_2, h_2, h_2') \in
  \sym_n\times H^2_g$ are mutually independent if
  \begin{itemize}
    \setlength\itemsep{0mm}
  \item $\sigma_2 h_1 = h_1$,
  \item $\sigma_2 \sigma_1 h_1 = \sigma_1 h_1$,
  \item $\sigma_2 h_1' = h_1'$,
  \item $\sigma_2 \sigma_1 h_1' = \sigma h_1'$,
  \end{itemize}
  
  We say that a set $S \subseteq \sym_n \times H^2$ is useful if each pair in it
  is useful. We say that $S$ is independent if each pair of distinct pairs in
  $S$ are mutually independent.
\end{definition}

\begin{lem}
  Let $g \in C_n$ and $\sigma_1, \sigma_1 \in \sym_n$ such that $\sigma_1 \cdot
  L(g)$ is sort-equivalent to $\sigma_2 \cdot L(g)$. Then for any $h,h' \in H_g$,
  $(h,h')$ is compatible with $\sigma_1$ iff $(h,h')$ is compatible with
  $\sigma_2$.
\end{lem}
\begin{proof}
  Suppose $\sigma_1 \cdot L(g)$ is sort-equivalent to $\sigma_2 \cdot L(g)$.
  Then there exists $\pi \in \sym_{A_1} \times \ldots \times \sym_{A_s}$ such
  that $\sigma_1 \cdot L(g) (\pi \cdot \vec{x}) = \sigma_2 \cdot L(g)
  (\vec{x})$.

  Suppose $(h, h')$ is compatible with $\sigma_1$. Let $\vec{x} = L(g)^{-1}(h)$
  and $\vec{x}' = L(g)^{-1}(h')$. Then $\sigma_2 \cdot \vec{x} = (L(g)^{-1}
  \cdot \sigma_2 \cdot L(g)) (\vec{x}) = L(g)^{-1} \cdot \sigma_1 \cdot L(g)(\pi
  \cdot \vec{x}) = \sigma_1 \cdot (\pi \cdot \vec{x})$. Similarly $\sigma_2
  \cdot \vec{x}' = \sigma_2 \cdot (\pi \cdot \vec{x}')$. Let $i \in \dom
  (\vec{x})$, then $\vec{x}(i) = \vec{x}'(i)$ iff $(\sigma_1 \cdot \vec{x})(i) =
  (\sigma_2 \cdot \vec{x}')(i)$ iff $(\sigma_1 \cdot (\pi \vec{x}))(i) =
  (\sigma_2 \cdot (\pi \vec{x}'))(i)$ iff $(\sigma_2 \cdot \vec{x})(i) =
  (\sigma_2 \cdot \vec{x}')(i)$. The other direction follows by symmetry.
\end{proof}

\begin{claim}
  \label{claim:useful-independant-set}
  Let $g$ be a rank gate with labeling $\omega$ and child set $H$. Let $S$ be a
  useful and independent. We then have that $\vert \sym_n: \matstab(\omega)
  \vert \leq 2^{\vert S \vert}$.
\end{claim}

\begin{proof}
  Let $R \subseteq S$ and define $\sigma_R = \Pi_{(\sigma, h, h') \in R} \sigma$
  (with some arbitrary order assumed on $S$). Let $R$ and $Q$ be distinct
  subsets of $S$ and WLOG let $\vert R \vert \geq \vert Q \vert$. We want to
  show that we don't have $\sigma_R \omega \sim_\omega \sigma_Q \omega$. Pick
  any $(\sigma, h, h') \in R/Q \neq \emptyset$. Given that $\sigma_R h = \sigma
  h$ and $\sigma_R h' = \sigma h'$ it is easy to see that the usefulness of
  $(\sigma, h,h')$ implies the incompatibility of $(h,h')$ with with $(\omega,
  \sigma_R)$. Moreover, the fact that $\sigma_Q h = h$ and $\sigma_Q h' = h'$
  makes it easy to see $(h,h')$ is compatible with $(\omega, \sigma_Q)$. From
  the above lemma we may conclude that that we do not have $\sigma_R \sim_\omega
  \sigma_Q$, and the result follows.
\end{proof}

The following two lemmas proved by Anderson and Dawar \cite{AndersonD17} are
both of use in proving the following theorem.

\begin{lem}
  \label{lem:big-or-small}
  For any $\epsilon$ and $n$ such that $0 < \epsilon < 1$ and $\log n \geq
  \frac{4}{\epsilon}$, if $\mathcal{P}$ is a partition of $[n]$ with $k$ parts,
  $s = [\sym_n : \setstab (\mathcal{P})]$ and $n \leq s leq 2^{n^{1-\epsilon}}$,
  then $\min \{k, n-k\} \leq \frac{8}{\epsilon} \frac{\log s}{\log n}$.
\end{lem}

\begin{lem}
  \label{lem:small-means-support}
  For any $\epsilon$ and $n$ such that $0 < \epsilon < 1$ and $\log n \geq
  \frac{8}{\epsilon^2}$, if $\mathcal{P}$ is a partition of $[n]$ with $\vert
  \mathcal{P} \vert \leq \frac{n}{2}$, $s:= [\sym_n : \setstab (\mathcal{P})]$
  and $n \leq s \leq 2^{n^{1-\epsilon}}$, then $\mathcal{P}$ contains a part $P$
  with at least $n - \frac{33}{\epsilon} \cdot \frac{\log s} {\log n}$.
\end{lem}

If $g$ is a symmetric gate (i.e. the usual gates in a circuit) we note that
$\orb(g) = [\sym_n : \stab (g)]$ by the orbit-stabilizer theorem.

If $g$ is a matrix-symmetric gate then $\orb(g) = [\sym_n:\matstab (\omega)]$,
where $\omega$ is the matrix labelling associated with $g$.



\begin{thm}
  \label{thm:support_thm}
  For any $\epsilon$ and $n$ such that $\frac{2}{3} \leq \epsilon \leq 1$ and $n
  \geq \frac{128}{\epsilon^2}$, if $C$ is a symmetric, rigid circuit on
  structures of size $n$ and $s := \max_{g \in C} \vert \orb (g)\vert \leq
  2^{n^{1-\epsilon}}$, then, $\SP(C) \leq \frac{33}{\epsilon}\frac{log s}{log
    n}$.
\end{thm}

\begin{proof}
  It is easy to see that if $g$ is a gate in $C$ then $\stab (g) \subseteq
  \setstab(\SP(g))$, and so $s \geq \orb(g) = [\sym_n : \stab(g)] \geq [\sym_n :
  setstab(\SP(g))]$. Thus if $\vert \SP(g) \vert \leq frac{n}{2}$, then from
  Lemma \ref{lem:small-means-support}, we have $\| SP (g) \| \leq
  \frac{33}{\epsilon} \cdot \frac{\log s} {\log n}$. The result thus follows
  from showing that for each $g$ in $C$ we have that $\vert \SP (g) \vert \leq
  \frac{n}{2}$.
  
  % Include some detail here for constant and relational gates
  The cases where $g$ is a constant or relational gate are easy to handle.

  We now consider the case for internal gates. Let $g$ be the topologically
  first internal gate with $\vert \SP(g) \vert > \frac{n}{2}$. If $g$ is not a
  matrix-symmetric gate then the result follows from the argument presented by
  Anderson and Dawar \cite{AndersonD17}. Suppose $g$ is a matrix-symmetric gate,
  and suppose $g$ has a labelling $\omega$. We now argue that this leads to a
  contradiction.

  Let $k' := \lceil \frac{8 \log s}{\epsilon \log n} \rceil$. From the
  assumptions on $s, n$ and $\epsilon$ we have that $k' \leq
  \frac{1}{4}n^{1-\epsilon} < \frac{n}{2}$. Lemma \ref{lem:big-or-small} implies
  that $n - \vert \SP(g) \vert \leq k'$
  
  From Claim \ref{claim:useful-independant-set} it remains to show that we can
  construct a sufficiently large useful and independent set of gate-automorphism
  pairs $S$. Divide $[n]$ into $\lfloor \frac{n}{k' + 2} \rfloor$ disjoint sets
  $S_i$ of size $k' + 2$ and ignore the elements left over. It follows that for
  each $i$ there is a permutation $\sigma_i$ which fixes $[n] / S_i$ pointwise
  but moves $g$. Suppose there was no such $\sigma_i$ it follows that every
  every permutation that fixes $[n]/S_i$ pointwise fixes $g$. Thus the partition
  of all the singletons in $[n]/S_i$ and $S_i$ is a supporting partition. As
  $\SP(g)$ is the coarsest partition it follows that $\vert \SP(g) \vert \leq n
  - (k'+2) + 1 = n - k' - 1$, which contradicts the inequality $n - \vert \SP(g)
  \vert \leq k'$.

  Since $g$ is moved by each $\sigma_i$ and $C$ is rigid it follows that we
  don't have $\sigma_i \notin \matstab(\omega)$. Thus there exists $(h_i, h_i')$
  that is inconsistent with $(\sigma_i, \omega)$, and so the triple $(\sigma_i,
  h_i, h_i')$ is useful.

  Let $\SP (h)^*$ be the union of all parts of $\SP(h)$ except for the largest
  part. Let $Q_i = \SP(h_i)^* \cup \SP(\sigma_i h_i)^* \cup \SP (h_i')^* \cup
  \SP (\sigma_i h_i')^*$. Then note that if $\sigma_j$ fixes $Q_i$ then by
  construction, we have that $\sigma_j \in \stab_n(\SP(h_i)) \cap
  \stab_n(\SP(\sigma_i h_i)) \cap \stab_n(\SP(h_i')) \cap \stab_n(\SP(\sigma_i
  h_i'))$

  Define a graph $K$ with vertices given by the sets $S_i$ and an edge from
  $S_i$ to $S_j$ (with $i \neq j$) if $Q_i \cap S_j \neq \emptyset$. It follows
  then that if there is no edge between $S_i$ and $S_j$ then $(\sigma_i, h_i,
  h_i')$ and $(\sigma_j, h_j, h_j')$ are mutually independent. It remains to
  argue that $K$ has a large independent set. This is possible as the out-degree
  of $S_i$ in $K$ is bounded by
  \begin{align*}
    \vert Q_i \vert \leq \|SP(h_i) \| + \|SP(\sigma_i h_i) \| + \|SP(h_i') \| + \|SP(\sigma_i h_i') \leq 4 \cdot \frac{33\log s}{\epsilon \log n}
  \end{align*}. 

  This follows as the sets $S_i$ are disjoint and we may apply Lemma
  \ref{lem:small-means-support} to each of the child gates. It follows that the
  average total degree (in + out degree) of $K$ is at most $2 \cdot \vert Q_i
  \vert \leq 34 \cdot k'$. Now greedily select a maximal independent set in $K$
  by repeatedly selecting $S_i$ with the lowest total degree and eliminating it
  and its neighbours. This action does not affect the bound on the average total
  degree of $K$ and hence determines an independent set $I$ in $K$ of size at
  least
  \begin{align*}
    \frac{\lfloor \frac{n}{k' + 2} \rfloor}{34k' + 1} \geq \frac{n - (k'+2)}{34k'+1k'+2} \geq \frac{n\frac{7}{16}}{34k'^2 + 69k' +2} \geq \frac{n}{(16k')^2}.
  \end{align*}

  Take $S = \{(\sigma, h, h') : S_i \in I \}$. Then from the above argument we
  have that $S$ is useful and independent.
  
  Moreover, from Claim \ref{claim:useful-independant-set}, we have that $s \geq
  \vert \orb(g) \vert \geq 2^{\vert S \vert} \geq 2^{\frac{n}{(16k')^2}}$ then
  $n^{1-\epsilon} \geq \log s \geq n \cdot (\frac{128}{\epsilon}\frac{\log
    s}{\log n})^{-2} > n \cdot (n^{1-\epsilon})^{-2} = n^{2\epsilon -1} \geq
  n^{1-\epsilon}$. This is a contradiction.
\end{proof}

Let $\mathcal{C} = (C_n)_{n \in \mathbb{N}}$ be a $P$-uniform family of
symmetric circuits with matrix symmetric gates. Let $s(n)$ be the $s$ for $C_n$
from the hypothesis of Theorem \ref{thm:support_thm}. The theorem implies that,
for $n$ large enough, if $s(n)$ is bounded by a polynomial then there exists a
constant $k$ such that for all $n \in \mathbb{N}$ and $g \in C_n$, $\vert
\consp(g) \vert \leq k$. In other words, gates have constant size supports.


\end{document}
