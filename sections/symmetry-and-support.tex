\documentclass[../paper.tex]{subfiles}

\begin{document}
In order to analyse families of symmetric circuits we need to develop some
theory for analysing symmetries. In this section we define the terms needed
(e.g. the notion of a support) and prove basic results. We then prove the first
important theorem in this paper, the support theorem, using these tools. It is
worth noting that we prove the support theorem for arbitrary symmetric circuits
with matrix-symmetric gates, not just for rank gates.

\subsection{Supports and Supporting Partitions}
In this section we formally define a support and the related notion of a
supporting partition.

\begin{definition}
  Let $G \leq \sym_n$ and let $S \subseteq [n]$. Then $S$ is a \emph{support}
  for $G$ if $\stab(S) \leq G$.
\end{definition}
%AD - \stab needs to be defined in background section

\begin{definition}
  Let $G \leq \sym_n$ and $\mathcal{P}$ be a partition of $[n]$. Then
  $\mathcal{P}$ is a \emph{supporting partition} for $G$ if $\stab(\mathcal{P})
  \leq G$.
\end{definition}

Notice that if $\mathcal{P}$ is a supporting partition for $G$ and $P \in
\mathcal{P}$ then $\stab([n] \setminus P) \leq \stab(\mathcal{P}) \leq
G$, i.e. $[n] \setminus P$ is a support for $G$.

Let $\mathcal{P}, \mathcal{P}'$ be partitions of $[n]$.  We say that $\mathcal{P}'$ is as
\emph{coarse} as $\mathcal{P}$ (denoted by $\mathcal{P} \preceq \mathcal{P}'$)
if for all $x \in \mathcal{P}$ there exists $y \in \mathcal{P}'$ such that $x
\subseteq y$.
Anderson and Dawar~\cite{AndersonD17} show that for any permutation group $G
\leq \sym_n$ a unique coarsest supporting partition can always be found.  We call this
partition the \emph{canonical supporting partition}, and denote it by $\SP (G)$.
This canonical supporting partition, along with the above observation about
supports, allows us to construct for each group a canonical support.

\begin{definition}
  Let $G \leq \sym_n$. Let $\| \SP(G) \| = \min \{\vert [n] \setminus P \vert :
  P \in \SP(G) \}$. We say that $G$ has \emph{small support} if $\| \SP(G) \| <
  \frac{n}{2}$.
\end{definition}
Note that if $G$ has small support then there is a part in its
canonical supporting partition of size greater than $\frac{n}{2}$ and this
part is necessarily the unique largest part in the partition.

\begin{definition}
  Let $G \leq \sym_n$ such that $G$ has small support. Let $\consp(G) = [n]
  \setminus P$, where $P$ is the largest element of $\SP(G)$. We call
  $\consp(G)$ the \emph{canonical support} of $G$.
\end{definition}

% \begin{remark}
%   Include a remark about the importance of asserting the size consideration.
%   Use the alternating group as an example.
% \end{remark}

\begin{lem}\label{lem:subgroup-coarse}
Let $G_1, G_2 \leq \sym_n$ with $G_1 \leq G_2$.  Then $\SP(G_1)
  \preceq \SP(G_2)$
\end{lem}
\begin{proof}
  We have that $\stab(\SP(G_1)) \leq G_1 \leq G_2$, and so $\SP(G_1)$
  supports $G_2$. Since $\SP(G_2)$ is the coarsest support of $G_2$, $\SP(G_1)
  \preceq \SP(G_2)$.
\end{proof}

\begin{lem}
  Let $G_1, G_2 \leq \sym_n$ such that $G_1 \leq G_2$ and $G_1$ has small
  support. Then $G_2$ has small support and $\consp(G_2) \subseteq \consp(G_1)$.
  \label{lem:support-containment}
\end{lem}
\begin{proof}
  By Lemma~\ref{lem:subgroup-coarse}, we have that $\SP(G_1)
  \preceq \SP(G_2)$. Let $P_1$ be the largest element of $\SP(G_1)$, and note
  that since $G_1$ has small support, $\vert P_1 \vert > \frac{n}{2}$. Then
  there exists a (unique) $P_2 \in \SP(G_2)$ such that $P_1 \subseteq P_2$. Thus
  $G_2$ has small support, and $\consp(G_2) = [n] \setminus P_2 \subseteq [n]
  \setminus P_1 = \consp(G_1)$.
\end{proof}

\begin{lem}
  Let $G, H, K \leq \sym_n$ such that $G$ has small support and $G = H \cap K$.
  Then $H$ and $K$ have small support and $\consp(G) = \consp(H) \cup
  \consp(K)$.
  \label{lem:row-column-supports-well-behaved}
\end{lem}
\begin{proof}
The fact that $H$ and $K$ have small support follows from Lemma~\ref{lem:support-containment}.

  Let $\mathcal{Q} = \{P_H \cap P_K : P_H \in \SP(H), P_K \in \SP(K) \text{ and
  } \exists P \in \SP(G), P \subseteq P_H \cap P_K \}$. We first show that
  $\mathcal{Q}$ is a supporting partition of $G$. We have that for any $P_H \cap
  P_K, P_H' \cap P_K' \in \mathcal{Q}$, $P_H \cap P_K \cap P_H' \cap P_K' \neq
  \emptyset$ if, and only if, $P_H = P_H'$ and $P_K = P_K'$ if, and only if,
  $P_H \cap P_K = P_H' \cap P_K'$.  By
  Lemma~\ref{lem:subgroup-coarse}, we have that $\SP(G) \preceq
  \SP(H)$ and $\SP(G) \preceq \SP(K)$. It follows that for each $P \in SP(G)$ there exists $P_K \in \SP(K)$ and $P_H \in
  \SP(H)$ such that $P \subseteq P_H \cap P_K$.  So, for each $a \in [n]$ there is
  $P_a \in \SP(G)$, $P_H \in \SP(H)$ and $P_K \in \SP(K)$ such that $a \in P_a
  \subseteq P_H \cap P_K$. It follows that $\mathcal{Q}$ is a partition.

  Moreover, we note that $\stab(\mathcal{Q}) \leq \stab(\SP(H))$ and
  $\stab(\mathcal{Q}) \leq \stab(\SP(K))$, and thus $\stab(\mathcal{Q})
  \leq \stab(\SP(H)) \cap \stab(\SP(K)) \leq H \cap K = G$. It follows
  that $\mathcal{Q}$ is a supporting partition of $G$, and so by definition of
  the canonical supporting partition $\mathcal{Q} \preceq \SP (G)$.
  Since $G$ has small support, there is a part $P_G$ in $\SP(G)$ with
  $|P_G| > \frac{n}{2}$. Then there exists $P_H \in \SP(H)$ and
  $P_K \in \SP(K)$ such that $P_G \subseteq P_H \cap P_K$, and so $\vert P_H \cap P_K
  \vert > \frac{n}{2}$. Thus $P_H \cap P_K$ is the unique largest
  element in $\mathcal{Q}$.   It follows from $\mathcal{Q} \subseteq \SP(G)$
  that $P_G \subseteq P_H \cap P_K \subseteq P_G$.

  % From the fact that $G$ has small support, $P_H$ and $P_K$

  % Thus if $P_Q$ is the maximal element of $\mathcal{Q}$ then there exists $P_G
  % \in \SP(G)$ such that $P_Q \subsetq P_G$. But $\SP(G)$ supports

  % Since It follows that there exists $P \in \SP(G)$ suLet $P_G$ be the maximal
  % element of $\SP(G)$. Since $G$ has small support, $\vert P_G \vert >
  % \frac{n}{2}$. $We have that there exists unique maximal elements $P_H \in
  % \SP(H)$ and $P_K \in \SP(K)$ such that $P_G P_H \cap P_K$
\end{proof}


We state the following two results proved by Anderson and Dawar \cite{AndersonD17}.

\begin{lem}
  \label{lem:SP_conjugation}
  Let $G \leq \sym_n$ and $\sigma \in \sym_n$ then $\sigma \SP (G) = \SP(\sigma
  G \sigma^{-1})$.
\end{lem}

\begin{lem}
  For any $G \leq \sym_n$ we have that $\stab (\SP (G)) \leq G \leq
  \setstab(\SP(G))$.
\end{lem}

\subsection{Group Action on Supports}
In this paper, unlike in the case of Anderson and Dawar \cite{AndersonD17}, we
often wish to speak of supports and group actions on elements on the circuit other than gates. As such, we need to develop theory and terminology for dealing with group actions (and
supports) in general. In this subsection we introduce some useful terminology.


\begin{definition}
  Let $G \leq X$ be a set on which a left group action of $G$ on $X$ is defined.
  Then $\stab_G (x) = \{\pi \in G : \pi x = x\}$ and $\orb_G (x) = \{\pi x : \pi
  \in G\}$. In the event that the context makes the group obvious we omit the
  subscript. We use $\stab_n$ and $\orb_n$ to abbreviate $\stab_{\sym_n}$ and
  $\orb_{\sym_n}$.
\end{definition}

\begin{definition}
  Let $G \leq \sym_n$ and $X$ be a set on which a left group action of $G$ on
  $X$ is defined. We denote the \emph{canonical supporting partition} of $x\in
  X$ by $\SP (x) = \SP (\stab(x))$. Similarly we let $\| \SP (x) \| = \| \SP
  (\stab(x)) \|$. We say that $x \in X$ has \emph{small support} if $\stab(x)$
  has small support. We say that $X$ has small supports if all $x \in X$
  have small support.

  If $x \in X$ has small support, we denote the \emph{canonical support} of $x$
  by $\consp(x) = \consp (stab(x))$.
\end{definition}

\begin{lem}
  \label{lem:stab_conjugation}
  Let $G \leq \sym_n$, $X$ be a set on which a left group action of $G$ on $X$
  is defined and $\sigma \in G$. Then for any $x \in X$ it follows that $\sigma
  \stab (x) \sigma^{-1} = \stab(\sigma x)$.
\end{lem}

\begin{proof}
  Let $\pi \in \stab(x)$, then $\sigma \pi \sigma^{-1}(\sigma x) = \sigma \pi
  (x) = \sigma (x)$, and so $\sigma \pi \sigma^{-1} \in \stab(\sigma x)$. Let
  $\pi \in \stab(\sigma (x))$ then $\pi \sigma (x) = \sigma (x)$ and so
  $\sigma^{-1} \pi \sigma (x) = x$. It follows that $\sigma^{-1} \pi \sigma \in
  \stab(x)$ and so $\pi = \sigma (\sigma^{-1} \pi \sigma) \sigma ^{-1} \in
  \sigma \stab(x) \sigma^{-1}$.
\end{proof}

\begin{lem}
  \label{lem:support_mapping}
  Let $G \leq \sym_n$, $X$ be a set on which a left group action of $G$ on $X$
  is defined and $\sigma \in G$. Then for any $x \in X$ it follows that $\sigma
  \SP (x) = \SP (\sigma x)$ and, if $x$ has small support, $\sigma \consp (x) =
  \consp (\sigma x)$.
\end{lem}
\begin{proof}
  From Lemma \ref{lem:SP_conjugation} and Lemma \ref{lem:stab_conjugation} have
  that $\sigma \SP (x) = \SP(\sigma \stab(h) \sigma^{-1}) = \SP (\stab(\sigma
  x)) = \SP (\sigma x)$. Thus proving the first part of the statement.

  From the fact that $\| \SP(x) \| < \frac{n}{2}$ it follows there exists a
  unique $P_1 \in \SP(x)$ such that $\vert P_1 \vert > \frac{n}{2}$ and
  $\consp(x) = [n] \setminus P_1$. But then $\sigma \consp (x) = \sigma([n]
  \setminus P_1) = [n] \setminus (\sigma P_1)$. We note that $\sigma P_1 \in
  \sigma \SP (x) = \SP(\sigma x)$ and $\vert \sigma P_1 \vert > \frac{n}{2}$.
  Thus $\sigma P_1$ is the unique largest partition in $\SP(\sigma x)$, and so
  $\consp(\sigma x) = [n]\setminus (\sigma P_1) = \sigma (\consp(x))$.
  Thus  proving the second part of the statement.
\end{proof}

\begin{lem}
  Let $G \leq \sym_n$, $X$ be a set on which a left group action of $G$ on $X$
  is defined.  Then for any $x \in X$ with small support, and $\sigma, \sigma'
  \in G$ such that $\sigma (\vec{\consp}(x)) = \sigma' (\vec{\consp}((x))$ we
  have that $\sigma (x) = \sigma'(x)$.
\end{lem}
\begin{proof}
  We have that $(\sigma')^{-1}\sigma(\vec{\consp}(x)) = \vec{\consp}(x)$ and so
  $(\sigma')^{-1} \sigma (x) = x$ and thus $\sigma (x) = \sigma' (x)$.
\end{proof}

% \begin{lem}
%   Let $G \leq \sym_n$, $X$ be a set on which a left group action of $G$ on $X$
%   is defined and $\sigma \in G$. Then if $y \in \orb_{G} (x)$ and $\consp (x)
%   = \consp (y)$ then $x = y$.
% \end{lem}

% \begin{proof}
%   Let $\sigma \in G$ be such that $\sigma x = y$. Then $\consp(x) = \consp (y)
%   = \consp (\sigma x) = \sigma \consp (x)$. Thus $\sigma (\sp (x))$
% \end{proof}

% \begin{definition}
%   Let $h \in H$. Let $\stab_r(h)$ be a subgroup of $\stab(\consp(g))$
%   consisting of exactly those permutations $\sigma$ such that $\row (\sigma h)
%   = \row (h)$. Let $\stab_c(h)$ be a subgroup of $\stab(\consp(g))$ consisting
%   of exactly those permutations $\sigma$ such that $\column (\sigma h) =
%   \column(h)$.
% \end{definition}

% \begin{definition}
%   A set $S \subseteq [n]$ is called a row-support for $h$ if $S$ is a support
%   for $\stab_r(h)$. A set $S \subseteq [n]$ is called a column-support for $h$
%   iff $S$ is a support for $\stab_c(h)$.
% \end{definition}

% \begin{definition}
%   A partition $\mathcal{P}_r$ is called a row-supporting partition iff
%   $\mathcal{P}_r$ is a support for $\stab_r(h)$. A partition $\mathcal{P}_c$
%   is called a column-supporting partition iff $\mathcal{P}_c$ is a support for
%   $\stab_c(h)$.
% \end{definition}

% \begin{definition}
%   Let $\SP^r(h)$ denote the conical supporting partition and $r_h$ the
%   canonical support of $\stab_r(h)$. Similarly, let $\SP^c(h)$ denote the
%   conical supporting partition and $c_h$ the canonical support of
%   $\stab_c(h)$.
% \end{definition}

% \begin{definition}
%   Let $G \subseteq \sym_n$ be a subgroup. Then let $\SPs(G)$ be the set of all
%   supporting partitions of $G$ and $\consps(G)$ be the set of all supports of
%   $G$.
% \end{definition}

% We now prove a few important lemmas on supports.

% \begin{lem}
%   Let $A, B \leq \sym_n$ then $\SP(A \cap B) \subseteq \SP (A) $.
% \end{lem}
% \begin{proof}
%   Let $\mathcal{P}$ be a supporting partition of $A \cap B$. Then
%   $\mathcal{P}$ is a supporting partition of $A$. Since $\SP(A)$ is coarser
%   than all such partitions the result follows.
% \end{proof}

% \begin{lem}
%   Let $h \in H$, then $\SP(\stab(h)) \subseteq \SP (\stab (\row (h)))$ and
%   $\SP (\stab(h)) \subseteq \SP (\stab (\column(h)))$.
% \end{lem}
% \begin{proof}
%   It can be shown that $\stab (h)$
% \end{proof}

% The following lemma will be used to prove the relationship between the support
% of a gate and the supports of its row and columns.
% % Use this lemma + a support type theorem in order to get the result.
% \begin{lem}
%   Let $G \leq \sym_n$, $X_1$ and $X_2$ be sets on which left group actions of
%   $G$ on $X_1$ and $X_2$ are defined. Let $x_1 \in X_1$ and $x_2 \in X_2$ and
%   suppose $\vert \consp(x_1) \vert < \frac{n}{2}$ and $ \vert \consp(x_2)
%   \vert < \frac{n}{2}$. Then if $\SP(\stab(x_1)) \subseteq \SP (\stab(x_2))$
%   then $\consp (x_2) \subseteq \consp (x_1)$.
% \end{lem}
% \begin{proof}
%   Since $\vert \consp(x_1) \vert = \| \SP (\stab(x_1)) \| < \frac{n}{2}$ it
%   follows that there a unique maximal $P_1 \in \SP (\stab(x_1))$ such that
%   $\vert P_1 \vert > \frac{n}{2}$. Similarly there is a unique maximal $P_2
%   \in \SP (\stab(x_2))$. Moreover, from the fact that $\SP (\stab(x_2))$ is
%   coarser than $\SP (\stab(x_1))$, it follows that there exists $P \in
%   \SP(\stab(x_2))$ such that $P_1 \subseteq P$. From the fact that
%   $\frac{n}{2} < \vert P_1 \vert \leq \vert P \vert$ and $P_2$ is the only
%   part such that $\frac{n}{2} < \vert P_2 \vert$, it follows that $P_1
%   \subseteq P = P_2$ and so $\consp(x_2) = [n] - P_2 \subseteq [n] - P_1 =
%   \consp(x_1)$.
% \end{proof}

\subsection{Support Theorems}
The support theorem of Anderson and Dawar~\cite{AndersonD17} asserts
that in a symmetric circuit (over the basis $\mathbb{B}_{\maj}$), if
the circuit is not too large, then the stabiliser group of any gate hs
small support.  In particular, in a family of such circuits of
polynomial size, there is a constant bound on the size of supports.
The proof proceeds in two parts.  First it is shown that if the index
of a group $G \leq \sym_n$ is small then $\SP(G)$ either has very few
or very many parts and, moreover, in the former case there is one
large part (these are reproduced as Lemmas~\ref{lem:big-or-small}
and~\ref{lem:small-means-support} below).  Secondly, it is shown by
induction on the structure of the circuit that the latter case does
not occur as the stabiliser group of any gate.
While the first part of the argument is generically about permutation
groups, the induction in the second part relies crucially on the fact
that each gate computes a \emph{symmetric} Boolean function.  Our aim
in this section is to extend this argument to circuits with
arbitrary structured functions.  We first begin by establishing some
preliminary results we need about group actions on index sets and then
proceed to the main result, Theorem~\ref{thm:support_thm}.


% In this subsection we prove a support theorem for symmetric circuits
% analogous to the support theorem of Anderson and Dawar
% \cite{AndersonD17}, showing that the stabiliser groups of individual
% gates in such circuits have small supports, provided that the circuits
% themselves are not too large.  In general, we refer to the support of
% a gate as a shorthand for the support of the stabiliser group of the gate.
% As a consequence of the theorem we get that for
% any polynomial-size family of symmetric circuits $(C_n)_{n \in \nats}$
% there is a constant $k$ such that every gate in every circuit in the
% family has a canonical support of size at most $k$.



% Informally, we prove the support theorem by showing that, under certain
% conditions on the circuit, if, for a gate $g$, $\SP(g)$ has few parts, then
% $\consp{g}$ has bounded size. We show that $\SP(g)$ indeed has few parts by
% contradiction by establishing a lower-bound on the size of the orbit of $g$ and
% showing that if $\SP(g)$ has too many parts then the size of the orbit of $g$
% becomes too large for the circuit.

% In the first part of this subsection we prove results needed to establish the
% lower-bound on the orbit of a gate. In the second part of this subsection we
% apply this theory to circuits and prove the support theorem.

% In this section let $\mathbb{B}$ be a basis of sort-symmetric functions and
% $(C_n)_{n \in \nats}$ always denote afamily family of symmetric, rigid
% $(\mathbb{B}, \tau)$-circuits with bijective labels. For each $g \in C_n$ let
% $(\tau_g, A_g) := \type(g)$, with sorts $\tau_g = (R, [s], \nu)$ and $A_g =
% A_1 \sqcup \ldots \sqcup A_s$.

% We note that for each $g \in C_n$ there is a natural action of $\stab(g)$ on
% $\ind(g)$ defined by $\sigma \cdot \vec{x} := L(g) \cdot \sigma \cdot
% L(g)^{-1} (\vec{x})$ for each $\vec{x} \in \ind(g)$.



% \begin{lem}[Support Lemma]
%   For any $\epsilon$ and $n$ such that $\frac{2}{3} \leq \epsilon \leq 1$ and
%   $n \geq \frac{128}{\epsilon^2}$, if $C$ is a symmetric, rigid circuit on
%   structures of size $n$ and $s := \max_{g \in C} \vert \orb (g)\vert \leq
%   2^{n^{1-\epsilon}}$, then, for all $g \in C$, $\vert \SP (g) \vert \leq
%   \frac{n}{2}$.
% \end{lem}

% \begin{thm}[Support Theorem]
%   For any $\epsilon$ and $n$ such that $\frac{2}{3} \leq \epsilon \leq 1$ and
%   $n \geq \frac{128}{\epsilon^2}$, if $C$ is a symmetric, rigid circuit on
%   structures of size $n$ and $s := \max_{g \in C} \vert \orb (g)\vert \leq
%   2^{n^{1-\epsilon}}$, then, $SP(C) \leq \frac{33}{\epsilon}\frac{\log s}{\log
%   n}$.
% \end{thm}

% \begin{lem}
%   For any $\epsilon$ and $n$ such that $\frac{2}{3} \leq \epsilon \leq 1$ and
%   $n \geq \frac{128}{\epsilon^2}$, if $C$ is a symmetric, rigid circuit on
%   structures of size $n$ and $s := \max_{g \in C} \vert \orb (g)\vert \leq
%   2^{n^{1-\epsilon}}$, then, for all $g \in C$, we have for all $(i,j) \in
%   \dom (L_g)$ that $\consp(i) \subseteq \consp (L_g(i,j))$ and $\consp(j)
%   \subseteq \consp(L_g (i,j)))$.
% \end{lem}



% \begin{lem}
%   Let $\mathcal{P}_1$ and $\mathcal{P}_2$ be partitions of $[n]$ such that
%   $\mathcal{P}_1 = A \cup S_1$ and $\mathcal{P}_2 = B \cup S_2$, and
%   \begin{enumerate}
%   \item for all $X \in A$ and $Y \in B$ we have that $X \cap Y = \emptyset$,
%     and
%   \item for all $x \in S_1 \cup S_2$, x is a singleton.
%   \end{enumerate}
%   It follows that $\varepsilon(\mathcal{P}_1, \mathcal{P}_2) = A \cup B \cup
%   S$, where $S$ is a set of singletons.
% \end{lem}
% \begin{proof}
% \end{proof}

% \begin{lem}
%   Let $G_1$ and $G_2$ be subgroups of $\sym_n$. Then $\SP^S(G_1) = \SP^S(G_2)$
%   iff $\consp^S(G_1) = \consp^S(G_2)$.
% \end{lem}
% \begin{proof}
  
% \end{proof}

% \begin{lem}
%   Let $G_1$ and $G_2$ be subgroups of $\sym_n$. Then $\consp^S(G_1) =
%   \consp^S(G_2)$ implies that $\consp (G_1) = \consp (G_2)$.
% \end{lem}
% \begin{proof}
% \end{proof}

% The following lemma asserts a few basic facts about the canonical row and
% column supports.

% \begin{lem}
%   Let $h \in H$, then $r_h \subset \consp(h)$, $c_h \subset \consp(h)$ and
%   $c_r \cup r_h = \consp(h)$.
% \end{lem}

% \begin{definition}
%   Let $\omega$ be a matrix labeling for $g$. Define the matrix stabilizer for
%   $\omega$, denoted by $\matstab(\omega)$, to be the set of all $\sigma \in
%   \sym_n$ such that $\sigma H = H$ and there exists $(\alpha, \beta) \in
%   \sym_A \times \sym_B$ such that for all $(i,j) \in A \times B$ we have that
%   $\omega (\alpha i, \beta j) = \sigma \omega (i,j)$.
% \end{definition}

% \begin{definition}
%   Let $\omega$ be a matrix labeling for $g$. Define the matrix stabilizer for
%   $\omega$, denoted by $\matstab(\omega)$, to be the set of all $\sigma \in
%   \sym_n$ such that $\sigma H = H$ and there exists $(\alpha, \beta) \in
%   \sym_A \times \sym_B$ such that for all $(i,j) \in A \times B$ we have that
%   $\omega (\alpha i, \beta j) = \sigma \omega (i,j)$.
% \end{definition}


% \begin{definition}
%   Let $g$ be a gate with matrix labeling $\omega$, $h,h' \in H$ and $\sigma
%   \in \sym_n$. We say that a pair $(h, h')$ is compatible with $(\sigma,
%   \omega)$ if $\sigma h, \sigma h' \in H$ and there exists $(\gamma_1,
%   \gamma_2), (\gamma_1', \gamma_2') \in \sym_A \times \sym_B$ s.t.
%   \begin{align*}
%     &(\gamma_1, \gamma_2) \cdot \omega^{-1} (h) = \omega^{-1}(\sigma h), \text{ and} \\ 
%     &(\gamma_1', \gamma_2') \cdot \omega^{-1} (h') = \omega^{-1}(\sigma h'),
%   \end{align*}
%   and for all $\vec{x} \in \L(g)^{-1}(h)$, $\vec{x}' \in \omega^{-1}(h')$ and
%   all $a \in $
%   \begin{align*}
%     &i =i' \implies \gamma_1(i) = \gamma_1'(i'), \text{ and} \\
%     &j =j' \implies \gamma_2(j) = \gamma_2'(j').
%   \end{align*}

% \end{definition}

% \begin{definition}
%   Let $g \in C_n$ and let $h, h' \in H_g$ and $\sigma \in \sym_n$. We say that
%   a pair $(h, h')$ is \emph{compatible with} $\sigma$ if $\sigma h, \sigma h'
%   \in H_g$ and there exists $\lambda, \lambda' \in \sym_{A_1} \times \ldots
%   \times \sym_{A_s}$ such that
%   \begin{align*}
%     & \lambda \cdot L(g)^{-1} (h)  = L(g)^{-1} (\sigma \cdot h), \text{ and} \\ 
%     & \lambda' \cdot L(g)^{-1} (h') = L(g)^{-1} (\sigma \cdot  h'),
%   \end{align*} 
%   and for all $\vec{x} \in L(g)^{-1}(h)$. $\vec{x}' \in L(g)^{-1}(h')$, and $i
%   \in \dom (\vec{x})$, $\vec{x}(i) = \vec{x}'(i)$ iff $\lambda (\vec{x})(i) =
%   \lambda' (\vec{x}')(i)$.
% \end{definition}

In this subsection we always use $\tau$ to denote a many-sorted vocabulary, and $D$ to denote the universe of a $\tau$-structure.

% Let $H$ be a set on which an action of $\sym_n$ is defined and let $L :
% \ind(\mathcal{D} \rightarrow H$ be a bijection. Let $\isostab(L)$ be the set
% of all $\sigma \in \sym_n$ such that $\sigma \cdot L$ and $L$ are
% isomorphism-equivalent. For the moment we present these ideas abstractly, but
% later on $H$ will denote the set of child gates of some gate $g$ and $L$ will
% be the index for $g$.

\begin{definition}
  Let $\sigma \in \sym_{\ind(\tau, D)}$ and $\vec{x}, \vec{y} \in
  \ind(\tau, D)$, and let $R_i$ and $R_j$ be relation symbols in $\tau$ such that $\vec{x} \in R^{D}_i$ and
  $\vec{y} \in R^{D}_j$. We say that $(\vec{x}, \vec{y})$ is
  \emph{compatible} with $\sigma$ if $\sigma (\vec{x}) \in R^{D}_i$
  and $\sigma (\vec{y}) \in R^D_j$, and for all $a \in [\arty(R_i)]$
  and $b \in [\arty(R_j)]$, $\vec{x}(a) = \vec{y}(b)$ if, and only if, $(\sigma
  (\vec{x}))(a) = (\sigma(\vec{y}))(b)$.

  % Let $H$ be a set on which an action of $\sym_n$ is defined and let $L :
  % \ind(\mathcal{D} \rightarrow H$ be a bijection, and let $\sigma \in \sym_n$.
  % In this case we say that $(h, h') \in H$ is \emph{compatible} with $\sigma$
  % for $L$ if $(L^{-1}(h), L^{-1}(h'))$ is compatible with $L^{-1} \cdot \sigma
  % \cdot L$. In the event that the function $L$ is obvious from the context we
  % just say that $(h, h')$ is compatible with $\sigma$.
\end{definition}

Suppose there is an action of $G \leq \sym_n$ defined on $\ind(\tau, D)$. We say
that $\sigma \in G$ \emph{acts like a sorted permutation} if there exists $\pi
\in \sortsym(\tau, D)$ such that $\pi \cdot \vec{x} = \sigma \cdot \vec{x}$ for
all $\vec{x} \in \ind(\tau, D)$. In this case it is easy to define an action of $G$ on $D$, defined for $\sigma \in G$ by $\sigma \cdot x = \pi (x)$, for all $x \in D$. The intuition then is that a pair of vectors is
compatible with a permutation on $\ind(\tau, D)$ if the permutation is acting like a sorted permutation (i.e. acting like an element of $\sortsym(\tau, D)$) when restricted to those elements appearing in the two vectors. In the following lemma we
show that a permutation acts like a sorted permutation if, and only
if, every pair of tuples is compatible with it. This gives us a useful
sufficient local condition to check if a permutation is acting like a sorted permutation.
%
% The bijection $L$ also allows us to translate an action on the co-domain to an
% action on the domain, translating $\sigma \in \sym_n$ to $L^{-1} \cdot \sigma
% \cdot L$. This translation allows us to define an equivalent compatibility
% condition on the pairs of elements of the co-domain.



% us to define an action of $\sym_n$ on $\ind(\mathcal{D})$, defined for $\sigma
% \in \sym_n$ by $L^{-1}\cdot \sigma \cdot L (\vec{x})$ for all $\vec{x} \in
% \mathcal{D}$. If we think of $L$ as indexing the set $H$ by
% $\ind(\mathcal{D})$, then the compatibility of $(h, h') \in H^2$ with $\sigma
% \in \sym_n$, just tells us that $\sigma$ acts as an automorphism of
% $\mathcal{D}$ on those vectors indexing $h$ and $h'$.

% We have that $L$ is isomorphism-equivalent to $\sigma \cdot L$ if, and only
% if, there exists $\pi \in \aut(\mathcal{D})$ such that $L \cdot \pi = \sigma
% \cdot L$. In other words, $L$ is isomorphism-equivalent to $\sigma$ if, and
% only if, $\sigma$ acts on $\ind(\mathcal{D})$ through $L$ as an automorphism.
% The following lemma tells us that if all the pairs in $H$ are compatible with
% $\sigma \in \sym_n$ for $L$ then $\sigma$ acts on $\ind(\mathcal{D})$ (through
% $L$) as an automorphism.


% \begin{definition}
%   Let $g \in C_n$, $\vec{x}, \vec{x}' \in \ind(g)$ and $\sigma \in \sym_n$. We
%   say that a pair $(\vec{x}, \vec{x}')$ is \emph{compatible with} $\sigma$ if
%   $(\sigma L(g)(\vec{x}), \sigma L(g)(\vec{x}') \in H_g$ and for all $i \in
%   \dom (\vec{x})$, $\vec{x}(i) = \vec{x}'(i)$ iff $(\sigma \cdot \vec{x})(i) =
%   (\sigma \cdot \vec{x}')(i)$.

%   We say that a pair $(h, h')$ is \emph{compatible with} $\sigma$ if
%   $(L(g)^{-1}(h), L(g)^{-1}(h'))$ is compatible with $\sigma$.
% \end{definition}

% \begin{lem}
%   Let $g \in C_n$, and $\sigma \in \ind(g)$. Then $\sigma \in \stab(L(g))$ iff
%   for all $\vec{x}, \vec{y} \in \ind(g)$, $(\vec{x}, \vec{y})$ is compatible
%   with $\sigma$.
%   \label{lem:sortstab-compatible}
% \end{lem}

% \begin{lem}
%   Let $H$ be a set on which an action of $\sym_n$ is defined and let $L :
%   \ind(\mathcal{D} \rightarrow H$ be a bijection. Let $\sigma \in \sym_n$.
%   Then $\sigma \in \isostab(L)$ iff for all $h, h' \in H$, $(h, h')$ is
%   compatible with $\sigma$.
%   \label{lem:isostab-compatible}
% \end{lem}

\begin{lem}
  Suppose there is a defined group action of $G \leq \sym_n$ on
  $\ind(\tau, D)$. Let $\sigma \in G$. Then $\sigma$ acts like a sorted permutation if, and only if, for all $\vec{x}, \vec{y} \in
  \ind(\tau, D)$, $(\vec{x}, \vec{y})$ is compatible with $\sigma$.
  \label{lem:isostab-compatible}
\end{lem}

\begin{proof}
  $`\Rightarrow'$: Let $\vec{x}, \vec{y} \in \ind(\tau, D)$ and
  suppose $\sigma$ acts like a sorted permutation.  Let $\pi \in
  \sortsym(\tau, D)$ be the permutation that witnesses this.  Then $\sigma$
  acts on $\ind (\tau, D)$ as an isomorphism, and so maps tuples in a
  relation to tuples of the same relation. Let $R_i$ and $R_j$ be relation
  symbols in $\tau$ such that $\vec{x} \in
  R^D_i$ and $\vec{y} \in R^D_j$. Let $w \in [\arty
  (R_i)]$ and $z \in [\arty (R_j)]$, then $\sigma \cdot \vec{x} (w) = \sigma
  \cdot \vec{y}(z)$ iff $\pi \cdot \vec{x}(w) = \pi \cdot \vec{y}(z)$ iff
  $\vec{x}(w) = \vec{y}(z)$.

  % Let $i \in \dom (\vec{x})$. If $\vec{x}(i) = \vec{x}'(i)$ then $(\sigma
  % \cdot
  % \vec{x}) (i) = (L(g)^{-1} \cdot \sigma \cdot \L(g) (\vec{x}))(i) =
  % (L(g)^{-1}
  % \cdot \sigma \cdot \L(g) (\vec{x}'))(i) = (\sigma \cdot \vec{x}) (i)$. If
  % $(\sigma \cdot \vec{x})(i) = (\sigma \cdot \vec{x}')(i)$ then $(L(g)^{-1}
  % \cdot \sigma \cdot \L(g) (\vec{x}))(i) = (L(g)^{-1} \cdot \sigma \cdot \L(g)
  % (\vec{x}'))(i)$. Cancelling bijections leaves $\vec{x}(i) = \vec{x}'(i)$.

  
  $`\Leftarrow'$: We have that for all $\vec{x}, \vec{y} \in \ind(\tau, D)$,
  $(\vec{x}, \vec{y})$ is compatible with $\sigma$. We now define $\pi$ such
  that $\pi \in \sortsym(D)$ and $\sigma \cdot \vec{x} = \pi \cdot$. Let $\vec{x} \in \ind(\tau, D)$ be a tuple containing $a$ and
  let $\pi (a) = \sigma \cdot \vec{x} (\vec{x}^{-1}(a))$. The compatibility
  condition ensures that this definition is independent of the particular
  choice of $\vec{x}$.
  
  We show that $\pi$ is injective. Suppose $\pi(a) = \pi(b)$. Suppose $a$ appears in $\vec{x} \in
  \ind(\tau, D)$ and $b$ appears in $\vec{y} \in \ind(\tau, D)$, then
  $\sigma \cdot \vec{x} (\vec{x}^{-1}(a)) = \pi (a) = \pi(b) = \sigma \cdot
  \vec{y}(\vec{y}^{-1}(b))$. From the hypothesis $(\vec{x}, \vec{y})$ is
  compatible with $\sigma$, and so $a = \vec{x}(\vec{x}^{-1}(a)) =
  \vec{y}(\vec{y}^{-1}(b)) = b$. It follows
  $\pi$ is an injection.

%   If $a$ appears in
%   some tuple $\vec{x} \in \ind(\tau, D)$ then $\pi (a) = \sigma \cdot
%   \vec{x} (\vec{x}^{-1}(a)) = b$, and so $b$ also appears in some tuple. Thus we
%   may assume that either both $a$ and $b$ appear in tuples in
%   $\ind(\tau, D)$ or neither of them do. 
  
  We now prove surjectivity. Let $b \in D$. Suppose $b$ appears in
  $\vec{x} \in \ind(\tau, D)$ and let $i \in \dom(\vec{x})$ be such that
  $\vec{x}(i) = b$. Let $\vec{y} = \sigma \cdot \vec{x}$. Then $\pi (\vec{y}(i))
  = \sigma \cdot \vec{y} (\vec{y}^{-1}(\vec{y}(i))) = \vec{x}(i) = b$. It
  follows that $\pi$ is surjective, and thus bijective. The compatibility
  condition gives us that $\pi (R^D_i) = R^D_i$ for any
  relation symbol $R_i$ in $\tau$. Thus $\pi \in \sortsym(D)$.

  Clearly $\pi \cdot \vec{x} = \sigma \cdot \vec{x}$ for all $\vec{x} \in
  \ind(\tau, D)$, and the result follows.
\end{proof}

Let $G \leq \sym_n$ and suppose there is a group action of $G$ on
$\ind(\tau, D)$. We also use $\sortsym_G (D)$ (or
$\sortsym_n(D)$ if $G = \sym_n$) to refer to the subgroup of $G$
consisting of those permutations that act like a sorted permutation
(or, equivalently, are compatible with every pair  of vectors in
$\ind(\tau, D)$).

% $`\Rightarrow'$: We have that there exists $\pi \in \aut (\mathcal{D})$ such
% that $L \cdot \pi = \sigma \cdot L$, and so $\pi = L^{-1} \cdot \sigma \cdot L
% \in \aut(\mathcal{D})$. Let $h, h' \in H$, and let $\vec{x} = L^{-1}(h)$ and
% $\vec{y} = L^{-1}(h')$. Let $R_i$ and $R_j$ be relation symbols in the
% signature of $\mathcal{D}$ such that $\vec{x} \in R^{\mathcal{D}}_i$ and
% $\vec{y} \in R^{\mathcal{D}}_j$. Clearly we have that $L^{-1} \cdot \sigma
% \cdot L (\vec{x}) \in R^{\mathcal{D}}_i$ and $L^{-1} \cdot \sigma \cdot L
% (\vec{y}) \in R^{\mathcal{D}}_j$. Let $a \in [\arty(R_i)]$ and $b \in
% [\arty(R_j)]$ and suppose $\vec{x} (a) = \vec{y} (b)$. We thus have that
% $(L^{-1} \cdot \sigma \cdot L (\vec{x})) (a) = \pi (\vec{x}(a)) = \pi
% (\vec{y}(b)) = (L^{-1} \cdot \sigma \cdot L (\vec{x})) (a)$. The opposite
% direction follows easily, proving the compatibility of $(h, h')$ and $\sigma$.

% % Let $i \in \dom (\vec{x})$. If $\vec{x}(i) = \vec{x}'(i)$ then $(\sigma
% % \cdot
% % \vec{x}) (i) = (L(g)^{-1} \cdot \sigma \cdot \L(g) (\vec{x}))(i) =
% % (L(g)^{-1}
% % \cdot \sigma \cdot \L(g) (\vec{x}'))(i) = (\sigma \cdot \vec{x}) (i)$. If
% % $(\sigma \cdot \vec{x})(i) = (\sigma \cdot \vec{x}')(i)$ then $(L(g)^{-1}
% % \cdot \sigma \cdot \L(g) (\vec{x}))(i) = (L(g)^{-1} \cdot \sigma \cdot
% % \L(g)
% % (\vec{x}'))(i)$. Cancelling bijections leaves $\vec{x}(i) = \vec{x}'(i)$.

  
% $`\Leftarrow'$: We have that for all $\vec{x}, \vec{y} \in \ind(\mathcal{D})$,
% $(\vec{x}, \vec{y})$ is compatible with $L^{-1} \cdot \sigma \cdot L$. We now
% define $\pi$ such that $\pi \in \aut(\mathcal{D})$ and $\sigma \cdot L = L
% \cdot \pi$. Let $a \in \mathcal{D}$. If there is no tuple in
% $\ind(\mathcal{D})$ containing $a$ then let $\pi (a) = a$, otherwise let
% $\vec{x} \in \ind(\mathcal{D}$ be an tuple containing $a$ and let $\pi (a) =
% (L^{-1} \cdot \sigma \cdot L (\vec{x})) (\vec{x}^{-1}(a))$. Notice that the
% compatibility condition ensures that this definition is independent of that
% particular choice of $\vec{x}$. Suppose $\pi(a) = \pi(b)$. Suppose that if $a$
% appears in $\vec{x} \in \ind(\mathcal{D})$ then $\pi (a) = (L^{-1} \cdot
% \sigma \cdot L (\vec{x})) (\vec{x}^{-1}(a)) = b$, and so $b$ also appears in a
% tuple. Thus we may assume that either both $a$ and $b$ appear in a tuple in
% $\ind(\mathcal{D})$ or neither of them do. Suppose $a$ appears in $\vec{x} \in
% \ind(\mathcal{D})$ and $b$ appears in $\vec{y} \in \ind(\mathcal{D})$, then
% $(L^{-1} \cdot \sigma \cdot L (\vec{x})) (\vec{x}^{-1}(a)) = \pi (a) = \pi(b)
% = (L^{-1} \cdot \sigma \cdot L (\vec{y})) (\vec{y}^{-1}(b))$. But $(\vec{x},
% \vec{y})$ is compatible with $L^{-1} \cdot \sigma \cdot L$, and so $a =
% \vec{x}(\vec{x}^{-1}(a)) = \vec{y}(\vec{y}^{-1}(b)) = b$. Suppose instead
% neither $a$ nor $b$ appear in tuples in $\ind(\mathcal{D})$, then $a = \pi (a)
% = \pi (b) = b$. It follows $\pi$ is an injection. Take $b \in \mathcal{D}$. If
% $b$ does not appear in any tuple in $\ind(\mathcal{D})$ then $\pi(b) = b$.
% Suppose $b$ appears in $\vec{x} \in \ind(\mathcal{D})$ and let $i \in
% \dom(\vec{x})$ be such that $\vec{x}(i) = b$. Let $\vec{y} = (L^{-1} \cdot
% \sigma \cdot L)^{-1} (\vec{x})$. Then $\pi (\vec{y}(i)) = ((L^{-1} \cdot
% \sigma \cdot L)^{-1} (\vec{y}))(\vec{y}^{-1}(\vec{y}(i))) = \vec{x}(i) = b$.
% It follows that $\pi$ is surjective, and thus bijective. The compatibility
% condition gives us that $\pi (R^{\mathcal{D}}_i) = R^{\mathcal{D}}_i$ for any
% relation symbol $R_i$ in the signature of $\mathcal{D}$. Thus $\pi \in \aut
% (\mathcal{D})$.
  
% Notice that for $\vec{x} \in \ind(\mathcal{D})$, $L \cdot \pi (\vec{x})) =
% L(\pi (\vec{x})) = L \cdot L^{-1} \cdot \sigma \cdot L (\vec{x}) = \sigma
% \cdot L (\vec{x})$. The result follows.


  
% Then there exists $\vec{x}, \vec{y} \in \ind(\mathcal{D})$ such that $a$
% appears in $\vec{x}$ and $b$ such that $\pi (a) = (L^{-1} \cdot \sigma \cdot L
% (\vec{x})) (\vec{x}^{-1}(a))


% Since $\vec{y}$ and $\vec{x}$ are comp

% Let $R_i$ and $R_j$ be a relation symbols in the signature of $\matcal{D}$ and
% let $\vec{x} \in R^{\mathcal{D}}_i$ and $\vec{y} \in R^{\mathcal{D}}_j$.


% define a permutation $\pi \in \sym_X$ and show that $\pi \in
% \aut(\mathcal{D})$ and $L \cdot \pi = \sigma \cdot L$.


% If there is no tuple in $\ind (g)$ containing an element of sort $t$ then let
% $\pi_t$ be the trivial permutation. Suppose there is an element of sort $t$.
% Let $a \in A_t$. Let $\vec{x}_a \in \ind(g)$ be such that for each $i \in \dom
% (\vec{x}_a)$, if $\vec{x}_a(i)$ has sort $t$ then $\vec{x}_a(i) = a$. Let
% $\pi_t (a) = (\sigma \cdot \vec{x}_a) (i)$. The compatibility condition
% guarantees that $\pi_t$ is well defined.

% We now show $\pi_t$ is indeed a bijection. Let $a, b \in A_t$ and suppose
% $\pi_t (a) = \pi_t(b)$, then $(\sigma \cdot \vec{x}_a)(i) = (\sigma \cdot
% \vec{x}_b)(i)$, and from compatibility $a = \vec{x}_a(i) = \vec{x}_b(i) = b$.
% Moreover, for $a \in A_t$, let $b = \sigma^{-1} \cdot \vec{x}_a$, then $\pi_t
% (b) = (\sigma \cdot \vec{x}_b)(i) = (\sigma \cdot (\sigma^{-1} \cdot
% \vec{x}_a))(i) = a$. So we have that $\pi_t$ is indeed a bijection.

% Let $\pi = \pi_1 \times \ldots \times \pi_s$. Let $\vec{x} \in \ind(g)$, let
% $i \in \dom (\vec{x})$, let $t$ be the sort of $\vec{x}(i)$ and let $a =
% \vec{x}(i)$. Then $(\pi \cdot \vec{x})(i) = \pi_t (\vec{x}(i)) = (\sigma \cdot
% \vec{x}_a)(i) = (L(g)^{-1}\cdot \sigma \cdot L(g)(\vec{x}_a))(i) =
% (L(g)^{-1}\cdot \sigma \cdot L(g)(\vec{x}))(i)$. It follows that $L(g) \cdot
% \pi = L(g)^{-1} \cdot \sigma \cdot L(g)$, and the result follows
  
  

% $`\Leftarrow'$: We now define an element $\pi \in \sym_{A_1} \times \ldots
% \sym_{A_s}$. Let $t \in [s]$. We now defined $\pi_t \in \sym_{A_t}$. If there
% is no tuple in $\ind (g)$ containing an element of sort $t$ then let $\pi_t$
% be the trivial permutation. Suppose there is an element of sort $t$. Let $a
% \in A_t$. Let $\vec{x}_a \in \ind(g)$ be such that for each $i \in \dom
% (\vec{x}_a)$, if $\vec{x}_a(i)$ has sort $t$ then $\vec{x}_a(i) = a$. Let
% $\pi_t (a) = (\sigma \cdot \vec{x}_a) (i)$. The compatibility condition
% guarantees that $\pi_t$ is well defined.

% We now show $\pi_t$ is indeed a bijection. Let $a, b \in A_t$ and suppose
% $\pi_t (a) = \pi_t(b)$, then $(\sigma \cdot \vec{x}_a)(i) = (\sigma \cdot
% \vec{x}_b)(i)$, and from compatibility $a = \vec{x}_a(i) = \vec{x}_b(i) = b$.
% Moreover, for $a \in A_t$, let $b = \sigma^{-1} \cdot \vec{x}_a$, then $\pi_t
% (b) = (\sigma \cdot \vec{x}_b)(i) = (\sigma \cdot (\sigma^{-1} \cdot
% \vec{x}_a))(i) = a$. So we have that $\pi_t$ is indeed a bijection.

% Let $\pi = \pi_1 \times \ldots \times \pi_s$. Let $\vec{x} \in \ind(g)$, let
% $i \in \dom (\vec{x})$, let $t$ be the sort of $\vec{x}(i)$ and let $a =
% \vec{x}(i)$. Then $(\pi \cdot \vec{x})(i) = \pi_t (\vec{x}(i)) = (\sigma \cdot
% \vec{x}_a)(i) = (L(g)^{-1}\cdot \sigma \cdot L(g)(\vec{x}_a))(i) =
% (L(g)^{-1}\cdot \sigma \cdot L(g)(\vec{x}))(i)$. It follows that $L(g) \cdot
% \pi = L(g)^{-1} \cdot \sigma \cdot L(g)$, and the result follows.


% \begin{proof}
%   $`\Rightarrow'$: From the hypothesis there exists $\pi \in \sym_{A_1} \times
%   \ldots \times \sym_{A_s}$ such that $L(g) \cdot \pi = \sigma \cdot L(g)$.
%   Then let $h, h' \in H_g$ and let $\vec{x} = L(g)^{-1}(h)$ and $\vec{x}' =
%   L(g)^{-1}(h')$. Let $i \in \dom (\vec{x})$. If $\vec{x}(i) = \vec{x}'(i)$
%   then $(\sigma \cdot \vec{x}) (i) = (L(g)^{-1} \cdot \sigma \cdot \L(g)
%   (\vec{x}))(i) = (L(g)^{-1} \cdot \sigma \cdot \L(g) (\vec{x}'))(i) = (\sigma
%   \cdot \vec{x}) (i)$. If $(\sigma \cdot \vec{x})(i) = (\sigma \cdot
%   \vec{x}')(i)$ then $(L(g)^{-1} \cdot \sigma \cdot \L(g) (\vec{x}))(i) =
%   (L(g)^{-1} \cdot \sigma \cdot \L(g) (\vec{x}'))(i)$. Cancelling bijections
%   leaves $\vec{x}(i) = \vec{x}'(i)$.
  
%   $`\Leftarrow'$: We now define an element $\pi \in \sym_{A_1} \times \ldots
%   \sym_{A_s}$. Let $t \in [s]$. We now defined $\pi_t \in \sym_{A_t}$. If
%   there is no tuple in $\ind (g)$ containing an element of sort $t$ then let
%   $\pi_t$ be the trivial permutation. Suppose there is an element of sort $t$.
%   Let $a \in A_t$. Let $\vec{x}_a \in \ind(g)$ be such that for each $i \in
%   \dom (\vec{x}_a)$, if $\vec{x}_a(i)$ has sort $t$ then $\vec{x}_a(i) = a$.
%   Let $\pi_t (a) = (\sigma \cdot \vec{x}_a) (i)$. The compatibility condition
%   guarantees that $\pi_t$ is well defined.

%   We now show $\pi_t$ is indeed a bijection. Let $a, b \in A_t$ and suppose
%   $\pi_t (a) = \pi_t(b)$, then $(\sigma \cdot \vec{x}_a)(i) = (\sigma \cdot
%   \vec{x}_b)(i)$, and from compatibility $a = \vec{x}_a(i) = \vec{x}_b(i) =
%   b$. Moreover, for $a \in A_t$, let $b = \sigma^{-1} \cdot \vec{x}_a$, then
%   $\pi_t (b) = (\sigma \cdot \vec{x}_b)(i) = (\sigma \cdot (\sigma^{-1} \cdot
%   \vec{x}_a))(i) = a$. So we have that $\pi_t$ is indeed a bijection.

%   Let $\pi = \pi_1 \times \ldots \times \pi_s$. Let $\vec{x} \in \ind(g)$, let
%   $i \in \dom (\vec{x})$, let $t$ be the sort of $\vec{x}(i)$ and let $a =
%   \vec{x}(i)$. Then $(\pi \cdot \vec{x})(i) = \pi_t (\vec{x}(i)) = (\sigma
%   \cdot \vec{x}_a)(i) = (L(g)^{-1}\cdot \sigma \cdot L(g)(\vec{x}_a))(i) =
%   (L(g)^{-1}\cdot \sigma \cdot L(g)(\vec{x}))(i)$. It follows that $L(g) \cdot
%   \pi = L(g)^{-1} \cdot \sigma \cdot L(g)$, and the result follows.

% \end{proof}
% If there is no type in $\ind (g)$ containing an element of type $t$ then let
% $\pi_t$ be the trivial permutation. Let $a \in A_t$. Let $\vec{x}_a,
% \vec{x}_a' \in \ind(g)$ be such that for every $i \in \dom (\vec{x}_a)$ if
% $\vec{x}_a(i)$ has sort $t$ then $\vec{x}_a(i) = a = \vec{x}_a'(i)$.

 
% $`\Leftarrow'$: We now define an element $\pi \in \sym_{A_1} \times \ldots
% \sym_{A_s}$. Let $t \in [s]$. We now defined $\pi_t \in \sym_{A_t}$. If there
% is no type in $\ind (g)$ containing an element of type $t$ then let $\pi_t$ be
% the trivial permutation. Let $a \in A_t$. Let $\vec{x}_a, \vec{x}_a' \in
% \ind(g)$ be such that for every $i \in \dom (\vec{x}_a)$ if $\vec{x}_a(i)$ has
% sort $t$ then $\vec{x}_a(i) = a = \vec{x}_a'(i)$. From the hypothesis we have
% that $L(g)(\vec{x}_a)$ and $L(g)(\vec{x}_a')$ are compatible with $\sigma$,
% and let this compatibility be witnessed by $\lambda_a$ and $\lambda_a'$. Let
% $\pi_t (a) = \lambda_a (\vec{x}_a)(i)$. Note that $\lambda_a (\vec{x}_a)(i) =
% \lambda_a' (\vec{x}_a'(i)$ and so this function is well defined.

% We have that $\pi_t$ is surjective, as for every $a \in A_t$,
% $\pi_t(\lambda^{-1}_a(a)) = a$. Let $a, b \in A_t$, and suppose $a \neq b$.
% Let $\vec{x}_a, \vec{x}_b$ be defined as above, but choose them such that they
% agree nowhere. We have that $(L(g)(\vec{x}_a), \L(g)(\vec{x}_b))$ are
% compatible with $\sigma$, and it is easy to see that $\lambda_a$ and
% $\lambda_b$ as defined above witness this compatibility. But then
% $\lambda(\vec{x_b})$

% $\pi_t (a) = \pi_t (b)$ implies $\lambda_a (a) = \lambda_b (b)$, but
% $L(g)(\vec{x}_a)$ is compatible with $L(g)(\vec{x}_b)$
  
% Let $\vec{x} \in \ind (g)$ and $i \in \dom(\vec{x})$ and let $t$ be the type
% of $\vec{x}(i)$. Then $\pi_t (\vec{x}(i)) = \lambda (\vec{x}(i))$

% $(\vec{x})$, then $\pi_t (a) = \lambda (\vec{x}_a)(i) = \lambda
% (L(g)^{-1}(h_a))(i) = L(g)^{-1}(\sigma \cdot h_a') (i)$.

% Let $\pi := \pi_1 \times \ldots \times \pi_s$. Let $\vec{x} \in \ind(g)$, then
% $L(g) (\pi \cdot \vec{x}) = $

% Note that for any pair of $\vec{x}$, $\vec{x}'$ chosen with all entries of
% type $t$ being $a$ it is easy to see that $\lambda (a)$
  

% It remains to show that $\pi$ witnesses the sort-equivalence of $\sigma \cdot
% L(g)$ and $L(g)$. First note that $\sigma \cdot L(g)(\vec{x}) = \sigma
% (h_{\vec{x}}) = $


% Then there exists $\lambda_1, \lambda_2 \in \sym_{A_1} \times \ldots \times
% \sym_{A_s}$ witnessing Let $\lambda (\vec{x}) = \$

% we have $\lambda_1$ and $\lambda_2$ witnessing that $(h, h')$ is compatible
% with $\sigma$. We now define $\lambda \in \sym_{A_1} \times \ldots \times
% \sym_{A_s}$ by $\lambda (\vec{a}) = \lambda_1(\vec{a})$


% \begin{lem}[not reviewed]
%   Let $g \in C_n$ be a gate, $\sigma \in \sym_n$. Then $\sigma \in
%   \sortstab(g)$ iff for all $h,h' \in H$ we have that $(h,h')$ is compatible
%   with $(\sigma, \omega)$.
% \end{lem}

% \begin{proof}
%   $`\Rightarrow'$: We have that $\sigma \in \matstab(\omega)$ and so there
%   exists $(\alpha, \beta) \in \sym_A \times \sym_B$ such that for all $(i,j)
%   \in A \times B$ we have $\sigma \omega (i,j) = \omega (\alpha i, \beta j)$.
%   From compatibility we have that $h,h' \in H$ Let $h, h' \in H$ . Let
%   $(\gamma_1, \gamma_2) := (\gamma_1', \gamma_2') := (\alpha, \beta)$. This
%   assignment is sufficient to prove the direction.

%   $`\Leftarrow'$: Suppose for all $h,h' \in H$ we have $(\gamma_1, \gamma_2),
%   (\gamma_1, \gamma_2') \in \sym_A \times \sym_B$ satisfying the above
%   requirements. Notice that for a given $i \in A$ and any $j, j' \in B$, let
%   $h = \omega(i,j)$ and $h' = \omega(i,j')$, then we have that $\gamma_1 (i) =
%   \gamma_1'(i)$. It follows that we can define a $\alpha \in \sym_A$ by
%   $\alpha(i) = \gamma_1 (i)$. Similarly we can define $\beta \in \sym_B$ by
%   $\beta (j) = \gamma_2 (j)$.

% \end{proof}

\begin{definition}
  Suppose there is a group action of $G \leq \sym_n$ on $\ind(\tau, D)$. We
  say that $(\sigma, \vec{x}, \vec{y}) \in G \times \ind(\tau, D)^{2}$ is
  \emph{useful} if $(\vec{x}, \vec{y})$ is not compatible with $\sigma$.

  We say that two distinct pairs $(\sigma_1, \vec{x}_1, \vec{y}_1), (\sigma_2,
  \vec{x}_2, \vec{y}_2) \in G\times \ind(\tau, D)^2$ are \emph{mutually
    independent} if
  \begin{itemize}
    \setlength\itemsep{0mm}
  \item $\sigma_2 \vec{x}_1 = \vec{x}_1$,
  \item $\sigma_2 \sigma_1 \vec{x}_1 = \sigma_1 \vec{x}_1$,
  \item $\sigma_2 \vec{y}_1 = \vec{y}_1$, and
  \item $\sigma_2 \sigma_1 \vec{y}_1 = \sigma \vec{y}_1$.
  \end{itemize}
  
  We say that a set $S \subseteq G \times \ind(\tau, D)^2$ is \emph{useful}
  if each pair in it is useful. We say that $S$ is \emph{independent} if any two
  distinct elements of $S$ are mutually independent.
\end{definition}
%AD: isn't this definition missing something about y_2 and x_2?  I
%also think it needs some intuitive explanation to accompany it.

% \begin{definition}
%   Let $H$ be a set on which the action of $\sym_n$ is defined, and $L:
%   \ind{\mathcal{D}} \rightarrow H$ be a bijection. We say that $(\sigma, h,
%   h') \in \sym_n \times H^2$ is \emph{useful} if $(h, h')$ is not compatible
%   with $\sigma$.

%   We say that two distinct pairs $(\sigma_1, h_1, h_1'), (\sigma_2, h_2, h_2')
%   \in \sym_n\times H^2_g$ are \emph{mutually independent} if
%   \begin{itemize}
%     \setlength\itemsep{0mm}
%   \item $\sigma_2 h_1 = h_1$,
%   \item $\sigma_2 \sigma_1 h_1 = \sigma_1 h_1$,
%   \item $\sigma_2 h_1' = h_1'$, and
%   \item $\sigma_2 \sigma_1 h_1' = \sigma h_1'$.
%   \end{itemize}
  
%   We say that a set $S \subseteq \sym_n \times H^2$ is \emph{useful} if each
%   pair in it is useful. We say that $S$ is \emph{independent} if any two
%   distinct elements of $S$ are mutually independent.
% \end{definition}

% \begin{lem}
%   \label{lem:duel-sort-equivalence-compatible}
%   Let $H$ be a set on which an action of $\sym_n$ is defined and let $L :
%   \ind(\mathcal{D} \rightarrow H$ be a bijection. Let $\sigma_1, \sigma_2 \in
%   \sym_n$ be such that $\sigma_1 \cdot L$ is isomorphism-equivalent to
%   $\sigma_2 \cdot L$. Then for any $h, h' \in H$, $(h, h')$ is compatible with
%   $\sigma_1$ iff $(h, h')$ is compatible with $\sigma_2$.
% \end{lem}
% \begin{proof}
%   We have that $\sigma_1 \cdot L$ is isomorphism-equivalent to $\sigma_2 \cdot
%   L$, and thus there exists $\pi \in \aut(\mathcal{D})$ such that $\sigma_1
%   \cdot L \cdot \pi = \sigma_2 \cdot L $.

%   Suppose $h, h' \in H$, and suppose $(h, h')$ is compatible with $\sigma_1$.
%   Let $\vec{x} := L^{-1}(h)$ and $\vec{y} := L^{-1} (h')$. We then have that
%   $(\vec{x}, \vec{y})$ is compatible with $L^{-1} \cdot \sigma_1 \cdot L$. Let
%   $R_i$ and $R_j$ be relation symbols in the vocabulary of $\mathcal{D}$ such
%   that $\vec{x} \in R^{\mathcal{D}}_i$ and $\vec{y} \in R^{\mathcal{D}}_j$.
%   Then $L^{-1} \cdot \sigma_2 \cdot L (\vec{x}) = L^{-1} \cdot \sigma_1 \cdot
%   L (\pi \cdot \vec{x})$. But since $\pi$ is an automorphism of $\mathcal{D}$
%   we have that $\pi \cdot \vec{x} \in R_i$

%   Similarly $\sigma_2 \cdot \vec{x}' = \sigma_2 \cdot (\pi \cdot \vec{x}')$.
%   Let $i \in \dom (\vec{x})$, then $\vec{x}(i) = \vec{x}'(i)$ iff $(\sigma_1
%   \cdot \vec{x})(i) = (\sigma_2 \cdot \vec{x}')(i)$ iff $(\sigma_1 \cdot (\pi
%   \vec{x}))(i) = (\sigma_2 \cdot (\pi \vec{x}'))(i)$ iff $(\sigma_2 \cdot
%   \vec{x})(i) = (\sigma_2 \cdot \vec{x}')(i)$. The other direction follows by
%   symmetry.
% \end{proof}

% \begin{lem}
%   \label{lem:duel-sort-equivalence-compatible}
%   Let $g \in C_n$ and $\sigma_1, \sigma_1 \in \sym_n$ such that $\sigma_1
%   \cdot L(g)$ is sort-equivalent to $\sigma_2 \cdot L(g)$. Then for any $h,h'
%   \in H_g$, $(h,h')$ is compatible with $\sigma_1$ iff $(h,h')$ is compatible
%   with $\sigma_2$.
% \end{lem}
% \begin{proof}
%   Suppose $\sigma_1 \cdot L(g)$ is sort-equivalent to $\sigma_2 \cdot L(g)$.
%   Then there exists $\pi \in \sym_{A_1} \times \ldots \times \sym_{A_s}$ such
%   that $\sigma_1 \cdot L(g) (\pi \cdot \vec{x}) = \sigma_2 \cdot L(g)
%   (\vec{x})$.

%   Suppose $(h, h')$ is compatible with $\sigma_1$. Let $\vec{x} =
%   L(g)^{-1}(h)$ and $\vec{x}' = L(g)^{-1}(h')$. Then $\sigma_2 \cdot \vec{x} =
%   (L(g)^{-1} \cdot \sigma_2 \cdot L(g)) (\vec{x}) = L(g)^{-1} \cdot \sigma_1
%   \cdot L(g)(\pi \cdot \vec{x}) = \sigma_1 \cdot (\pi \cdot \vec{x})$.
%   Similarly $\sigma_2 \cdot \vec{x}' = \sigma_2 \cdot (\pi \cdot \vec{x}')$.
%   Let $i \in \dom (\vec{x})$, then $\vec{x}(i) = \vec{x}'(i)$ iff $(\sigma_1
%   \cdot \vec{x})(i) = (\sigma_2 \cdot \vec{x}')(i)$ iff $(\sigma_1 \cdot (\pi
%   \vec{x}))(i) = (\sigma_2 \cdot (\pi \vec{x}'))(i)$ iff $(\sigma_2 \cdot
%   \vec{x})(i) = (\sigma_2 \cdot \vec{x}')(i)$. The other direction follows by
%   symmetry.
% \end{proof}


\begin{lem}
  \label{lem:useful-independant-set}
  Suppose there is a group action of $G \leq \sym_n$ on $\ind(\tau, D)$. Let
  $S \subseteq G \times \ind(\tau, D)^2$ be a useful and independent set (with respect to the group $G$).
  Then $[G : \sortsym_n(D)] \geq 2^{\vert S \vert}$.
\end{lem}

\begin{proof}
  For any $R \subseteq S$ define $\sigma_R = \Pi_{(\sigma, \vec{x}, \vec{y}) \in
    R} \sigma$ (with some arbitrary order assumed on $S$). Let $R$ and $Q$ be
  distinct subsets of $S$ and WLOG let $\vert R \vert \geq \vert Q \vert$. It
  suffices to show that $\sigma^{-1}_Q \sigma_R \notin \sortsym (D)$. Pick
  any $(\sigma, \vec{x}, \vec{y}) \in R \setminus Q \neq \emptyset$.
%AD: the claim below doesn't follow just from independence.
 From
  independence we have $\sigma_R \cdot \vec{x} = \sigma \vec{x}$, $\sigma_R
  \cdot \vec{y} = \sigma \cdot \vec{y}$, $\sigma_Q \sigma \cdot \vec{x} = \sigma
  \cdot \vec{x} $, and $\sigma_Q \sigma \cdot \vec{y} = \sigma \cdot \vec{y}$.
  Therefore $\sigma^{-1}_Q \sigma_R \cdot \vec{x} = \sigma^{-1}_Q \sigma \cdot
  \vec{x} = \sigma \cdot \vec{x}$, and similarly $\sigma^{-1}_Q\sigma_R \cdot
  \vec{x} = \sigma \cdot \vec{x}$. Thus, since $(\vec{x}, \vec{y})$ is
  incompatible with $\sigma$, it follows $(\vec{x}, \vec{y})$ is incompatible
  with $\sigma^{-1}_Q\sigma_R$. The result follows from Lemma
  \ref{lem:isostab-compatible}.
\end{proof}

% \begin{lem}
%   \label{lem:useful-independant-set}
%   Let $H$ be a set on which the action of $\sym_n$ is defined, and $L:
%   \ind{\mathcal{D}} \rightarrow H$ be a bijection. Let $S$ be a useful and
%   independent set. We then have that $[\sym_n : \isostab(L)] \geq 2^{\vert S
%   \vert}$.
% \end{lem}

% \begin{proof}
%   For any $R \subseteq S$ define $\sigma_R = \Pi_{(\sigma, h, h') \in R}
%   \sigma$ (with some arbitrary order assumed on $S$). Let $R$ and $Q$ be
%   distinct subsets of $S$ and WLOG let $\vert R \vert \geq \vert Q \vert$. We
%   want to show that $\sigma_R L$ is not isomorphism-equivalent to $ \sigma_Q
%   L$, which is equivalent to showing that $\sigma^{-1}_Q \sigma_R L$ is not
%   isomorphism-equivalent to $L$. Pick any $(\sigma, h, h') \in R \setminus Q
%   \neq \emptyset$. From independence we have $\sigma_R h = \sigma h$,
%   $\sigma_R h' = \sigma h'$, $\sigma_Q (sigma (h)) = \sigma (h) $, and
%   $\sigma_Q (sigma (h)) = \sigma (h)$. Thus $\sigma^{-1}_Q \sigma_R (h) =
%   \sigma^{-1}_Q (\sigma (h)) = \sigma(h)$, and similarly
%   $\sigma^{-1}_Q\sigma_R (h') = \sigma(h')$. Since $(h, h')$ is incompatible
%   with $\sigma$, it follows $(h, h')$ is incompatible with
%   $\sigma^{-1}_Q\sigma_R$. The result follows from Lemma
%   \ref{lem:isostab-compatible}.
% \end{proof}




%
% $determines a set of automorphisms of $\mathcal{D}$, all of which are equal
% when their domain in restricted to $\mathcal{D}_R$. Let $s$ be a sort symbol
% in the vocabulary of $\mathcal{D}$. Let $\sigma \in \isostab(L(g))$ determine
% $\sigma' \in \aut (\mathcal{D})$. Then

% Let $R_i$ be a relation symbol in the vocabulary of $\Sigma(g)$. Then there is
% an action of $\sigma$[\arty(R_i)]$


% \begin{lem}
%   \label{lem:useful-independant-set}
%   Let $g \in C_n$. Let $S$ be a useful and independent. We then have that $[
%   \sym_n: \isostab(L(g)) ] \geq 2^{\vert S \vert}$.
% \end{lem}

% \begin{proof}
%   For any $R \subseteq S$ define $\sigma_R = \Pi_{(\sigma, h, h') \in R}
%   \sigma$ (with some arbitrary order assumed on $S$). Let $R$ and $Q$ be
%   distinct subsets of $S$ and WLOG let $\vert R \vert \geq \vert Q \vert$. We
%   want to show that $\sigma_R L(g)$ is not isomorphism-equivalent to $
%   \sigma_Q L(g)$. Pick any $(\sigma, h, h') \in R/Q \neq \emptyset$. From
%   independence we have $\sigma_R h = \sigma h$ and $\sigma_R h' = \sigma h'$,
%   and from the fact that $(\sigma, h,h')$ is useful, we have that $(h,h')$ is
%   incompatible with $\sigma_R$. Moreover, the fact that $\sigma_Q h = h$ and
%   $\sigma_Q h' = h'$ makes it easy to see $(h,h')$ is compatible with
%   $\sigma_Q$. The result then follows from Lemma
%   \ref{lem:duel-sort-equivalence-compatible}.
% \end{proof}

% \begin{claim}
%   \label{claim:useful-independant-set}
%   Let $g \in C_n$. Let $S$ be a useful and independent. We then have that
%   $\vert \sym_n: \sortstab(g) \vert \leq 2^{\vert S \vert}$.
% \end{claim}

% \begin{proof}
%   Let $R \subseteq S$ and define $\sigma_R = \Pi_{(\sigma, h, h') \in R}
%   \sigma$ (with some arbitrary order assumed on $S$). Let $R$ and $Q$ be
%   distinct subsets of $S$ and WLOG let $\vert R \vert \geq \vert Q \vert$. We
%   want to show that we don't have $\sigma_R \omega \sim_\omega \sigma_Q
%   \omega$. Pick any $(\sigma, h, h') \in R/Q \neq \emptyset$. Given that
%   $\sigma_R h = \sigma h$ and $\sigma_R h' = \sigma h'$ it is easy to see that
%   the usefulness of $(\sigma, h,h')$ implies the incompatibility of $(h,h')$
%   with with $(\omega, \sigma_R)$. Moreover, the fact that $\sigma_Q h = h$ and
%   $\sigma_Q h' = h'$ makes it easy to see $(h,h')$ is compatible with
%   $(\omega, \sigma_Q)$. From the above lemma we may conclude that that we do
%   not have $\sigma_R \sim_\omega \sigma_Q$, and the result follows.
% \end{proof}



The following two lemmas are proved by Anderson and Dawar \cite{AndersonD17},
and are both useful for proving the support theorem. The next lemma is used to
establish a size bound on the supporting partition of a gate. In
particular, it shows that for a partition $\mathcal{P}$ of $[n]$, if the index
of $\setstab (\mathcal{P})$ in $\sym_n$ is small enough, then $\mathcal{P}$
either contains very few or very many parts.

\begin{lem}[{\cite[Lemma 5]{AndersonD17}}]
  \label{lem:big-or-small}
  For any $\epsilon$ and $n$ such that $0 < \epsilon < 1$ and $\log n \geq
  \frac{4}{\epsilon}$, if $\mathcal{P}$ is a partition of $[n]$ with $k$ parts,
  $s = [\sym_n : \setstab (\mathcal{P})]$ and $n \leq s \leq
  2^{n^{1-\epsilon}}$, then $\min \{k, n-k\} \leq \frac{8}{\epsilon} \frac{\log
    s}{\log n}$.
\end{lem}

The following lemma tell us that, under the same assumptions as the previous
lemma, if the number of parts in $\mathcal{P}$ is less then $\frac{n}{2}$ then
$\mathcal{P}$ contains a very large part. This lemma is used both to establish
that the stabiliser group of a gate has small support, and hence a canonical support, but also that
this canonical support has bounded size (and is in fact constant for
polynomial-size circuits).

\begin{lem}[{\cite[Lemma 6]{AndersonD17}}]
  \label{lem:small-means-support}
  For any $\epsilon$ and $n$ such that $0 < \epsilon < 1$ and $\log n \geq
  \frac{8}{\epsilon^2}$, if $\mathcal{P}$ is a partition of $[n]$ with $\vert
  \mathcal{P} \vert \leq \frac{n}{2}$, $s:= [\sym_n : \setstab (\mathcal{P})]$
  and $n \leq s \leq 2^{n^{1-\epsilon}}$, then $\mathcal{P}$ contains a part $P$
  with at least $n - \frac{33}{\epsilon} \cdot \frac{\log s} {\log n}$
  elements.
\end{lem}

Let $C$ be a rigid symmetric circuit with bijective labels that takes in
structures of size $n$. Let $\SP(C)$ denote the maximum $\| \SP(g) \|$ over
all gates $g$ in $C$. Let $g$ be a gate in $C$, and suppose $\Sigma (g)$ is a
$(\tau, D)$-structured function. The action of $\sym_n$ on the circuit induces
an action on $\ind(\tau,D)$, defined for $\sigma \in \sym_n$ by $\sigma \cdot \vec{x}
:= L(g)^{-1} \cdot \sigma \cdot L(g) (\vec{x})$ for all $\vec{x} \in \ind(\tau,D)$.
Moreover, since $C$ is rigid, we have for $\sigma \in \sym_n$, $\sigma \cdot g =
g$ if, and only if, $\sigma \cdot L(g)$ is isomorphism-equivalent to $L(g)$ if,
and only if, $\sigma \in \sortsym_n(D)$. It follows from the
orbit-stabiliser theorem that $\vert \orb (g) \vert = [\sym_n : \stab(g)] =
[\sym_n : \sortsym_n(D)]$.

\begin{thm}
  \label{thm:support_thm}
  For any $\epsilon$ and $n$ such that $\frac{2}{3} \leq \epsilon \leq 1$ and $n
  \geq \frac{128}{\epsilon^2}$, if $C$ is a rigid symmetric circuit with
  bijective labels that takes takes in structures of size $n$ and $s := \max_{g
    \in C} \vert \orb (g)\vert \leq 2^{n^{1-\epsilon}}$, then, $\SP(C) \leq
  \frac{33}{\epsilon}\frac{log s}{log n}$.
\end{thm}

\begin{proof}
  First we note that if $1 \leq s < n$, then $C$ cannot have a relational gate,
  as the orbit of a relational gate has at least $n$ elements. Since $C$ has no
  relational gates, the only input gates in the circuit are the constant gates.
  Since constant gates are fixed by all permutations, it follows that any gate
  $g$ whose children are constant gates must similarly be fixed under all
  permutations. Furthermore, from the fact that $C$ is rigid, this property
  inductively extends to the rest of the circuit. Thus for each gate $g$ in $C$
  the partition $\{[n]\}$ supports $g$, and since this is trivially the coarsest
  such partition it follows that $\| \SP(g) \| = 0 = \SP(C)$. We therefore
  assume that $s \geq n$.

  If $g$ is a gate in $C$ then $\stab (g) \subseteq \setstab(\SP(g))$, and so $s
  \geq \vert \orb(g) \vert = [\sym_n : \stab(g)] \geq [\sym_n :
  \setstab(\SP(g))]$. Thus if $\vert \SP(g) \vert \leq \frac{n}{2}$, then from
  Lemma \ref{lem:small-means-support}, we have $\| \SP (g) \| \leq
  \frac{33}{\epsilon} \cdot \frac{\log s} {\log n}$. The result thus follows
  from showing that for each $g$ in $C$ we have that $\vert \SP (g) \vert \leq
  \frac{n}{2}$.
  
  If $g$ is a constant gate, then as argued above, it follows $\vert \SP(g)
  \vert = 0 < \frac{n}{2}$. If $g$ is a relational gate, then $g$ is fixed by a
  permutation $\sigma \in \sym_n$ if, and only if, $\sigma$ fixes all elements
  that appear in $\Lambda(g)$. It follows that $\{a\} \in \SP(g)$ for each $a$
  appearing in $\Lambda(g)$ and all other elements of $[n]$ are contained in a
  single part of $\SP(g)$. But suppose $\vert \SP(g) \vert > \frac{n}{2}$. Then
  the number of singletons in $\SP(g)$ must be larger than $\frac{n}{2}$, which
  in turn, from the orbit-stabiliser theorem, gives us that $ s \geq \vert \orb
  (g) \vert \geq \frac{n!}{(n-\vert \Lambda (g) \vert)!} \geq
  \frac{n!}{{(\frac{n}{2}})!} \geq 2^{\frac{n}{4}} > 2^{n^{1- \epsilon}} $. This
  is a contradiction, and so $\vert \SP(g) \vert \leq \frac{n}{2}$.

  We now consider the case of internal gates. Let $g$ be the topologically first
  internal gate with $\vert \SP(g) \vert > \frac{n}{2}$. Let $k' := \lceil
  \frac{8 \log s}{\epsilon \log n} \rceil$. From the assumptions on $s$, $ n$
  and $\epsilon$ we have that $k' \leq \frac{1}{4}n^{1-\epsilon} < \frac{n}{2}$.
  We note that Lemma \ref{lem:big-or-small} implies that $n - \vert \SP(g) \vert
  \leq k'$.
  
  We now construct a sufficiently large useful and independent set of
  triples.  Using Lemma
  \ref{lem:useful-independant-set}, this set allows us to place a lower
  bound on the orbit size of $g$, and from this bound we derive the required
  contradiction.  Divide $[n]$ into $\lfloor \frac{n}{k' + 2} \rfloor$ disjoint
  sets $S_i$ of size $k' + 2$ and ignore the elements left over. It follows that
  for each $i$ there is a permutation $\sigma_i$ which fixes $[n] \setminus S_i$
  pointwise but moves $g$. Suppose there was no such $\sigma_i$, but then every
  permutation that fixes $[n]\setminus S_i$ pointwise fixes $g$. Thus the partition of
  all the singletons in $[n]\setminus S_i$ and $S_i$ is a supporting
  partition of $g$.  As
  $\SP(g)$ is the coarsest such partition it follows that $\vert \SP(g) \vert \leq n
  - (k'+2) + 1 = n - k' - 1$, which contradicts the inequality $n - \vert \SP(g)
  \vert \leq k'$.

  Since $g$ is moved by each $\sigma_i$, and $C$ is rigid, it follows that we
  have $\sigma_i \notin \sortsym_n(D)$. Thus, from Lemma
  \ref{lem:isostab-compatible} and the fact that the circuit has bijective
  labels, there exists $(h_i, h_i') \in H^{-1}_g$ such that $(L(g)^{-1}(h_i),
  L(g)^{-1}(h_i'))$ is inconsistent with $sigma_i$, and so the triple
  $(\sigma_i, L(g)^{-1}(h_i), L(g)^{-1}(h_i'))$ is useful.

%AD: I haven't changed this below in case I'm just confused.  But,
%isn't \SP(h)^* just a new notation for \sp(h)?
  Let $\SP (h)^*$ be the union of all parts of $\SP(h)$ except for the largest
  part. Let $Q_i = \SP(h_i)^* \cup \SP(\sigma_i \cdot h_i)^* \cup \SP (h_i')^*
  \cup \SP (\sigma_i \cdot h_i')^*$. Then note that if $\sigma_j$ fixes $Q_i$
  then by construction we have that $\sigma_j \in \stab(\SP(h_i)) \cap
  \stab(\SP(\sigma_i h_i)) \cap \stab(\SP(h_i')) \cap \stab(\SP(\sigma_i h_i'))$

  Define a graph $K$ with vertices given by the sets $S_i$ and an edge from
  $S_i$ to $S_j$ (with $i \neq j$) iff $Q_i \cap S_j \neq \emptyset$. It follows
  then that if there is no edge between $S_i$ and $S_j$ then $(\sigma_i,
  L(g)^{-1}(h_i), L(g)^{-1}(h_i'))$ and $(\sigma_j, L(g)^{-1}(h_j),
  L(g)^{-1}(h_j'))$ are mutually independent. It remains to argue that $K$ has a
  large independent set. This is possible as the out-degree of $S_i$ in $K$ is
  bounded by
  \begin{align*}
    \vert Q_i \vert \leq \|\SP(h_i) \| + \|\SP(\sigma_i h_i) \| + \|\SP(h_i') \| + \|\SP(\sigma_i h_i') \leq 4 \cdot \frac{33\log s}{\epsilon \log n}.
  \end{align*}
  These inequalities follow from the fact that the sets $S_i$ are disjoint and we may apply Lemma
  \ref{lem:small-means-support} to each of the child gates. From these inequalities
  we have that the average total degree (in + out degree) of $K$ is at most $2 \cdot \vert Q_i
  \vert \leq 34 \cdot k'$. Now greedily select a maximal independent set in $K$
  by repeatedly selecting $S_i$ with the lowest total degree and eliminating it
  and its neighbours. This action does not affect the bound on the average total
  degree of $K$ and hence determines an independent set $I$ in $K$ of size at
  least
  \begin{align*}
    \frac{\lfloor \frac{n}{k' + 2} \rfloor}{34k' + 1} \geq \frac{n - (k'+2)}{34k'+1k'+2} \geq \frac{n\frac{7}{16}}{34k'^2 + 69k' +2} \geq \frac{n}{(16k')^2}.
  \end{align*}

  Take $S = \{(\sigma_i, L(g)^{-1}(h_i), L(g)^{-1}(h_i')) : S_i \in I \}$. Then
  from the above argument we have that $S$ is useful and independent.
  
  Moreover, from Lemma \ref{lem:useful-independant-set}, we have that $s \geq
  \vert \orb(g) \vert \geq 2^{\vert S \vert} \geq 2^{\frac{n}{(16k')^2}}$ then
  $n^{1-\epsilon} \geq \log s \geq n \cdot (\frac{128}{\epsilon}\frac{\log
    s}{\log n})^{-2} > n \cdot (n^{1-\epsilon})^{-2} = n^{2\epsilon -1} \geq
  n^{1-\epsilon}$. This is a contradiction, and the result follows.
\end{proof}
 
Let $\mathcal{C} = (C_n)_{n \in \mathbb{N}}$ be a polynomial-size family of
rigid symmetric circuits with bijective labels. Then $s(n) := \max_{g \in C_n}
\vert \orb (g)\vert$ must be polynomially bounded, and so the theorem implies
that there exists $k \in \mathbb{N}$ such that for all $n$ large enough, for all
$g \in C_n$, $\vert \consp(g) \vert \leq k$.

\begin{cor}
  Let $\mathcal{C} := (C_n)_{n \in \mathbb{N}}$ be a polynomial-size rigid
  symmetric circuit family with bijective labels. Then $\SP(\mathcal{C}) \in
  O(1)$.
\end{cor}

% If $\mathcal{C} := (C_n)_{n \in \mathbb{N}}$ with bijective labels, we can
% extend the action of $\sigma \in \sym_n$ on the gates of $C_n$ to the sorts of
% the indexes of those gates. Let $g \in C_n$ and $(\tau, A) := \type(g)$ and
% let $A_1, \ldots, A_s$ be the sorts of $A$. Note that for any $h_1, h_2 \in
% H_g$, and $i \in [r_i]$ $L(g)(h_1)(i, R_j)$ For all $i \in [s]$, we can define
% the action of $\sigma \in \sym_n$ on $A_i$ by $\sigma (a) = L(g)()$

% $a_i \in A_$

Let $C_n$ be a rigid symmetric circuit with bijective labels. We now extend the given action of $\sym_n$ on the gates of $C_n$ to the indexes. Let $g$ be a gate in $C_n$, and let $(\tau, D) := \type(g)$. Note that the action of any permutation in $\stab(g)$ on the circuit fixes $H_g$ setwise. Therefore the action of $\stab(g)$ on $H_g$ induces an action of $\stab(g)$ on $\ind(\tau, D)$. Moreover, since $\stab(g) = \sortsym_n(D)$, it follows that $\stab(g)$ acts on $\ind(\tau, D)$ as a sorted permutation.

For $x \in D$ let $\stab_g(x) = \{\sigma \in \stab(\consp(g)) : \sigma (x) = x\}$ and let $\consp_g(x)$ be the canonical support of $\stab_g(x)$. We call $\consp_g(x)$ the canonical support of $x$ with respect to $g$. In the event that the gate $g$ is clear from context we omit the subscript. If $h \in H_g$ and $(x_1, \ldots ,x_r) := L(g)^{-1}(h)$ then $\stab(h) \cap \stab(g) = \stab_{\stab(g)}(h) = \bigcap_{i \in [r]} \stab_g(x_i)$. If $\stab(h) \cap \stab(g)$ has small support, then from Lemma \ref{lem:row-column-supports-well-behaved} it has canonical support $\consp(h) \cup \consp(g)$. Moreover, if $\stab(h) \cap \stab(g)$ has small support, then $\stab_g(x_i)$ for all $i \in [r]$ and $\consp(g) \cup \consp(h) = \bigcup_{i \in [r]}\consp_g(x_i)$. This follows from repeated application of Lemma \ref{lem:row-column-supports-well-behaved}. As such, we may apply the support theorem to the elements of $D$ for large enough $n$.

%REVIEW THIS BIT, MORE WORK MAY BE NEEDED.

% It follows that for each gate $g$ and $h \in H_g$, $\stab(h) \subseteq
% \stab(L(g)^{-1}(h))$. Thus all supports and supporting partitions of $h$ are also supports or supporting partitions of $L(g)^{-1} (h)$, which allows us to translate the support theorem to elements of the index of $g$.
% Suppose $\Sigma(g)$ is a $(\tau, D)$-structured function. Notice that $\stab
% (\consp(g)) \subseteq \stab (g) = \sortsym_n(D)$, and so each 
% $\sigma \in \stab(g)$ may be thought of as a sorted-permutation of $D$, defining an action of $\stab(g)$ on $D$. Furthermore, if $h \in H_g$ and $L(g)^{-1}(h) = (x_1, \ldots , x_{r_i}) \in \ind(\tau, D)$ then $\stab_{\stab(g)}(h) = \stab_{\stab(g)}(x_1) \cap \ldots \cap \stab_{\stab(g)}(x_k)$


% there exists $\pi_{\sigma} \in \sortsym(D)$
% such that $\pi_\sigma \cdot \vec{x} = \sigma \cdot \vec{x}$ for all $\vec{x} \in
% \ind(\tau, D)$. It follows that we can define an action of $\stab(\consp(g))\leq \sym_n$ on $D$  defined for $\sigma \in \stab(g)$ by $\sigma \cdot a := \pi (a)$
% for all $a \in D$. Mo


% Thus there is a well-defined action of $\sym_n$ on the set of all elements in
% the universe of $\mathcal{D}$ that appears in some relation. It follows that
% $\sigma$ defines a action on the elements of the universe of $\mathcal{D}$
% that appear in some relation, and also this action extends to an automorphism
% of $\mathcal{D}$. Thus for each $a \in \mathcal{D}$ we may speak For any
% $\sigma \in \stab(\consp(g)) \subseteq \aut_n(\mathcal{D})$, $\sigma$ fixes
% $g$ and hence permutes $H_g$ and acts as an automorphism. Then $\sigma$
% $\mathcal{D}$

\end{document}
