\documentclass[../paper.tex]{subfiles}

\begin{document}
In this section we present basic background on logics, first-order structures,
and symmetric circuits.

\subsection{Logic and Structures}
A \emph{relational vocabulary} (otherwise just called a \emph{vocabulary} or
\emph{signature}) is a finite sequence of relation symbols $(R_1, \ldots, R_k)$,
each of which has a fixed \emph{arity}, denoted by $\arty(R_i) \in \nats$. We
often let $r_i \in \nats$ denote the arity of the relation symbol $R_i$. A
\emph{many-sorted vocabulary} is a tuple of the form $(R, S, \nu)$, where $R$ is
a relational vocabulary, $S$ is a finite sequence of \emph{sort} symbols $(s_1,
\ldots s_p)$, and $\nu$ is a function that assigns to each $R_i \in R$ a tuple
$\nu(R_i) := (s^i_1, \ldots, s^i_{r_i})$, where for each $j \in [r_i]$, $s^i_j
\in S$. We call $\nu$ the \emph{\type function} and $\nu(R_i)$ the \emph{type}
of $R_i$. We notice that a relational vocabulary can be thought of as a
single-sorted vocabulary, i.e. a many-sorted vocabulary where the set of sorts
is a singleton. As such, if $\tau := (R, S, \nu)$ is single-sorted we identify
$\tau$ with $R$.

Let $\tau := (R, S, \nu)$ be a many-sorted vocabulary, where $R = (R_1, \ldots,
R_k)$ and $S = (s_1, \ldots , s_p)$ A \emph{$\tau$-structure} $\mathcal{A}$ is a
tuple $(U , R^{\mathcal{A}}_1 , \ldots , R^{\mathcal{A}}_k)$ where $U =
\uplus_{s \in S } U_{s}$ a disjoint union of non-empty sets, and is called the
\emph{universe} of $\mathcal{A}$, and for all $i \in [k]$, $R^{\mathcal{A}}_i
\subseteq R^{\mathcal{A}}_i \subseteq U_{s^i_1} \times \ldots \times
U_{s^i_{r_i}}$, where $(s^i_1 , \ldots , s^i_{r_i}) = \nu (R_i)$. The size of
$\mathcal{A}$, denoted by $\vert \mathcal{A} \vert$ is the cardinality of its
universe. The \emph{sorted-size} of $\mathcal{A}$ is a vector $\vec{n} :=
(n_{s_1}, \ldots , n_{s_p}) \in \mathbb{N}^{p}$ such that $\vert U_{s_i} \vert =
n_{s_i}$ for all $i \in [p]$. We say $a \in U$ has \emph{sort} $s \in S$ if $a
\in U_s$.

All structures in this paper are assumed to have finite size unless stated
otherwise. For more details on these basic definitions please
see~\cite{Gradel:2005}, \cite{immerman1999descriptive},
or~\cite{grohe2017descriptive}.

% \begin{definition}
%   Let $S = \{s_1, \ldots , s_p\}$ be a set of sort symbols and $A = A_{s_1}
%   \uplus \ldots \uplus A_{s_p}$ and $B = B_{s_1}\uplus \ldots \uplus B_{s_p}$
%   be disjoint unions of non-empty sets. A \emph{sorted bijection} between $A$
%   and $B$ is a bijection $f: A \rightarrow B$ such that $f (A_s) = B_s$ for
%   all $s \in S$.
% \end{definition}

\subsection{First Order and Fixed-Point Logic}
Let $L$ be a logic and $\tau$ be a vocabulary, we let $L[\tau]$ denote the set
of all $\tau$-formulas of $L$ (i.e. the set of all $L$-formulas with respect to
$\tau$). Let $\FO[\tau]$ denote \emph{first-order logic} with respect to the
vocabulary $\tau$. A formula in $\FO[\tau]$ is formed from atomic formulas, each
formed using variables from some countable sequence of (first-order) variable
symbols $(x, y, \ldots)$, the relation symbols in $\tau$, and the equality
symbol $=$, and then closing the set of atomic formulas under the Boolean
connectives, negation, and universal and existential quantification (i.e.
$\land$, $\lor$, $\neg$, $\forall$, and $\exists$).

Let $\FP[\tau]$ denote \emph{fixed-point logic} with respect to $\tau$. The
logic $\FP$ is an extension of $\FO$ by an inflationary fixed-point operator. In
other words, $\FP[\tau]$ includes all formulas in $\FO[\tau]$, along with all
formulas of the form $[\ifp_{V, \vec{x}}. \phi]$, where $V$ is a relation symbol
not in $\tau$, $\phi \in \FO[\tau \cup \{V\}]$, and $\vec{x}$ is a sequence of
variables of length $\arty(V)$. We assume the standard syntax and semantics for
$\FO[\tau]$ and $\FP[\tau]$. For more detail on the syntax and semantics of
these logics please see~\cite{grohe2017descriptive} and~\cite{GradelP15a}.

Let $L$ be a logic and $\tau$ a vocabulary. For $\phi \in L[\tau]$ and a
sequence of variables $\vec{x}$, we write $\phi(vec{x})$ if the free variables
in $\phi$ are among those in $\vec{x}$. Suppose $\vec{x} = (x_1, \ldots , x_k)$,
and let $\mathcal{A}$ be a $\tau$-structure with universe $U$, and let $\vec{a}
\in U^k$. We write $\mathcal{A} \models_L \phi[\vec{a}]$ if, and only if, $\phi$
is true in $\mathcal{A}$ with respect to $L$ where for all $i \in [k]$ each
variable $x_i$ is assigned to $a_i$. If the logic is obvious from context we
omit the subscript and just write $\mathcal{A} \models \phi[\vec{a}]$.

\subsection{Fixed-Point with Counting}
Let $\FPC[\tau]$ denote \emph{fixed-point logic with counting} with respect to
the vocabulary $\tau$. The logic $\FPC$ is an extension of $\FP$ that includes a
counting operator that allows the logic to determine the size of a definable
set, as well as other mechanisms for reasoning about quantities. We now briefly
describe the structure of a formula of $\FPC[\tau]$. In order to facilitate
counting we define terms which are assigned to natural numbers. Every
first-order variable is assigned a type, and so each variable is either a
\emph{vertex} or \emph{number} variable. Vertex variables are assigned to
elements of the universe of the structure, and play a similar role to the
variables that appear in formulas in $\FO[\tau]$ or $\FP[\tau]$. Number
variables are assigned to elements of $\nats$. We usually use Latin letters
(e.g. $x, y, z, \ldots$) to denote vertex variables and Greek letters (e.g.
$\mu, \nu, \epsilon, \ldots$) to denote number variables. All atomic formulas in
$\FP[\tau]$ are atomic formulas in $\FPC[\tau]$. We say that $t$ is a
\emph{number term} if $t$ a number variable or if $t$ is an application of the
\emph{counting operator}, i.e. $t = \#x. \phi(x)$, where $\phi \in \FPC[\tau]$
and $x$ is a vertex variable free in $\phi$. If $t_1$ and $t_2$ are number terms
then $t_1 \leq t_2$ and $t_1 = t_2$ are atomic formulas. The formulas of $\FPC$
are then formed by closing the set of atomic formulas under the usual Boolean
connectives, the first-order quantifiers (for vertex variables), and the
inflationary fixed-point operator. The second-order variables, such as those
that appear in an application of the fixed-point operator, are allowed to have
mixed-type, i.e. we associate with a second-order variable $R$ a type $(k, l)
\in \nats^2$ intended to mean that $R$ is assigned to a subset of $U^k \times
\nats^l$.

In order to ensure that the data complexity of $\FPC$ is in polynomial time, we
only allow bounded quantification when using number variables. In other words,
if $Q \in \{\forall, \exists \}$, $\phi \in \FPC[\tau]$, $\mu$ is a number
variable free in $\phi$, and $t$ is a number term, then $Q \mu \leq t . \phi$ is
a formula of $\FPC[\tau]$. Similarly, if $\phi$ is a number term then $\# \mu
\leq t .\phi$ is a number term in $\FPC[\tau]$.

If $\mathcal{A}$ is a $\tau$-structure with universe $U$, the counting term
$\#x. \phi(x)$ evaluates to $\vert \{a \in U : \mathcal{A} \models \phi[a]\}
\vert$ in $\mathcal{A}$, assuming a fixed assignment to the other free variables
in $\phi$. The intended semantics for bounded quantification is that the
variable may only be assigned to a value at most equal to the the evaluation of
the number term bounding the variable. The semantics of the $=$ and $\leq$
symbols are defined for number terms in the obvious manner. For a more detailed
description of the syntax and semantics of $\FPC$ please
see~\cite{grohe2017descriptive} or the introduction of~\cite{GradelP15a}.

It is possible to define the usual arithmetic operations on number terms using
the constructs thus far introduced. We do not show how to explicitly define
these operations in $\FPC$, but we do make use of them in this paper. Please
see~\cite{grohe2017descriptive} for the explicit constructions.

Let $\FOC[\tau]$ denote \emph{first-order logic with counting} with respect to
the vocabulary $\tau$, and be the set of all formulas in $\FPC[\tau]$ that do
not contain the an application of the fixed-point operator.


% We also recall the definition of \emph{fixed-point logic with counting}
% $\FPC$, an extension $\FP$ with counting terms (see
% \cite{grohe2017descriptive} for details). Variables and terms in $\FPC$ must
% be of one of two sorts, vertex or numeric, where vertex variables are assigned
% to elements of the universe the structure and numeric variables to the
% elements of $\mathbb{N}$. All atomic formula as in $\FP$ are formulas of
% $\FPC$ (using only vertex sort variables). Let $\phi(x)$ be a formula of
% $\FPC$ with free variables $x$, then $\# x \phi$ is a numeric, with the term's
% semantics being the number of elements satisfying $\phi$. All numeric
% variables as well as the symbols 0 and 1 are numeric terms with the expected
% semantics. If $\mu_1 and \mu_2$ are numeric terms, then $\mu_1 + \mu_2$ and
% $\mu_1 * \mu_2$ are numeric terms with the expected semantics. Moreover, if
% $\mu_1$ and $\mu_2$ are numeric terms, then $\mu_1 < \mu_2$ and $\mu_1 =
% \mu_2$ are formulas with the expected semantics. If $\vec{x}$ is a $k$-tuple
% of numeric variables and $\vec{\mu}$ is a $k$-tuple of closed numeric term
% (i.e. ones without any free variables) then $\exists \vec{x} \leq \vec{\mu}
% \phi(\vec{x})$ and $\forall \vec{x} \leq \vec{mu} \phi(\vec{x})$ are formulas.
% In the existential case, for a $\tau$-structure $\mathcal{A}$, we say
% $\mathcal{A} \models \exists \vec{x} \leq \vec{\mu} \phi(\vec{x})$ iff there
% exists $\vec{n} \in [\mu^{\mathcal{A}}_1] \times \ldots \times
% [\mu^{\mathcal{A}}_2]$, such that $phi^{\mathcal{A}}$ holds for the assignment
% $\vec{x} \mapsto \vec{n}$. The semantics of the universal case follows
% similarly. The application of the fixed-point operator (but now with the
% relation variables possibly being of mixed type), and Boolean connectives
% similarly define formulas as in $\FP$. The semantics of these formulas is also
% similarly defined.

% Let $\FO(\tau)$ denote \emph{first order logic} with respect to the vocabulary
% $\tau$. \emph{Fixed-Point Logic} is the extension of $\FO(\tau)$ with an
% inflationary fixed-point operator (see \cite{} for details). For logic
% $\mathcal{L}$, a formula $\phi \in \mathcal{L}(\tau)$, $x$ a $k$-length tuple
% of variables, $\mathcal{A}$ a (finite) structure, and $a \in U^k$, we have
% that $\phi (x)$ denotes that the variables in $x$ are free in $\phi$ and we
% write $\mathcal{A} \models_{\mathcal{L}} \phi[a]$ iff the the formula $\phi$
% is true in $\mathcal{A}$ with respect to $\mathcal{L}$ with the variables $x$
% assigned to $a$. The context usually makes the logic obvious, and in that case
% we usually drop the subscript.

% We also recall the definition of \emph{fixed-point logic with counting}
% $\FPC$, an extension $\FP$ with counting quantifiers. For a structure
% $\mathcal{A}$, Formulas of this logic are evaluated over a structure
% $\mathcal{A}$ by extending that structure to $\mathcal{A}^\leq$, a two sorted
% extension of $\mathcal{A}$ by an ordered number sort $\{1, \ldots, \vert
% U\vert \}$. Variables are taken to be either point variables, ranging over $U$
% or number variables ranging over $[\vert U \vert]$. The counting quantifiers
% are of the form $\#_x$, where $x$ is either a point or number variable and for
% $\phi(x) \in \FPC(\tau)$, $\#_x \phi(x)$ is a term denoting the element of the
% number sort corresponding to $\vert \{a \in \mathcal{A} | \mathcal{A} \models
% \phi[a]\}\vert$. Note additionally that $k$-tuples of number variables can be
% used to encode numbers in $\FPC$ of size up to $\vert U\vert^k$. For more
% details on $\FPC$ please see \cite{}.

\subsection{Rank Logic}
Let $\FPR[\tau]$ denote \emph{fixed-point logic with rank} (or just \emph{rank
  logic}) with respect to the vocabulary $\tau$. The logic $\FPR$ is an
extension of $\FP$ to include an operator that decides the rank of a definable
matrix over a finite field, as well as other mechanisms for reasoning about
quantity. In this way we can think of the structure of a formula of $\FPR$ as
being similar to that of $\FPC$, with rank operators replacing counting
operators.

We now briefly describe the structure of a formula in $\FPR[\tau]$. For more
detail on the syntax and semantics of $\FPR$ please see~\cite{GradelP15a}
and~\cite{Dawar09logicswith}. As in the case of $\FPC$, we suppose the
first-order variable are assigned one of two types. We call the first type
\emph{vertex variables} and the second \emph{number variables} variables. Vertex
variables are assigned to elements of the universe and number variables are
assigned to natural numbers. All atomic formulas in $\FP[\tau]$ are atomic
formulas in $\FPC[\tau]$. We say that $t$ is a \emph{number term} if $t$ is a
number variable or if $t$ is an application of the \emph{rank operator}, i.e. $t
= [\rank (\vec{x}, \vec {\nu} \leq \vec{t}, \vec{y}\vec{\mu} \leq \vec{s}, \pi
\leq \eta). \phi]$, where $\phi$ is a number term, $\vec{t}$ and $\vec{s}$ are
tuples of number terms bounding the sequences of number variables $\vec{\mu}$
and $\vec{\nu}$, and $\eta$ is a number term bounding the number variable $\pi$.
If $t_1$ and $t_2$ are number terms then $t_1 \leq t_2$ and $t_1 = t_2$ are
atomic formulas. The formulas of $\FPR$ are then formed by closing the set of
atomic formulas under the usual Boolean connectives, the first-order
quantifiers, and the inflationary fixed-point operator. As in the case of
$\FPC$, the second-order variables may have mixed-type and we only allow bounded
quantification for number variables. We note additionally that formulas can be
treated as number terms that evaluate to zero or one.

We now briefly explain the intended semantics of the rank operator. Let $\phi
(\vec{x}, \vec {\nu}, \vec{y}\vec{\mu})$ be a number term, where $\vec{x}$ and
$\vec{y}$ are tuples of vertex variables, $\vec{\nu}$ and $\vec{\mu}$ are tuples
of number variables. Let and $\vec{s}$ and $\vec{t}$ be tuples of number terms
that bound the ranges of $\vec{\nu}$ and $\vec{\mu}$, respectively. Let $\pi$ be
a number variable and $r$ be a number term bounding the range of $\pi$. Let
$\mathcal{A}$ be a structure with universe $A$, and suppose we have a given
assignment to the all of the free variables appearing in $r$ and in the terms in
$\vec{t}$ and $\vec{s}$. Let $\mathbb{N}^{\leq\vec{t}} := \{\vec{n} \in
\mathbb{N}^{\vert \vec{v} \vert}: n_i \leq t^{\mathcal{A}}_i\}$, and let
$\mathbb{N}^{\leq \vec{s}}$ be defined similarly. Let $I:= A^{\vert \vec{x}
  \vert} \times \mathbb{N}^{\leq \vec{t}}$ and $J := A^{\vert \vec{y} \vert}
\times \mathbb{N}^{\leq \vec{s}}$. Let $M^{\mathcal{A}}_\phi : I \times J
\rightarrow \mathbb{N}$ be defined by $M^{\mathcal{A}}_\phi (\vec{a}\vec{n},
\vec{b}\vec{m}):= \phi^{\mathcal{A}}(\vec{a}\vec{n},\vec{b}\vec{m})$. For the
structure $\mathcal{A}$ and assignment $\pi \mapsto p$, for $p \leq r^{A}$ prime,
$[\rank (\vec{x}, \vec {\nu} \leq \vec{t}, \vec{y}\vec{\mu} \leq \vec{s}, \pi
\leq r). \phi]$ evaluates to the rank of $M^{\mathcal{A}}_\phi$ mod $p$, where
the matrix is understood to have entries in $\mathbb{F}_p$.

We should note that there are two different of rank logics defined in the
literature. The first, introduced by Dawar et al.~\cite{Dawar09logicswith},
differs from the one defined above in that it does not take the characteristic
of the field as a parameter, instead including a different operator for every
prime field over which compute rank. It was shown by Grad{\"e}l and
Pakusa~\cite{GradelP15a} that this rank logic does not capture $\PT$. Grad{\"e}l
and Pakusa introduced the rank logic we have presented above and showed that
this logic, with the uniform rank operator, is strictly more expressive then the
rank logic introduced by Dawar et al. For this reason when we refer to rank
logic we are always referring to the rank logic introduced by Grad{\"e}l and
Pakusa~\cite{GradelP15a}.

We may also define an extension of first order logic by rank operators. Let
$\FOR[\tau]$ denote \emph{first-order with rank} with respect to the vocabulary
$\tau$, and be the set of formulas in $\FPR[\tau]$ that do not contain an
application of the fixed-point operator.

Let $\FPrk[\tau]$ denote \emph{fixed-point logic with rank quantifiers} with
respect to the vocabulary $\tau$. This logic, first introduced by Dawar et
al~\cite{Dawar09logicswith}, is an extension of fixed-point logic by \emph{rank
  quantifiers}. For a prime $p$ and $r$ a natural number, a rank quantifier
$\rank^r_p$ is defined such that $\rank^r_p\vec{x}\vec{y}. \phi$ is interpreted
as $[\rank (\vec{x}, \vec{y}, \pi) . \phi] \leq t$, where $\pi$ is assigned to
$p$ and $t$ evaluates to $r$. The arity of this quantifier is $\vert \vec{x}
\vert + \vert \vert{y} \vert$. We thus have a rank quantifier for every prime
$p$, natural number $r$ and arity $k$. Let $\mathcal{R}$ be the set of all such
quantifiers. Let $\FPrk[\tau]$ be the closure of $\FP[\tau]$ under the
quantifiers in $\mathcal{R}$.

We may also define an extension of first-order logic with rank quantifiers. Let
$\FOrk[\tau]$ denote \emph{first-order with rank quantifiers} with respect to
the vocabulary $\tau$, and be the set of all formulas in $\FPR[\tau]$ that do
not contain an application of the fixed-point operator.

For more detail on the syntax and semantics of these logics with rank
quantifiers please see~\cite{Dawar09logicswith}. It is easy to see that $\FPrk
\leq \FPR$ and $\FOrk \leq FOR$. In fact, Dawar et al.~\cite{Dawar09logicswith}
shows that these inclusions are strict. Moreover, we note that $[\rank(\vec{x},
\vec{y}). (\vec{x} = \vec{y}, \pi \leq r) \land \phi(\vec{x})]$ evaluates, for a
structure $\mathcal{A}$ and any assignment $\pi \mapsto p$, to the number of
assignments to $\vec{x}$ for which $\phi$ holds (i.e.\ $\vert \{\vec{a} \in
U^{\vert \vec{x} \vert} : \mathcal{A} \models \phi[\vec{a}]\} \vert$). It
follows that $\FPC \leq \FPR$ and $\FOC \leq \FOR$. Dawar et
al.\cite{Dawar09logicswith} show that these inclusions are strict.

\subsection{Group Theory}
Let $X$ be a set, we write $\sym_X$ to denote the symmetric group on the set
$X$. For $n \in \nats$ we write $\sym_n$ to abbreviate $\sym_{[n]}$. We assume
the reader is familiar with the basic theory of group actions.

Let $G$ be a group and let $H \leq G$. Let $X$ be a set on which a group action
of $G$ is defined. This group action defines a natural action on
$\mathcal{P}(X)$ (and hence $\mathcal{P}(\ldots(\mathcal{P}(X)))$) defined by
$\sigma \cdot S := \{\sigma x : x \in S\}$ for $\sigma \in G$, $S \subseteq X$.
Let $S \subseteq X$. We call $\stab_G(S) = \{\sigma in G : \forall x \in S \,
\sigma x = x\}$ the \emph{stabiliser} of $S$ with respect to $G$. We call
$\setstab_G(S) = \stab_G(\{S\}) = \{\sigma \in G : \sigma S = S\}$ the
\emph{set-wise stabiliser} of $S$ with respect to $G$. We call $\orb_G(S) :=
\{\sigma S : \sigma \in G\}$ the \emph{orbit} of $S$ with respect to $G$. If $S$
is a singleton we omit the set bracers, e.g.\ we write $\orb_G(x)$ for
$\orb_G(\{x\})$. If $G = \sym_n$ for some $n \in \nats$ we replace the subscript
with the number $n$, e.g.\ we write $\orb_n(x)$ for $\orb_{\sym_n}(x)$. If the
group $G$ is obvious from context we omit the subscript, e.g.\ we write
$\orb(x)$ for $\orb_G(x)$.

\subsection{Boolean Functions, Bases, and Circuits}
A (finite) \emph{Boolean function} is a function of the form $f: \{0,1\}^X
\rightarrow \{0,1\}$, for some finite index set $X$. We often take $X = [n]$,
and think of $f$ as taking $n$-length strings as inputs. We call a Boolean
function $f: \{0,1\}^X \rightarrow \{0,1\}$ \emph{symmetric} if for all $\sigma
\in \sym_X$ and $\vec{a} \in \{0,1\}^X$, $f (\vec{a} \sigma) = f(\vec{a})$. It
follows that a Boolean function is symmetric if, and only if, the output of the
function is entirely determined by the number of ones in the input. We say a
function $f : \{0,1\}^{*} \rightarrow \{0,1\}$ is \emph{symmetric} if its
restriction to inputs of size $n$ is symmetric for all $n$.

A \emph{Boolean basis} (or \emph{basis}) is a set of Boolean functions. We
always denote a basis by $\BB$. It is common in the literature to define a basis
using a finite sequence of functions $f_1, \ldots , f_k:\{0,1\}^* \rightarrow
\{0,1\}$. We write $\BB = \{f_1, \ldots, f_k\}$ to denote the basis consisting
of all Boolean functions $f^n_i : \{0,1\}^n \rightarrow \{0,1\}$ where $n \in
\nats$, $i \in [k]$, and $f^n_i$ is the restriction of $f_i$ to strings of
length $n$. We have $\AND, \OR, \NAND, \MAJ : \{0,1\}^* \rightarrow \{0,1\}$,
where $\AND, \OR, \NAND$ are the usual (unbounded fan-in) logical operators, amd
$\MAJ$ is the majority function. We write $\BS := \{\AND, \OR, \NOT\}$ to denote
the \emph{standard basis} and $\BS := \BS \cup \{\MAJ\}$ to denote the
\emph{majority basis}.

Let $\BB$ be a basis. A \emph{Boolean circuit} $C$ defined over $\BB$ is a
labelled directed acyclic graph (DAG) with designated input gates $x_1, \ldots,
x_n$, each with in-degree $0$, internal gates labelled by elements of $\BB$, and
a single internal gate with out-degree $0$ designated as the output gate. We say
a circuit with $n$ input gates has \emph{order $n$}. Let $C$ be a Boolean
circuit of order $n$ and let $\vec{a} := (a_1, \ldots, a_n) \in \{0,1\}^n$. The
evaluation of $C$ for the input $\vec{a}$ is denoted by $C[\vec{a}]$ and is
defined by induction on the structure of the circuit. We denote the evaluation
of a gate $g$ in $C$ for the input $\vec{a}$ by $C[\vec{a}](g)$. If $g$ is an
input gate, then $g = x_i$ and $C[\vec{a}](g) = a_i$. If $g$ is an internal gate
then $g$ is labelled by a symbol denoting a Boolean operation, and
$C[\vec{a}](g)$ is the result of applying that operation the string formed from
the evaluations of those gates input to $g$. If $g$ is the output gate, let
$C[\vec{a}] = C[\vec{a}](g)$.

The \emph{size} of a circuit $C$, denoted by $\vert C \vert$ is the number of
gates in the circuit. The \emph{depth} of a gate is the longest path from an
input gate to that gate, and the \emph{depth} if the circuit is the depth of the
output gate. The \emph{width} of a circuit is the maximum size of a set of gates
with the same depth.

If $C$ is a circuit of order $n$, then $C$ computes a Boolean function $f_C :
\{0,1\}^n \rightarrow \{0,1\}$ defined by $f_C(\vec{a}) = C[\vec{a}]$. Let
$(C_n)_{n \in \nats}$ be a family of circuits, where $C_n$ has order $n$. We say
$(C_n)_{n \in \nats}$ \emph{decides} a language $L : \{0,1\}^{*} \rightarrow
\{0,1\}$ if for all $\vec{a} \in \{0,1\}^{*}$, $C_{\vert \vec{a} \vert}
[\vec{a}] = L(\vec{a})$.


\begin{definition}
  Let $(C_n)_{n \in \mathbb{N}}$ be a family of Boolean circuits. We say that
  $(C_n)_{n \in \mathbb{N}}$ is \emph{$\PT$-uniform} if the mapping $n \mapsto
  C_n$ is polynomial-time computable in $n$.
\end{definition}

\subsection{Circuits for Structures}
Anderson and Dawar~\cite{AndersonD17} define a circuit that takes as input an
encoding of a finite relational structure. Let $\tau$ be a vocabulary, $\BB$ be
a basis of symmetric functions and $q, n \in \nats$. A \emph{circuit for
  structures} (or \emph{circuit}) $C$ is a DAG with a designated set of input
gates each labelled by a relation symbol $R \in \tau$ and a tuple $\vec{a} \in
[n]^{\arty(R)}$, internal gates each labelled by an element of $\BB$, and a
designated set of output gates, each labelled by a unique element of $[n]^q$.

Let $\mathcal{A}$ be a $\tau$-structure of size $n$ and fix a bijection $\gamma$
from the universe of $\mathcal{A}$ to $[n]$. This bijection defines an encoding
of $\mathcal{A}$ as a $\tau$-structure with universe $[n]$. This encoding
defines an assignment to the input gates of the circuit, assigning the gate
labelled by the relation symbol $R$ and tuple $\vec{a})$ to one if, and only if,
$\vec{a}$ is an element of the interpretation of $R$ in the encoding of
$\mathcal{A}$. The circuit may then be recursively evaluated, with the
evaluation of the output gates taken as the output of the circuit. We note that
an evaluation of the output gates (and hence the output of the circuit) is an
element of $\{0,1\}^{[n]^q}$. The circuit then associates with every
$\tau$-structure and bijection $\gamma$ a set of $q$-tuples in $\mathcal{A}$.

A circuit is called \emph{invariant} if its output does not depend on the choice
of $\gamma$. In this case the circuit defines a $q$-ary query of
$\tau$-structures. If $C$ is an invariant circuit with $q = 0$, and the circuit
has a single output gate and decides a property of $\tau$-structures. A circuit
is called \emph{symmetric} if every permutation on the universe $[n]$, each of
which induces a permutation on the input gates, extends to an automorphism of
the circuit. A symmetric circuit is necessarily invariant.

Anderson and Dawar use families of symmetric circuits to characterise $\FPC$.

\begin{thm}[Anderson and Dawar~\cite{AndersonD17}]
  For a relational vocabulary $\tau$, a property of finite $\tau$-structures is
  decided by a P-uniform family of symmetric circuits defined over the basis
  $\BS$ if, and only if, it is expressible in $\FPC[\tau]$.
\end{thm}

% \subsection{Basic Notation}

% Let $f : X \rightarrow Y$ be a function and $\sim$ be an equivalence relation
% on Y. We define $f/{\sim} : X \rightarrow Y/{\sim}$ by $f(x) = [f(x)]_{\sim}$.
% We call this the quotient of the function by $\sim$. For any two sets $X, Y$
% we let $X^{\underline{Y}}$ denote the set of all injective functions from $Y$
% to $X$.

% INCLUDE STUFF ON SUPPORTS SET-WISE SUPPORTS

\end{document}

