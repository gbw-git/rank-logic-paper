\documentclass[../paper.tex]{subfiles}

\begin{document}
In this section we present basic background on logics, first-order structures,
and symmetric circuits.

\subsection{Logic and Structures}
A \emph{relational vocabulary} (usually denoted by $\tau$) is a finite sequence
of relation symbols $(R_1, \ldots, R_k)$, such that for all $i \in [k]$, $R_i$
has arity of $r_i \in \nats$. All vocabularies discussed in this paper are
relational vocabularies. A \emph{many-sorted vocabulary} is a tuple of the form
$(R, S, \nu)$, where $R$ is a relational vocabulary, $S$ is a finite sequence of
\emph{sort} symbols $(s_1, \ldots s_p)$, and $\nu$ is a function that assigns to
each $R_i \in R$ a tuple $\nu(R_i) := (s^i_1, \ldots, s^i_{r_i})$, where for
each $j \in [r_i]$, $s^i_j \in S$. We call $\nu$ the \emph{\type function} and
$\nu(R_i)$ the \emph(type) of $R_i$.

Let $\tau$ be a many-sorted vocabulary. A \emph{$\tau$-structure} $\mathcal{A}$
is a tuple $(U , R^{\mathcal{A}}_1 , \ldots , R^{\mathcal{A}}_k )$, where $U$,
the \emph{universe} of $\mathcal{A}$, is a disjoint union of non-empty sets
$U_{s_j}$, one for each sort symbol $s_j \in S$. For each $i \in [k]$, where
$\nu(R_i) = (s^i_1 , \ldots , s^i_{r_i})$, $R^{\mathcal{A}}_i \subseteq
U_{s^i_1} \times \ldots \times U_{s^i_{r_i}}$. The elements of $U$ are called
the elements of $\mathcal{A}$. If $tau$ is a relational vocabulary, a
$\tau$-structure is a structure over $(\tau, S, \nu)$, where $S$ is a singleton
and $\nu$ is the only allowable type function. The size of a structure
$\mathcal{A}$, denoted by $\vert \mathcal{A} \vert$, is the cardinality of its
universe. The sorted-size of a structure is a vector $\vec{n} := (n_{s_1},
\ldots , n_{s_p}) \in \mathbb{N}^{S}$ such that $\vert U_{s} \vert = n_{s}$ for
all $s \in S$. If $a \in U_{s_j}$ for some sort $s_j \in S$, we say that $a$ has
sort $s_j$. We often use $A$ to denote the universe of a structure
$\mathcal{A}$. All structures in this paper are finite unless stated otherwise.

\begin{definition}
  Let $S = \{s_1, \ldots , s_p\}$ be a set of sort symbols and $A = A_{s_1}
  \uplus \ldots \uplus A_{s_p}$ and $B = B_{s_1}\uplus \ldots \uplus
  B_{s_p}$ be disjoint unions of non-empty sets. A \emph{sorted bijection}
  between $A$ and $B$ is a bijection $f: A \rightarrow B$ such that $f (A_s) =
  B_s$ for all $s \in S$.
\end{definition}

If $\tau = (R, S, \nu)$ is a single-sorted vocabulary then we abuse notation and
identify $\tau$ with $R$.

\subsection{First Order, Fixed-Point and Counting}
Let $\FO(\tau)$ denote \emph{first order logic} with respect to the vocabulary
$\tau$. \emph{Fixed-Point Logic} is the extension of $\FO(\tau)$ with an
inflationary fixed-point operator (see \cite{} for details). For logic
$\mathcal{L}$, a formula $\phi \in \mathcal{L}(\tau)$, $x$ a $k$-tuple of
variables, $\mathcal{A}$ a (finite) structure, and $a \in U^k$, we write the
formulas $phi (x)$ to denote that the free variables of $\phi$ are among and we
write $\mathcal{A} \models_{\mathcal{L}} \phi[a]$ iff the the formula $\phi$ is
true in $\mathcal{A}$ with respect to $\mathcal{L}$ with the variables $x$
assigned to $a$. The context often makes the logic obvious, and in that case we
usually omit the subscript.

We also recall the definition of \emph{fixed-point logic with counting} $\FPC$,
an extension $\FP$ with counting terms (see \cite{grohe2017descriptive} for
details). Variables and terms in $\FPC$ must be of one of two sorts, vertex or
numeric, where vertex variables are assigned to elements of the universe the
structure and numeric variables to the elements of $\mathbb{N}$. All atomic
formula as in $\FP$ are formulas of $\FPC$ (using only vertex sort variables).
Let $\phi(x)$ be a formula of $\FPC$ with free variables $x$, then $\# x \phi$
is a numeric, with the term's semantics being the number of elements satisfying
$\phi$. All numeric variables as well as the symbols 0 and 1 are numeric terms
with the expected semantics. If $\mu_1 and \mu_2$ are numeric terms, then $\mu_1
+ \mu_2$ and $\mu_1 * \mu_2$ are numeric terms with the expected semantics.
Moreover, if $\mu_1$ and $\mu_2$ are numeric terms, then $\mu_1 < \mu_2$ and
$\mu_1 = \mu_2$ are formulas with the expected semantics. If $\vec{x}$ is a
$k$-tuple of numeric variables and $\vec{\mu}$ is a $k$-tuple of closed numeric
term (i.e. ones without any free variables) then $\exists \vec{x} \leq \vec{\mu}
\phi(\vec{x})$ and $\forall \vec{x} \leq \vec{mu} \phi(\vec{x})$ are formulas.
In the existential case, for a $\tau$-structure $\mathcal{A}$, we say
$\mathcal{A} \models \exists \vec{x} \leq \vec{\mu} \phi(\vec{x})$ iff there
exists $\vec{n} \in [\mu^{\mathcal{A}}_1] \times \ldots \times
[\mu^{\mathcal{A}}_2]$, such that $phi^{\mathcal{A}}$ holds for the assignment
$\vec{x} \mapsto \vec{n}$. The semantics of the universal case follows
similarly. The application of the fixed-point operator (but now with the
relation variables possibly being of mixed type), and Boolean connectives
similarly define formulas as in $\FP$. The semantics of these formulas is also
similarly defined.

% Let $\FO(\tau)$ denote \emph{first order logic} with respect to the vocabulary
% $\tau$. \emph{Fixed-Point Logic} is the extension of $\FO(\tau)$ with an
% inflationary fixed-point operator (see \cite{} for details). For logic
% $\mathcal{L}$, a formula $\phi \in \mathcal{L}(\tau)$, $x$ a $k$-length tuple
% of variables, $\mathcal{A}$ a (finite) structure, and $a \in U^k$, we have
% that $\phi (x)$ denotes that the variables in $x$ are free in $\phi$ and we
% write $\mathcal{A} \models_{\mathcal{L}} \phi[a]$ iff the the formula $\phi$
% is true in $\mathcal{A}$ with respect to $\mathcal{L}$ with the variables $x$
% assigned to $a$. The context usually makes the logic obvious, and in that case
% we usually drop the subscript.

% We also recall the definition of \emph{fixed-point logic with counting}
% $\FPC$, an extension $\FP$ with counting quantifiers. For a structure
% $\mathcal{A}$, Formulas of this logic are evaluated over a structure
% $\mathcal{A}$ by extending that structure to $\mathcal{A}^\leq$, a two sorted
% extension of $\mathcal{A}$ by an ordered number sort $\{1, \ldots, \vert
% U\vert \}$. Variables are taken to be either point variables, ranging over $U$
% or number variables ranging over $[\vert U \vert]$. The counting quantifiers
% are of the form $\#_x$, where $x$ is either a point or number variable and for
% $\phi(x) \in \FPC(\tau)$, $\#_x \phi(x)$ is a term denoting the element of the
% number sort corresponding to $\vert \{a \in \mathcal{A} | \mathcal{A} \models
% \phi[a]\}\vert$. Note additionally that $k$-tuples of number variables can be
% used to encode numbers in $\FPC$ of size up to $\vert U\vert^k$. For more
% details on $\FPC$ please see \cite{}.

\subsection{Rank Logic}
In this paper we use the definition of $\FPR$ given by Grad\"{e}l and Pakusa
\cite{Gradel2015}, an strictly more expressive extension of the rank logic
introduced by Dawar et. al. cite{Dawar09logicswith}. We briefly define this
logic here. For more detail please see \cite{Gradel2015}. $\FPR$ is an extension
of $\FPC$ with rank operators, which we now define.

Let $\phi (\vec{x}, \vec {\nu} \leq \vec{t}, \vec{y}\vec{\mu} \leq \vec{s})$ be
a numeric term, where $\vec{x}$ and $\vec{y}$ are tuples of variables ranging
over the universe of the structure, $\vec{\nu}$ and $\vec{\mu}$ are tuples of
numeric variables , and $\vec{s}$ and $\vec{t}$ are tuples of closed numeric
terms that bound the range of $\vec{\nu}$ and $\vec{\mu}$ respectively. Then let
$\mathcal{A}$ be a structure and let $\mathbb{N}^{\leq\vec{t}} := \{\vec{n} \in
\mathbb{N}^{\vert \vec{v} \vert}: n_i \leq t^{\mathcal{A}}_i\}$, and let
$\mathbb{N}^{\leq \vec{s}}$ be defined similarly. Let $I:= A^{\vert \vec{x}
  \vert} \times \mathbb{N}^{\leq \vec{t}}$ and $J := A^{\vert \vec{y} \vert}
\times \mathbb{N}^{\leq \vec{s}}$ Then let $M^{\mathcal{A}}_\phi : I \times J
\rightarrow \mathbb{N}$ be defined by $M^{\mathcal{A}}_\phi (\vec{a}\vec{n},
\vec{b}\vec{m}):= \phi^{\mathcal{A}}(\vec{a}\vec{n},\vec{b}\vec{m})$.

The rank operator is then defined as $[rk (\vec{x}, \vec {\nu} \leq \vec{t},
\vec{y}\vec{\mu} \leq \vec{s}, \pi \leq r). \phi]$, where $\pi$ is a numeric
variable and $r$ is a closed numeric term that bounds $\pi$. The value of the
rank term for a structure $\mathcal{A}$ and an assignment $\pi \mapsto p$, where
$p$ is prime, is given by the rank  of $M^{\mathcal{A}}_\phi$ mod $p$, where the
matrix is understood to have entries in $\mathbb{F}_p$.

\subsection{Group Theory}
--Include details on how to define a groupa action. Particularly a group action
as a homomorphism from $G$ to $\sym_X$

Let $X$ be a set and $G$ be a group acting on $X$. We let the stabiliser of $S
\subseteq X$ for $G$ be defined as $\stab_G(S) := \{ \sigma \in G : \forall x
\in S \, \sigma \cdot x = x \}$ and the orbit of $S$ for $G$ to be defined as
$\orb_G (S) := \{ T \subseteq X : \exists \sigma \in G \, \sigma \cdot S = T
\}$. If $S = \{x\}$, for some $x \in X$, we write $\stab_G (x)$ and $\orb_G(x)$
in order to abbreviate $\stab_G(\{x\})$ and $\orb_G(\{x\})$, respectively. In
the event that the choice of $G$ is obvious from context we omit the subscript.

We are often interested in the case where $G$ is the symmetric group $\sym_n$ on
$n$ elements. In this case we write $\stab_n(x)$ and $\orb_n(x)$ in order to
abbreviate $\stab_{\sym_n}(x)$ and $\orb_{\sym_n}(x)$, respectively.

We are also often interested in the case where $G$ is the stabiliser group for
set of elements, or single element, from $X$. For $S, T \subseteq X$, we write
$\stab_{S}(T)$ and $\orb_{S}(T)$ in order to abbreviate $\stab_{\stab(S)}(T)$
and $\orb_{\stab(S)}(T)$.

\subsection{Circuits}
A \emph{Boolean circuit} $C$ defined over some basis $\mathbb{B}$ is a directed
acyclic graph (DAG) with designated input gates $x_1, \ldots, x_n$, each with
in-degree $0$, internal gates labelled by elements of $\mathbb{B}$, and a single
designated output gate with out-degree 0. A Boolean basis consists either of a
set of Boolean functions or a set of symbols which implicitly define a family of
Boolean functions.

Let $C$ be a circuit with input gates $x_1, \ldots, x_n$ and output gate $g_o$.
The evaluation of $C$ for an input $\vec{x}\in \{0,1\}^n$ proceeds inductively
from the assignment of $\vec{x}$ to the input gates. with the internal and
output gates evaluated in accordance with the element of the basis labelling
that gate. We denote the evaluation of a gate $g$ in by $C[\vec{x}](g)$, with
the out of the circuit then given by $C[\vec{x}](g_o)$. In this way $C$ computes
a Boolean function of the form $f:\{0,1\}^n \rightarrow \{0,1\}$. It is easy to
see that for any complete Basis and any Boolean function there is a circuit over
that basis that computes the function.

% It is worth noting that the input to a gate is an unstructured set, and as
% such it is implicitly required that the label associated with that gate define
% a \emph{symmetric function} (a function whose output depends on the number of
% 1's in its input).

There are many natural properties of a circuit.The \emph{size} of a circuit $C$,
denoted by $\vert C \vert$ is the number of gates in the circuit. The
\emph{depth} of a circuit is the length of the longest path from an input gate
to the output gate. The \emph{width} of a circuit is the maximum size of a set
of gates all the same distance from the input gates.

We say a family of circuits $(C_n)_{\mathbb{N}}$ \emph{decides} a language $L
\{0,1\}^{*} \rightarrow \{0,1\}$ if $C_n$ computes the same function as $L$
restricted to inputs of size $n$. It is common to ask about the class of
languages that a can be decided by circuit families with certain restrictions.
The following restriction is of importance in this paper.

Two Boolean bases often considered in this paper are $\mathbb{B}_{\std} = \{
\neg , \wedge , \lor \}$ and $\mathbb{B}_\maj = \{ \maj \} \cup \mathbb{B}$,
where $\neg, \wedge and \lor$ are the usual logical connectives and $\maj$ is
the majority gate, where for a Boolean sequence $\vec{x}$, $\maj(\vec{x}) = 1$
iff the number of $1$'s in $\vec{x}$ is greater than or equal to
$\frac{\vert\vec{x}\vert}{2}$.


\begin{definition}
  Let $(C_n)_{n \in \mathbb{N}}$ be a family of Boolean circuits. We say that
  $(C_n)_{n \in \mathbb{N}}$ is \emph{$\PT$-uniform} if the mapping $n \mapsto
  C_n$ is polynomial-time computable in $n$.
\end{definition}

Let $C$ be a circuit with the sets of gates $G$ and edge relation $W$. Let $g
\in G$, we denote the \emph{children} or \emph{input gates} of $g$ by the set
$H_g := \{h \in G : W(h,g)\}$.

\subsection{Symmetric Circuits}
A circuit for structures, as given by Anderson and Dawar, is defined for a fixed
relational vocabulary $\tau$, a Boolean basis of $\mathbb{B}$ \emph{symmetric
  functions} (function's whose output depends only on the number of 1's in the
input) and some fixed universe of size $n$. The input gates are taken to be the
formal graph of the relational symbols over the universe $[n]$, i.e. gates
labelled by $R(\vec{x})$ where $\vec{x} \in [n]^{\arty(R)}$. The input to the
circuit is a finite $\tau$-structure $\mathcal{A}$ of cardinality $n$.

Evaluation of the circuit proceeds by first choosing a bijection $\gamma$ from
the universe of $\mathcal{A}$ to $[n]$ and then, for each input gate, using the
interpretation of the relevant relational symbol in $\mathcal{A}$ and the
bijection $\gamma$ to evaluate the gate. Recursing through the circuit and
evaluating the internal gates using the given semantics for the symbols in the
Boolean basis allows each gate in the circuit to be evaluated.

A circuit is \emph{invariant} if its output does not depend on the choice of
$\gamma$. A \emph{symmetric} circuit is a circuit such that every permutation on
the universe $[n]$, each of which induces a permutation on the input gates,
extends to an automorphism of the circuit. A symmetric circuit is necessarily
invariant.

It is important to note that the word `symmetric' is used in two different ways
here. When applied to a circuit the above definition is used, but when applied
to a Boolean function it denotes a function whose output depends only on the
number of 1's in its input.

The main result of~\cite{AndersonD17} is the following characterisation theorem.

\begin{thm}
  For any relational vocabulary $\tau$, a property of finite $\tau$-structures
  is decided by a P-uniform family of symmetric circuits with majority gates if,
  and only if, it is expressible in FPC.
\end{thm}

\subsection{Basic Notation}

Let $f : X \rightarrow Y$ be a function and $\sim$ be an equivalence relation on
Y. We define $f/{\sim} : X \rightarrow Y/{\sim}$ by $f(x) = [f(x)]_{\sim}$. We
call this the quotient of the function by $\sim$. For any two sets $X, Y$ we let
$X^{\underline{Y}}$ denote the set of all injective functions from $Y$ to $X$.

INCLUDE STUFF ON SUPPORTS SET-WISE SUPPORTS

\end{document}

