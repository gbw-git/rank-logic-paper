\documentclass[../paper.tex]{subfiles}

\begin{document}
In this section we present basic background on logics, first-order structures,
and symmetric circuits.

\subsection{Logic and Structures}
A \emph{relational vocabulary} (otherwise just called a \emph{vocabulary} or
\emph{signature}) is a finite sequence of relation symbols $(R_1, \ldots, R_k)$,
each of which has a fixed \emph{arity}, denoted by $\arty(R_i) \in \nats$. We
often let $r_i \in \nats$ denote the arity of the relation symbol $R_i$. A
\emph{many-sorted vocabulary} is a tuple of the form $(R, S, \nu)$, where $R$ is
a relational vocabulary, $S$ is a finite sequence of \emph{sort} symbols $(s_1,
\ldots s_p)$, and $\nu$ is a function that assigns to each $R_i \in R$ a tuple
$\nu(R_i) := (s^i_1, \ldots, s^i_{r_i})$, where for each $j \in [r_i]$, $s^i_j
\in S$. We call $\nu$ the \emph{\type function} and $\nu(R_i)$ the \emph{type}
of $R_i$. We notice that a relational vocabulary can be thought of as a
single-sorted vocabulary, i.e. a many-sorted vocabulary where the set of sorts
is a singleton. As such, if $\tau := (R, S, \nu)$ is single-sorted we identify
$\tau$ with $R$.

Let $\tau := (R, S, \nu)$ be a many-sorted vocabulary, where $R = (R_1, \ldots,
R_k)$ and $S = (s_1, \ldots , s_p)$ A \emph{$\tau$-structure} $\mathcal{A}$ is a
tuple $(U , R^{\mathcal{A}}_1 , \ldots , R^{\mathcal{A}}_k)$ where $U =
\uplus_{s \in S } U_{s}$ a disjoint union of non-empty sets, and is called the
\emph{universe} of $\mathcal{A}$, and for all $i \in [k]$, $R^{\mathcal{A}}_i
\subseteq R^{\mathcal{A}}_i \subseteq U_{s^i_1} \times \ldots \times
U_{s^i_{r_i}}$, where $(s^i_1 , \ldots , s^i_{r_i}) = \nu (R_i)$. The size of
$\mathcal{A}$, denoted by $\vert \mathcal{A} \vert$ is the cardinality of its
universe. The \emph{sorted-size} of $\mathcal{A}$ is a vector $\vec{n} :=
(n_{s_1}, \ldots , n_{s_p}) \in \mathbb{N}^{p}$ such that $\vert U_{s_i} \vert =
n_{s_i}$ for all $i \in [p]$. We say $a \in U$ has \emph{sort} $s \in S$ if $a
\in U_s$.

All structures in this paper are assumed to have finite size unless stated
otherwise. For more details on these basic definitions please
see~\cite{Gradel:2005}, \cite{immerman1999descriptive},
or~\cite{grohe2017descriptive}.

% \begin{definition}
%   Let $S = \{s_1, \ldots , s_p\}$ be a set of sort symbols and $A = A_{s_1}
%   \uplus \ldots \uplus A_{s_p}$ and $B = B_{s_1}\uplus \ldots \uplus B_{s_p}$
%   be disjoint unions of non-empty sets. A \emph{sorted bijection} between $A$
%   and $B$ is a bijection $f: A \rightarrow B$ such that $f (A_s) = B_s$ for
%   all $s \in S$.
% \end{definition}

\subsection{First Order and Fixed-Point Logic}
Let $\FO[\tau]$ denote \emph{first-order logic} with respect to the vocabulary
$\tau$. A formula in $\FO[\tau]$ is formed from atomic formulas, each formed
using variables from some countable sequence of (first-order) variable symbols
$(x, y, \ldots)$, the relation symbols in $\tau$, and the equality symbol $=$,
and then closing the set of atomic formulas under the Boolean connectives,
negation, and universal and existential quantification (i.e. $\land$, $\lor$,
$\neg$, $\forall$, and $\exists$).

Let $\FP[\tau]$ denote \emph{fixed-point logic} with respect to $\tau$. The
logic $\FP$ is an extension of $\FO$ by an inflationary fixed-point operator. In
other words, $\FP[\tau]$ includes all formulas in $\FO[\tau]$, and is closed
under applications of the fixed-pint operator, i.e. if $\phi \in \FP[\tau]$, $V$
is a second-order variable, and $\vec{x}$ is a tuple of variables of length
$\arty(V)$, then $[\ifp_{V, \vec{x}}. \phi] \in FP[\tau]$. We assume the
standard syntax and semantics for $\FO[\tau]$ and $\FP[\tau]$. For more detail
on the syntax and semantics of these logics please
see~\cite{grohe2017descriptive} and~\cite{GradelP15a}.

Let $L$ be a logic and $\tau$ a vocabulary. For $\phi \in L[\tau]$ and a
sequence of variables $\vec{x}$, we write $\phi(\vec{x})$ if the free variables
in $\phi$ are among those in $\vec{x}$. Suppose $\vec{x} = (x_1, \ldots , x_k)$,
and let $\mathcal{A}$ be a $\tau$-structure with universe $U$. We call a
function $\alpha : {x_1, \ldots, x_k} \rightarrow U$ an \emph{assignment}. If
$\vec{a}\in U^k$, we write $\alpha^{\vec{x}}_{\vec{a}}$ to denote the assignment
that maps the variable $x_i$ to $a_i$ for $i \in [k]$. We say that two
assignments $\alpha$ and $\beta$ are \emph{compatible} if $\alpha$ and $\beta$
agree on the intersection of their domain. We write $\mathcal{A} \models_L
\phi[\alpha]$ if, and only if, $\phi$ is true in $\mathcal{A}$ with respect to
$L$ where for all $i \in [k]$ each variable $x_i$ is assigned to $\alpha(x_i)$.
If the sequence of variables $\vec{x}$ that we are assigning to $\vec{a}$ is
obvious from context we write $\mathcal{A} \models_L \phi[\vec{a}]$ instead. If
the logic is obvious from context we omit the subscript and just write
$\mathcal{A} \models \phi[\alpha]$.

\subsection{Fixed-Point with Counting}
Let $\FPC[\tau]$ denote \emph{fixed-point logic with counting} with respect to
the vocabulary $\tau$. The logic $\FPC$ is an extension of $\FP$ that includes a
counting operator that allows us to define the size of a definable set, as well
as other mechanisms for reasoning about quantities. We now briefly describe the
structure of a formula of $\FPC[\tau]$. Every first-order variable is assigned a
type, and so each variable is either a \emph{vertex} or \emph{number} variable.
Vertex variables are assigned to elements of the universe of the structure, and
play a similar role to the variables that appear in formulas in $\FO[\tau]$ or
$\FP[\tau]$. Number variables are assigned to elements of $\nats$. We usually
use Latin letters (e.g. $x, y, z, \ldots$) to denote vertex variables and Greek
letters (e.g. $\mu, \nu, \epsilon, \ldots$) to denote number variables. All
atomic formulas in $\FP[\tau]$ are atomic formulas in $\FPC[\tau]$. We say that
$t$ is a \emph{number term} if $t$ is a number variable or if $t$ is an application
of the \emph{counting operator}, i.e. $t = \#x. \phi(x)$, where $\phi \in
\FPC[\tau]$ and $x$ is a vertex variable free in $\phi$. If $t_1$ and $t_2$ are
number terms then $t_1 \leq t_2$ and $t_1 = t_2$ are atomic formulas.
$\FPC[\tau]$ includes all formulas formed by closing the set of atomic formulas
under the usual Boolean connectives, the first-order quantifiers (for vertex
variables), and the inflationary fixed-point operator. The second-order
variables, such as those that appear in an application of the fixed-point
operator, are allowed to have mixed-type, i.e. we associate with a second-order
variable $R$ a type $(k, l) \in \nats^2$ intended to mean that $R$ is assigned
to a subset of $U^k \times \nats^l$.

We also allow quantification over the number sort, but in order to ensure that
the data complexity of $\FPC$ is in polynomial time we only allow bounded
quantification. In other words, if $Q \in \{\forall, \exists \}$, $\phi \in
\FPC[\tau]$, $\mu$ is a number variable free in $\phi$, and $t$ is a number
term, then $Q \mu \leq t . \phi$ is a formula of $\FPC[\tau]$. Similarly, if
$\phi$ is a number term then $\# \mu \leq t .\phi$ is a number term in
$\FPC[\tau]$.

If $\mathcal{A}$ is a $\tau$-structure with universe $U$, the counting term
$\#x. \phi(x)$ evaluates to $\vert \{a \in U : \mathcal{A} \models \phi[a]\}
\vert$ in $\mathcal{A}$, assuming a fixed assignment to the other free variables
in $\phi$. The intended semantics for bounded quantification is as for ordinary
quantification but with the additional restriction that the assignment to the
number variable be bounded by the evaluation of the bounding number term. The
semantics of the `$=$' and `$\leq$' symbols are defined for number terms in the
obvious manner. For a more detailed description of the syntax and semantics of
$\FPC$ please see~\cite{grohe2017descriptive} or the introduction
of~\cite{GradelP15a}.

It is possible to define the usual arithmetic operations on number terms using
the constructs thus far introduced. We do not show how to explicitly define
these operations in $\FPC$, but we do make use of them in this paper. Please
see~\cite{grohe2017descriptive} for the explicit constructions.

Let $\FOC[\tau]$ denote \emph{first-order logic with counting} with respect to
the vocabulary $\tau$, and let $\FOC[\tau]$ be set of all formulas in
$\FPC[\tau]$ that do not contain an application of the fixed-point operator.


% We also recall the definition of \emph{fixed-point logic with counting}
% $\FPC$, an extension $\FP$ with counting terms (see
% \cite{grohe2017descriptive} for details). Variables and terms in $\FPC$ must
% be of one of two sorts, vertex or numeric, where vertex variables are assigned
% to elements of the universe the structure and numeric variables to the
% elements of $\mathbb{N}$. All atomic formula as in $\FP$ are formulas of
% $\FPC$ (using only vertex sort variables). Let $\phi(x)$ be a formula of
% $\FPC$ with free variables $x$, then $\# x \phi$ is a numeric, with the term's
% semantics being the number of elements satisfying $\phi$. All numeric
% variables as well as the symbols 0 and 1 are numeric terms with the expected
% semantics. If $\mu_1 and \mu_2$ are numeric terms, then $\mu_1 + \mu_2$ and
% $\mu_1 * \mu_2$ are numeric terms with the expected semantics. Moreover, if
% $\mu_1$ and $\mu_2$ are numeric terms, then $\mu_1 < \mu_2$ and $\mu_1 =
% \mu_2$ are formulas with the expected semantics. If $\vec{x}$ is a $k$-tuple
% of numeric variables and $\vec{\mu}$ is a $k$-tuple of closed numeric term
% (i.e. ones without any free variables) then $\exists \vec{x} \leq \vec{\mu}
% \phi(\vec{x})$ and $\forall \vec{x} \leq \vec{mu} \phi(\vec{x})$ are formulas.
% In the existential case, for a $\tau$-structure $\mathcal{A}$, we say
% $\mathcal{A} \models \exists \vec{x} \leq \vec{\mu} \phi(\vec{x})$ iff there
% exists $\vec{n} \in [\mu^{\mathcal{A}}_1] \times \ldots \times
% [\mu^{\mathcal{A}}_2]$, such that $phi^{\mathcal{A}}$ holds for the assignment
% $\vec{x} \mapsto \vec{n}$. The semantics of the universal case follows
% similarly. The application of the fixed-point operator (but now with the
% relation variables possibly being of mixed type), and Boolean connectives
% similarly define formulas as in $\FP$. The semantics of these formulas is also
% similarly defined.

% Let $\FO(\tau)$ denote \emph{first order logic} with respect to the vocabulary
% $\tau$. \emph{Fixed-Point Logic} is the extension of $\FO(\tau)$ with an
% inflationary fixed-point operator (see \cite{} for details). For logic
% $\mathcal{L}$, a formula $\phi \in \mathcal{L}(\tau)$, $x$ a $k$-length tuple
% of variables, $\mathcal{A}$ a (finite) structure, and $a \in U^k$, we have
% that $\phi (x)$ denotes that the variables in $x$ are free in $\phi$ and we
% write $\mathcal{A} \models_{\mathcal{L}} \phi[a]$ iff the the formula $\phi$
% is true in $\mathcal{A}$ with respect to $\mathcal{L}$ with the variables $x$
% assigned to $a$. The context usually makes the logic obvious, and in that case
% we usually drop the subscript.

% We also recall the definition of \emph{fixed-point logic with counting}
% $\FPC$, an extension $\FP$ with counting quantifiers. For a structure
% $\mathcal{A}$, Formulas of this logic are evaluated over a structure
% $\mathcal{A}$ by extending that structure to $\mathcal{A}^\leq$, a two sorted
% extension of $\mathcal{A}$ by an ordered number sort $\{1, \ldots, \vert
% U\vert \}$. Variables are taken to be either point variables, ranging over $U$
% or number variables ranging over $[\vert U \vert]$. The counting quantifiers
% are of the form $\#_x$, where $x$ is either a point or number variable and for
% $\phi(x) \in \FPC(\tau)$, $\#_x \phi(x)$ is a term denoting the element of the
% number sort corresponding to $\vert \{a \in \mathcal{A} | \mathcal{A} \models
% \phi[a]\}\vert$. Note additionally that $k$-tuples of number variables can be
% used to encode numbers in $\FPC$ of size up to $\vert U\vert^k$. For more
% details on $\FPC$ please see \cite{}.

\subsection{Rank Logic}
Let $\FPR[\tau]$ denote \emph{fixed-point logic with rank} (or just \emph{rank
  logic}) with respect to the vocabulary $\tau$. The logic $\FPR$ is an
extension of $\FP$ to include an operator that denotes the rank of a definable
matrix over a finite field, as well as other mechanisms for reasoning about
quantity. In this way we can think of the structure of a formula of $\FPR$ as
being similar to that of $\FPC$, with rank operators replacing counting
operators.

We now briefly describe the structure of a formula in $\FPR[\tau]$. For more
detail on the syntax and semantics of $\FPR$ please see~\cite{GradelP15a}
and~\cite{Dawar09logicswith}. As in the case of $\FPC$, each variable is either
a number or vertex variable, with vertex variables assigned to elements of the
universe and number variables assigned to natural numbers. All atomic formulas
in $\FP[\tau]$ are atomic formulas in $\FPR[\tau]$. We say that $t$ is a
\emph{number term} if $t$ is a number variable or if $t$ is an application of
the \emph{rank operator}, i.e. $t = [\rank (\vec{x}, \vec {\nu} \leq \vec{t},
\vec{y}\vec{\mu} \leq \vec{s}, \pi \leq \eta). \phi]$, where $\phi$ is a number
term, $\vec{t}$ and $\vec{s}$ are tuples of number terms bounding the sequences
of number variables $\vec{\mu}$ and $\vec{\nu}$, and $\eta$ is a number term
bounding the number variable $\pi$. If $t_1$ and $t_2$ are number terms then
$t_1 \leq t_2$ and $t_1 = t_2$ are atomic formulas. The formulas of $\FPR$ are
then formed by closing the set of atomic formulas under the usual Boolean
connectives, the first-order quantifiers, and the fixed-point operator. As in
the case of $\FPC$, the second-order variables may have mixed-type and we only
allow bounded quantification for number variables. We note additionally that
formulas may be treated as number terms that evaluate to zero or one.

We now briefly explain the intended semantics of the rank operator. Let $\phi
(\vec{x} \vec {\nu}, \vec{y}\vec{\mu})$ be a number term, where $\vec{x}$ and
$\vec{y}$ are tuples of vertex variables, $\vec{\nu}$ and $\vec{\mu}$ are tuples
of number variables. Let $\vec{s}$ and $\vec{t}$ be tuples of number terms such
that $\vert \vec{s} \vert = \vert \vec{\nu} \vert$ $\vert \vec{t}\vert = \vert
\vec{mu} \vert$. Let $\pi$ be a number variable and $r$ be a number term. Let
$\mathcal{A}$ be a structure with universe $A$, and suppose we have a given
assignment to the those free variables appearing in $r$ or in the terms in
$\vec{t}$ and $\vec{s}$. Let $\mathbb{N}^{\leq\vec{t}} := \{\vec{n} \in
\mathbb{N}^{\vert \vec{\nu} \vert}: n_i \leq t^{\mathcal{A}}_i\}$, and let
$\mathbb{N}^{\leq \vec{s}}$ be defined similarly. Let $I:= A^{\vert \vec{x}
  \vert} \times \mathbb{N}^{\leq \vec{t}}$ and $J := A^{\vert \vec{y} \vert}
\times \mathbb{N}^{\leq \vec{s}}$. Let $M^{\mathcal{A}}_\phi : I \times J
\rightarrow \mathbb{N}$ be defined by $M^{\mathcal{A}}_\phi (\vec{a}\vec{n},
\vec{b}\vec{m}):= \phi^{\mathcal{A}}(\vec{a}\vec{n},\vec{b}\vec{m})$. For the
structure $\mathcal{A}$ and assignment $\pi \mapsto p$, where $p \leq
r^{\mathcal{A}}$, if $p$ is prime then the rank term $[\rank (\vec{x}, \vec
{\nu} \leq \vec{t}, \vec{y}\vec{\mu} \leq \vec{s}, \pi \leq r). \phi]$ is the
rank of $M^{\mathcal{A}}_\phi$ mod $p$, where the entries in $M^{\mathcal{A}}$
are interpreted as elements of $\mathbb{F}_p$. If $p$ is not prime then the rank
term evaluates to zero.

We should note that there are two distinct rank logics defined in the
literature. The first was introduced by Dawar et al.~\cite{Dawar09logicswith},
and differs from the rank logic defined above in that it includes a sequence of
rank operator, rather than a single uniform operator, one for each prime $p$
expressing the rank over the field $\ff_p$. It was shown by Grad{\"e}l and
Pakusa~\cite{GradelP15a} that this non-uniform rank logic does not capture
$\PT$. After proving this result, Grad{\"e}l and Pakusa introduced the rank
logic with a uniform rank operator that we have presented above, and showed that
this rank logic is strictly more expressive then the rank logic introduced by
Dawar et al. This uniform rank logic remains a candidate for capturing
polynomial-time. For this reason when we refer to rank logic in this paper we
are always referring to the rank logic introduced by Grad{\"e}l and
Pakusa~\cite{GradelP15a}.

We may also define an extension of first order logic by rank operators. Let
$\FOR[\tau]$ denote \emph{first-order with rank} with respect to the vocabulary
$\tau$, and let $\FOR[\tau]$ be the set of formulas in $\FPR[\tau]$ that do not
contain an application of the fixed-point operator.

Let $\FPrk[\tau]$ denote \emph{fixed-point logic with rank quantifiers} with
respect to the vocabulary $\tau$. This logic, first introduced by Dawar et
al~\cite{Dawar09logicswith}, is similar to $\FPR$, except that it extends $\FP$
by a set of \emph{rank quantifiers}, rather than a rank operator. We define for each prime
$p$, and natural number $r$, a rank quantifier $\rank^r_p$, such that
$\rank^r_p\vec{x}\vec{y}. \phi$ is interpreted as $[\rank (\vec{x}, \vec{y},
\pi) . \phi] \leq t_r$, where $\pi$ is assigned to $p$ and $t_r$ is a number
term that evaluates to $r$. The arity of this quantifier is $\vert \vec{x} \vert
+ \vert \vec{y} \vert$. Let $\mathcal{R}$ be the set of all such quantifiers and
$\FPrk[\tau]$ be the closure of $\FP[\tau]$ under $\mathcal{R}$.

We may also define an extension of first-order logic with rank quantifiers. Let
$\FOrk[\tau]$ denote \emph{first-order with rank quantifiers} with respect to
the vocabulary $\tau$, and let $\FOrk[\tau]$ be the set of all formulas in
$\FPR[\tau]$ that do not contain an application of the fixed-point operator.

For more detail on the syntax and semantics of logics with rank quantifiers please
see~\cite{Dawar09logicswith}. It is easy to see that $\FPrk \leq \FPR$ and
$\FOrk \leq \FOR$. Moreover, we note that $[\rank(\vec{x}, \vec{y}). (\vec{x} =
\vec{y}, \pi \leq r) \land \phi(\vec{x})]$ evaluates, for a structure
$\mathcal{A}$ and any assignment $\pi \mapsto p$, to the number of assignments
to $\vec{x}$ for which $\phi$ holds (i.e.\ $\vert \{\vec{a} \in U^{\vert \vec{x}
  \vert} : \mathcal{A} \models \phi[\vec{a}]\} \vert$). It follows that $\FPC
\leq \FPR$ and $\FOC \leq \FOR$. Dawar et al.\cite{Dawar09logicswith} show that
all of these inclusions are strict.

\subsection{Group Theory}
Let $X$ be a set, we write $\sym_X$ to denote the symmetric group on the set
$X$. For $n \in \nats$ we write $\sym_n$ for $\sym_{[n]}$.

Let $G$ be a group and let $H \leq G$. Let $X$ be a set on which a group action
of $G$ is defined. This group action can be lifted to a group action on
$\mathcal{P}(X)$ (and hence $\mathcal{P}(\ldots(\mathcal{P}(X))\ldots)$) defined
by $\sigma \cdot S := \{\sigma x : x \in S\}$ for $\sigma \in G$, $S \subseteq
X$. Let $S \subseteq X$. We let $\stab_G(S) := \{\sigma \in G : \forall x \in S
\,, \sigma x = x\}$ be the \emph{stabiliser} of $S$ with respect to $G$. We let
$\setstab_G(S) := \stab_G(\{S\}) = \{\sigma \in G : \sigma S = S\}$ be the
\emph{set-wise stabiliser} of $S$ with respect to $G$. We let $\orb_G(S) :=
\{\sigma S : \sigma \in G\}$ be the \emph{orbit} of $S$ with respect to $G$. If
$S$ is a singleton we omit the set bracers, e.g.\ we write $\orb_G(x)$ for
$\orb_G(\{x\})$. If $G = \sym_n$ for some $n \in \nats$ we replace the subscript
with the number $n$, e.g.\ we write $\orb_n(x)$ for $\orb_{\sym_n}(x)$, and if
the group $G$ is obvious from context we omit the subscript entirely.

\subsection{Boolean Functions, Bases, and Circuits}
A (finite) \emph{Boolean function} is a function of the form $f: \{0,1\}^X
\rightarrow \{0,1\}$, for some finite index set $X$. We call a Boolean function
$f: \{0,1\}^X \rightarrow \{0,1\}$ \emph{symmetric} if for all $\sigma \in
\sym_X$ and $\vec{a} \in \{0,1\}^X$, $f (\vec{a} \sigma) = f(\vec{a})$. It
follows that a Boolean function is symmetric if, and only if, the output of the
function is entirely determined by the number of ones in the input. We say a
Boolean function $f : \{0,1\}^{*} \rightarrow \{0,1\}$ is \emph{symmetric} if
its restriction to inputs of size $n$ is symmetric for all $n$.

We depart slight from the conventional notion of a basis in this paper. We
define a \emph{Boolean basis} (or \emph{basis}) to be a set of finite Boolean
functions. We always denote a basis by $\BB$. It is common in the literature to
define a basis as a finite sequence of functions $f_1, \ldots , f_k:\{0,1\}^*
\rightarrow \{0,1\}$. In this paper we instead think of this sequence as
\emph{generating} a basis, and write $\BB = \langle f_1, \ldots, f_k \rangle$ to
denote the basis generated by these functions, i.e.\ $\BB$ consists of all the
Boolean functions $f_i[n] : \{0,1\}^n \rightarrow \{0,1\}$ where $n \in \nats$,
$i \in [k]$, and $f_i[n]$ is the restriction of $f_i$ to strings of length $n$.
In other words $\langle f_1, \ldots , f_k \rangle$ is the union of $\{f_i[n] : n
\in \nats\}$ for each $f_i$ in the generating sequence. We have $\AND, \OR,
\NAND, \MAJ : \{0,1\}^* \rightarrow \{0,1\}$, where $\AND, \OR, \NAND$ are the
usual (unbounded fan-in) logical operators, and $\MAJ$ is the majority function
(i.e.\ the function that outputs one if more than half of bits in the inputs are
one). We write $\BS := \langle \AND, \OR, \NAND\} \rangle$ to denote the
\emph{standard basis} and $\BM := \langle \AND, \OR, \NAND , \MAJ \rangle$ to
denote the \emph{majority basis}.

Let $\BB$ be a basis. A \emph{Boolean circuit} $C$ defined over $\BB$ is a
labelled directed acyclic graph (DAG) with designated input gates $x_1, \ldots,
x_n$, each with in-degree $0$, internal gates labelled by elements of $\BB$, and
a single internal gate with out-degree $0$ designated as the output gate. We say
a circuit with $n$ input gates has \emph{order $n$}. The evaluation of circuit
$C$ of order $n$ for the input $\vec{a} \in \{0,1\}^n$ is denoted by
$C[\vec{a}]$ and computed by recursively evaluating the gates in the circuit. We
denote the evaluation of a gate $g$ in $C$ for the input $\vec{a}$ by
$C[\vec{a}](g)$. If $g$ is an input gate, then $g = x_i$ for some $i \in [n]$,
and $C[\vec{a}](g) = a_i$. If $g$ is an internal gate then $g$ is labelled by a
symbol denoting a Boolean operation, and $C[\vec{a}](g)$ is the result of
applying that operation to the string formed from the evaluations of those gates
input to $g$. If $g$ is the output gate, let $C[\vec{a}] = C[\vec{a}](g)$.

The \emph{size} of a circuit $C$, denoted by $\vert C \vert$, is the number of
gates in the circuit. The \emph{depth} of a gate $g$ is the longest path from an
input gate to $g$, and the \emph{depth} if the circuit is the depth of the
output gate. The \emph{width} of a circuit is the maximum size of a set of gates
with the same depth.

If $C$ is a circuit of order $n$, then $C$ computes a Boolean function $f_C :
\{0,1\}^n \rightarrow \{0,1\}$ defined by $f_C(\vec{a}) = C[\vec{a}]$. Let
$(C_n)_{n \in \nats}$ be a family of circuits, where $C_n$ has order $n$. We say
$(C_n)_{n \in \nats}$ \emph{decides} a language $L : \{0,1\}^{*} \rightarrow
\{0,1\}$ if for all $\vec{a} \in \{0,1\}^{*}$, $C_{\vert \vec{a} \vert}
[\vec{a}] = L(\vec{a})$.


\begin{definition}
  Let $(C_n)_{n \in \mathbb{N}}$ be a family of Boolean circuits. We say that
  $(C_n)_{n \in \mathbb{N}}$ is \emph{$\PT$-uniform} if the mapping $n \mapsto
  C_n$ is polynomial-time computable in $n$.
\end{definition}

\subsection{Circuits for Structures}
A \emph{circuit for structures} (or just \emph{circuit}), defined by Anderson
and Dawar~\cite{AndersonD17}, takes as input an encoding of a $\tau$-structure
and outputs a set of tuples from that structure. More formally, for a vocabulary
$\tau$, a basis $\BB$ of symmetric functions, and $q, n \in \nats$, a
\emph{circuit for structures} is a DAG with (i) a designated set of input gates,
each labelled by a relation symbol $R \in \tau$ and a tuple $\vec{a} \in
[n]^{\arty(R)}$, (ii) a set of internal gates, each labelled by an element of
$\BB$, and (iii) a designated set of output gates, each labelled by a unique
element of $[n]^q$.

Let $\mathcal{A}$ be a $\tau$-structure of size $n$ and $\gamma$ be a bijection
from the universe of $\mathcal{A}$ to $[n]$. The bijection $\gamma$ defines an
encoding of $\mathcal{A}$ as a $\tau$-structure with universe $[n]$. The input
gate labelled by the relation symbol $R$ and tuple $\vec{a}$ is assigned to one
if, and only if, $\vec{a}$ is an element of the interpretation of $R$ in the
encoding of $\mathcal{A}$. The circuit may then be recursively evaluated, and
the evaluation of the output gates is taken as the output of the circuit. Since
each output gate is labelled by an element of $[n]^q$, the output of the circuit
is an element of $\{0,1\}^{[n]^q}$. In other words, a circuit maps a
$\tau$-structure $\mathcal{A}$ and bijection $\gamma$ to a set of $q$-tuples in
$\mathcal{A}$.

A circuit is called \emph{invariant} if its output does not depend on the choice
of $\gamma$. In this case the circuit defines a $q$-ary query on
$\tau$-structures. If $C$ is an invariant circuit with $q = 0$, and the circuit
has a single output gate, then $C$ decides a property of $\tau$-structures. A circuit
is called \emph{symmetric} if every permutation on the universe $[n]$, each of
which induces a permutation on the input gates, extends to an automorphism of
the circuit. A symmetric circuit is necessarily invariant.

Anderson and Dawar use families of symmetric circuits to characterise $\FPC$.

\begin{thm}[Anderson and Dawar~\cite{AndersonD17}]
  For a relational vocabulary $\tau$, a property of finite $\tau$-structures is
  decided by a P-uniform family of symmetric circuits defined over the basis
  $\BM$ if, and only if, it is expressible in $\FPC[\tau]$.
\end{thm}

\subsection{Equivalence Relations}
Let $\sim$ be an equivalence relation on the set $X$. For $x \in X$ we write
$[x]_\sim$ to denote the equivalence class of $x$ with respect to $\sim$. We
omit the subscript if the equivalence relation is obvious from context. We write
$X/_{\sim}$ to denote the set of equivalence classes of $X$. Let $A = \Pi_{i \in
  I} A_i$ and let $J \subseteq I$ be such that for $i \in J$, $A_i \subseteq X$.
Let $\sim_A$ be an equivalence relation on $A$ defined such that $\vec{a} \sim_A
\vec{b}$ if, and only if, $\vec{a}(i) = \vec{b}(i)$ for all $i \in J$. We call
$\sim_A$ the \emph{extension} of $\sim$ to $A$. We often abuse notation and
write $\sim$ for the extension of the equivalence relation $\sim$. We notice
that $(\Pi_{i \in I} A_i) /_{\sim} = \Pi_{i \in I} (A_i /_\sim)$.

Let $Y$ and $Z$ be sets and let $f : Y \rightarrow Z$ be a function. We say $f$
is \emph{invariant} under $\sim$ if for all $a, b \in Y$ if $a \sim_Y b$ then
$f(a) \sim_Z f(b)$. If $f$ is an invariant under $\sim$ then $f /_{\sim}$ is a
function mapping $Y /_{\sim_Y} $ to $ Z /_{\sim_Z}$ such that $f /_{\sim} ([a]) =
[f(a)]$ for all $a \in Y$. We call $f_{\sim}$ the \emph{quotient} of $f$ with
respect to $\sim$.


% \subsection{Basic Notation}

% Let $f : X \rightarrow Y$ be a function and $\sim$ be an equivalence relation
% on Y. We define $f/{\sim} : X \rightarrow Y/{\sim}$ by $f(x) = [f(x)]_{\sim}$.
% We call this the quotient of the function by $\sim$. For any two sets $X, Y$
% we let $X^{\underline{Y}}$ denote the set of all injective functions from $Y$
% to $X$.

% INCLUDE STUFF ON SUPPORTS SET-WISE SUPPORTS

\end{document}

