\documentclass[../paper.tex]{subfiles}

\begin{document}
In this section we present basic background on logics, first-order structures,
and symmetric circuits.

\subsection{Logic and Structures}
A \emph{relational vocabulary} (usually denoted by $\tau$) is a finite sequence
of relation symbols $R_1, \ldots, R_k$, such that for all $i \in [k]$, $R_i$ has
arity of $r_i \in \nats$.

A $\tau$-structure $\mathcal{A}$ is a tuple $\langle U, R^{\mathcal{A}}_1 ,
\ldots , R^{\mathcal{A}}_k \rangle$, where $U$ is a set and called the
\emph{universe} of $\mathcal{A}$ and for all $i \in [k]$ the relations
$R^{\mathcal{A}}_i \subseteq U^{r_i}$. The elements of $U$ are called the
elements of $\mathcal{A}$. A \emph{multi-sorted} structure is one whose universe
is a disjoint union, where each set in the union is called a \emph{sort}. The
size of a structure is the cardinality of its universe.

All structures in this paper are finite.

\subsection{First Order, Fixed-Point, Counting}

Let $\FO(\tau)$ denote \emph{first order logic} with respect to the vocabulary
$\tau$. \emph{Fixed-Point Logic} is the extension of $\FO(\tau)$ with an
inflationary fixed-point operator (see \cite{} for details). For logic
$\mathcal{L}$, a formula $\phi \in \mathcal{L}(\tau)$, $x$ a $k$-length tuple of
variables, $\mathcal{A}$ a (finite) structure, and $a \in U^k$, we have that
$\phi (x)$ denotes that the variables in $x$ are free in $\phi$ and we write
$\mathcal{A} \models_{\mathcal{L}} \phi[a]$ iff the the formula $\phi$ is true
in $\mathcal{A}$ with respect to $\mathcal{L}$ with the variables $x$ assigned
to $a$. The context usually makes the logic obvious, and in that case we usually
drop the subscript.

We also recall the definition of \emph{fixed-point logic with counting} $\FPC$,
an extension $\FP$ with counting quantifiers. For a structure $\mathcal{A}$,
Formulas of this logic are evaluated over a structure $\mathcal{A}$ by extending
that structure to $\mathcal{A}^\leq$, a two sorted extension of $\mathcal{A}$ by
an ordered number sort $\{1, \ldots, \vert U\vert \}$. Variables are taken to be
either point variables, ranging over $U$ or number variables ranging over
$[\vert U \vert]$. The counting quantifiers are of the form $\#_x$, where $x$ is
either a point or number variable and for $\phi(x) \in \FPC(\tau)$, $\#_x
\phi(x)$ is a term denoting the element of the number sort corresponding to
$\vert \{a \in \mathcal{A} | \mathcal{A} \models \phi[a]\}\vert$. Note
additionally that $k$-tuples of number variables can be used to encode numbers
in $\FPC$ of size up to $\vert U\vert^k$. For more details on $\FPC$ please see
\cite{}.

\subsection{Rank Logic}
In this paper we use the definition of $\FPR$ given by Grad\"{e}l and Pakusa
\cite{}, an strictly more expressive extension of the rank logic introduced by
Dawar et. al. cite{}. We briefly define this logic here. For more detail please
see \cite{}. $\FPR$ is an extension of $\FPC$ with rank operators, which we now
define.

Let $\phi (\vec{x}, \vec {\nu} \leq \vec{t}, \vec{y}\vec{\mu} \leq \vec{s})$ be
a numeric term, where $\vec{x}$ and $\vec{y}$ are tuples of variables ranging
over the universe of the structure, $\vec{\nu}$ and $\vec{\mu}$ are tuples of
variables ranging over the number type, and $\vec{s}$ and $\vec{t}$ are tuples
of closed numeric terms that bound the range of $\vec{\nu}$ and $\vec{\mu}$
respectively. Then let $\mathcal{A}$ be a structure and let
$\mathbb{N}^{\leq\vec{t}} := \{\vec{n} \in \mathbb{N}^{\vert \vec{v} \vert}: n_i
\leq t^{\mathcal{A}}_i\}$, and let $\mathbb{N}^{\leq \vec{s}}$ be defined
similarly. Let $I:= A^{\vert \vec{x} \vert} \times \mathbb{N}^{\leq \vec{t}}$
and $J := A^{\vert \vec{y} \vert} \times \mathbb{N}^{\leq \vec{s}}$ Then let
$M^{\mathcal{A}}_\phi : I \times J \rightarrow \mathbb{N}$ be defined by
$M^{\mathcal{A}}_\phi (\vec{a}\vec{n}, \vec{b}\vec{m}):=
\phi^{\mathcal{A}}(\vec{a}\vec{n},\vec{b}\vec{m})$.

The rank operator is then defined as $[rk (\vec{x}, \vec {\nu} \leq \vec{t},
\vec{y}\vec{\mu} \leq \vec{s}, \pi \leq r). \phi]$, where $\pi$ is a variable
ranging over the number sort and $r$ is a closed numeric term that bounds $\pi$.
The value of the rank term for a structure $\mathcal{A}$ and an assignment $\pi
\mapsto p$, where $p$ is prime, is given by the rank of $M^{\mathcal{A}}_\phi$
mod $p$, as understood as a matrix over $\mathbb{F}_p$.

\subsection{Symmetric Circuits}
This still needs to be filled in
\end{document}

