\documentclass[../paper.tex]{subfiles}
%\usepackage{../mymacros}
\begin{document}
% Note: include circuits in background
Circuits are a well-studied model in complexity theory and, more recently, they
have been studied in the context of finite model theory. Circuits in the
literature () are usually defined as DAG's, with each vertex labelled by some
element of a basis. While not explicitly stated, a consequence of this
definition is that, when evaluating a gate, the Boolean function labelling this
gate should not depend on any ordering or labelling of its input. In other
words, it must be a symmetric function.

Motivated by the desire to study circuits with gates that compute rank, a
function which is not symmetric, we develop a general framework for studying
functions whose output depends on the structure of the input. This structure
also allows us to define and explore appropriate symmetries on the function.

With the necessary structure on the input of each Boolean function explicitly given, we then
develop a circuit model which incorporates this structure, and so allows for
circuits to be defined over Basis of functions that are not symmetric.

Finally, we prove that no family of Basis of symmetric functions can improve on
the expressive power of the circuit model of Anderson and Dawar \cite{AndersonD17}.
Together with the main theorem of this paper this implies that the more general
circuit models that are defined over basis of functions which are not symmetric
(with respect to the whole symmetric group) are strictly more powerful.

% consider the problem of explicitly labelling the input of the function both so
% that the semantics of the function can be reasonably defined and so as to
% ensure that appropriate symmetry conditions on the function can be enforced.


% labelled by functions that are not symmetric, we need to think explicitly
% about how the input is to be labelled and how that structure should be
% mirrored in the circuit definition. In particular, we should like to consider
% functions which invariant under some action and so choose labellings which
% appropriately reflect the required symmetry conditions of the function.

% In this section we first discuss how to structure the input to a Boolean
% function appropriately and how symmetries on that input structure induce
% useful symmetries on the function.

\subsection{$\tau$-Symmetric Functions}
% note: include many-sorted signature in background
Let $\tau = (R, S, \nu)$ be a many-sorted, relational signature. Suppose some
ordering on the set of relations and sorts, so $R = \{R_1 \ldots, R_t\}$ and $S
= \{s_1, \ldots s_p\}$, and let $\nu(R_i) = (a^i_1, \ldots , a^i_{\arty(R_i)})$,
where $a^i_q \in S$ for all $q \in [\arty(R_i)]$. For each $s \in S$ let $A_{s}$
be a non-empty set, and let $A = A_{s_1} \times \ldots \times A_{s_p}$. Let
\[
  X^{\tau}_A:= A_{a^1_1} \times \ldots \times A_{a^1_{\arty(R_1)}} \times \ldots
  \times A_{a^t_1} \times A_{a^t_{\arty(R_t)}}.
\]

Let $G \leq \sym_{A_{s_1}} \times \ldots \times \sym_{A_{s_p}}$. There is a
natural action of $G$ on $X$. It is easy to see that if $\mathcal{C}_A$ is the
category with objects given by $\tau$-structures over $A$ and morphisms by the
action of $G$, and $\mathcal{C}_X$ is the category of subsets of $X^{\tau}_A$
with morphisms similarly given by the action of $G$, then these two categories
are equivalent. More informally, the subsets of $X^{\tau}_A$ encode the
$\tau$-structures over $A$. This relationship is bijective and the action of $G$
factors through this bijection.

\begin{remark}
  Is the categorical language unnecessary? It seemed the quickest way of saying
  what I wanted to say formally.
\end{remark}

\begin{remark}
  In the above section (and just below) we talk about the action of
  $\sym_{A_{s_1}} \times \ldots \times \sym_{A_{s_p}}$ on elements of $X^\tau_A$
  and structured functions. In the proof of the support theorem, when defining a
  definable matrix for converting circuits into formulas, and later on in this
  section, it is useful to speak more generally and instead look at bijections
  from $A_1, \ldots A_{s_p}$ to $B_1, \ldots, B_{s_p}$, thus allowing us to map
  between $X^\tau_A$ and $X^\tau_B$ and so between the associated structured
  functions. I've added this in later on in the section, but I still need to
  incorporate it into the original definition (it's a more general notion in a
  sense, and so I think it should be incorporated). I'll wait for feedback
  before doing this as I'm not sure this formulation will survive.
\end{remark}

Note that there is also a natural action on functions of the form $\vec{i}:
X^{\tau}_A \rightarrow \{0,1\}$ given by $\sigma \cdot \vec{i}(x) =
\vec{i}(\sigma x)$.

We now introduce a few useful notions of symmetry. We call a function of the
form $F:\{0,1\}^{X^{\tau}_A} \rightarrow \{0,1\}$ a \emph{$(\tau, A)$-structured
  function}, with $\tau$ called the \emph{signature of $F$} and $A$ \emph{the
  universe of $F$}. Call $F$ a \emph{structured function} if it is a $(\tau,
A)$-structured function for some appropriate $\tau$ and $A$.

We say a $(\tau, A)$-function is \emph{$(\tau, G)$-symmetric} if for all
$\vec{i} \in \{0,1\}^{X^\tau_A}$ and $\sigma \in G$, $f(\sigma \cdot \vec{i}) =
f(\vec{i})$. We say $F$ is \emph{$\tau$-symmetric} if it is $(\tau,
\sym_{A_{s_1}} \times \ldots \times \sym_{A_{s_p}})$-symmetric. Let $\tau = (S,
R, \nu)$, with $\vert S \vert \leq 2$, $R = {R'}$ and $\nu (R') = (s_1, s_2)$
(with $s_1$ and $s_2$ being the first and last elements of S in the ordering
respectively); if $F$ is $\tau$-symmetric we say $F$ is \emph{matrix-symmetric}.
If $\tau$ is single-sorted we call $F$ \emph{simple} and if it consists of a
single unary relation, we call $F$ \emph{unary}. We say $F$ is
\emph{graph-symmetric} if $F$ is simple and matrix-symmetric.

\begin{lem}
  Let $X^\tau_A$ be defined as above and let $F: \{0,1\}^{X^{\tau}_A}
  \rightarrow \{0,1\}$, then $F$ is a symmetric function (in the sense of
  Anderson and Dawar \cite{AndersonD17}) if and only if $F$ is a simple and
  unary.
\end{lem}

We briefly introduce notions of equivalence useful for comparing functions.
\begin{definition}
  Let $\tau = (R, S, \nu)$ be a many-sorted signature and let $A = A_1 \times
  \ldots A_s$ and $B = B_1 \times \ldots \times B_s$ be a product of non-empty
  sets. A \emph{sorted bijection} between $A$ and $B$ is a function bijections
  $f: A \rightarrow B$ such that $f (A_i) = B_i$ for all $i \in S$. There is an
  obvious action of $f$ that maps $X^\tau_A$ to $X^\tau_B$ and
  $\{0,1\}^{X^\tau_A}$ to $\{0,1\}^{X^\tau_B}$.

  Let $L: X^\tau_A \rightarrow H$ and $L': X^\tau_B \rightarrow H$ be two
  functions for some finite set $H$. We say that $L \sim L'$, or \emph{$L$ is
    equivalent to $L'$}, if there is a sorted bijection $f: A \rightarrow B$
  such that for all $x \in X^\tau_A$, $L(x) = L'(f (x))$.

  Let $F$ be a $(\tau, A)$-structured function and $G$ be a $(\tau,
  B)$-structured function. Then we say that $F \sim G$, or \emph{$F$ is
    equivalent to $G$}, if there is a sorted bijection $f: A rightarrow B$ such
  that for all $\vec{i} \in \{0,1\}^{X^\tau_A}$, $F (\vec{i}) = G(f(\vec{i}))$.
\end{definition}

\begin{remark}
  There is an obvious connection between $\tau$-symmetric functions and
  generalised quantifiers (or closed classes of structures). Should I include
  details about this connection? I also feel a lot of dirtiness might be
  avoidable if we instead recast everything in terms of generalised quantifiers.
  For one, generalised quantifiers give a natural way of defining a Boolean
  function for each input universe.
\end{remark}

\subsection{Symmetric Circuits}
Having developed the notion of a function that accepts input structured in
accordance with some vocabulary, we now develop a circuit model which
incorporates this structure on the inputs of a gate. We use this general
framework to define the notion of a matrix-symmetric circuit, a circuit with
gates labelled by matrix-symmetric Boolean functions, and finally we develop the
notion of a matrix-symmetric circuit with rank.

% Importantly, many natural functions of interest are matrix symmetric. For
% example, the function that computes the rank of the matrix over
% $\mathbb{F}_2$. or a thresholded rank function, for example the rank of the
% matrix over $\mathbb{F}_p$ being larger then $r$, for some particular $(p, r)
% \in \mathbb{N}$.


\begin{definition}[Circuits on Structures]
  Let $\mathbb{B}$ be a basis of structured functions and $\tau$ a set of
  relation symbols, we define a \emph{$(\mathbb{B}, \tau)$-circuit} $C_n$
  computing a $q$-ary query $Q$ is a structure $\langle G, W, \Omega, \Sigma,
  \Lambda, L\rangle$.
  \begin{itemize}
    \setlength\itemsep{0mm}
  \item $G$ is called the set of gates of $C_n$ and $\vert C_n \vert := \vert G
    \vert$.
  \item $W \subseteq G \times G$, where $W$ is called the wires of the circuit.
    $(G,W)$ must be a directed acyclic graph. For $g \in G$ we $H_g := \{ h \in
    G: W(h,g)\}$ be the set of children of $g$.
  \item $\Omega$ is an injective function from $[n]^q$ to $G$. The gates in the
    image of $\Omega$ are called the output gates. When $q = 0$, $\Omega$ is a
    constant function mapping to a single output gate.
  \item $\Sigma$ is a function from $G$ to $\mathbb{B} \uplus \tau \uplus
    \{0,1\} $ which maps input gates to $\tau \uplus \{0,1\}$ and where $\vert
    \Sigma^{-1} (0) \vert \leq 1$ $\vert \Sigma^{-1} (1) \vert \leq 1$ and the
    internal gates get mapped into $\mathbb{B}$. Gates mapped to $\tau$ are
    called relational gates and gates mapped to 1 or 0 are called constant
    gates.
  \item $\Lambda$ is a sequence of injective functions $(\Lambda_R)_{R \in
      \tau}$ where for each $R \in \tau$, $\Lambda_R$ maps each relational gate
    $g$ with $R = \Sigma (g)$ to the tuple $\Lambda_R (g) \in [n]^r$, where $r$
    is the arity of the symbol $R$. When no ambiguity arises we write $\Lambda
    (g)$ for $\Lambda_R (g)$.
  \item $L$ to each gate a labelling on its children. Let $\tau_g$ and $A_g$ be
    such that $\Sigma(g)$ is a $(\tau_g, A_g)$-structured function. Then $L(g)$
    is a surjection from $X^{\tau_g}_{A_g}$ to $H_g$. We call $L(g)$ the
    \emph{structure-labelling} of $g$.
  \end{itemize}
\end{definition}

Given some finite $\tau$-structure $\mathcal{A}$ of size $n$ and some bijection
$\gamma: U \rightarrow [n]$ and let $\gamma \mathcal{A}$ be the structure formed
by mapping the universe of $\mathcal{A}$ in accordance with $\gamma$. The
evaluation of some $(\mathbb{B}, \tau)$-circuit $C$ proceeds by recursively
evaluating gates. The evaluation of the gate $g$ using $\gamma$ with input
$\mathcal{A}$ is denoted by $C[\gamma \mathcal{A}](g)$. The evaluation of a gate
$g$ is given as follows.
\begin{enumerate}
  \setlength\itemsep{0mm}
\item If $g$ is a constant gate then it evaluates to the bit given by
  $\Sigma(g)$.
\item If $g$ is a relational gate then $g$ evaluates to true iff $\gamma
  \mathcal{A} \models \Sigma(g)(\Lambda (g))$.
\item If $g$ is an internal gate labelled by a $(\tau_g, A_g)$-structured
  function $\Sigma(g)$, let $L^{\gamma}_g: X^{\tau_g}_{A_g} \rightarrow \{0,1\}$
  be defined by $L^{\gamma}_g(x) = C[\gamma \mathcal{A}](L(g)(x))$. Then $g$
  evaluates to true iff $\Sigma(g) (L^{\gamma}_g)$.
\end{enumerate}
$C$ defines the $q$-ary query $Q \subseteq \mathcal{A}^q$ where $a \in Q$ iff
$C[\gamma \mathcal{A}](\Omega (\gamma a)) = 1$.

The following definition is an important circuit property introduced by Anderson
and Dawar \cite{AndersonD17}. We introduce it for the sake of comparison.

\begin{definition}[Invariant Circuit]
  Let $C_n$ be a $(\mathbb{B}, \tau)$-circuit, computing some $q$-ary query. We
  say $C_n$ is \emph{invariant} if for every $\tau$-structure $\mathcal{A}$ of
  size $n$, $a \in \mathcal{A}^q$, and bijections $\gamma_1, \gamma_2: U
  \rightarrow [n]$ we have that $C[\gamma_1 \mathcal{A}](\Omega (\gamma_1 a)) =
  C[\gamma_2 \mathcal{A}](\Omega (\gamma_2 a))$.
\end{definition}

This property ensures that a circuit (or a family of circuits) is invariant
under isomorphism. In other words, the circuit (or family of circuits) decides a
graph property. The following lemma allows us to recast this notion in terms of
the language developed in this paper.

\begin{definition}
  Let $C_n$ be a $(\mathbb{B}, \tau)$-circuit computing a Boolean query. We have
  that $C_n$ computes a $(\tau, [n])$-structured function $F$. We say $C_n$ is
  $\tau$-symmetric if $F$ is $\tau$-symmetric.
\end{definition}

\begin{lem}
  Let $C_n$ be a $(\mathbb{B}, \tau)$-circuit. $C_n$ is $\tau$-symmetric iff
  $C_n$ is invariant.
\end{lem}

The following definition gives us a way of saying that two structured functions
with the same signature are equivalent up to a choice of their universe.

\begin{definition}[Automorphism]
  let $C = \langle G, W, \Omega, \Sigma, \Lambda, L\rangle$ be a
  $(\mathbb{B},\tau)$-circuit computing at $q$-ary query on structures of size
  $n$. Let $\sigma \in \sym_n$ and $\pi: G \rightarrow G$ be a bijection such
  that
  \begin{itemize}
    \setlength\itemsep{0mm}
  \item for all gates $g, h \in G$, $W(g,h)$ iff $W(\pi g, \pi h)$,
  \item for all output tuples $x \in [n]^q$, $\pi \Omega (x) = \Omega (\sigma
    x)$,
  \item for all gates $g \in G$, let $\Sigma (g) \sim \Sigma (\pi g)$,
  \item for each relational gate $g \in G$, $\sigma \Lambda (g) = \Lambda (\pi
    g)$, and
  \item for each internal gate $g$ if $\Sigma (g) \in \mathbb(B)$ then we have
    that $L(\pi g) \sim \pi \cdot L(g)$.
  \end{itemize}

  We call $\pi$ an \emph{automorphism} of $C$, and we say that $\sigma$
  \emph{induces the automorphism} $\pi$. The group of automorphisms of $C$ is
  called $\aut_n (C)$.
\end{definition}

\begin{definition}[Symmetry]
  A circuit $C$ on structures of size $n$ is called \emph{symmetric} if every
  $\sigma \in \sym_n$ induces an automorphism on $C$.
\end{definition}

It follows for any symmetric circuit $C_n$ there is a homomorphism $h$ that maps
$\sym_n$ to $\aut_n(C)$. This homomorphism is injective so long as a single
element of $[n]$ appears in the in the label of some input gate of $C$ (as then
all elements appear by symmetry)\cite{AndersonD17}. In this paper we always
assume that there is at least one such element, as otherwise all inputs are
constant, and so the circuit just computes a constant function. In order to
assure this homomorphism is surjective Anderson and Dawar \cite{AndersonD17}
introduce the notion of a \emph{rigid} circuit.

\begin{definition}[Rigidity]
  Let C be a $(\mathbb{B}, \tau)$-circuit, where $C = \langle G, W, \Omega,
  \Sigma, \Lambda, L\rangle$. Say that $C$ is rigid if there are no distinct
  internal gates $g, g' \in G$ such that $\Sigma(g) \sim \Sigma (g')$,
  $\Omega^{-1}(g) = \Omega^{-1}(g')$, $H_g = H_{g'}$ and $L(g) \sim L'(g)$.
\end{definition}

Another property which simplifies our analysis is the property of having
\emph{bijective labels}.

\begin{definition}
  We say that a circuit $C$ has \emph{bijective labels} if for each gate $g$ in
  $C$, $L(g)$ is a bijection.
\end{definition}

We prove in a later section that a circuit may be transformed in polynomial time
into an equivalent circuit that is both rigid and has bijective labelings. Hence
we may assume these two properties without a loss of generality.

With the assumption of rigidity in place, we abuse notation for permutations and
let $\sigma \in \sym_n$ also denote the induced automorphism.

We finally want to define a circuits that allow for gates labelled by
matrix-symmetric functions.
\begin{definition}
  Let $\mathbb{B}$ be a basis containing all simple unary structured functions
  that compute $\land$, $\lor$, $\neg$ and $\maj$ and such that all elements of
  this basis are either matrix-symmetric or simple unary. We call a
  $(\mathbb{B}, \tau)$-circuit a \emph{matrix-symmetric circuit}.
\end{definition}

We define a circuit that has gates that compute rank.
\begin{definition}
  Let the $(\mathbb{B}, \tau)$-circuit $C$ be a matrix-symmetric circuit. Let
  $\mathbb{B}_m$ be the subset of $\mathbb{B}$ consisting of all those functions
  that are matrix-symmetric but not symmetric. For any non-empty sets $A_1, A_2$
  and $r, p \in \mathbb{N}$, where $p$ prime define a $F_{p,r}: \{0,1\}^{A_1
    \times A_2} \rightarrow \{0,1\}$ such that $f_{p,r}(M) = 1$ iff the matrix
  $M: A_1 \times B_1 \rightarrow \{0,1\}$ has rank at least $r$ over the field
  $\mathbb{F}_p$. We call such a function a \emph{bounded-rank function}. If
  $\mathbb{B}_m$ is equal to the set of all such bounded-rank functions, then we
  call $C$ a \emph{symmetric circuit with rank gates}.
\end{definition}

% The natural restriction to consider on families of circuits is uniformity.

% \begin{definition}
%   Let $(C_n)_{n \in \mathbb{N}}$ be a family of Boolean circuits. We say that
%   $(C_n)_{n \in \mathbb{N}}$ is \emph{$P$-uniform} if the mapping $n \mapsto
%   C_n$ is computable in polynomial time.
% \end{definition}

We are now ready to state what will be the main theorem of this paper.
\begin{thm}[Main Theorem]
  A graph property is decidable by a $P$-uniform family of symmetric circuits
  with rank gates if, and only if, it is definable by an FPR sentence.
\end{thm}

\subsection{Limitations of Symmetric Bases}
When looking at symmetric circuits with matrix symmetric functions, it is
natural to ask if this extension to matrix-symmetric functions is really
necessary. In this section we show that Symmetric circuits, as developed by
Anderson and Dawar, defined over the usual basis with majority gates can compute
any symmetric function over polynomial size circuits. As such no additional
symmetric function added to the Basis improves the power of the model.

Recall from Anderson and Dawar \cite{AndersonD17} we have that
$\mathbb{B}_{\std} = \{ \neg , \wedge , \lor \}$ and $\mathbb{B}_\maj = \{ \maj
\} \cup \mathbb{B}$.

Let $F: \{0,1\}^* \rightarrow \{0,1\}$ be a symmetric Boolean function. Since
$F$ is symmetric we note that for a fixed size input the output of $F$ is
entirely determined by the number of 1's in its input. Then let
$c_{F}:\mathbb{N} \rightarrow 2^{\mathbb{N}}$ define a function where $c_{F}(n)$
is the set of all $m \leq n$ such that for all $\vec{x} \in \{ 0,1 \}^n$ with
$m$ 1's we have $F (\vec{x}) = 1$. Clearly any symmetric Boolean function $F$ is
entirely determined by $c_{F}$.
 
\begin{prop}
  \label{prop:fuctions-maj}
  There is a polynomial $p(k)$ such that for any symmetric function $F$ and a
  given $k \in \mathbb{N}$ there is a circuit $C_k$ on $k$ inputs over the basis
  $\mathbb{B}_\maj$ which is symmetric, constant depth and with width bounded by
  $p(k)$.
\end{prop}

\begin{proof}
  We define the circuit $C_k$ for inputs $\vec{x} = ( x_1, \ldots, x_k )$.

  For $a \in \mathbb{N}$ we define a gate $\countgate_a$ by
  \begin{align*}
    \countgate_a = \begin{cases} \maj (x_1, \ldots, x_k, \underbrace {0, \ldots,
        0}_{2a - k}) \land \neg \maj (x_1, \ldots , x_k, \underbrace{0, \ldots,
        0}_{2a - k + 2}) &  a \geq \frac{k}{2} \\
      \maj (x_1, \ldots, x_k, \underbrace {1, \ldots, 1}_{x - 2a}) \land \neg
      \maj (x_1, \ldots , x_k, \underbrace{1, \ldots, 1}_{k - 2a -2}) & a <
      \frac{k}{2}.
    \end{cases}
  \end{align*}
  Then let $g = \bigvee_{a \in c_{F_i}(k)}\countgate_a$ and let $C_{k}$ be the
  circuit with input gates labeled by $\vec{x}$ and output gate $g$.

  t is easy to see that $C_k$ is constant depth and it's width is a polynomial
  in $k$. We have that in each layer of $C_{g'}$ each gate is connected to all
  gates in the previous layer, and as such the circuit is symmetric.
\end{proof}

The above proposition has a straight forward application to circuit
characterisations.

\begin{thm}
  Let $F = \{F_i : i \in I \}$ be a family of symmetric Boolean functions where
  $F_i: \{0,1\}^* \rightarrow \{ 0,1 \}$.

  Let $(C_n)_{n \in \mathbb{N}}$ be family of symmetric circuits over the
  Boolean basis $\mathbb{B} \cup F$, where $C_n$ is a circuit on structures of
  size $n$, and the size of each circuit in the family is bounded by some
  function $f(n)$. Then there exists a polynomial $q(n)$ and a family of
  symmetric circuits $(C_n')_{n \in \mathbb{N}}$ over $\mathbb{B}_\maj$, where
  $C_n'$ is a circuit on structures of size $n$ and $\vert C_n' \vert \leq
  q(f(n))$.
\end{thm}

\begin{proof}
  From $C_n$ we construct $C_n'$ in the obvious way. For each gate $g \in C_n$
  of type $F_i$ we have a symmetric circuit $C_g$ from Proposition
  \ref{prop:function-maj} that computes the same function as $g$. Then let
  $C_n'$ be $C_n$ but with each gate $g$ replaced by $C_g$. It is easy to see
  that $C_n'$ is symmetric. We also have that each gate $g$ must have at most
  $f(n)$ inputs, and the size of $C_g$ is bounded by $p(f(n))$. Thus the size of
  $C_n'$ is bounded by $f(n)p(f(n))$.
\end{proof}

This result, combined with the main result of this paper, gives us that
symmetric circuits with rank are a strictly more powerful model than any circuit
model defined over a basis of symmetric functions.

% \section{Circuits to FPR}

% In this section I discuss how to determine the rank a matrix labelling at some
% gate $g$. This is a rough copy written for the purpose of discussion. In this
% section we fix some circuit $C_n$ and some rank gate $g$ in $C_n$ with a set
% of children $H$ and matrix labelling $L: [a] \times [b] \rightarrow H$ (which
% we assume WLOG is a bijection).
\end{document}

