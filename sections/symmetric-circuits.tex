\documentclass[../paper.tex]{subfiles}
%\usepackage{../mymacros}
\begin{document}
% Note: include circuits in background
Circuits are a well-studied model in complexity theory, and more recently they
have been studied in the context of finite model theory. However, there are some
subtleties in the definition and evaluation of circuits which must be considered
carefully and have important implications for the notion of a symmetric circuit
studied in this paper. Let $C$ be a Boolean circuit $g$ be an internal gate
labelled by some function $f_g: \{0,1\}^{A} \rightarrow \{0,1\}$, for some index
set $A$ with $\vert A \vert = \vert H_g \vert$. Formally, evaluating $g$
requires a (usually bijective) indexing function $f_i: A \mapsto H_g$, which
organises the inputs of $g$ into an appropriate sequence. The evaluation of $g$
is then given by $C[\vec{x}](g) = f_g (a \mapsto C[\vec{x}](f_i(a)))$. As such
the evaluation of a gate in a circuit may depend on how the inputs to the gate
are indexed.

In most contexts in complexity theory, circuits may be taken to be implicitly
ordered objects and so this indexing function can be defined as taking the $i$th
element of $A$ to the $i$th element of $H_g$. However, in general this renders
the evaluation of a circuit for a given input a property both of the circuit and
an arbitrary order. In order to address this problem this problem many authors
only define circuits over Boolean bases of symmetric functions (e.g. the
standard basis or the majority basis). In this case the evaluation of a gate is
independent of the choice of indexing function.

However, in this paper we require gates labelled by Boolean rank functions, i.e.
a function that takes in a Boolean sequence indexed by `row' and `column' sets,
outputs 1 iff that matrix, when interpreted as having entries in $\mathbb{F}_p$,
has rank at least $r$. These rank functions clearly depend on how the input
sequence is labelled, and so are not symmetric. However, it is worth noting that
while such rank functions may be symmetric, they are symmetric in the weaker
sense of being constant under permutations on the rows and columns of the input
matrix.

Motivated by the desire to study circuits with gates that compute rank, We
develop a general framework for circuits that allows gates to be labelled by
Boolean functions of the form $F: \{0,1\}^{X} \rightarrow \{0, 1\}$, where the
labelling $X$ is the universe of a many-sorted structure over some vocabulary
$\tau$. The Moreover, we generalise the notion of a circuit for structures (see
Anderson and Dawar \cite{AndersonD17}), incorporate an appropriate labelling for
the inputs of each gate and allowing for a Boolean basis including non-symmetric
functions. Furthermore, the vocabulary $\tau$ will allow us to impose natural
symmetry conditions on $F$, and so allowing us to develop a useful notion of a
symmetric circuit.

We also prove that no family symmetric functions defined over a basis of
symmetric functions can improve on the expressive power of the circuit model of
Anderson and Dawar \cite{AndersonD17}. Together with the main theorem of this
paper this implies that these more general symmetric circuits are strictly more
powerful.

% consider the problem of explicitly labelling the input of the function both so
% that the semantics of the function can be reasonably defined and so as to
% ensure that appropriate symmetry conditions on the function can be enforced.


% labelled by functions that are not symmetric, we need to think explicitly
% about how the input is to be labelled and how that structure should be
% mirrored in the circuit definition. In particular, we should like to consider
% functions which invariant under some action and so choose labellings which
% appropriately reflect the required symmetry conditions of the function.

% In this section we first discuss how to structure the input to a Boolean
% function appropriately and how symmetries on that input structure induce
% useful symmetries on the function.

\subsection{Structured Functions and Symmetry}
Let $\tau = (R, S, \nu)$ be a many-sorted, relational signature, with $R = \{R_1
\ldots, R_t\}$ and $S = [s]$ for some $s \in \nats$, and $\nu(R_i) = (a^i_1,
\ldots , a^i_{r_i)})$, where $r_i = \arty(R_i)$ and $a^i_q \in S$ for all $q \in
[r_i]$. For each $s \in S$ let $A_s$ be a non-empty set, and let $A := A_{1}
\sqcup \ldots \sqcup A_{s} = \{(a,s) : a \in A_s\}$. Let
\[
  \ind (A, \tau) := A_{a^1_1} \times \ldots \times A_{a^1_{r_1}} \times \ldots
  \times A_{a^t_1} \times A_{a^t_{r_t}},
\]. We call $\ind (A, \tau)$ the \emph{index defined by} $(A, \tau)$.

% It is worth noting that there is a natural bijection $P$ that maps
% $\tau$-structures with universe $A$ to subsets of $\Ind(A, \tau)$, taking each
% $\tau$-structure $\mathcal{A}$ to the set $\{(\vec{X}_1, \ldots , \vec{X}_t) :
% \exists i \in [t] \(\vec{X_i} \in R^{\mathcal{A}}_i \) \}$.

For the remainder of the section let $X := \ind (A, \tau)$. Let $G \leq
\sym_{X}$, there is a natural action of $G$ on $X$. There is also a natural
action of $\sym_{A_1} \times \ldots \times \sym_{A_s}$ on $X$ and a subgroup
$G'$ of $\sym_X$ corresponding to the same action. We identify these two groups.

Furthermore, we can defined an action of $G$ on functions of the form $f: X
\rightarrow H$, for any set $H$, given by $\sigma \cdot f(x) = f(\sigma \cdot
x)$ for all $x \in X$. If $H = \{0,1\}$ we say that $f$ is a \emph{binary
  sequence indexed by $X$}. If $H$ is surjective, we say that \emph{$f$ indexes
  $H$ by $X$} or \emph{$f$ indexes $H$ by $(A,\tau)$}.

We develop a notion of equivalence from this action. Say that two functions $f,
g : X \rightarrow H$ are \emph{$G$-equivalent} iff there exists $\sigma \in G$
such that $f(x) = f(\sigma \cdot x)$ for all $x \in X$. We say $f$ and $g$ are
\emph{sort-equivalent} if they are $\sym_{A_1} \times \ldots
\sym_{A_p}$-equivalent.

We call a function of the form $F:\{0,1\}^{X} \rightarrow \{0,1\}$ a
\emph{$(\tau, A)$-structured function}. We call $(\tau, A)$ the \emph{type} of
$F$, $\tau$ the \emph{signature of $F$} and $A$ \emph{the universe of $F$}. We
also use $\ind(F)$ to abbreviate $\ind (\type (F))$.Call $F$ a \emph{structured
  function} if it is a $(\tau, A)$-structured function for some appropriate
$\tau$ and $A$.

% It is easy to see that if $\mathcal{C}_A$ is the category with objects given
% by $\tau$-structures over $A$ and morphisms by the action of $G$, and
% $\mathcal{C}_X$ is the category of subsets of $X^{\tau}_A$ with morphisms
% similarly given by the action of $G$, then these two categories are
% equivalent. More informally, the subsets of $X^{\tau}_A$ encode the
% $\tau$-structures over $A$. This relationship is bijective and the action of
% $G$ factors through this bijection.

% \begin{remark}
%   Is the categorical language unnecessary? It seemed the quickest way of
%   saying what I wanted to say formally.
% \end{remark}

% \begin{remark}
%   In the above section (and just below) we talk about the action of
%   $\sym_{A_{s_1}} \times \ldots \times \sym_{A_{s_p}}$ on elements of
%   $X^\tau_A$ and structured functions. In the proof of the support theorem,
%   when defining a definable matrix for converting circuits into formulas, and
%   later on in this section, it is useful to speak more generally and instead
%   look at bijections from $A_1, \ldots A_{s_p}$ to $B_1, \ldots, B_{s_p}$,
%   thus allowing us to map between $X^\tau_A$ and $X^\tau_B$ and so between the
%   associated structured functions. I've added this in later on in the section,
%   but I still need to incorporate it into the original definition (it's a more
%   general notion in a sense, and so I think it should be incorporated). I'll
%   wait for feedback before doing this as I'm not sure this formulation will
%   survive.
% \end{remark}

These structured functions give us a general framework for indexing the input to
functions. Moreover, the structure on these index sets allows us to develop a
natural notion of symmetry particular to the indexing used by the function. Let
$F$ be a $(\tau, A)$-structured function. We say $F$ is \emph{$G$-invariant} if
for all $G$-equivalent $\vec{x},\vec{y} \in \{0,1\}^{X}$, $F(\vec{x}) =
F(\vec{y})$. We say $F$ is \emph{sort-invariant} if it is $(\tau, \sym_{A_{1}}
\times \ldots \times \sym_{A_{s}})$-invariant. If $\tau = (\{R\}, [2], \nu)$,
with $\nu (R) = (1, 2)$ and $F$ is sort-invariant then we say $F$ is
\emph{matrix-invariant}.

% Let $F$ be a $(\tau, A)$-structured function with $\tau = (\{R\}, \{1\},
% \nu)$, where $R$ is a binary relation. Let $G \leq \sym_A$ such that for any
% $v, w \in A$ We say $F$ is \emph{graph-symmetric} if $F$ is $(\tau, A)$simple
% and matrix-symmetric.

% If $\tau$ is single-sorted we call $F$ \emph{simple} and if it consists of a
% single unary relation, we call $F$ \emph{unary-relational}.
We have developed this more general notion of symmetry for Boolean functions
that naturally takes into account the structure on the input sequence. The
following lemma shows that symmetry is just the requirement that the function be
invariant under permutations of its input, irrespective of structure.

\begin{lem}
  Let $F$ be a $(\tau, A)$-structured function. $F$ is a symmetric function if
  and only if $F$ is $\sym_A$-invariant.
\end{lem}



% We briefly introduce notions of equivalence useful for comparing functions.
% \begin{definition}
%   Let $\tau = (R, S, \nu)$ be a many-sorted signature and let $A = A_1 \times
%   \ldots A_s$ and $B = B_1 \times \ldots \times B_s$ be a product of non-empty
%   sets. A \emph{sorted bijection} between $A$ and $B$ is a function bijections
%   $f: A \rightarrow B$ such that $f (A_i) = B_i$ for all $i \in S$. There is
%   an obvious action of $f$ that maps $X^\tau_A$ to $X^\tau_B$ and
%   $\{0,1\}^{X^\tau_A}$ to $\{0,1\}^{X^\tau_B}$.

%   Let $L: X^\tau_A \rightarrow H$ and $L': X^\tau_B \rightarrow H$ be two
%   functions for some finite set $H$. We say that $L \sim L'$, or \emph{$L$ is
%   equivalent to $L'$}, if there is a sorted bijection $f: A \rightarrow B$
%   such that for all $x \in X^\tau_A$, $L(x) = L'(f (x))$.

%   Let $F$ be a $(\tau, A)$-structured function and $G$ be a $(\tau,
%   B)$-structured function. Then we say that $F \sim G$, or \emph{$F$ is
%   equivalent to $G$}, if there is a sorted bijection $f: A rightarrow B$ such
%   that for all $\vec{i} \in \{0,1\}^{X^\tau_A}$, $F (\vec{i}) =
%   G(f(\vec{i}))$.
% \end{definition}

% \begin{remark}
%   There is an obvious connection between $\tau$-symmetric functions and
%   generalised quantifiers (or closed classes of structures). Should I include
%   details about this connection? I also feel a lot of dirtiness might be
%   avoidable if we instead recast everything in terms of generalised
%   quantifiers. For one, generalised quantifiers give a natural way of defining
%   a Boolean function for each input universe.
% \end{remark}

\subsection{Symmetric Circuits}
Having developed the notion of a function that accepts input structured in
accordance with some vocabulary, we now develop a circuit model which
incorporates this structure on the inputs of a gate. We use this general
framework to define the notion of a matrix-symmetric circuit, a circuit with
gates labelled by matrix-symmetric Boolean functions, and finally we develop the
notion of a matrix-symmetric circuit with rank.

% Importantly, many natural functions of interest are matrix symmetric. For
% example, the function that computes the rank of the matrix over
% $\mathbb{F}_2$. or a thresholded rank function, for example the rank of the
% matrix over $\mathbb{F}_p$ being larger then $r$, for some particular $(p, r)
% \in \mathbb{N}$.

\begin{definition}[Circuits on Structures]
  Let $\mathbb{B}$ be a basis of structured functions and $\tau := (R, \{s_1,
  \ldots, s_p\}, \nu)$ be a many-sorted vocabulary and let $\vec{n} := (n_{s_1},
  \ldots , n_{s_p}) \in \mathbb{N}^{S}$, we define a \emph{$(\mathbb{B},
    \tau)$-circuit} $C_{\vec{n}}$ computing a $q$-ary query $Q$ of type
  $(s^Q_{1}, \ldots , s^Q_{q})$ is a structure $\langle G, W, \Omega, \Sigma,
  \Lambda, L\rangle$.
  \begin{itemize}
    \setlength\itemsep{0mm}
  \item $G$ is called the set of gates of $C_{\vec{n}}$ and $\vert C_{\vec{n}}
    \vert := \vert G \vert$.
  \item $W \subseteq G \times G$, where $W$ is called the wires of the circuit.
    $(G,W)$ must be a directed acyclic graph. For $g \in G$ we $H_g := \{ h \in
    G: W(h,g)\}$ be the set of children of $g$.
  \item $\Omega$ is an injective function from $[n_{i_1}] \times \ldots \times
    [n_{i_q}]$ to $G$. The gates in the image of $\Omega$ are called the output
    gates. When $q = 0$, $\Omega$ is a constant function mapping to a single
    output gate.
  \item $\Sigma$ is a function from $G$ to $\mathbb{B} \uplus \tau \uplus
    \{0,1\} $ which maps input gates to $\tau \uplus \{0,1\}$ and where $\vert
    \Sigma^{-1} (0) \vert \leq 1$ $\vert \Sigma^{-1} (1) \vert \leq 1$ and the
    internal gates get mapped into $\mathbb{B}$. Gates mapped to $R$ are called
    relational gates and gates mapped to 1 or 0 are called constant gates.
  \item $\Lambda$ is a sequence of injective functions $(\Lambda_{R'})_{R' \in
      R}$ where for each $R' \in R$ with arity $r$ and type $(s^{R'}_{1},
    \ldots, s^{R'}_{r})$, $\Lambda_{R'}$ maps each relational gate $g$ with $R'
    = \Sigma (g)$ to the tuple $\Lambda_{R'} (g) \in [n_{s^{R'}_1}] \times
    \ldots \times [n_{S^{R'}_r}]$. When no ambiguity arises we write $\Lambda
    (g)$ for $\Lambda_{R'} (g)$.
  \item $L$ maps to each internal gate a labelling on its children. Let $g$ be
    an internal gate. We have that $L(g)$ is a surjection from $\ind (\Sigma (g))$
    to $H_g$. We call $L(g)$ the \emph{child-labelling} for $g$.
  \end{itemize}
\end{definition}

We write $C_{\vec{n}}$ to emphasise that the circuit accepts input structures
with sorted-size $\vec{n}$. If $\tau$ is single-sorted vocabulary then we write
$C_n$ to emphasise that $C_n$ accepts input structures of size $n$. For a gate
$g \in C_{\vec{n}}$ we abuse notation and define the \emph{type} of $g$, denoted
by $\type (g)$, to be type of $\Sigma(g)$, and the \emph{index} of $g$, denoted by
$\ind(g)$, to be the the index for $\Sigma(g)$.

Let $\tau = (\{s_1, \ldots, s_p\}, \{R_1, \ldots , R_t\}, \nu)$. Let
$\mathcal{A}$ be a many-sorted $\tau$-structure with sorts $U_1, \ldots, U_p$
and sort-size $\vec{n}$, and let $\gamma: U \rightarrow [n_{s_1}] \sqcup \ldots
\sqcup [n_{s_p}]$ be a sorted bijection. Let $\gamma \mathcal{A}$ be the
structure formed by mapping the universe of $\mathcal{A}$ in accordance with
$\gamma$. The evaluation of some $(\mathbb{B}, \tau)$-circuit $C$ computing a
q-ary query $Q$ of type $(s^Q_{1}, \ldots , s^Q_{q})$ proceeds by recursively
evaluating gates. The evaluation of the gate $g$ using $\gamma$ with input
$\mathcal{A}$ is denoted by $C[\gamma \mathcal{A}](g)$. The evaluation of a gate
$g$ is given as follows.
\begin{enumerate}
  \setlength\itemsep{0mm}
\item If $g$ is a constant gate then it evaluates to the bit given by
  $\Sigma(g)$.
\item If $g$ is a relational gate then $g$ evaluates to true iff $\gamma
  \mathcal{A} \models \Sigma(g)(\Lambda (g))$.
\item If $g$ is an internal gate labelled by a $(\tau_g, A_g)$-structured
  function $\Sigma(g)$, let $L^{\gamma}_g: \ind(F) \rightarrow \{0,1\}$ be
  defined by $L^{\gamma}_g(x) = C[\gamma \mathcal{A}](L(g)(x))$. Then $g$
  evaluates to true iff $\Sigma(g) (L^{\gamma}_g)$.
\end{enumerate}
$C$ defines the $q$-ary query $Q \subseteq U_{s^Q_1} \times \ldots \times
U_{s^Q_q}$ where $\vec{a} \in Q$ iff $C[\gamma \mathcal{A}](\Omega (\gamma
\vec{a})) = 1$.

The following definition is an important circuit property introduced by Anderson
and Dawar \cite{AndersonD17}. We introduce it for the sake of comparison.

\begin{definition}[Invariant Circuit]
  Let $C_{\vec{n}}$ be a $(\mathbb{B}, \tau)$-circuit, computing some $q$-ary
  query of type $(s^Q_1, \ldots , s^Q_q)$. We say $C_{\vec{n}}$ is
  \emph{invariant} if for every $\tau$-structure $\mathcal{A}$ of size
  $\vec{n}$, $\vec{a} \in U_{i_1} \times \ldots \times U_{i_q}$, and sorted
  bijections $\gamma_1, \gamma_2: U \rightarrow [n_{s_p}] \sqcup\ldots \sqcup
  [n_{s_p}]$ we have that $C[\gamma_1 \mathcal{A}](\Omega (\gamma_1 \vec{a})) =
  C[\gamma_2 \mathcal{A}](\Omega (\gamma_2 \vec{a}))$.
\end{definition}

This property ensures that the function computed by a circuit (or a family of
circuits) is invariant under isomorphisms on the input structure. In other
words, the circuit (or family of circuits) decides a property of the structure.
The following lemma allows us to recast this notion in terms of the language
developed in this paper.

\begin{lem}
  Let $C_n$ be a $(\mathbb{B}, \tau)$-circuit. We have that the function
  computed by $C_n$ is $\tau$-symmetric iff $C_n$ is invariant.
\end{lem}

For the sake of simplicity, from now on we assume the that all circuits take in
structures over a single-sorted vocabulary.

\begin{definition}[Automorphism]
  let $C = \langle G, W, \Omega, \Sigma, \Lambda, L\rangle$ be a
  $(\mathbb{B},\tau)$-circuit computing a $q$-ary query on structures of size
  $n$, and where $\mathbb{B}$ is a bases of sort-symmetric functions. Let
  $\sigma \in \sym_n$ and $\pi: G \rightarrow G$ be a bijection such that
  \begin{itemize}
    \setlength\itemsep{0mm}
  \item for all gates $g, h \in G$, $W(g,h)$ iff $W(\pi g, \pi h)$,
  \item for all output tuples $x \in [n]^q$, $\pi \Omega (x) = \Omega (\sigma
    x)$,
  \item for all gates $g \in G$, let $\Sigma (g) = \Sigma (\pi g)$,
  \item for each relational gate $g \in G$, $\sigma \Lambda (g) = \Lambda (\pi
    g)$, and
  \item for each internal gate $g$ then we have that $L(\pi g)$ and $ \pi \cdot
    L(g)$ are sort-equivalent.
  \end{itemize}
  We call $\pi$ an \emph{automorphism} of $C$, and we say that $\sigma$
  \emph{induces the automorphism} $\pi$. The group of automorphisms of $C$ is
  called $\aut_n (C)$.
\end{definition}

\begin{definition}[Symmetry]
  A circuit $C$ on structures of size $n$ is called \emph{symmetric} if every
  $\sigma \in \sym_n$ induces an automorphism on $C$.
\end{definition}

It follows for any symmetric circuit $C_n$ there is a homomorphism $h$ that maps
$\sym_n$ to $\aut_n(C)$. This homomorphism is injective so long as a single
element of $[n]$ appears in the in the label of some input gate of $C$ (as then
all elements appear by symmetry)\cite{AndersonD17}. In this paper we always
assume that there is at least one such element, as otherwise all inputs are
constant, and so the circuit just computes a constant function. In order to
assure this homomorphism is surjective Anderson and Dawar \cite{AndersonD17}
introduce the notion of a \emph{rigid} circuit.

\begin{definition}[Rigidity]
  Let $C_n$ be a $(\mathbb{B}, \tau)$-circuit, where $C_n = \langle G, W,
  \Omega, \Sigma, \Lambda, L\rangle$. Say that $C_n$ is \emph{rigid} if there
  are no distinct internal gates $g, g' \in G$ such that $\Sigma(g) = \Sigma
  (g')$, $\Omega^{-1}(g) = \Omega^{-1}(g')$, $H_g = H_{g'}$ and $L(g')$ and
  $L(g)$ are sort-equivalent.
\end{definition}

Another property which simplifies our analysis is the property of having
\emph{bijective labels}.

\begin{definition}
  We say that a circuit $C$ has \emph{bijective labels} if for each gate $g$ in
  $C$, $L(g)$ is a bijection.
\end{definition}

We prove in a later section that a circuit may be transformed in polynomial time
into an equivalent circuit that is both rigid and has bijective labelings. Hence
we may assume these two properties without a loss of generality.

With the assumption of rigidity in place, we abuse notation for permutations and
let $\sigma \in \sym_n$ also denote the induced automorphism.

We finally want to define a circuits that allow for gates labelled by
matrix-symmetric functions.

\begin{definition}
  Let $\mathbb{B}$ be a basis containing the single-sorted structured functions
  that compute $\land$, $\lor$, $\neg$ and $\maj$ and such that all elements of
  this basis are either matrix-symmetric or simple unary. We call a
  $(\mathbb{B}, \tau)$-circuit a \emph{symmetric matrix-circuit}.
\end{definition}

We define a circuit that has gates that compute rank.
\begin{definition}
  Let the $(\mathbb{B}, \tau)$-circuit $C$ be a matrix-symmetric circuit. Let
  $\mathbb{B}_m \subseteq \mathbb{B}$ consist of all matrix-symmetric functions
  that are not symmetric. For any non-empty sets $A_1, A_2$ and $r, p \in
  \mathbb{N}$, where $p$ prime define a $F_{p,r}: \{0,1\}^{A_1 \times A_2}
  \rightarrow \{0,1\}$ such that $f_{p,r}(M) = 1$ iff the matrix $M: A_1 \times
  B_1 \rightarrow \{0,1\}$ has rank at least $r$ over the field $\mathbb{F}_p$.
  We call such a function a \emph{bounded-rank function}. If $\mathbb{B}_m$ is
  equal to the set of all such bounded-rank functions, then we call $C$ a
  \emph{symmetric rank-circuit}.
\end{definition}

% The natural restriction to consider on families of circuits is uniformity.

% \begin{definition}
%   Let $(C_n)_{n \in \mathbb{N}}$ be a family of Boolean circuits. We say that
%   $(C_n)_{n \in \mathbb{N}}$ is \emph{$P$-uniform} if the mapping $n \mapsto
%   C_n$ is computable in polynomial time.
% \end{definition}

We are now ready to state what will be the main theorem of this paper.
\begin{thm}[Main Theorem]
  A graph property is decidable by a $P$-uniform family of symmetric circuits
  with rank gates if, and only if, it is definable by an $\FPR$ sentence.
\end{thm}

\subsection{Limitations of Symmetric Bases}
When looking at symmetric circuits with matrix symmetric functions, it is
natural to ask if this extension to matrix-symmetric functions is really
necessary. In this section we show that Symmetric circuits, as developed by
Anderson and Dawar, defined over the usual basis with majority gates can compute
any symmetric function over polynomial size circuits. As such no additional
symmetric function added to the Basis improves the power of the model.

Recall from Anderson and Dawar \cite{AndersonD17} we have that
$\mathbb{B}_{\std} = \{ \neg , \wedge , \lor \}$ and $\mathbb{B}_\maj = \{ \maj
\} \cup \mathbb{B}_{\std}$.

Let $F = (F_n: \{0,1\}^n \rightarrow \{0,1\})_{n \in \mathbb{N}}$ be a family of
symmetric Boolean functions. Recall that for any $n \in \mathbb{N}$, the output
of $F_n$ is entirely determined by the number of 1's in its input. Then let
$c_{F}:\mathbb{N} \rightarrow 2^{\mathbb{N}}$ define a function where $c_{F}(n)$
is the set of all $m \leq n$ such that for all $\vec{x} \in \{ 0,1 \}^n$ with
$m$ 1's we have $F_n (\vec{x}) = 1$. Clearly any symmetric Boolean function
$F_n$ is entirely determined by $c_{F}(n)$, and so $F$ is entirely determined by
$C_F$.
 
\begin{prop}
  \label{prop:fuctions-maj}
  Let $F = (F_n: \{0,1\}^n \rightarrow \{0,1\})_{n \in \mathbb{N}}$ be a family
  of symmetric functions. There exists a family of Boolean circuits $(C_n)_{n
    \in \mathbb{N}}$ defined over $\mathbb{B}_\maj$ computing $F$. Moreover,
  $(C_n)_{n \in \nats}$ has depth at most $5$ and width at most $2n+2$ and size
  at most $8n + 3$.
\end{prop}

\begin{proof}
  We define the circuit $C_n$ with $n$ input gates denoted by $x_1, \ldots, x_k$
  by successively adding gates. For $a \in \mathbb{N}$ we add an `and' gate
  $\countgate_a$, and a number of `majority' and `not' gates, wiring them up as
  follows:
\[
  \countgate_a = \begin{cases} \land ( \maj  ( x_1, \ldots, x_n, \underbrace {0, \ldots,
        0}_{2a - n} ), \neg ( \maj (x_1, \ldots , x_n, \underbrace{0, \ldots,
        0}_{2a - n + 2} ) )) &  a \geq \frac{n}{2} \\
     \land ( \maj ( x_1, \ldots, x_n, \underbrace {1, \ldots, 1}_{n -
           2a} ),  \neg (
      \maj ( x_1, \ldots , x_n, \underbrace{1, \ldots, 1}_{n - 2a -2}
      ) )) & a <
      \frac{n}{2}.
    \end{cases}
  \]
  
  Clearly $\vert c_F(n) \vert \leq n$, for each $a \in c_F(n)$ we have added
  four gates, so thus far the circuit's width is at most $2n + 2$ (2 extra for
  the possible inclusion of constant gates) and depth at most $4$. Now add to
  $C_n$ a `or' gate $g$ and connect the output of $\countgate_a$ to the input of
  $g$ for each $a \in c_F(n)$. Mark $g$ as the output gate for $C_n$.

  It is easy to see that $C_n$ has depth at most $5$, width at most by $2n + 2$
  and size at most $1 + 4(2n) + 2 = 8n+3$. We have that in each layer of $C_{g'}$
  each gate is connected to all gates in the previous layer, and as such the
  circuit is symmetric.
\end{proof}

The above proposition has a straight forward application to circuit
characterisations.

\begin{thm}
  Let $\mathcal{F} = \{F_i : i \in I \}$, where $F_i$ is a family of symmetric
  Boolean functions.
  
  Let $(C_n)_{n \in \mathbb{N}}$ be family of symmetric circuits over the
  Boolean basis $\mathbb{B}_{\std} \cup (\bigcup_{i \in I} F_i)$, where $C_n$ is
  a circuit on structures of size $n$, and the size of each circuit in the
  family is bounded by some function $f(n)$. Then there exists a family of
  symmetric circuits $(C_n')_{n \in \mathbb{N}}$ over $\mathbb{B}_\maj$, where
  $C_n'$ is a circuit on structures of size $n$ and $\vert C_n' \vert \leq
  (8f(n) + 3) f(n))$.
\end{thm}

\begin{proof}
  From $C_n$ we construct $C_n'$ as follows. For each gate $g \in C_n$ of
  labelled by a member of $F_i$ we have a symmetric circuit $C_g$ from
  Proposition \ref{prop:function-maj} that computes the same function as $g$.
  Then let $C_n'$ be $C_n$ but with each gate $g$ replaced by $C_g$. It is easy
  to see that $C_n'$ is symmetric. We also have that each gate $g$ must have at
  most $f(n)$ inputs, and so the size of $C_g$ is bounded by $8f(n) + 3$. Thus
  the size of $C_n'$ is bounded by $(8f(n)+3) f(n)$.
\end{proof}

This result gives us that for any family of circuits over an arbitrary basis of
symmetric functions we can can construct another circuit family computing the
same function over the majority basis without a blowup in size. As such, we know
that we cannot extend the power of the symmetric circuits studied by Anderson
and Dawar \cite{AndersonD17} by simply considering alternative bases of
symmetric functions. This result, combined with the main result of this paper,
gives us that symmetric circuits with rank are a strictly more powerful model
than any circuit model defined over a basis of symmetric functions.

\end{document}

