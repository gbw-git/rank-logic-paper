\documentclass[../paper.tex]{subfiles}
\begin{document}
% In contrast with the work of Anderson and Dawar~\cite{AndersonD17}, the
% circuit families we use to establish the main result of this paper are
% additionally required to be transparent. In this subsection we briefly discuss
% this condition, and present results suggesting its necessity.

In the statement of the main theorem of this paper, rather than working with
general symmetric rank-circuits, we instead restrict ourselves to families of
\emph{transparent} circuits. In this section we justify this restriction,
showing first that the a great number of very useful circuit properties (e.g.
syntactic-equivalence) are polynomial-time decidable for transparent circuits
and, second, that deciding most of these circuit properties for general circuits
(with non-symmetric gates) is at least as hard as the graph-isomorphism
problem. Since the proof of the main theorem makes use of the polynomial-time
decidability of many of these properties, these results together justify the
inclusion of the transparency restriction.
% We note additionally that the circuit properties that we show

% In this section we discuss the transparency condition on circuits. It is worth
% noting that, since the symmetric circuits of Anderson and
% Dawar~\cite{AndersonD17} have no non-symmetric internal gates, these circuits
% are transparent. We have from the first part of this subsection that a great
% number of useful properties (and transformations) are polynomial-time
% computable on transparent circuits. However, it is still reasonable to ask if
% the explicit inclusion of the transparency restriction is strictly necessary
% for proving these results, or indeed proving the main result of this paper. We
% provide evidence for necessity of the transparency condition, presenting
% reductions from the graph-isomorphism problem to the problem of deciding some
% of this important properties (e.g. symmetry, syntactic-equivalence, computing
% the orbit of a gate, etc.) over general, not necessarily transparent,
% circuits. We note that in all of these cases the decision problem restricted
% to transparent circuits is polynomial-time decidable.

% We further argue that, since most natural algorithms for transforming an
% arbitrary symmetric circuit into an equivalent transparent circuit make use of
% the polynomial-time decidability of these properties, we should consider these
% hardness results evidence that computing this transformation may be as hard as
% the graph-isomorphism problem
% We also prove that a number of related decision problems (e.g. synatcatic)
% important decision problems are reducible to the graph-isomorphism
% xthat the problem of deciding the orbit of a gate is reduca
% why (or whether) the transparency condition is strictly required when
% considering circuits that contain non-symmetric gates. In the second
% subsection we address this question, proving



% when considering circuits

% It is worth noting that the analogous results of Anderson and Dawar~\ref{}
% make no explicit restriction to transparent circuits,


% mentio It is worth asking if this additional transparency requirement is
% structing It is worth noting that Anderson and Dawar~\ref{} when developing
% their analogous translation, define a polynomial-time algorithm that
% transforms symmetric circuits into equivalent circuits with unique extensions
% -- making no mention of any transparency condition. Of course, the circuits
% they consider contain only symmetric internal gates, and as such are
% transparent already, but we may still ask if the transparency restriction is
% strictly necessary for proving the useful results mentioned above.





% It is worth noting that the translation defined by Anderson and Dawar~\cite{}
% between $P$-uniform families of symmetric circuits (with symmetric gates) and
% formulas of $\FPC$ makes no

% In the second subsection we discuss the necessity of the transparency
% requirement. Anderson and Dawar~\cite{AndersonD17} importance of this trans


% into In the main result of this paper we restrict ourselves to transparent
% circuits, a restriction not found in the analogous result of Anderson and
% Dawar~\cite{}. In this section we develop a number of technical that are
% important for defining and developing the basic technical tools used in this
% paper.



% Anderson and Dawar~\cite{} develop an correspondence between symmetric
% circuits (with symmetric gates) and formulas of fixed point with counting. In
% contrast, in the main theorem of this paper we explicitly require that the
% circuits under discussion be \emph{transparent}. This section we develop a
% number of results about transparent circuits. In the first subsection we
% discuss


% In this section we present a number of useful results for transparent circuits
% that together enable our analysis

% This condition does not appear explicitly in the analogous result of Anderson
% and Dawar~\cite{}, and in this section we discuss the importance of this
% requirement on circuits with non-symmetric gates.

% In the first subsection we demonstrate the usefulness of transparency. We
% begin by showing that we may transform transparent circuits into circuits with
% unique extensions in polynomial-time, allowing us assume, without a loss of
% generality, that $P$-uniform families of transparent circuits also have unique
% labels, and hence (from Proposition~\ref{}) unique extensions. We are also
% able to show that if a circuit has unique labels, which transparency allows us
% to assume, then many natural circuit properties, including the action of a
% permutation on the circuit, as well as the orbit and coarsest supporting
% partition of a gate, are polynomial-time computable.

% These results suggest the importance fundamental to our analysis of circuits.
% The ability to assume that the circuits under discussion have unique
% extensions (and unique labels) allows us to define many important theoretical
% tools, such as a canonical action of $\sym_n$ on the circuit and hence the
% notions of support and supporting partition. Moreover, the polynomial-time
% algorithms for computing the orbit and support of a gate play an important
% role (through the Immerman-Vardi Theorem~\cite{}) in our translation of
% circuits into formulas (see Section \ref{})


% we also show that if a circuit has unique labels (which transparency allows us
% to assume) then there exists polynomial-time algorithms for computing the
% action of $\sym_n$ on the circuit, and the canonical supporting partition and
% the orbit a gate (or elements of a universe of a gate). These algorithms play
% (along with the


% polynomial-time algorithms are thus crucial for developing our translation
% from circuits to formulas.


% and prove the existence of polynomial-time algorithms computing important
% functions, such as the coarsest supporting partition or the orbit of a gate
% (see Proposition~\ref{prop:c} and Lemma~\ref{}).

% Thus the transparency condition enables our analysis of circuits by allowing
% us to assume the circuits also have unique extensions. However, Anderson and
% Dawar~\cite{AndersonD17} prove that symmetric circuit can be transformed into
% circuits with unique extensions,without making any mention of any condition
% analogous to transparency. Of course, our result generalises theirs as they
% only consider circuits with symmetric gates, which are transparent, but it is
% reasonable to ask if the transparency requirement on circuits is strictly
% necessary.

% We address this question in this subsection, presenting reductions
% graph-isomorphism problem to the many related and important problems. We show,
% for example, that there is a polynomial-time many-one reduction from
% graph-isomorphism problem to the problem of deciding the orbit of a gate in an
% arbitrary symmetric circuit, and contrast this result with the known
% polynomial-time algorithm computing orbits of gates in transparent circuits.
% We also present reductions from the graph-isomorphism problem to many related
% problems (e.g. checking syntactic-equivalence or unique label for gates).
% Since most natural algorithms for transforming any an arbitrary symmetric
% circuit into an equivalent family of transparent circuits entail solving these
% related problems, we consider this evidence that computing this transformation
% may be as hard as the graph isomorphism problem.

% Moreover, we show that non-transparent circuits

% an ay be as hard as the graph isomorphism problem. However, we

% show that that being able to transform a circuit into an equivalent
% transparent circuit in polynomial-time would




% make no mention In contrast, Anderson and Dawar~\ref{} do not include any
% specific condition on the circuit families they consider


% consider only symmetric circuits with symmetric gates, which are transparent,
% and as such include no such condition.


% make use of an explicit condition on circuits we call \emph{transparency}. In
% this section we show that transparency is a sufficient condition in that it
% allows us to assume, without a loss of generality, that $P$-uniform families
% of In this subsection we show that if a circuit is transparent then it may be
% transformed in polynomial time into an equivalent circuit with unique labels.
% It follows from Proposition~\ref{prop:unique-labels-unique-extensions}) that
% we may assume, without a loss of generality, that $P$-uniform families of
% transparent circuits have unique labels, and hence unique extensions.

% We also discuss the importance of the transparency condition on circuits
% introduced in this paper. Indeed, while Anderson and Dawar~\cite{AndersonD17},
% place no similar requirement on the symmetric circuits in their paper, in our
% generalisation of their result, transparency is required.

% We also discuss relevant special cases and motivate the explicit requirement
% that the $P$-uniform families of circuit referenced in the main result of this
% paper consist of transparent circuits.

% Anderson and Dawar~\cite{AndersonD17} introduce the notion of a `rigid'
% circuit, a condition analogous to the rigidity condition defined in this
% paper. The following

% \begin{definition}
%   We say a circuit $C_n$ is que labels it has unique extensions. We finally
%   discuss th\emph{rigid} if it has injective labels and for every pair of
%   gates $(g, g')$ in $C_n$ such that $H_g = H_{g'}$ and $g \equiv g'$.
% \end{definition}
\subsection{The Usefulness of Transparency}
In the this subsection we show that a number of important circuit properties are
polynomial-time decidable for transparent circuits (e.g. symmetry, the
syntactic-equivalence relation, etc.). We also provide polynomial-time
algorithms for computing the action of a permutation on a symmetric circuit, as
well as for computing the orbits and supports of gates, and elements of
universes of gates, in a circuit.

We use the polynomial-time decidability of many of these properties throughout
this paper, and particularly for our translation of circuits into formulas. For
example, the polynomial-time decidability of the syntactic-equivalence relation
enables us to transform transparent circuits into circuits with unique labels
(and so unique extensions) in polynomial-time. This result allows us, when
working with $\PT$-uniform families of transparent circuits, to assume without a
loss of generality that these circuits have unique labels and unique extensions.
This is a crucial step in the proof of our main result, as the unique labels
requirement is needed to apply the support theorem, and the unique extensions
property is necessary for applying (and even defining) a great many technical tools (e.g. supports
and supporting partitions) used extensively throughout this paper.

We now show that the syntactic-equivalence relation can be computed in
polynomial time for transparent circuits.

% \begin{definition}
%   We say a class of circuits $\mathcal{C}$ is \emph{syntactically transparent}
%   if there is an algorithm that takes in a circuit $C \in \mathcal{C}$ and
%   outputs the syntactic equivalence relation on the gates of $C$ and runs in
%   time polynomial in the size of the input circuit.
% \end{definition}

\begin{lem}
  There is an algorithm that takes as input a transparent circuit $C$ and
  outputs the syntactic-equivalence relation on the gates of $C$. The algorithm
  runs in time polynomial in the size of $C$.
  \label{lem:unique-labels-syntactic-equiv}
\end{lem}
\begin{proof}
  Let $C := \langle G, \Omega, \Sigma, \Lambda, L \rangle$ be a transparent
  $(\mathbb{B}, \rho)$-circuit of order $n$. We define a relation $\sim$ by
  induction on height (i.e. distance from an input gate). This relation will be
  the syntactic-equivalence relation on $G$. It is easy to check the conditions
  for syntactic-equivalence for gates of height 0 as all such gates are input
  gates and so are either constant gates or relational gates. Suppose $g$ and
  $h$ are gates of height greater than zero. We only need to check conditions
  (i), (iv) and (v) of syntactic-equivalence. It is easy to check (i) and (v).
  If either condition is false then $g$ and $h$ are not syntacticly equivalent.
  So suppose $g$ and $h$ are two internal gates such that $\Sigma (g) =
  \Sigma(h)$ and suppose that if they are both output gates then $\Omega^{-1}(g)
  = \Omega^{-1}(h)$. It remains to check if $L(g)/{\sim}$ and $L(h)/{\sim}$ are
  isomorphism equivalent. First check if $H_h /{\sim} = H_g /{\sim}$. If not,
  $g$ and $h$ are not syntactically equivalent. So suppose $H_h /{\sim} = H_g
  /{\sim}$. If $g$ and $h$ are symmetric gates then $H_h /{\sim} = H_g/{\sim}$
  if, and only if, $L(g)/{\sim}$ and $L(h)/{\sim}$ are isomorphism equivalent,
  so let $g \sim h$. If $g$ and $h$ are not symmetric gates then, since $C$ is
  transparent, $g$ and $h$ have unique labels. We thus have that $L(g)/{\sim}$
  and $L(h)/{\sim}$ are isomorphism equivalent if, and only if, $L(g)^{-1}L(h)$
  acts like an isomorphism. This can be checked by first checking that the
  function induces a sorted permutation on the universe of $g$ (which can be
  checked efficiently). This is sufficient as $L(g)(L(g)^{-1}L(h)) = L(h)$.

  It is easy to see that, given a reasonable binary encoding of a circuit, the
  above procedure can be implemented so as to run in polynomial time, and that
  the relation $\sim$ is the syntactic-equivalence relation on the gates of $C$.
 
\end{proof}

We now prove that transparent circuits can be transformed in polynomial time
into equivalent circuits with unique extensions, and that this transformation
preserves important properties of the circuit.

\begin{lem}
  Let $C := \langle G, \Omega, \Sigma, \Lambda, L \rangle$ be a transparent
  $(\mathbb{B}, \rho)$-circuit. There is an algorithm that that takes in such a
  circuit and outputs a $(\mathbb{B} \cup \mathbb{B}_{\std}, \rho)$-circuit $C'$
  such that $C$ and $C'$ compute the same function, $C'$ has unique labels, and
  if $C$ is symmetric then $C'$ is symmetric. Moreover, this algorithm runs in
  time polynomial in the size of the input circuit.
  \label{lem:transparent-unique}
\end{lem}
\begin{proof}
  Let $C' := \op{make-injective-all}(\op{merge-all} (C, G))$ (see
  Definition~\ref{def:make-injective} and~\ref{def:merge-all} for the
  definitions of these functions). It follows that if $C$ is symmetric then $C'$
  is symmetric. We have from Lemma~\ref{lem:unique-labels-syntactic-equiv} that
  the syntactic-equivalence relation can be computed in polynomial time for
  transparent circuits. It follows that the construction of $C'$ from $C$ can be
  implemented so as to run in time polynomial in the size of $C$. Clearly every
  gate $g$ in $C'$ belongs to a singleton syntactic-equivalence equivalence
  class, and as such $C'$ has unique labels.
\end{proof}
% GW: read this when fresh

We now show that it is possible to compute the action of $\sym_n$ on the gates
of a circuit of order $n$ with unique labels in polynomial time. Recall that
from Proposition~\ref{prop:transparent-unique} we may, without a loss of
generality, restrict of attention from transparent circuits to circuits with
unique labels. The proof technique used here is similar to the one used by
Anderson and Dawar~\cite{AndersonD17}.

\begin{lem}
  There is an algorithm takes as input a $(\mathbb{B}, \rho)$-circuit $C$ of
  order $n$ with unique labels and $\sigma \in \sym_n$ and outputs for each gate
  $g$ its image under the unique automorphism extending $\sigma$ (if it exists).
  This algorithm runs in time polynomial in the combined size of the input
  circuit and the encoding of the permutation.
  \label{lem:compute-automorphisms}
\end{lem}
\begin{proof}
  Let $C := \langle G, \Omega, \Sigma, \Lambda, L \rangle$. We recursively build
  up the mapping $\pi$ extending $\sigma$. If at some point in the recursive
  construction we arrive at a point where no mapping for $g$ can be found we
  halt at that point and return that no automorphism exists.

  Let $h$ be any gate in the circuit. Suppose $h$ is an input gate. If $h$ is a
  constant gate then let $\pi (h) = h'$. If $h$ is a relational gate such that
  $R := \Sigma(h)$, then check if there exists $h'$ such that $\Sigma (h') = R$,
  $\sigma \Lambda_R(h) = \Lambda_R(h')$, and either both $h$ and $h'$ are output
  gates or neither are. If no such $h'$ exists then halt and output that no
  automorphism exists. If both $h$ and $h'$ are not output gates then set $\pi
  (h) = h'$. If both are output gates then check if $\sigma \Omega^{-1}(h) =
  \Omega^{-1}(h')$. If equality holds set $\pi(h) = h'$, otherwise halt and
  output that no automorphism exists. We have from Lemma~\ref{} that there is at
  most one gate $h'$ that meets the above criteria, and so $\pi(h)$ is well
  defined.

  Let $h$ be an internal gate in the circuit and assume we have defined $\pi
  (g)$ for every gate $g$ of height less than $h$. Let $h'$ be a gate in the
  circuit such that $\Sigma(h) = \Sigma (h')$, $\pi L(h)$ is
  isomorphism-equivalent to $L(h')$ and, if $h$ is an output gate then $h'$ is
  an output gate such that $\sigma \Omega^{-1}(h) = \Omega^{-1}(h')$. Since $C$
  has unique labels (and so unique extensions) we have that $\pi L(h)$ is
  isomorphism-equivalent to $L(h')$ if, and only if, $L(h')^{-1}\pi L(h)$ acts
  on $\ind(h)$ like an automorphism of $\str{h}$. This can be determined easily.
  If no such $h'$ exists halt and output that there is no automorphism extending
  $\sigma$. Again, we have that if $C$ has unique labels then there is at most
  one such $h'$.

  This recursive approach can be implemented as an algorithm that runs in time
  polynomial in the combined size of the inputs and outputs the required
  automorphism if it exists.
\end{proof}

We now that, using the algorithm from Lemma~\ref{lem:compute-automorphisms}, we
can similarly compute the action of $\sym_n$ on the elements of the universes of
the gates of a circuit of order $n$ in polynomial time.

\begin{lem}
  There is an algorithm takes as input a $(\mathbb{B}, \rho)$-circuit $C$ of
  order $n$, $\sigma \in \sym_n$, $g$ a gate in $C$, and $a \in \universe{g}$
  and, if there exists an automorphism of $C$ extending $\sigma$ such that
  $\sigma in \stab(g)$, outputs image of $a$ under $\sigma$. The algorithm runs
  in time polynomial in the size of $C$ and the encoding of $\sigma$.
  \label{lem:compute-automorphisms-labels}
\end{lem}
\begin{proof}
  Let $C = \langle G, \Omega, \Sigma, \Lambda, L \rangle$. Use the algorithm
  from Lemma \ref{lem:compute-automorphisms} to check if $\sigma$ extends to an
  automorphism on $C$ and $\sigma \in \stab(g)$. If not, return that no such
  automorphism exists. We thus have that $C$ is symmetric. Let $h \in H_g$ and
  $\vec{a} := L(g)^{-1}(h)$ be such that $a \in \vec{a}$, and let $i$ be the
  index of $a$ in $\vec{a}$. Output $\sigma \cdot a := L(g)^{-1}(\sigma h)(i)$.
\end{proof}

We now show that it is possible to compute the orbits and canonical supporting
partitions of gates in circuits with unique labels in polynomial-time. We also
show that it is possible to compute the orbits and canonical supporting
partitions of elements on the universes of gates in circuits with unique labels
in polynomial-time. In order to prove both of these results we first prove a
more general result for group actions.

\begin{lem}
  \label{lem:computing-support-orbit}
  There is an algorithm that takes in a subset $S \subseteq [n]$ and $x \in X$,
  where $X$ is some finite set such that there is a group action of $\stab(S)$
  on $X$ computable in time polynomial in both $n$ and $\vert X \vert$ and
  outputs $\orb_{\stab(S)} (x)$ and $\SP_{\stab(S)}(x)$. This algorithm runs in
  time polynomial in both $\vert X \vert$ and $n$.
\end{lem}
\begin{proof}
  Let $(u, v) \in \sym_{[n] \setminus S}$ be a transposition. We note that
  $(u,v) \in \stab(S)$ and there are ${n - \vert S \vert}\choose{2}$ many such
  transpositions. For a transposition $(u, v) \in \sym_{[n] \setminus S}$, let
  \begin{align*}
    \mathcal{P}_{(u,v)} := \{ \{u,v\}\} \bigcup_{w \in [n] \setminus \{ u,v \}} \{ \{ w \} \}
  \end{align*}
  be a partition of $[n]$. Then $\mathcal{P}_{(u,v)}$ supports
  $\stab_{\stab(S)}(x)$ if, and only if, $(u,v) \cdot x = x$.

  Let $\mathcal{P}$ be the partition formed by applying $\mathcal{E}$ to all
  $\mathcal{P}_{(u,v)}$ such that $(u,v) \cdot x = x$ (which can be checked in
  time polynomial in $\vert X \vert$ and $n$). From
  Proposition~\ref{prop:combining-supporting-patitions} we have that
  $\mathcal{P}$ supports $\stab_{\stab(S)}(x)$. Suppose that $\mathcal{P}$ is
  not the coarsest supporting partition of $\stab_{\stab(S)}(x)$. Then there
  exists a partition $\mathcal{P}'$ supporting $\stab_{\stab(S)}(x)$ such that
  $\mathcal{P}' \preceq P$ and $\mathcal{P}' \neq \mathcal{P}$. And so there
  exists $P \in \mathcal{P}$ and $P' \in \mathcal{P}'$ such that $P \subsetneq
  P'$. But then there exists $a , b \in P'$ such that $a \not\in P$. Note that
  $(a,b)$ fixes $\mathcal{P}'$ and, since $\mathcal{P}'$ supports
  $\stab_{\stab(S)}(x)$, it follows that $(a,b) \in \stab_{\stab(S)}(x)$ and $a
  \not\in S$ and $b \not\in S$. But then we have that $(a,b) \cdot x = x$, and
  so $\mathcal{P}_{(a,b)}$ supports $\stab_{\stab(S))}(x)$ and thus, from the
  construction of $\mathcal{P}$, $\mathcal{P}$ is fixed by $(a,b)$. But we
  selected $a$ and $b$ such that $\mathcal{P}$ is not fixed by $(a,b)$, and so
  we have a contradiction. We thus have that $\mathcal{P}$ is the coarsest
  supporting partition of $\stab_{\stab(S)}(x)$.

  It remains to compute $\orb_{\stab(S)}(x)$. Let $M_0 := \{x\}$ and for each $i
  \geq 0$ let $M_{i+1} := M_i\cup \bigcup_{(u,v) \in \sym_{[n] \setminus S}}
  ((u,v) \cdot M_i)$. Let $M$ be the least fixed point of this sequence. It is
  easy to see that $M \subseteq \orb_{\stab(S)}(x)$ as every element of $M$ is
  equal to the action of some finite sequence of transpositions acting on $x$.
  Moreover, if $y \in \orb_{\stab(S)}(x)$, then there exists $\pi \in \stab(S)$
  such that $y = \pi \cdot x$. But then, since $\sym_{[n] \setminus S}$, is
  generated by the set of all transpositions in $\sym_{[n] \setminus S}$, it
  follows that $\pi$ can be written as a sequence of $t$ transpositions (for
  some $t \in \nats$). Thus $y \in M_t \subseteq M$, and hence
  $\orb_{\stab(S)}(x) \subseteq M$, and so $\orb_{\stab(S)}(x) = M$.

  Note that the set of all transpositions in $\sym_{[n] \setminus S}$ can be
  computed in time $\mathcal{O}(n^{2})$ and whether a transposition fixes $x$
  can be computed in time polynomial in both $\vert X \vert$ and $n$. Moreover,
  since it is easy to show that $\mathcal{E}$ can be computed in time polynomial
  in $n$, it follows that $\mathcal{P}$ (defined above) can be computed in
  $\mathcal{O} (n^{2} q(\vert X \vert, n))$, for some polynomial $q$.

  Furthermore, since, when computing the orbit, the least fixed point must be
  reached in fewer than $\vert X \vert$ iterations, and each iteration requires
  at most $\mathcal{O} (n^{2})$ applications of the group action, it follows
  that the procedure terminates in time $\mathcal{O}(\vert X \vert n^{2} q'
  (\vert X \vert, n))$, for some polynomial $q'$. The result follows.
\end{proof}

We now use Lemma~\ref{lem:computing-support-orbit} to prove the result for the
orbits and canonical supporting partitions of gates in the circuit.

\begin{lem}
  There is an algorithm that takes in a circuit $C$ of order $n$ with unique
  labels and outputs if the circuit is symmetric. If it is symmetric then it
  outputs the orbit and coarsest supporting partition of every gate. This
  algorithm runs in time polynomial in the size of the circuit.
  \label{lem:computing-support-orbit-gate}
\end{lem}

\begin{proof}
  For each transposition $(u,v) \in \sym_n$ and each gate $g$ in $C$ use
  Lemma~\ref{lem:compute-automorphisms} to check if the image of $g$ under
  $(u,v)$ exists. If for any transposition and gate the algorithm returns that
  no image exists then output that the circuit is not symmetric.

  We note that if all gates have an image under all transpositions then, since
  $\sym_n$ is generated by the set of transpositions, we have that $C$ is
  symmetric.
  
  We have from Lemma~\ref{lem:compute-automorphisms} that computing the action
  of $\sym_n$ on the gates of $C$ can be done in time polynomial in the size of
  the circuit. For each gate $g$ in $C$ we run the algorithm from
  Lemma~\ref{lem:computing-support-orbit} with $S := \emptyset$, $X := G$ (the
  set of gates in C), and $x := g$, and output the results.
  
  Since there are ${{n}\choose{2}} \leq n^2$ transpositions in $\sym_n$, the
  initial symmetry check can be completed in time polynomial in the size of the
  circuit. Moreover, from Lemmas~\ref{lem:compute-automorphisms} and
  \ref{lem:computing-support-orbit}, and the fact that $\vert C \vert \geq n$
  and $\vert C \vert \geq \vert \universe{g} \vert$, we have that the rest of
  the algorithm also runs in time polynomial in the size of the circuit.
\end{proof}

We now use Lemma~\ref{lem:computing-support-orbit} to prove the result for the
orbits and canonical supporting partitions of elements of the universes of gates
in the circuit.

\begin{lem}
  There is an algorithm that takes in a circuit $C$ of order $n$ with unique
  labels, a gate $g$ in $C$ with small support, and $a \in \universe{g}$, and
  outputs if the circuit is symmetric. If $C$ is symmetric it outputs the orbit
  $\orb_{\consp(g)} (a)$ and coarsest supporting partition $\SP_{\consp(g)}(a)$.
  This algorithm runs in time polynomial in the size of the circuit.
  \label{lem:computing-support-orbit-index}
\end{lem}
\begin{proof}
  Use the algorithm from Lemma~\ref{lem:computing-support-orbit-gate} in order
  to compute the canonical support of $g$. If the algorithm returns that $C$ is
  not symmetric, output that $C$ is not symmetric.

  Use the algorithm from Lemma~\ref{lem:computing-support-orbit}, with $S :=
  \consp(g)$, $X := \universe{g}$ and $x:= a$, and output the results.

  Notice that, since $\vert \universe{g} \vert \leq \vert C \vert$ and $n \leq
  \vert C \vert$, this algorithm runs in polynomial time.
\end{proof}

Anderson and Dawar~\cite{AndersonD17}, in their study of circuits with symmetric
gates, show that such circuits may be transformed in polynomial time into
equivalent circuits with unique extensions. They do this by showing that a
circuit with symmetric gates may be transformed into an equivalent \emph{rigid}
circuit in polynomial time, and that all rigid circuits have unique extensions.
Moreover, they also prove the existence of polynomial-time algorithms computing
the action of a permutation on a rigid circuit, and the orbits and coarsest
supporting partitions of gates in a rigid circuits. We have shown that
transparent circuits can be translated into equivalent circuits with unique
labels in polynomial time, and that circuits with unique labels have unique
extensions (see Proposition~\ref{}). Moreover, we have similarly shown that, for
transparent circuits, the action of a permutation, as well as the coarsest
supporting partition and orbit of a gate (or element of the universe of a gate),
can be computed in polynomial-time. Since all circuits with symmetric gates are
transparent, this is direct generalisation of the results of Anderson an Dawar.

\subsection{The Necessity of Transparency}
% The reader will notice that while we have included a specific requirement that
% each non-symmetric internal gate in a circuit have unique labels (i.e. that
% the circuit be transparent) no similar condition In this section we discuss
% the transparency condition on circuits. It is worth noting that, since the
% symmetric circuits of Anderson and Dawar~\cite{AndersonD17} have no
% non-symmetric internal gates, these circuits are transparent.

% we provide evidence f (which from lemma~\ref{} may be assumed to have unique
% extensions)or necessity of the transparency condition, presenting reductions
% from the graph-isomorphism problem to the problem of deciding some of this
% important properties (e.g. symmetry, syntactic-equivalence, computing the
% orbit of a gate, etc.) over general, not necessarily transparent, circuits. we
% note that in all of these cases the decision problem restricted to transparent
% circuits is polynomial-time decidable.

% we further argue that, since most natural algorithms for transforming an
% arbitrary symmetric circuit into an equivalent transparent circuit make use of
% the polynomial-time decidability of these properties, we should consider these
% hardness results evidence that computing this transformation may be as hard as
% the graph-isomorphism problem.
We have from the first subsection that a great number of useful properties are
polynomial-time computable on transparent circuits. However, it is still
reasonable to ask if the explicit inclusion of the transparency restriction is
strictly necessary for proving these results. In this subsection we present a
number of results suggesting the importance of the transparency requirement. In
particular, we present reductions from the graph-isomorphism problem to many
important decision problems addressed in the first subsection (e.g. deciding if
a circuit is symmetric, if a gate has unique labels, if two gates are
syntactically-equivalent, of if two gates are in the same orbit).

These results together strongly suggest the necessity of the transparency
condition for enabling the polynomial-time decidability of many basic properties
of circuits with non-symmetric gates. Moreover, while we do not show that there
is no polynomial-time computable transformation from a general circuit to an
equivalent transparent circuit, the difficulty associated with computing basic
circuit properties that seem essential for defining such a transformation
provide evidence in favour of this conjecture. (Indeed, the translation defined
by Anderson and Dawar~\cite{AndersonD17} from circuits with symmetric gates to
equivalent circuits with unique extensions makes explicit use of the
polynomial-time computability of syntactic-equivalence on such circuits).

As a final point, we note that while we present evidence against the ability to
transform arbitrary circuits into transparent circuits in polynomial-time, we do
show that the class of transparent circuits is polynomial-time decidable. In contrast, we show
that deciding the class of circuits with unique extensions is at least as hard
as the graph-isomorphism problem.

% We particularly note the hardness results for syntactic-equivalence note The
% difficulty of computing the orbit for non-transparent circuits is particularly
% important to note, as we make use of the existence of the polynomial-time
% decidability of the orbit in the proof of our main result.

% In particular, we show that the graph-isomorphism problem reduces to the
% problem of computing the syntactic-equivalence relation for general circuits.

% As we noted, Anderson and Dawar~\cite{AndersonD17} show that circuits (with
% symmetric gates) can be transformed in polynomial time into equivalent rigid
% circuits (in fact, their algorithm transforms these circuits into equivalent
% circuits with unique labels, a stronger condition than rigidity). Their proof
% of this result makes explicit use of the fact that the syntactic-equivalence
% relation on the gates of these circuits can be computed in polynomial-time






% These results together suggest that the a polynomial-time definable



% use of to compute the orbit is a particularly important limitation, as we make
% use of a polynomial-time algorithm for deciding the orbit of a gate (or
% element of a universe of a gate) in our proof of the main result.

% Moreover, we also show that

% However, if we allow gates that compute non-symmetric functions, it is not
% clear that the syntactic-equivalence relation remains polynomial-time
% decidable.


% . prove results analogous to Lemma~\cite{} The reader will notice that, while
% we require that the circuit families in our main result consist of transparent
% circuits, no similar explicit condition is introduced in Anderson and
% Dawar~\cite{}.


% When Anderson and Dawar~\cite{} prove these results for symmetric circuits
% (with symmetric gates) they explicitly use the fact that the syntactic
% equivalence relation can be computed in polynomial time.

% The proofs of these results by Anderson and Dawar~\cite{} can prove all of
% these results


% We now turn to the question of the importance of the transparency condition.
% It could be that, just as for the case of circuits with symmetric gates, a
% general circuit (not necessarily transparent) can be transformed in polynomial
% time into an equivalent transparent circuit. If so, from the argument above,
% we could assume without a loss of generality that $P$-uniform families of
% symmetric circuit have unique extensions, eliminating the need for the
% explicit transparency requirement present in the main theorem.

% While we do not show that no such transformation exists, we do show that many
% related problems are as hard as the graph isomorphism problem for a general,
% not-necessarily transparent, circuit). In particular, we show that the related
% problems of computing the syntactic equivalence relation for a circuit, or
% whether a gate in the circuit has unique labels, or unique extensions (in the
% sense defined below), are all also at least as hard as the graph isomorphism
% problem. We also show that computing other basic properties of a symmetric
% circuit, such as the orbit of a gate, is also at least as hard as the graph
% isomorphism problem. This is a particularly important limitation, as we make
% use of a polynomial-time algorithm for deciding the orbit of a gate (or
% element of a universe of a gate) in our proof of the main result.

% ,; address two related decision problems in Proposition \ref{} and Corollaries
% \ref{} and \ref{}. In partic

% As we will show in Proposition \ref{}, formulas of $\FPR$ can be translated
% into. For this transparent circuits. However, the transformation is
% complicated by the requirement that the resultant circuit be transparent.
% However, it is worth asking if the transparency condition is required, i.e. if
% it can be shown that

% \emph{rigid} circuit. They show that all circuits such circuits can be
% transformed into equivalent rigid circuits . They show that if a circuit is
% rigid then it has unique extensions and that every circuit with injective
% labels and symmetric gates can be transformed into an equivalent rigid circuit
% in polynomial time. We have similarly shown that a transparent circuit can be
% transformed into an equivalent one with unique labels in polynomial time, and
% that circuits with unique labels have unique extensions. Moreover, since all
% circuits with symmetric gates and injective labels are transparent, this is a
% strict generalisation of their result.

% and they show that a symmetric gate


% Using the language developed in this paper, a circuit $C_n$ is \emph{rigid} if
% it has injective labels and there are no two gates $g$ and $h$ in $C_n$ such
% that $g$ and $h$ are syntactically equivalent. Since their discussion is
% limited to circuits with symmetric gates and injective labels this is equ


% circuit, a condition analogous to the rigidity condition defined in this
% paper. The following

% It is worth noting that all symmetric circuits are transparent, and

% prove that it's a sufficient condition for establishing the uniqueness of the
% induced automorphism on the circuit. Moreover, they prove that any circuit
% (noting that they limit themselves to a discussion of circuits over a
% symmetric basis) may be transformed into an equivalent rigid circuit in
% polynomial time. As such, they may assume $P$-uniform symmetric circuit
% families are rigid without a loss of generality, and hence be may assume
% without a loss of generality that each permutation in $\sym_n$ uniquely
% extends to an automorphism of the circuit.



% However, in this subsection we show that deciding the rigidity of circuits
% over bases which may include non-symmetric functions is at least as hard as
% the graph isomorphism problem. However, we shoudl that on classes

% Although the rigidity condition we present in this paper is similarly defined,
% the existence of a polynomial-time computable transformation of a circuit into
% an equivalent rigid circuit is not as straight forward. Indeed, it is possible
% to show that the problem of determining if a given $(\mathbb{B},
% \tau)$-circuit is rigid is as least as hard as the graph isomorphism problem.
% This is a direct consequence of the following result.

% \begin{definition}
%   The \emph{syntactic equivalence class problem} is the decision problem for
%   the set of all triples $(C, g_1, g_2)$, where $C$ is a $(\mathbb{B},
%   \tau)$-circuit, and $g_1$ and $g_2$ are two syntactically equivalent gates
%   in $C$.
% \end{definition}

% \begin{definition}
%   The \emph{rigidity problem} is the decision problem for the set of all
%   circuits $C$ where $C$ is a rigid circuit.
% \end{definition}

% \begin{definition}
%   The \emph{unique labels problem} is the decision problem for the set of all
%   circuits $C$ where $C$ has unique labels.
% \end{definition}

% \begin{prop}
%   Let $\tau$ be any non-empty relational vocabulary and let $n, r, p \in
%   \nats$ with $p$ prime. There is a a polynomial time reduction from the graph
%   isomorphism problem to the problem of deciding if a given pair of gates in a
%   given symmetric rank-circuit for structures of size $n$ that (i) has
%   injective labels, (ii) contains no constant gates, and (iii) contains at
%   most two rank gates with bound $r$ and prime $p$, are syntactically
%   equivalent.
%   \label{prop:syntactic-graph-iso}
% \end{prop}
% \begin{proof}
%   We have from \cite{} that there is a polynomial time reduction from the
%   graph isomorphism problem to the bipartite isomorphism problem. The
%   bipartite isomorphism problem is the decision problem for the set of all
%   pairs of partitioned bipartite graphs $B_1 := (U_1, V_1, E_1)$ and $B_2 :=
%   (U_2, V_2, E_2)$ such that there exists a graph isomorphism $\pi : B_1
%   \rightarrow B_2$ with $\restr{\pi}{U_1} = U_2$ and $\restr{\pi}{V_1} = V_2$.

%   Suppose we are given two partitioned bipartite graphs $B_1 := (U_1, V_1,
%   E_1)$ and $B_2 := (U_2, V_2, E_2)$. We assume, without a loss of generality,
%   that there exists $a, b \in \nats$ such that $U_1 = U_2 = [a]$ and $V_1 =
%   V_2 = [b]$.

%   Let $R$ be a relation symbol in $\tau$ and let $k := \arty{R}$. Let $G_{R}
%   := \{g_{R, \vec{c}} : \vec{c} \in [n]^k\}$, $G_{\text{mid}} := \{\land_0,
%   \land_1, \neg_0, \land_{\text{out}} \}$, $G_{\rank} : = \{ g_{\rank, 1},
%   g_{\rank, 2}\}$, and $G_{\land} := \{g_{i, (u,v)} : i \in [2] \, (u,v) \in
%   E_i \}$. Let $C_n = \langle G, \Omega, \Sigma, \Lambda, L \rangle$ be a
%   $(\{\rank^{1}_2\}, \tau)$-circuit defined as follows. Let $G = G_R \cup
%   G_{\text{mid}} \cup G_{\rank} \cup G_{\land}$ and $\Omega$ be the $0$-ary
%   function $\land_{\text{out}}$. For each $\vec{c} \in [n]^{\arty{R}}$ let
%   $\Lambda(\vec{c}) = g_{R, \vec{c}}$. Define $\Sigma$ as follows. For each $g
%   \in G$,
%   \begin{itemize}
%     \setlength\itemsep{0mm}
%   \item if $g \in G_R$ then $\Sigma(g) = R$,
%   \item if $g$ equals $\land_0$ or $\land_1$ then $\Sigma(g) = \land[n^k]$
%   \item if $g$ equals $\land_{\text{out}}$ then $\Sigma(g) = \land[2]$,
%   \item if $g \in G_\rank$ let $\Sigma(g) = \rank^r_p [a,b]$, and
%   \item if $g \in G_{\text{in}}$ and $g$ then $\Sigma(g) = \land[1]$.
%   \end{itemize}
%   Define $L$ as follows. For each $g \in G$,
%   \begin{itemize}
%     \setlength\itemsep{0mm}
%   \item if $g \in G$ and $g = g_{\text{out}}$ then for each $i \in [2]$,
%     $L(g)(i) := g_{\rank, i}$,
%   \item if $g \in G_\rank$ and $g = g_{\rank, i}$ then $L(g)(p,q) = g_{i,
%     (p,q)}$,
%   \item if $g \in G_{\land}$ and $g = g_{i, (p,q)}$ then $L(g)(1) = \land_1$
%     if $(p,q) \in E_i$ and $L(g)(1) = \neg_0$ otherwise,
%   \item if $g = \neg_0$ then $L(g)(1) = \land_0$, and
%   \item if $g = \land_0$ or $g = \land_1$ then for all $p \in [n^k]$ we have
%     that $L(g)(p)$ equals $\Lambda^{-1}(\vec{c}_p)$, where $\vec{c}_p$ is the
%     $p$th element of $[n]^k$ in the lexicographical ordering on $[n]^k$.
%   \end{itemize}

%   We note that for $i \in [2]$, $L(g_{\rank, i})(p,q) = \land_1$ if $(p,q)$ is
%   an edge in in $B_i$ and $L(g_\rank, i)(p,q) = \neg_0$ if $(p,q)$ is not an
%   edge in $B_i$. We thus have that $B_1$ and $B_2$ are bipartite isomorphic
%   (in the sense defined above) if, and only if, $g_{\rank, 1}$ is
%   syntactically equivalent to $g_{\rank, 2}$.

%   Since the construction of $C_n$ can be implemented in time polynomial in the
%   combined sizes of the input graphs, the mapping of $(B_1, B_2)$ to the tuple
%   $(C_n, (g_{\rank, 1}, g_{\rank, 2}))$ witnesses the reduction. The result
%   follows.
% \end{proof}

We now present a reduction from the graph-isomorphism problem to the problem of
deciding if two gates in a circuit are syntactically-equivalent. In fact, we
prove a stronger result, presenting a reduction from the graph-isomorphism
problem to the problem of computing the syntactic-equivalence relation over an
ostensibly simpler class of circuits.

\begin{prop}
  Let $\rho$ be any non-empty relational vocabulary and let $n, r, p \in \nats$,
  with $p$ prime. There is a a polynomial time reduction from the
  graph-isomorphism problem to the problem of deciding if a given pair of gates
  in a given symmetric rank-circuit for structures of size $n$ that (i) has
  injective labels, (ii) contains no constant gates, and (iii) contains at most
  two rank gates with bound $r$ and prime $p$, are syntactically equivalent.
  \label{prop:syntactic-graph-iso}
\end{prop}
\begin{proof}
  We have from \cite{} that there is a polynomial time reduction from the graph
  isomorphism problem to the bipartite isomorphism problem. The bipartite
  isomorphism problem is the decision problem for the set of all pairs of
  partitioned bipartite graphs $B_1 := (U_1, V_1, E_1)$ and $B_2 := (U_2, V_2,
  E_2)$ such that there exists a graph isomorphism $\pi : B_1 \rightarrow B_2$
  with $\restr{\pi}{U_1} = U_2$ and $\restr{\pi}{V_1} = V_2$.

  Suppose we are given two partitioned bipartite graphs $B_1 := (U_1, V_1, E_1)$
  and $B_2 := (U_2, V_2, E_2)$. We assume, without a loss of generality, that
  there exists $a, b \in \nats$ such that $U_1 = U_2 = [a]$ and $V_1 = V_2 =
  [b]$.

  Let $R$ be a relation symbol in $\rho$ and let $k := \arty(R) > 0$. Let $G_{R}
  := \{g_{R, \vec{c}} : \vec{c} \in [n]^k\}$, $G_{\text{mid}} := \{\land_0,
  \land_1, \neg_0, \land_{\text{out}} \}$, $G_{\rank} : = \{ g_{\rank, 1},
  g_{\rank, 2}\}$, and $G_{\text{nodes}} := \{g_{i, (u,v)} : i \in [2] \, (u,v)
  \in E_i \}$. Let $C = \langle G, \Omega, \Sigma, \Lambda, L \rangle$ be a
  $(\{\rank^{1}_2\}, \rho)$-circuit of order $n$ defined as follows. Let $G =
  G_R \cup G_{\text{mid}} \cup G_{\rank} \cup G_{\text{nodes}}$ and $\Omega$ be
  the $0$-ary function $\land_{\text{out}}$. For each $\vec{c} \in
  [n]^{\arty(R)}$ let $\Lambda(\vec{c}) = g_{R, \vec{c}}$. Define $\Sigma$ as
  follows. For each $g \in G$,
  \begin{itemize}
    \setlength\itemsep{0mm}
  \item if $g = \land_{\text{out}}$ then $\Sigma(g) = \land[2]$,
  \item if $g \in G_\rank$ then $\Sigma(g) = \rank^r_p [a,b]$,
  \item if $g \in G_{\text{nodes}}$ then $\Sigma(g) = \land[1]$,
  \item if $g= \neg_0$ then $\Sigma(g) = \neg$,
  \item if $g$ equals $\land_0$ or $\land_1$ then $\Sigma(g) = \land[n^k]$, and
  \item if $g \in G_R$ then $\Sigma(g) = R$.
  \end{itemize}
  Define $L$ as follows. For each $g \in G$,
  \begin{itemize}
    \setlength\itemsep{0mm}
  \item if $g = \land_{\text{out}}$ then for each $i \in [2]$, $L(g)(i) :=
    g_{\rank, i}$,
  \item if $g \in G_\rank$ and $g = g_{\rank, i}$ then $L(g)(p,q) = g_{i,
      (p,q)}$,
  \item if $g \in G_{\text{nodes}}$ and $g = g_{i, (p,q)}$ then if $(p, q \in
    E_i)$ then $L(g)(1) = \land_1$, otherwise $L(g)(1) = \neg_0$,
  \item if $g = \neg_0$ then $L(g)(1) = \land_0$, and
  \item if $g = \land_0$ or $g = \land_1$ then for all $p \in [n^k]$ we have
    that $L(g)(p)$ equals $\Lambda^{-1}(\vec{c}_p)$, where $\vec{c}_p$ is the
    $p$th element of $[n]^k$ in the lexicographical ordering on $[n]^k$.
  \end{itemize}

  We note that for $i \in [2]$, $L(g_{\rank, i})(p,q) = \land_1$ if, and only
  if, $(p,q)$ is an edge in in $B_i$ and $L(g_\rank, i)(p,q) = \neg_0$ if, and
  only if, $(p,q)$ is not an edge in $B_i$. We thus have that $B_1$ and $B_2$
  are bipartite-isomorphic (in the sense defined above) if, and only if,
  $g_{\rank, 1}$ is syntactically equivalent to $g_{\rank, 2}$.

  Since the construction of $C$ can be implemented in time polynomial in the
  combined sizes of the input graphs, the mapping of $(B_1, B_2)$ to the tuple
  $(C, (g_{\rank, 1}, g_{\rank, 2}))$ is a reduction, and the result follows.
\end{proof}

Since there is a trivial reduction from the problem of computing the
syntactic-equivalence relation over a restricted class of circuits to the
problem over the general class of circuits, we have the following corollary.

\begin{cor}
  There is a polynomial-time reduction from the graph isomorphism problem to the
  problem of determining if a given pair of gates in a given circuit are
  syntactically-equivalent.
  \label{lem:syntactically-equivilent-general-hard}
\end{cor}

We also have a reduction from the problem of computing the syntactic-equivalence
relation to the problem of deciding if a given gate in a circuit has unique
labels. From then transitivity of polynomial-time many-one reductions, this gives
us a reduction from the graph-isomorphism problem to the problem of deciding if
a gate has unique labels.

\begin{lem}
  There are a polynomial-time reduction from the problem of determining if a
  given pair of gates in a given circuit are syntactically equivalent to the
  problems of determining if a given gate in a given circuit has unique labels.
  \label{lem:syntactically-equivalent-unique-labels}
\end{lem}

\begin{proof}
  Let $C := \langle G, \Omega, \Sigma, \Lambda, L \rangle$ be a circuit of order
  $n$ and let $g_1, g_2 \in G$. Let $D$ be the circuit formed from $C$ by
  removing every gate $g \in G \setminus\{g_1, g_2\}$ such that $\neg W_t(g,
  g_1) \land \neg W_t(g, g_2)$ (i.e. all those gates in the circuit such that
  there is no path from them to either $g_1$ or $g_2$). Here $W_t$ is the
  relation defined in the proof of Lemma~\ref{}. Let $C'$ be the circuit formed
  from $D$ by adding in a single two-input $\AND$ gate $g'$ and connecting the
  outputs of $g_1$ and $g_2$ to the inputs of $g'$. Moreover, we let this $g'$
  be the single output gate of $C'$.

  It follows that $g_1$ and $g_2$ are syntactically equivalent in $C$ if, and
  only if, $g'$ has unique labels in $C'$. Since the construction of $C'$ from
  $C$ can be completed in polynomial time, the mapping of $(C, (g_1, g_2))$ to
  $(C', g')$ is a reduction.
\end{proof}

We can construct a similar argument for reducing the problem of deciding if two
gates are syntactically equivalent to the problem of deciding if a given gate
$g$ in a circuit $C$ does not have \emph{unique extensions}. We say a gate $g$
has unique extensions if there is no permutation such that two automorphisms of
the circuit extend the permutation and disagree with each other on $g$ (i.e. $g$
is not a counterexample to $C$ having unique extensions).

\begin{lem}
  There is a polynomial-time reduction from the problem of determining if a
  given pair of gates gates in a given $(\mathbb{B}, \rho)$-circuit are
  syntactically-equivalent to the problem of determining if for a given triple
  $(C, g)$, where $C$ is a circuit of order $n$, $g$ is a gate in $C$, that
  there exists $\sigma \in \sym_n$ and automorphisms $\pi, \pi' \in \aut(C)$
  extending $\sigma$ such that $\pi (g) \neq \pi' (g)$.
  \label{lem:syntactically-equivilent-unique-extensions}
\end{lem}
\begin{proof}
  Let $C$ be a circuit of order $n$ and let $g_1$ and $g_2$ be two gates in $C$.
  Note that for any gate $g$ in $\sigma \in sym_n$, if $\pi, \pi' \in \aut(C)$
  extend $\sigma$ and $\pi_e := \pi'\pi{-1}$ then $\pi(g) \neq \pi'(g)$ if, and
  only if, $\pi_e (g) \neq g$. It follows that there exists $\sigma \in sym_n$
  such that there are automorphisms $\pi, \pi' \in \aut(C)$ extending $\sigma$
  such that $\pi (g) \neq \pi'(g)$ if, and only if, there exists $\pi_e$
  extending the trivial permutation such that $\pi (g) \neq g$.

  Let $C'$ be the circuit constructed from $C$ as in the proof of
  Lemma~\ref{lem:syntactically-equivalent-unique-labels}. We now prove that the
  mapping $(C, (g_1, g_2))$ to $(C', g_1)$ is a reduction. Let $\pi_e$ be a
  function from the gates of $C'$ to the gates of $C'$ that swaps $g_1$ and
  $g_2$ and fixes all other gates. It follows that if $g_1$ and $g_2$ are
  syntactically-equivalent in $C$, then they are syntactically-equivalent in
  $C'$, and so $\pi_e$ is a non-trivial automorphism extending the trivial
  permutation, and $g_1$ does not have unique extensions in $C'$. In the other
  direction, suppose $g_1$ does not have unique extensions in $C'$. Then there
  exists an automorphism $\pi_e \in \aut(C')$ extending the trivial permutation
  and such that $\pi_e(g_1) \neq g_1$. But $g_1$ is a child of the single output
  gate $g'$ (which must be fixed by any automorphism), and the only other child
  of $g'$ is $g_2$. It follows $\pi_e$ swaps $g_1$ and $g_2$, and so $g_1$ and
  $g_2$ are syntactically-equivalent in $C'$. The result follows.
\end{proof}

We now show that there is a reduction from the graph-isomorphism problem to the
problem of deciding if a given circuit is symmetric.

\begin{prop}
  The graph-isomorphism problem is polynomial-time reducible to the problem of
  determining if a circuit is symmetric.
  \label{prop:graph-iso-symmetric}
\end{prop}

\begin{proof}
  We use a similar approach as in Proposition~\ref{prop:syntactic-graph-iso},
  and so proving a reduction from the bipartite-graph-isomorphism to the problem
  of deciding if a circuit is symmetric.
 
  Suppose we are given two partitioned bipartite graphs $B_1 := (U_1, V_1, E_1)$
  and $B_2 := (U_2, V_2, E_2)$. We assume, without a loss of generality, that
  there exists $a, b \in \nats$ such that $U_1 = U_2 = [a]$ and $V_1 = V_2 =
  [b]$.

  Let $\rho:= \{R\}$ be a relational vocabulary, where $R$ is a unary relational
  symbol. Fix any number $r$ and $p$ prime. We define a $(\rho, \RB)$-circuit $C
  := \langle G, \Omega, \Sigma, \Lambda, L \rangle$ of order two as follows. Let
  $G_{R} := \{g_{R, 1}, g_{R_2}\}$ and $G_{\text{ands}} := \{\land^1_0,
  \land^2_0, \land^1_1, \land^2_1 , \land_{\text{out}}\}$, $G_{\text{negs}} :=
  \{,neg^1_0, \neg^2_0\}$, $G_{\rank} : = \{ g_{\rank, 1}, g_{\rank, 2}\}$, and
  $G_{\text{nodes}} := \{g_{i, (u,v)} : i \in [2] \, (u,v) \in E_i \}$. Let $G =
  G_R \cup G_{\text{ands}} \cup G_{\text{negs}} \cup G_{\rank} \cup
  G_{\text{nodes}}$ and $\Omega$ be the $0$-ary function $\land_{\text{out}}$.
  Let $\Lambda(1) := g_{R, 1}$ and $\Lambda(2) := g_{R,2}$. Define $\Sigma$ as
  follows. For each $g \in G$,
  \begin{itemize}
    \setlength\itemsep{0mm}
  \item if $g$ equals $\land_{\text{out}}$ then $\Sigma(g) = \land[2]$,
  \item if $g \in G_\rank$ let $\Sigma(g) = \rank^r_p [a,b]$,
  \item if $g \in G_{\text{nodes}}$ then $\Sigma(g) = \land[1]$,
  \item if $g \in G_{\text{ands}}$ then $\Sigma(g) = \land[1]$ and if $g \in
    G_{\text{negs}}$ then $\Sigma(g) = \neg$, and
  \item if $g \in G_R$ then $\Sigma(g) = R$.
  \end{itemize}
  Define $L$ as follows. For each $g \in G$,
  \begin{itemize}
    \setlength\itemsep{0mm}
  \item if $g = g_{\text{out}}$ then for each $i \in [2]$, $L(g)(i) := g_{\rank,
      i}$,
  \item if $g \in G_\rank$ and $g = g_{\rank, i}$ then $L(g)(p,q) = g_{i,
      (p,q)}$,
  \item if $g \in G_{\text{nodes}}$ and $g = g_{i, (p,q)}$ for some $i \in [2]$,
    then $L(g)(1) = \land^i_1$ if $(p,q) \in E_i$ and $L(g)(1) = \neg^i_0$
    otherwise ,
  \item if $g = \neg^i_0$ for some $i \in [2]$, then $L(g)(1) = \land^i_0$, and
  \item if $g = \land^i_0$ or $g = \land^i_1$ for some $i \in [2]$, then
    $L(g)(p) = \Lambda^{-1}(i)$.
  \end{itemize}

  We note that for $i \in [2]$, $L(g_{\rank, i})(p,q) = \land^i_1$ if, and only
  if, $(p,q)$ is an edge in $B_i$ and $L(g_\rank, i)(p,q) = \neg^i_0$ if, and
  only if, $(p,q)$ is not an edge in $B_i$. It can be shown that $C$ is
  symmetric if, and only if, there exists an extension of the transposition
  $(1,2)$ to an automorphism of the circuit, which holds if, and only if, there
  exists an extension of $(1,2)$ to an automorphism of the circuit such that
  $(1,2) (L(g_{\rank, 1}))$ is isomorphism-equivilent to $L(g_{\rank, 2})$,
  which holds if, and only if, $B_1$ and $B_2$ are biparetite-isomorphic.

  % Suppose $C$ is not symmetric. If $\vert L(g_{\rank,1})^{-1}(\land^1_1) \vert
  % \neq L_(g_{\rank,2})^{-1}(\land^2_1) \vert$ then clear $B_1$ and $B_2$ are
  % not biparetite-isomorphic as they have different numbers of edges. Suppose
  % $\vert L(g_{\rank,1})^{-1}(\land^1_1) \vert =
  % L_(g_{\rank,2})^{-1}(\land^2_1) \vert$. Then the only reason $C$ would not
  % be symmetric is that there is no extension of $\sigma$ to the circuit t

  % If $B_1$ and $B_2$ are


  % We thus have that if $B_1$ and $B_2$ are bipartite-isomorphic


  % (in the sense defined in the proof of
  % Proposition~\ref{prop:syntactic-graph-iso}) if, and only if, $g_{\rank, 1}$
  % is syntactically-equivalent to $g_{\rank, 2}$.

  Since the construction of $C$ can be implemented in time polynomial in the
  combined sizes of the input graphs, the mapping of $(B_1, B_2)$ to $C$ is a
  reduction, and the result follows.
\end{proof}

We now show that computing the orbit of a gate for a general circuit is as hard
as the graph isomorphism problem. We make use of the ability to compute the
orbit of a gate in polynomial-time in
Section~\ref{sec:translating-formulas-to-FPR}, when translating families of
circuits into equivalent formulas.

\begin{lem}
  There is a polynomial-time reduction from the graph isomorphism problem to the
  problem of computing if two gates in a given circuit are in the same orbit.
  \label{lem:graph-iso-to-orbit}
\end{lem}
\begin{proof}
  Let $C$ be a circuit. Let $g_1$ and $g_2$ be two gates in $C$ such that
  $H_{g_1} = H_{g_2}$. Then there exists $\pi \in \aut(C)$ such that $\pi (g_1)
  = g_2$ if, and only if, $g_1$ is syntactically-equivalent to $g_2$. As such,
  we may use the same construction as in
  Proposition~\ref{prop:syntactic-graph-iso} to prove this result.
\end{proof}
We have from Propositions~\ref{prop:syntactic-graph-iso}
and~\ref{prop:graph-iso-symmetric}, and
Lemmas~\ref{lem:syntactically-equivalent-unique-labels},
\ref{lem:graph-iso-to-orbit}
and~\ref{lem:syntactically-equivilent-unique-extensions}, that a number of basic
circuit properties are at least as hard for general circuits as the
graph-isomorphism problem. In contrast, all of these properties are known to be
polynomial-time computable for transparent circuits (as follows from
Lemmas~\ref{lem:unique-labels-syntactic-equiv}
and~\ref{lem:computing-support-orbit-gate}).

The hardness results for the problems of computing syntactic-equivalence
relation and deciding if a gate has unique labels are particularly worth noting
as many natural algorithms for translating a circuit into an equivalent
transparent circuit (or one with unique labels) make use of polynomial-time
decidability of these properties. Indeed, the translation defined by Anderson
and Dawar~\cite{AndersonD17} from circuits with symmetric gates to equivalent
circuits with unique extensions makes explicit use of the polynomial-time
computability of the syntactic-equivalence relation on such circuits. As such,
these results, along with the hardness results for many other basic circuit
properties (e.g. symmetry, unique extensions, or the orbits of a gate) provides
evidence against the existence of an easily-definable polynomial-time
translation from circuits to transparent circuits, or circuits with unique
labels.

% We note that in all of the above casesgive us that many circuit properties of
% interest are harder to compute than the graph-isomorphism problem. In
% particular, determining

% We have shown that basic properties of a circuit, including the ability to
% compute the syntactic equivalence relation or whether a gate has unique
% labels, are as hard to decide as the graph-isomorphism problem. Moreover,

% incluidng that various properties of the circuit that seem necessary for
% defining a translation of a circuit into an equivalent transparent circuit are
% as hard to compute as the graph-isomorphism problem. We note that all of these
% properties are polynomial-time computable for transparent circuits (as follows
% from Lemma~\ref{} and Lemma~\ref{}).

While we have provided evidence against the ability to efficiently transform a
circuit into an equivalent transparent circuit, we now show that, in contrast,
there is a polynomial-time algorithm for checking if a circuit is transparent.
This has a certain significance, as we now have transparency is not some
unverifiable condition on circuits, further motivating the usefulness of this
definition.

\begin{prop}
  There is an algorithm that takes as input a circuit, decides if that circuit
  is transparent and runs in time polynomial in the size of the circuit.
  \label{prop:transparent-polynomial-time}
\end{prop}
\begin{proof}
  Let $C = \langle G, \Omega, \Sigma, \Lambda, L \rangle$ be a $(\mathbb{B},
  \rho)$-circuit. We first check that, for each non-symmetric gate $g \in G$,
  $L(g)$ is an injection. If not, we return that $C$ is not transparent.

  The height of a gate $g$ in the circuit is the distance from $g$ to the set of
  input gates. For each $p \in \nats$ let $G^p \subseteq G$ be the set of all
  gates of height $p$ and let $G^{\leq p} = \bigcup_{0 \leq i \leq p}G^i$. Note
  that each $g \in G^1$, $g$ has unique labels. Thus, from
  Lemma~\ref{lem:unique-labels-syntactic-equiv}, we can compute the syntactic
  equivalence relation between gates in $G^{\leq 1}$ efficiently. We construct a
  loop as follows. Let $p$ be initialised to $1$. We have that every
  non-symmetric gate $g \in G^{\leq p}$ has unique gates. We compute the
  syntactic-equivalence relation on $G^{\leq p}$ and check if each non-symmetric
  gate in $g \in G^{i + 1}$ has unique labels. If not, halt and output that the
  circuit is not transparent. If $p = \depth(C) -1$ then halt and output that
  the circuit is transparent. If neither halt state is reached, then increment
  $p$ by one and continue to the next iteration of the loop.

  We note that in every iteration of the loop we have that every non-symmetric
  gate in $G^{\leq p}$ has unique labels. Thus, from Proposition
  \ref{prop:unique-labels-syntactic-equiv}, we may compute the
  syntactic-equivalence relation on this set of gates efficiently. It follows
  that the above algorithm can be implemented so as to run in polynomial time.
\end{proof}

We similarly have that the unique labels property on circuits can be decided in
time polynomial in the number of gates in the circuit.

\begin{cor}
  There is an algorithm that takes in a circuit and decides if that circuit has
  unique labels and runs in time polynomial in the size of the circuit.
\end{cor}
\begin{proof}
  Let $C$ be the input circuit. From Proposition
  \ref{prop:transparent-polynomial-time} we may check if $C$ is transparent in
  polynomial time. If $C$ is not transparent halt and output that $C$ does not
  have unique labels. If $C$ is transparent then from Lemma \ref{} we may
  compute the syntactice-equivalence relation on the gates of $C$ efficiently.
  This then allows us to check if each gate in $C$ has unique labels.
\end{proof}

% perhaps related to the problem of translating a circuit into an equivalent
% transparent circuit are at least as hard as the graph isomorphism problem. We
% now show that additionally recognising if a circuit has unique extensions is
% at least as hard as the graph isomorphism problem.

We have restricted ourselves to transparent circuits in large part because they
can be transformed in polynomial-time into circuits with unique extensions and
because the transparency property is polynomial-time verifiable. It may be asked
why we do not just include an explicit restriction to circuits with unique
extensions. We now show that, in contrast with transparency, the problem of
deciding if a circuit has unique-extensions is at least as hard as the
graph-isomorphism problem.

\begin{lem}
  There is a polynomial-time reduction from the graph isomorphism problem to the
  problem of determining if a given circuit does not have unique extensions.
  \label{lem:unique-extensions-hard}
\end{lem}
\begin{proof}
  We use a similar argument as for Proposition \ref{prop:syntactic-graph-iso}.
  We again argue using a reduction from thee bipartite isomorphism problem. Let
  $B_1 := (U_1, V_1, E_1)$ and $B_2 := (U_2, V_2, E_2)$ and let $C$ be the
  associated circuit defined in the proof of Proposition
  \ref{prop:syntactic-graph-iso}. Let $C' = \op{merge-all} (C, G_{\land})$.
  Notice that $B_1$ and $B_2$ are isomorphic if, and only if, $g_{\rank, 1}$ and
  $g_{\rank, 2}$ are syntactically equivalent, which holds if, and only if,
  there exists $\pi \in \aut(C')$ extending the trivial permutation such that
  $\pi (g_{\rank, 1}) = g_{\rank, 2}$, which is true if, and only if, $C'$ does
  not have unique extensions. it follows that the mapping of $(B_1, B_2)$ to
  $C'$ defines a reduction.
\end{proof}

\begin{remark}
  It's interesting to note that the circuit used in the reduction does not have
  injective labels. If we instead ask about a reduction from GI to the problem
  of deciding if a circuit has unique extensions and injective labels, then I
  can reduce from to the problem of deciding if two rigid graphs are isomorphic.
  It is unknown if this decision problem is as hard as the full graph
  isomorphism problem. After looking around a bit I discovered that it's a very
  important open problem in quantum computation. Is this worth mentioning?
\end{remark}

% We have that if a circuit is transparent then we may compute in polynomial
% time the syntactic equivalence relation on the gates of that circuit, and
% hence compute an equivalent circuit that has unique extensions. However, we've
% shown that absent the transparency condition, computing the the syntactic
% equivalence relation is at least as hard as the graph isomorphism problem.
% Moreover, other properties that can be computed in polynomial time for
% transparent circuits, e.g. deciding if particular gates have unique labels or
% might act as counterexamples to the unique extensions, have also been shown to
% be at least as the graph isomorphism problem.

% While we have not shown that there is no polynomial time algorithm that
% translates an arbitrary circuit into an equivalent transparent circuit, we
% have provided evidence in favour of the conjecture that such an algorithm
% might not exist.

% \begin{remark}
%   I really don't like the slightly vague language here. I would prefer to say
%   this differently. How does one provide `evidence' that a functional problem
%   is hard, when one has proved results about a few related decision problems?
% \end{remark}

% However, we must then address another question: If we must include an explicit
% requirement on circuits, why not simply require that the circuit have unique
% extensions? But we have from Lemma \ref{lem:unique-extensions-hard} that at
% the least verifying that a circuit have unique extensions is at least as hard
% as the graph isomorphism problem. We show that, in contrast, deciding if a
% circuit is transparent is in $\PT$.

% \begin{prop}
%   There is an algorithm that takes as input a circuit, decides if that circuit
%   is transitive and runs in time polynomial in the size of the circuit.
%   \label{prop:tansitive-polynomial-time}
% \end{prop}
% \begin{proof}
%   Let $C = \langle G, \Omega, \Sigma, \Lambda, L \rangle$ be a $(\mathbb{B},
%   \tau)$-circuit. We first check that, for each non-symmetric gate $g \in G$,
%   $L(g)$ is an injection. If not, we return that $C$ is not transitive.

%   The height of a gate $g$ in the circuit is the distance from $g$ to the set
%   of input gates. For each $p \in nats$ let $G^p \subseteq G$ be the set of
%   all gates of height $p$ and let $G^{\leq p} = \bigcup_{0 \leq i \leq p}G^i$.
%   Note that each $g \in G^1$, $g$ has unique labels. Thus, from Lemma \ref{},
%   we can compute the syntactic equivalence relation between gates in $G^{\leq
%   1}$ efficiently. We construct a loop as follows. Let $p$ be initialised to
%   $1$. We have that every non-symmetric gate $g \in G^{\leq p}$ has unique
%   gates. We compute the syntactic equivalence relation on $G^{\leq p}$ and
%   check if each non-symmetric gate in $g \in G^{i + 1}$ has unique labels. If
%   not, halt and output that the circuit is not transitive. If $p = \depth(C)
%   -1$ then halt and output that the circuit is transitive. If neither halt
%   state is reached, then increment $p$ by one and continue to the next
%   iteration of the loop.

%   We note that in every iteration of the loop we have that every non-symmetric
%   gate in $G^{\leq p}$ has unique labels. Thus, from Proposition
%   \ref{prop:unique-labels-syntactic-equiv}, we may compute the syntactic
%   equivalence relation on this set of gates efficiently. Thus it is easy to
%   see that the above algorithm can be implemented so as to run in polynomial
%   time.
% \end{proof}

% A similar result for circuits with unique labels.

% \begin{cor}
%   There is an algorithm that takes in a circuit and decides if that circuit
%   has unique labels and runs in time polynomial in the size of the circuit.
% \end{cor}
% \begin{proof}
%   Let $C$ be the input circuit. From Proposition
%   \ref{prop:transitive-polynomial-time} we may check if $C$ is transitive in
%   polynomial time. If $C$ is not transitive halt and output that $C$ does not
%   have unique labels. If $C$ is transitive then from Lemma \ref{} we may
%   compute the syntactic equivalence relation on the gates of $C$ efficiently.
%   This then allows us to check if each gate in $C$ has unique labels.
% \end{proof}

% We thus have that we can verify in polynomial time that a family of circuits
% indeed is transparent or has unique labels. The same cannot be said for the
% unique extensions property.


% \begin{lem}
%   There is an algorithm that takes in a symmetric circuit $C$ of order $n$
%   with unique labels, a gate $g$ in $C$ and $a \in \universe{g}$, the orbit
%   $\orb_g (a)$ and coarsest supporting partition $\SP_g (a)$.
% \end{lem}
% \begin{proof}
%   Let $(u,v) \in \sym_n$ be a transposition. We can check if $(u,v) \in
%   \stab(g)$ using Lemma~\ref{lem:compute-automorphisms}. Then $(u,v)$ acts
%   like an isomorphism at $g$, and hence $(u,v) \cdot a = L(g)^{-1}((\sigma
%   L(g)(\vec{x}))) (i)$, where $\vec{x} \in \ind(g)$ such that $\vec{x} (i) =
%   a$ for some $i \in [\vert \vec{x} \vert]$. Thus, since
%   Lemma~\ref{lem:compute-automorphisms} gives us that that we can compute
%   $\sigma L(g)(\vec{x})$ in polynomial time, we can compute $(u,v) \cdot a$ in
%   polynomial-time.

%   We note that
  
%   and hence we can compute $\sigma \cdot a$ effeciently. and hence compute
%   $\orb_g(a) = \{ \sigma \cdot a : \sigma \in \stab(g)\}$.

% \end{proof}

% Once again we note that computing the orbit of a gate for a general circuit is
% as hard as the graph isomorphism problem. We make use of the ability to
% compute the orbit of a gate in polynomial time in sub-section
% \ref{sec:translating-formulas-to-FPR} when translating families of circuits
% into equivalent formulas. As such the approach in this paper would fail for
% the general circuit.

% \begin{prop}
%   There is a polynomial-time reduction from the graph isomorphism problem to
%   the problem of computing if two gates in a given circuit are in the same
%   orbit.
%   \label{prop:graph-iso-to-orbit}
% \end{prop}
% \begin{proof}
%   Let $C$ be a circuit. Let $g_1$ and $g_2$ be two gates in $C$ such that
%   $H_{g_1} = H_{g_2}$. Then there exists $\pi \in \aut(C)$ such that $\pi
%   (g_1) = g_2$ if, and only if, $g_1$ is syntactically equivalent to $g_2$. As
%   such, we may use the same construction as in
%   Proposition~\ref{prop:syntactic-graph-iso} to prove this result.
% \end{proof}



% \begin{lem}
%   There is a polynomial-time Turing reduction from the problem of determining
%   if a given pair of gates in a given circuit are syntactically equivalent to
%   the problem of deciding if a circuit is transparent.
%   %   If the problem of deciding if a given circuit is transparent can be
%   %   decided
%   %   in
%   %   polynomial time then the graph isomorphism problem is in polynomial
%   %   time.
% \end{lem}
% \begin{proof}
%   Suppose there is an algorithm that takes as input, runs in time polynomial
%   in the size of the circuit, and decides whether or not the circuit is
%   transparent.

%   From Corollary \ref{} it is sufficient to show that there is a
%   polynomial-time algorithm for the problem of deciding if a pair of gates in
%   a given circuit are syntactically equivalent. Let $C_n$ be a circuit and
%   $g_1$ and $g_2$ be two gates in the circuit.

  
%   %   We simply need a polynomial-time reduction from the problem of computing
%   %   t
%   %   to the problem
%   %   of checking if a given circuit is transparent. Let $C_n =\langle G,
%   %   \Omega,
%   %   \Sigma, \Lambda, L\rangle$ be a circuit and let $g_1$ and $g_2$ be two
%   %   gates
%   %   in $G$. We now construct a circuit $C_n'$. If $g_1 = g_2$ then let
%   %   $C_n'$ be
%   %   any transparent circuit. If $g_1$ and $g_2$ do not satisfy any of
%   %   conditions
%   %   (i), (ii), (iii) or (v) of syntactic equivalence then let $C_n'$ be any
%   %   non-transparent circuit. So suppose all these conditions are met and
%   %   $g_1
%   %   \neq
%   %   g_2$, then let $D_n$ be the sub-circuit of $C_n$ consisting of exactly
%   %   those
%   %   gates $g$ in $C_n$ such that $W_t(g, g_1)$ or $W_t(g, g_2)$. Then let
%   %   $C_n'$
%   %   be $D_n'$ with an additional gate $g_\rank$, such that $g_\rank $ is
%   %   labelled
%   %   by the function $\rank^1_2[2,1]$ and $g_\rank$
% \end{proof}

% \begin{corollary}
% \end{corollary}

% We have from this proposition that even many nice properties Since there is a
% polynomial-time reduction from this problem reduces the general problem of
% deciding the syntatic equivilence relation for an arbitary



% \begin{corollary}
%   There is a polynomial-time reduction from the graph isomorphism problem to
%   the problem of determining if a given $(\mathbb{B}, \tau)$-circuit is rigid.
% \end{corollary}

% \begin{corollary}
%   There is a polynomial-time reduction from the graph isomorphism problem to
%   the problem of determining if a given gate in a given $(\mathbb{B},
%   \tau)$-circuit has unique labels.
% \end{corollary}


% reduction from the graph isomorphism from to the problem of of computing the
% the syntactic equivalence relation on the gates of an arbitrary $(\mathbb{B},
% \tau)$-circuit.

% It is worth noting that if we restrict the input to a class of circuits
% defined over symmetric bases then we can decide the syntactic equivalence
% relation in polynomial time.

% % \begin{prop}
% %   Let $C_n$ be a circuit with symmetric gates. There is an algorithm that
% %   takes in such a circuit, runs in time polynomial in the size of the circuit,
% %   and outputs the syntactic equivalence relation on the gates of $C_n$.
% %   \label{prop:symmetric-syntactic}
% % \end{prop}
% % \begin{proof}
% % \end{proof}

% Moreover, it turns out that being able to compute the syntactic equivalence
% relation in polynomial time for some class of circuits $\mathcal{C}$ implies
% that the existence of a polynomial-time algorithm for converting circuits in
% $C$ into equivalent rigid circuit.

% % \begin{prop}
% %   Let $\mathcal{C}$ be a set of circuits. Suppose the syntactic equivalence
% %   relation can be computed in polynomial time for circuits in $\mathcal{C}$.
% %   It follows that there is an algorithm that runs in polynomial time that
% %   takes in a circuit $C \in \mathcal{C}$ and outputs a rigid circuit $C''$
% %   such that if $C$ is symmetric then $C'$ is symmetric.
% %   \label{prop:syntactic-equivilence-rigid}
% % \end{prop}
% % \begin{proof}
% % \end{proof}

% The following result gives us that the unique labels condition is sufficient.

% % \begin{prop}
% %   Let $\mathcal{C}$ be syntactically transparent class of circuits. It follows
% %   that there exists an algorithm that takes in a $(\mathbb{B}, \tau)$-circuit
% %   $C \in \mathcal{C}$ and outputs a $(\mathbb{B} \cup \mathbb{B}_{\std},
% %   \tau)$-circuit $C'$ such that $C$ and $C'$ compute the same function, $C'$
% %   is rigid, and if $C$ is symmetric then $C'$ is symmetric. Moreover, this
% %   algorithm runs in time polynomial in the size of the input circuit.
% % \end{prop}
% % \begin{proof}
% %   Let $C' = \make-injective-all(\op{merge-all} (C, G)($). We have that $C'$ is
% %   symmetric if $C$ is symmetric. Since the syntactic equivalence relation can
% %   be computed in polynomial time for circuits in $\mathcal{C}$ it follows that
% %   this operation can be implemented so as to run in time polynomial in the
% %   size of $C$. Clearly every gate $g$ in $C'$ belongs to a singleton syntactic
% %   equivalence class.
% % \end{proof}

% \begin{lem}
%   Let $\mathbb{B}$ be a basis of symmetric functions. Then for any class of
%   circuits $\mathcal{C}$ defined over the basis $\mathbb{B}$, $\mathcal{C}$ is
%   a syntactically transparent class of circuits.
% \end{lem}
% In contrast to
% \begin{lem}
%   If the class of all circuits is syntactically transparent then then graph
%   isomorphism problem is in $\PT$.
% \end{lem}

% \begin{lem}
%   Let $\mathcal{C}$ be a syntactically transparent class of circuits. There is
%   an algorithm that takes as input a $(\mathbb{B}, \tau)$-circuit $C \in
%   \mathcal{C}$ and decides if $C$
% \end{lem}

% Since the class of The following result, also proved by Anderson and Dawar
% \cite{AndersonD17}, follows as a corollary.

% \begin{lem}
%   Let $\mathcal{C}$
% \end{lem}

% \begin{lem}
%   Let $C_n = \langle G, \Omega, \Sigma, \Lambda, L\rangle$ be a $(\mathbb{B},
%   \tau)$-circuit computing a $q$-ary query. If $C_n$ has unique labels then it
%   is $C_n$ rigid
%   \label{lem:unique-implies-rigid}
% \end{lem}
% \begin{proof}

%   %   Suppose there exists a gate $g'$ such that $W_t (g, g')$ and $W_t
%   %   (\pi(g),
%   %   g')$ and $\pi (g') = g'$, then $\pi H_{g'} = H_{g'}$, and since $g'$ has
%   %   unique labels it follows that $\pi$ acts trivially on $H_{g'}$ (i.e. for
%   %   all
%   %   $h \in H_{g'}$, $\pi(h) = h$). Moreover, for any pair of gates $h,h' \in
%   %   G$
%   %   such that $W(h, h')$ and $W_t(h', g')$, if $\pi$ acts on trivially on
%   %   the
%   %   children of $h'$ then $\pi (h) = h$ and so $\pi$ must act trivially

%   %   $\pi$ acts trivially on the children of $h$.

%   %   If $\Omega_g \cap \Omega_{\pi(g)} = \emptyset$ then, since
%   %   $\Omega_{\pi(g)}
%   %   =
%   %   \pi \Omega_g$, there exists $g' \in \Omega_g$ such that $\pi g' \neq
%   %   g'$.
%   %   But
%   %   $\pi$ must act trivially on output gates, so this is a contradiction. So
%   %   there
%   %   exists $g' \in \Omega_g \cap \Omega_{\pi(g)}$. But then $W_t (g, g')$
%   %   and
%   %   $W_t
%   %   (\pi (g), g')$.


%   %   We have that $g$ must be a non-output internal gate. So $g$ is
%   %   syntactically
%   %   equivalent to $\pi (g)$.

%   %   and let $h$ be a gate such that $W_t (g, h)$.
% \end{proof}

% \begin{prop}
%   Let $C_n$ be a circuit with unique labels. If
% \end{prop}

% \begin{prop}
  
% \end{prop}
% \begin{proof}
% \end{proof}

% \begin{prop}
%   Let $\mathcal{C}$ be a class on which the Let $C_n := \langle G, \Omega,
%   \Sigma, \Lambda, L \rangle$ be a $(\mathbb{B}, \tau)$-circuit with unique
%   labels. There is a deterministic algorithm that takes in such a circuit and
%   outputs the syntactic equivalence relation on $G$. Moreover, this algorithm
%   runs in time polynomial in the size of the circuit.
%   \label{prop:unique-labels-syntactic-equiv}
% \end{prop}

% % \begin{prop}
% %   There is an algorithm that takes in a $(\mathbb{B}, \tau)$-circuit $C$ with
% %   unique labels and outputs the syntactic equivalence relation on its gates.
% %   This algorithm runs in time polynomial in the size of $C$.
% %   \label{prop:unique-labels-syntactic-equiv}
% % \end{prop}
% % \begin{proof}
% %   We build of the relation by induction. Let $g$
% % \end{proof}

% Putting together Propositions \ref{prop:symmetric-syntactic},
% \ref{prop:syntactic-equivilence-rigid}, and
% \ref{prop:unique-labels-syntactic-equiv}, we have that there is a
% polynomial-time algorithm that takes as input a circuit with unique labels and
% outputs an equivalent rigid circuit. In fact, there is an equivalence of sorts
% between the unique labels condition and the rigidity condition.

% \begin{prop}
%   Let $\mathcal{C}$ be a family of circuits. There is an algorithm that runs
%   in polynomial-time that takes in a circuit $C \in \mathcal{C}$ and outputs a
%   rigid circuit $C'$ such that if $C$ is symmetric then $C'$ is symmetric if,
%   and only if, there is an algorithm that runs in polynomial-time and takes in
%   a circuit $C \in \mathcal{C}$ and outputs a circuit $C'$ with unique labels
%   such that if $C$ is symmetric then $C'$ is symmetric.
% \end{prop}

% It follows then that for the symmetric circuits defined over bases of
% symmetric functions discussed by Anderson and Dawar~\cite{} we may assume the
% circuits are rigid and/or have unique labels without a loss of generality. For
% the more general circuit discussed in this paper we cannot make such an
% assumption as deciding rigidity is harder then graph isomorphism. However, if
% we restrict ourselves to families of circuits with unique labels then again we
% may assume rigidity without a loss of generality.

% Of course, this restriction to circuits with unique labels is only useful if
% we can show that formulas in rank logic can be translated into $P$-uniform
% families circuits with unique labels. In the next subsection we prove that
% this is indeed the case.


\end{document}