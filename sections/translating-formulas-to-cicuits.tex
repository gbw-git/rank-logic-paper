\documentclass[../paper.tex]{subfiles}
\begin{document}

\subsection{Circuit Manipulation}
In this section and the next we develop a number of results that reference
$P$-uniform families of circuits. In order to prove these results we often need
to define polynomial-time algorithms for generating or manipulating circuits. In
this subsection we define a number of algorithms that perform commonly invoked
operations on circuits.

We often include quotes around the names of these algorithm in order to avoid
any confusion. We assume a reasonable encoding of the circuit.

The algorithm \emph{merge} takes a circuit and an appropriate set of gates and
produces a new circuit by combining those gates into a single gate.

\begin{definition}
  The algorithm \emph{merge} takes as input a circuit $C = \langle G, \Omega,
  \Sigma , \Lambda, L \rangle$ and a set of syntactically equivalent gates $H =
  \{h_1, \ldots , h_p\}$ and outputs a new circuit $C' = \langle G', \Omega',
  \Sigma', \Lambda', L' \rangle$, where
  \begin{itemize}
    \setlength\itemsep{0mm}
  \item $G' = G \setminus \{h_2, \ldots, h_q\}$,
  \item $\Omega' = \Omega$,
  \item $\Sigma' = \restr{\Sigma}{G'}$,
  \item $\Lambda' = \Lambda$, and
  \item For all $g \in G'$ and $\vec{a} \in Ind(g)$ if $L(g) (\vec{a}) \in H$
    then $L'(g)(\vec{a}) = h_1$ and if $L(g)(\vec{a}) \notin H$ then
    $L'(g)(\vec{a}) = L(g)(\vec{a})$.
  \end{itemize}
\end{definition}

It can be shown that `merge' can be implemented so as to run in time polynomial
in the size of the input circuit. We note that if the input circuit to `merge'
is symmetric (resp. has unique labels) then the output circuit will also be
symmetric (resp. have unique labels). The input and output circuits compute the
same function.

% Moreover, if we run merge on a circuit $C$ and a set $H$ then if $C$ has
% unique labels and $H$ has the property that there is no gate $g$ in $C$ such
% that two elements of $H$ are in $H_g$ then if $C$ has unique labels then so
% does $C'$. Moreover, if $C$ is symmetric then so is $C'$.

The algorithm \emph{separate-labels} takes a circuit and outputs an equivalent
circuit such that every symmetric-gate has unique labels.

\begin{definition}
  The algorithm \emph{separate-labels} takes as input a $(\mathbb{B},
  \tau)$-circuit $C := \langle G, \Omega, \Sigma , \Lambda, L \rangle$ and
  outputs circuit $C' = \langle G', \Omega', \Sigma', \Lambda', L' \rangle$. We
  now describe how these values are derived. We have that $\Omega' = \Omega'$
  and $\Lambda' = \Lambda$. We start with $G' = G$, $\Sigma' = \Sigma$ and $L' =
  L$ and updates their definitions iteratively according to the following
  procedure.

  Let $k$ denote a variable initialised to $1$. Let $S$ be the set of all gates
  of height $k$ that are symmetric-gates. For each $h_1 \in \bigcup_{g \in
    S}H_g$ let $g \in S$ be the first gate (using the order of the circuit
  derived from its encoding) such that $\vert L(g)^{-1}(h_1) \vert$ is maximal.
  Let $\{a_1, \ldots, a_p\} := L(g)^{-1}(h_1)$ (again, making use of the order).
  We add to $G'$ the gates $\{h_2, \ldots, h_p\}$ and for each $i \in \{2,
  \ldots, p\}$ we set $\Sigma'(h_i)$ such that $h_i$ is a $\land$ gate with a
  single output and $L'(h_i)$ such that $h_i$ takes only $h_{i-1}$ as an input.
  For each $g' \in S$ such that $h_1 \in H_g$ let $\{ b_1, \ldots, b_q \} :=
  L(g)^{-1}(h_1)$. For each $i \in [q]$ let $L'(g')(b_i) = h_i$. Iterate $k$ and
  repeat this procedure until $k$ exceeds the depth of the circuit.
\end{definition}

It can be shown that `separate-labels' can be implemented so as to run in time
polynomial in the size of the input circuit. We note that both the input and
output circuits compute the same function. Moreover, if all of the non-symmetric
gates in the input circuit have unique labels then the output circuit will have
unique labels. If the input circuit is symmetric and has the property that every
non-symmetric gate in the circuit has unique labels then the output circuit is
symmetric.

% Notice that for a given circuit $C$ and a symmetric-gate $g$ in $C$, running
% `merge' on each equivalence class of syntactically equivalent gates input to
% $g$ and then running `separate-labels' on the resultant circuit produces
% circuit $C'$ computing the same function as $C$ but the gate $g$ in $C$ has
% unique labels.

The algorithm \emph{insert} takes as input two circuits and creates a new
circuit formed by inserting the gates of the first circuit into the second
circuit.

\begin{definition}
  The algorithm \emph{insert} takes as input a $(\mathbb{B}_1, \tau_1)$-circuit
  $C^1 := \langle G_1, \Omega_1, \Sigma_1 , \Lambda_1, L_1 \rangle$ and a
  $(\mathbb{B}_2, \tau_2)$-circuit $C^2 := \langle G_2, \Omega_2, \Sigma_2 ,
  \Lambda_2, L_2 \rangle$ and outputs a $(\mathbb{B}_1 \cup \mathbb{B}_2, \tau_1
  \uplus \tau_2)$-circuit $C = \langle G, \Omega, \Sigma , \Lambda, L \rangle$
  such that
  \begin{itemize}
    \setlength\itemsep{0mm}
  \item $G = G_1 \uplus G_2$,
  \item $\Omega = \Omega_2$,
  \item $\Sigma = \Sigma_1 \uplus \Sigma_2$,
  \item $\Lambda = \Lambda_1 \uplus \Lambda_2$, and
  \item $L = L_1 \uplus L_2$.
  \end{itemize}
\end{definition}

It can be shown that `insert' can be implemented so as to run in time polynomial
in the size of the input circuit. If the input circuits are symmetric then the
output circuit will be symmetric.

The operation \emph{replace} removes a gate from a circuit and replaces it with
another gate.
  
\begin{definition}
  Formally, \emph{replace} is an algorithm that takes as input a circuit $C =
  \langle G, \Omega, \Sigma, \Lambda, L\rangle$ and a pair of gates $(g_1, g_2)$
  in $C$ and outputs a circuit $C' = \langle G', \Omega', \Sigma', \Lambda', L'
  \rangle$ such that
  \begin{itemize}
    \setlength\itemsep{0mm}
  \item $G' = G \setminus \{g_1\}$,
  \item $\Omega'$ is an function from the domain of $\Omega$ to $G'$ such that
    for each $\vec{a} \in \dom(Omega)$ if $\Omega(\vec{a}) = g_1$ then
    $\Omega'(\vec{a}) = g_2$ and if then $\Omega'(\vec{a}) = \Omega(\vec{a})$.
  \item $\Sigma' = \restr{\Sigma}{G'}$
  \item $\Lambda' = \restr{\Lambda}{G'}$
  \item For each $g \in G'$ and each $\vec{a} \in \ind(g)$ if $L(g)(\vec{a}) =
    g_1$ then $L'(g)(\vec{a}) = g_2$ and if not then $L'(g)(\vec{a}) =
    L(g)(\vec{a})$.
  \end{itemize}
\end{definition}

It can be shown that `replace' can be implemented so as to run in time
polynomial in the size of the input circuit.

The algorithm \emph{tag} takes in a circuit, a gate in that circuit, a relation
symbol and a tuple and outputs a circuit computing but with an additional
gadget, constructed using the gates labelled by the given relation symbol and
tuple, marking the designated gate. The idea is that marking a gate can be used
as a step to ensure unique labels in a circuit.

\begin{definition}
  The algorithm \emph{tag} takes in a $(\mathbb{B}, \tau)$-circuit $C_n :=
  \langle G, \Omega, \Sigma, \Lambda, L \rangle$, a relation symbol $R \in
  \tau$, a gate $g \in G$ and a tuple $\vec{a} \in [n]^k$ for some $k$ and
  outputs the circuit $C_n' = \langle G', \Omega', \Sigma', \Lambda', L'\rangle$
  defined by the following algorithm. Let $G' = G \uplus \{\land_g, \land_{g'},
  \land_1, \ldots , \land_k\}$, $\Omega' = \Omega$, and $\Lambda' = \Lambda$.
  Define $L'$ and $\Sigma'$ as follows. For each $g \in G$ $\Sigma'(g) =
  \Sigma(g)$. For each $g \in \{\land_0, \land_1, \ldots , \land_k \}$,
  $\Sigma(g)$ is an $\AND$-gate that takes in two inputs. $\Sigma(\land_g)$ is
  an $\AND$-gate that takes in a single input. Set $g$ to be the lone input to
  $\land_g$. Set the two inputs to $\land_0$ to be $\land_g$ and $\land_1$. For
  each $i \in [k-1]$ set the inputs of $\land_i$ to be $\land_{i+1}$ and the
  gate $g_i \in G'$ such that $\Sigma(g_i) = R$ and $\Lambda^{-1}(g_i) =
  \vec{a}_i$, where $\vec{a}_i$ is the $k$-tuple $(\vec{a}(i), \ldots,
  \vec{a}(i))$. Set the inputs of $\land_k$ to be the constant gate $1$ and the
  gate $g_k$ such that $\Sigma(g) = R$ and $\Lambda^{-1}(g_k) = \vec{a}_k$.

  For each gate $h \in G \setminus \{g\}$ and $\vec{b} \in \ind(h)$, if
  $L(h)(\vec{b}) = g$ then let $L'(h)(\vec{b}) = \land_0$, otherwise let
  $L'(h)(\vec{b}) = L(h)(\vec{b})$.

\end{definition}

It can be shown that `tag' can be implemented so as to run in time polynomial in
the size of the input. The input and output circuits compute the same function.
Moreover, if the input circuit had unique labels then the output circuit will
have unique labels. In the event that the input relation symbol is obvious from
the context we omit it when invoking this algorithm.

The algorithm \emph{glue-circuits} takes in a designated circuit and another set
of appropriate circuits and forms a new circuit by replacing the relational
gates of the designated circuit with the output gates of the circuits in the
given set.

\begin{definition}
  The algorithm \emph{glue-circuits} takes as input a set of relation symbols
  $\tau_R := \{R_1, \ldots, R_t\}$ and $t+1$ circuits $\{C^0, \ldots, C^{t}\}$,
  where $C^0 = \langle G_0, \Omega_0, \Sigma_0, \Lambda_0, L_0 \rangle$ is a
  $(\mathbb{B}, \tau)$-circuit with $\tau_R \subseteq \tau$ and for each $i \in
  [t]$, $C^i = \langle G_i, \Omega_i, \Sigma_i, \Lambda_i, L_i \rangle$ is a
  $(\mathbb{B}_i, \tau_i)$-circuit computing a query of arity $\arty(R_i)$. The
  algorithm outputs, for a given input, a circuit $C = \langle G, \Omega,
  \Sigma, \Lambda, L\rangle$ constructed as follows. Let $C^0' := C^0$. For each
  $i \in [t]$ fix a relation symbol $V_i \in \tau_i \setminnus \tau_R$. For each
  $i \in [t]$ construct the circuit $C^{i}' = \langle G_i', \Omega_i',
  \Sigma_i', \Lambda_i', L_i' \rangle $ by `inserting' $C^i$ into $C^{i-1}'$,
  and call the resultant circuit $D^{i-1}'$. For each relational gate $g$ in
  $D^{i-1}$ labelled by the symbol $R_i$, `replace' $g$ with the gate
  $\Omega_{i}(\Lambda_{i-1}'(g))$ in $D^{i-1}$ and `tag' the gate
  $\Omega_i(\Lambda_{i-1}'(g))$ in $D^{i-1}$ using the tuple $\Lambda_{i-1}'(g)$
  and the relation symbol $V_i$. Output the circuit $C := C^t'$.
\end{definition}

It can be shown that `glue-circuits' can be implemented so as to run in time
polynomial in the combined size of the input circuits. Moreover, if the input
circuits are all symmetric (resp. have unique labels) then the output circuit
will be symmetric (resp. have unique labels). We sometimes refer to `gluing'
$(C^1, \ldots, C^t)$ into $C^0$ along $(R_1 , \ldots , R_t)$ when invoking
`glue-circuit' on the input $(C^0, \ldots , C^t)$ and $\tau_R = \{R_1, \ldots ,
R_t\}$.

\subsection{Unique Labels and Rigidity}
In contrast with the work of Anderson and Dawar~\cite{}, the circuit families we
use to establish our main result are additionally required to have unique
labels. In this subsection we briefly discuss this inclusion, and present
results suggesting its importance.

It is important to note that our analysis of circuits depends crucially on the
fact that a circuit may be transformed (in polynomial time) into an equivalent
circuit with the property that each element of $\sym_n$ induces a unique
automorphism on the circuit $C_n$. This fact ensures a canonical action of
$\sym_n$ on the circuit $C_n$ and is a crucial requirement for defining needed
theoretical tools, such as supports, and computing various facts about the
circuit in polynomial time, such as the coarsest supporting partition or the
orbit of a gate.

Anderson and Dawar~\cite{AndersonD17} define a condition analogous to the one
define above and prove that it's a sufficient condition for establishing the
uniqueness of the induced automorphism on the circuit and that a circuit can be
transformed into an equivalent rigid circuit in polynomial time.

Although the rigidity condition we present in this paper is similarly defined,
the existence of a polynomial-time computable transformation of a circuit into
an equivalent rigid circuit is not as straight forward. Indeed, it is possible
to show that the problem of determining if a given $(\mathbb{B}, \tau)$-circuit
is rigid is as least as hard as the graph isomorphism problem. This is a direct
consequence of the following result.

\begin{prop}
  There is a a polynomial time reduction from the graph isomorphism problem to
  the problem of deciding if a specified gate in a matrix-circuit is syntacticly
  equivalent to any other gate in the circuit.
  \label{prop:syntactic-graph-iso}
\end{prop}
\begin{proof}
\end{proof}

Another obvious consequence of this result is that if the syntactic equivalence
relation for an input $(\mathbb{B}, \tau)$-circuit can be computed in polynomial
time then the graph isomorphism problem is in $\PT$. It is worth noting that if
we restrict the input to a class of circuits defined over symmetric bases then
we can decide the syntactic equivalence relation in polynomial time.

\begin{prop}
  Let $C_n$ be a circuit with symmetric gates. There is an algorithm that takes
  in such a circuit, runs in time polynomial in the size of the circuit, and
  outputs the syntactic equivalence relation on the gates of $C_n$.
  \label{prop:symmetric-syntactic}
\end{prop}
\begin{proof}
\end{proof}

Moreover, it turns out that being able to compute the syntactic equivalence
relation in polynomial time for some class of circuits $\mathcal{C}$ implies
that the existence of a polynomial-time algorithm for converting circuits in $C$
into equivalent rigid circuit.

\begin{prop}
  Let $\mathcal{C}$ be a set of circuits. Suppose the syntactic equivalence
  relation can be computed in polynomial time for circuits in $\mathcal{C}$. It
  follows that there is an algorithm that runs in polynomial time that takes in
  a circuit $C \in \mathcal{C}$ and outputs a rigid circuit $C'$ such that if
  $C$ is symmetric then $C'$ is symmetric.
  \label{prop:syntactic-equivilence-rigid}
\end{prop}
\begin{proof}
\end{proof}

The following result gives us that the unique labels condition is sufficient.

\begin{prop}
  There is an algorithm that takes in a $(\mathbb{B}, \tau)$-circuit with unique
  labels and outputs the syntactic equivalence relation on its gates. This
  algorithm runs in polynomial time.
  \label{prop:unique-labels-syntactic-equiv}
\end{prop}
\begin{proof}
\end{proof}

Putting together Propositions \ref{prop:symmetric-syntactic},
\ref{prop:syntactic-equivilence-rigid}, and
\ref{prop:unique-labels-syntactic-equiv}, we have that there is a
polynomial-time algorithm that takes as input a circuit with unique labels and
outputs an equivalent rigid circuit. In fact, there is an equivalence of sorts
between the unique labels condition and the rigidity condition.

\begin{prop}
  Let $\mathcal{C}$ be a family of circuits. There is an algorithm that runs in
  polynomial-time and takes in a circuit $C \in \mathcal{C}$ and outputs a rigid
  circuit $C'$ such that if $C$ is symmetric then $C'$ is symmetric if, and only
  if, there is an algorithm that runs in polynomial-time and takes in a circuit
  $C \in \mathcal{C}$ and outputs a circuit $C'$ with unique labels such that if
  $C$ is symmetric then $C'$ is symmetric.
\end{prop}

It follows then that for the symmetric circuits defined over bases of symmetric
functions discussed by Anderson and Dawar~\cite{} we may assume the circuits are
rigid and/or have unique labels without a loss of generality. For the more
general circuit discussed in this paper we cannot make such an assumption as
deciding rigidity is harder then graph isomorphism. However, if we restrict
ourselves to families of circuits with unique labels then again we may assume
rigidity without a loss of generality.

Of course, this restriction to circuits with unique labels is only useful if we
can show that formulas in rank logic can be translated into $P$-uniform families
circuits with unique labels. In the next subsection we prove that this is indeed
the case.

\subsection{Translating FPR to Symmetric Matrix Circuits}
Anderson and Dawar~\cite{} define (or reference) a translation of formulas of
$\FPC$ into equivalent $P$-uniform families of symmetric majority-circuits. More
formally, they show that for $\theta(\vec{x}) \in \FPC[\tau]$ there exists a
$P$-uniform family of symmetric majority-circuits $(C_n)_{n \in \nats}$ such
that the query computed by $(C_n)_{n \in \nats}$ is equal to the query defined
by $\theta (\vec{x})$. In this subsection we develop a similar translation from
formulas of $\FPR$ into $P$-uniform families of symmetric matrix-circuits.

% From Lemma \cite{} we may assume, without a loss of generality, that $(C_n)_{n
% \in nats}$ is a family of rigid circuits (and hence has unique labels).

We are now ready to prove the main theorem of this section.

\begin{thm}
  For any relational vocabulary $\tau$ and each $\theta(\vec{x}) \in \FPR[\tau]$
  there is an algorithm that takes as input a natural number $n$ and outputs a
  symmetric matrix-circuit $C_n$ with unique labels. This algorithm runs in time
  polynomial in $n$ and the family of circuits produced by this algorithm
  $(C_n)_{n \in \nats}$ computes the same query as the one defined by
  $\theta(\vec{x})$.
\end{thm}

\begin{proof}
  We say a formula $\psi(\vec{x}) \in \FPR[tau]$ is
  \emph{matrix-circuit-translatable} if there is a $P$-uniform family of
  symmetric matrix-circuits with unique labels that compute the same query as
  $\psi(\vec{x})$. If a formula $\psi(\vec{x}) \in \FPR[tau]$ is
  matrix-circuit-translatable we let $(C^\psi_n)_{n \in \nats}$ denote the
  corresponding $P$-uniform family of symmetric matrix-circuits with unique
  labels.

  We prove that $\theta(\vec{x}) \in \FPR[\tau]$ is matrix-circuit-translatable
  by structural induction. That is, we first prove that if every sub-formula of
  $\theta(\vec{x})$ is circuit-translatable then $\theta(\vec{x})$ is
  circuit-translatable and, second, we note that every atomic formula in
  $\FPR[\tau]$ is circuit-translatable. We have already that formulas of
  $\FPC[\tau]$ are circuit translatable (Anderson and Dawar~\cite{AndersonD17}),
  and in fact we may assume, without a loss of generality, that this translation
  yields a family of rigid circuits.
  
  % It is sufficient to prove that if $\theta(\vec{x})$ has the property that
  % all
  % of its subformulas have are matrix-circuit-translatable then there is an
  % algorithm that takes in a natural number $n$ and outputs a symmetric
  % matrix-circuit $C^\theta_n$ with unique labels that is equivalent to
  % $\theta(\vec{x})$ on structures of size $n$.

  Suppose $\phi_1, \phi_2 \in \FPR[\tau]$ are matrix-circuit-translatable and
  $\theta (\vec{x}) = \phi_1(\vec{x}) \land \phi_2(\vec{x})$. Let $R_{\phi_1}$
  and $R_{\phi_2}$ be two $\vert \vec{x} \vert$-ary relation symbols not in
  $\tau$. Let $\theta' (\vec{x}) := R_{\phi_1}(\vec{x}) \land
  R_{\phi_2}(\vec{x})$. This is a formula of $\FPC$, and so we may efficiently
  compute $C^{\theta'}_n$ using the construction in \cite{AndersonD17}. From the
  inductive hypothesis, for each $i \in \{1,2\}$ we may also efficiently compute
  the circuit $C^{\phi_i}_n = \langle G_i, \Omega_i, \Sigma_i Lambda_i, L_i
  \rangle$. We construct $C^{\theta}_n$ by `gluing' $C^{\phi_1}$ and
  $C^{\phi_2}$ into $C^{\theta'}_n$ along $R_1$ and $R_2$

  % We construct the circuit $C'_n = \langle G', \Omega', \Sigma',
  % \Lambda', L' \rangle$ by `inserting' $C^{\phi_1}_n$ and $C^{\phi_2}_n$ into
  % $C^{\theta'}_n$. We now construct $C^{\theta}_n$ from $C'_n$ as follows:
  % \begin{itemize}
  % \item For each $i \in \{1,2\}$ and each relational gate $g$ in $C_n$ such
  %   that $\Sigma'(g) = R_{\phi_i}$ `replace' $g$ with
  %   $\Omega_i(\Lambda_i(g))$,
  % \item then, for each syntactic equivalence class $H \subseteq
  %   \range(\Omega_1) \uplus \range(Omega_2)$, `merge' $H$, and finally
  % \item run `separate-labels' on the circuit.
  % \end{itemize}

  We note that, since $C^{\phi_1}_n$, $C^{\phi_2}_n$ and $C^{\theta'}_n$ have
  unique labels, the only gates that might not have unique labels in
  $C^{\theta}_n$ are those whose input gates include gates in the range of
  $\Omega_1$ or $\Omega_2$. However, all such gates are symmetric and steps (2)
  and (3) above guarantee that these gates have unique labels. Finally, we note
  that this algorithm can implemented such that the computation of
  $C^{\theta}_n$ from $n$ runs in time polynomial in $n$.

  Analogous algorithms can be used to prove that other formulas are
  circuit-translatable. Indeed, for circuit-translatable $\phi, \phi_1, \phi_2
  \in \FPR[\tau]$ an analogous algorithm allows us to prove that
  $\phi_1(\vec{x}) \lor \phi_2(\vec{x})$, $\neg \phi(\vec{x})$, $\forall y. \phi
  (\vec{x}, y)$, $\exists y. \phi (\vec{x}, y)$ are all circuit-translatable.

  Suppose $\theta(\vec{x}) = [\lfp_{V, \vec{y}} \phi(\vec{y}, V)](\vec{x})$ for
  some circuit-translatable formula $\phi(\vec{x}, \vec{y}) \in FPR[\tau \uplus
  \{V\}]$ and $k$-ary relation symbol $V$. We now describe an algorithm that
  takes in a natural number $n$ and outputs $C^{\theta}_n$. We have that we may
  efficiently compute $C^{\phi}_n$. For each $i \in [n^k]$ we construct a
  circuit $C^i = \langle G^i, \Omega^i, \Sigma^i, \Lambda^i, L \rangle $ as
  follows. Let $C^1$ be $C^{\phi}_n$ but with all relational gates labelled by
  the symbol $V$ replaced with a $0$. For each $i>1$ we construct $C^{i}$ by
  `gluing' $C^{i-1}$ into $C^\phi_n$, producing $C^i$. Set $C^{\theta}_n :=
  C^{n^k}$. Since $C^1$ and $C^\phi_n$ are symmetric and have unique labels, by
  induction for each $i > 1$ $C^i$ is symmetric and has unique labels. It
  follows that $C^{\theta}_n$ is symmetric and has unique labels. Moreover,
  since the least-fixed-point operator must converge in at most $n^k$ steps,
  this algorithm runs in time polynomial in $n$ and the family
  $(C^{\theta}_n)_{n \in \nats}$ computes the same query as $\theta(\vec{x})$.

  We have from \cite{} that a formula containing number terms may be converted
  into an equivalent family of formulas containing counting quantifiers. More
  formally if $\theta (\vec{x})$ contains number terms then there exists an
  algorithm that takes as input a natural number $n$ and outputs a formula
  $\theta_n(\vec{x}) \in C$ with counting quantifiers and no number terms which
  is equivalent to $\theta(\vec{x})$ for structures of size $n$ and does not
  contain any number terms. Moreover, this algorithm runs in time polynomial in
  $n$.

  Thus to handle the number-term case it is sufficient to show that if
  $\phi(\vec{x}) \in \FPR[\tau]$ is circuit-translatable then $\theta(\vec{x})
  := C^{k}y. \phi(\vec{x}, y)$ is circuit translatable. Again, an analogous
  construction as for the $\AND$-gate case described in detail above is
  sufficient.

  We have from \cite{} that a formula containing rank operator may be converted
  into an equivalent family of formulas containing rank quantifiers. In other
  words for a given formula $\theta(\vec{x}) \in \FPR[\tau]$ containing rank
  operators there exists an algorithm that takes as input a natural number $n$
  and outputs a formula $\theta_n(\vec{x})$ with rank quantifiers and no rank
  operators. It is thus sufficient to prove that if $\phi(\vec{z}, \vec{y},
  \vec{x})$ is circuit-translatable, $p$ is prime and $r$ is a natural number
  then $\theta(\vec{x}) = \rank^{\leq r}_p \vec{z} \vec{y}. \phi(\vec{z},
  \vec{y}, \vec{x})$ is circuit-translatable. We define an algorithm that takes
  as input a natural number $n$ outputs the required circuit $C^{\theta}_n$.

  We may efficiently compute the $(\mathbb{B}_{\rank}, \tau)$-circuit
  $C^{\phi}_n = \langle G^{\phi}, \Omega^{\phi}, \Sigma^{\phi}, \Lambda^{\phi},
  L^{\phi} \rangle$. Let $k:= \vert \vec{x} \vert$, $k_1 : = \vert \vec{z}
  \vert$ and $k_2 := \vert \vec{y} \vert$. We now define a $(\mathbb{B}_{\rank},
  \{ R \})$-circuit $C':= \langle G', \Omega', \Sigma', \Lambda', L'\rangle$,
  where $R$ is a $(k_1 + k_2 +k)$-arity relation symbol. Let $G_1' :=
  \{g_{\rank, \vec{c}} : \vec{c} \in [n]^k\}$ and $G_2' : = \{ g_{R, \vec{a},
    \vec{b}, \vec{c}} : \vec{a} \in [n]^{k_1}, \vec{b} \in [n]^{k_2}, \vec{c}
  \in [n]^{k}\}$.

  \begin{itemize}
    \setlength\itemsep{0mm}
  \item Let $G' := G_1' \uplus G_2'$,
  \item $\Omega'$ be an injection from $[n]^k$ to $G'$ defined by
    $\Omega'(\vec{c}) = g_{\rank, \vec{c}}$,

  \item $\Sigma' (g) := rk^r_p$ if $g \in G_1'$ and $\Sigma'(g) = R$ if $g \in
    G_2'$,
  \item For each $(\vec{a}, \vec{b}, \vec{c}) \in [n]^{k_1 +k_2 + k}$ let
    $\Lambda' (\vec{a}, \vec{b}, \vec{c}) := g_{R, \vec{a}, \vec{b}, \vec{c}}$,
    and

  \item For each gate $g_{\rank, \vec{a}} \in G_1'$ let $L(g_{\rank, \vec{a}}):
    [n]^{k_1} \times [n]^{k_2} \rightarrow G'$ be defined by $L(g_{\rank,
      \vec{c}})(\vec{a}, \vec{b}) := g_{R, \vec{a}, \vec{b}, \vec{c}}$ for each
    $\vec{a} \in [n]^{k_1}$ and $\vec{b} \in [n]^{k_2}$.

    \end{itemize}
    Let $C^{\theta}_n$ be the result of `gluing' $C^{\phi}_n$ into $C'$.

    It is easy to see that both $C'$ and $C^{\phi}_n$ are symmetric
    rank-circuits with unique labels. Thus $C^{\theta}_n$ is a symmetric
    rank-circuit with unique labels. The construction of $C'$, and hence
    $C^{\theta}_n$, can be implemented so as to run in time polynomial in $n$.
    The construction ensures that the circuit family $(C^{\theta}_n)_{n \in
      \nats}$ computes the same query as the one defined by $\theta(\vec{x})$.

    It remains only to note that any atomic formula of $\FPR[\tau]$ is trivially
    circuit-translatable. The result thus follows.
  \end{proof}

  % such that for any structure $\mathcal{A}$ of size $n$ and an assignment $$
  % to the tuple of variables $\vec{x}$

  % is symmetric and has unique labels.

  % an equality between two numerical terms or an application of the
  % least-fixed-point operator,



  % it follows from Lemma \ref{}, that the syntactic equivalence relation on
  % $G_1 \uplus G_2$ can be efficiently computed.

  % Note that the construction guarantees that $C^{\theta}_n$ is symmetric.
  % Moreover, since the only gates in $C^{\theta}_n$ are either output gates
  % or gates


  % Suppose $\theta(\vec{x})$ is of the form $\phi_1(\vec{x}) \land
  % \phi_2(\vec{x})$, $\phi_1(\vec{x})\lor \phi_2(\vec{x})$, $\neg
  % \phi(\vec{x})$, $\thr_k(y).\phi(\vec{x}, y)$, $\forall y. \phi (\vec{x},
  % y)$, $\exists y. \phi (\vec{x}, y)$, an equality between two numerical
  % terms or an application of the least-fixed-point operator, and each of
  % these sub-formulas is $(\mathbb{B}, \tau)$-circuit-translatable} then
  % $\theta(\vec{x})$ is $(\mathbb{B}, \tau)$-circuit-translatable}. This
  % follows from the following construction. Let $n \in \nats$. We take derive
  % the formula $\theta'(\vec{x})$ from $\theta(\vec{x})$ by replacing each
  % sub-formula $\phi(\vec{z})$ with the formula $R_\phi(\vec{z})$ (adding in
  % some new relational symbol $R_\phi$). We may then translate
  % $\theta'(\vec{x})$ into a $P$-uniform family of symmetric circuits with
  % threshold and unique labels $(C^{\theta'}_n)_{n \in \nats}$. For each $n$
  % we then replace the relational gates in $C^{\theta'}_n$ labelled by
  % $R_\phi$ with the circuit $C^{\phi}_n$ (that is, we add in the entire
  % circuit $C^\phi_n$ and replace each relational gate $g$ labelled by
  % $R_\phi$ such that $\vec{a}:= \Lambda(\vec{a})$ with with the output gates
  % of $C^{\phi}_n$ such that if $g$ is a relational gate computing
  % $R_\phi(\vec{a})$ then $g$ is replaced by the gate $\Omega^{-1}(\vec{a})$
  % in $C^{\phi}_n$). We call this new circuit $C^\theta_n$. It is easy to see
  % that this circuit is symmetric. To see that it has unique labels, notice
  % that we need only check

  % , if $L;$ $L'(g) := L(g)$ for all $g \in G' \cap G$ and for all $g \in H$
  % $L'(H) := L(g)$ if $g \in H$



  % \begin{lem}
  %   We say a
  % \end{lem}

  % We should like to argue that for any $\theta(\vec{x}) \in \FPR[\tau]$ if
  % every sub-formula of $\theta(\vec{x})$ is $(\mathbb_{\mat}, \tau)$-circuit
  % translatable then $\theta(\vec{x})$ is $(\mathbb_{\mat}, \tau)$-circuit
  % translatable. The existence of the desired translation then follows by
  % induction on the structure of the formula.

  % It is easy to see that if $\theta(\vec{x})$ is of the form $\phi_1(\vec{x})
  % \land \phi_2(\vec{x})$, $\phi_1(\vec{x})\lor \phi_2(\vec{x})$, $\neg
  % \phi(\vec{x})$, $\thr_k(y).\phi(\vec{x}, y)$, $\forall y. \phi (\vec{x},
  % y)$, $\exists y. \phi (\vec{x}, y)$, an equality between two numerical terms
  % or an application of the least-fixed-point operator, and each of these
  % sub-formulas is $(\mathbb{B}, \tau)$-circuit-translatable} then
  % $\theta(\vec{x})$ is $(\mathbb{B}, \tau)$-circuit-translatable}. This
  % follows from the following construction. Let $n \in \nats$. We take derive
  % the formula $\theta'(\vec{x})$ from $\theta(\vec{x})$ by replacing each
  % sub-formula $\phi(\vec{z})$ with the formula $R_\phi(\vec{z})$ (adding in
  % some new relational symbol $R_\phi$). We may then translate
  % $\theta'(\vec{x})$ into a $P$-uniform family of symmetric circuits with
  % threshold and unique labels $(C^{\theta'}_n)_{n \in \nats}$. For each $n$ we
  % then replace the relational gates in $C^{\theta'}_n$ labelled by $R_\phi$
  % with the circuit $C^{\phi}_n$ (that is, we add in the entire circuit
  % $C^\phi_n$ and replace each relational gate $g$ labelled by $R_\phi$ such
  % that $\vec{a}:= \Lambda(\vec{a})$ with with the output gates of $C^{\phi}_n$
  % such that if $g$ is a relational gate computing $R_\phi(\vec{a})$ then $g$
  % is replaced by the gate $\Omega^{-1}(\vec{a})$ in $C^{\phi}_n$). We call
  % this new circuit $C^\theta_n$. It is easy to see that this circuit is
  % symmetric. To see that it has unique labels, notice that we need only check


  % We solve this problem completely for one case and argue that the other cases
  % follow similarly. Suppose $\theta (\vec{x}) = \phi_1(\vec{x}) \land
  % \phi_2(\vec{x})$. We now describe an algorithm that takes in $n\in \nats$
  % and outputs a formula $C^\theta_n$ with the required properties. We first
  % introduce two relation symbols $R_{\phi_1}$ and $R_{\phi_2}$ not in $\tau$
  % and with arity $\vert\vec{x}\vert$. We define $\theta' (\vec{x}) =
  % R_{\phi_1}(\vec{x}) \land R_{\phi_2}(\vec{x})$ and compute the associated
  % symmetric threshold circuit with unique labels $C^{\theta'}_n$. Similarly,
  % for each $i\ \in \{1,2\}$ we may compute the symmetric $(\mathbb{B},
  % \tau)$-circuit with unique labels corresponding to $\phi_i$ that takes in
  % inputs of size $n$, $C^{\phi_i}_n = \langle G_i, \Omega_i, \Sigma_i
  % Lambda_i, L_i \rangle$ algorithms given by the inductive hypothesis. We then
  % construct $C^{\theta}_n$ by inserting into $C^{\theta'}_n$ the circuits
  % $C^{\phi_1}_n$ and $C^{\phi_2}_n$ (that is, taking a disjoint union of the
  % sets of gates and the functions that define each circuit -- with the
  % exception of $\Omega$ which is specified by $C^{\theta'}_n$ alone). Then for
  % all $i \in \{1,2\}$ and each relational gate $g$ in $C^{\theta}_n$ such that
  % $\Sigma(g) = R_{\phi_i}$ we replace $g$ with $\Omega_i(\Lambda_i(g))$. Since
  % $C^{\phi_1}_n$ and $C^{\phi_2}_n$ have unique labels, it follows from Lemma
  % \ref{} that we can efficiently decide if any two gates $g_1, g_2 \in G_1
  % \uplus G_2$ are syntactical equivalent. For each pair of gates $g_1, g_2$ in
  % the range of $\Omega_1 \uplus \Omega_2$ we check if $g_1 \equiv g_2$, and if
  % so we merge them.

  % The resultant circuit $C^{\theta}_n$ is clearly a symmetric $(\mathbb{B},
  % \tau)-circuit$-circuit. Moreover, it is easy to see that if all pairs of
  % gates in the range of $\Omega_1 \uplus Omega_2$ are not syntactical
  % equivalent $g$ is a gate such that $h_1, h_2 \in H_g$ are syntactically
  % equivalent then . it has unique labels (as the only point where two input
  % gates could be equivalent

  % we can from Lemma \ref{} that we can efficiently decide if two any two gates
  % from the union of these circuits are syntactical equivalent. If any of their
  % output gates are syntactically equivalent, merge these gates. This new
  % circuit has disjoint


  % we replace the relational gate $g$ in $C^{\theta'}_n$ such that $\Sigma(g) =
  % R_\phi$ and $\Lambda(g) = \vec{a}$ with the circuit $C^\phi_n$ $\vec{a}$, an
  % $\arty(R_\pi)$ each relational gate $R_\phi(\vec{a})$ in the circuit
  % $C^{\theta'}_n$


  % It remains to show that for any $k \in \nats$ and formula $\phi \in
  % \FPR[\tau]$ such that $\phi$ formula of the form $\theta (\vec{z}): = [rk
  % (\vec{x}, \vec {\nu} \leq \vec{t}, \vec{y}\vec{\mu} \leq \vec{s}, \pi \leq
  % r). \phi(\vec{x}, \vec{nu}, \vec{y}, \vec(mu), \vec{s}, \vec{z})] \leq k$
  % there is an equivalent $P$-uniform family of symmetric rank-circuits with
  % unique labels $(C_n)_{n \in \nats}$ that compute the same query as $\theta$.


  % \begin{lem}
  %   Let $\tau$ be a relational vocabulary and let $\phi(\vec{x})$ be a formula
  %   of $\FPR[\tau]$. There is an algorithm that takes as input a natural
  %   number $n$ and outputs a formula $\phi_n$ such that $\phi_n$ contains no
  %   rank operators or counting terms and for any $\tau$-structure
  %   $\mathcal{A}$ of size $n$ and $\vert \vec{x} \vert$-tuple of elements
  %   $\vec{a}$ from the universe of $\mathcal{A}$, $\mathcal{A}\models
  %   \phi[\vec{a}]$ if, and only if, $\mathcal{A}\models \phi_n[\vec{a}]$.
  %   Moreover, this algorithm runs in time polynomial in $n$.
  % \end{lem}

  % \begin{thm}
  %   Let $\tau$ be a relational vocabulary and let $\phi$ be a formula of
  %   $\FPR[\tau]$. There is an algorithm that takes as input a natural number
  %   $n$ and outputs a symmetric matrix-circuit $C_n$. This algorithm runs in
  %   time polynomial in $n$. Moreover, the query computed by the circuit family
  %   $(C_n)_{n \in \nats}$ and the query defined by the formula $\phi$ are
  %   equal.
  % \end{thm}
  % \begin{proof}
  %   We describe a transformation inductively on the structure of the formula.
  %   This procedure can then be implemented algorithmically using recursion.
  
  %   If $\phi(\vec{a})$ is an atomic formula
  
  % \end{proof}

  % We define this transformation by recursion on the structure of the formula
  % and note that and note that such a transformation can be implemented so as
  % to run in time polynomial in $n$.

  % Let $\phi$ be a formula




  % \begin{thm}
  %   Formulas of $\FPR$ can be translated into symmetric matrix-circuits
  % \end{thm}
  % \begin{proof}
  %   The idea here is to do exactly the same thing as for FPC formulas except
  %   when we arrive at a rank quantifier we `tag' each formula with
  % \end{proof}
 \end{document}