\documentclass[../paper.tex]{subfiles}

\begin{document}
Symmetric circuits, as developed by Anderson and Dawar, are, importantly defined
over (complete) Boolean basis consisting of symmetric functions. The following
result shows that once a majority gate has been included in this basis any
symmetric function can be computed using polynomial-size symmetric circuits, and
as such no additional symmetric function added to the Basis improves the power
of the model.

Recall from Anderson and Dawar \cite{AndersonD17} we have that $\mathbb{B}_{\std} = \{
\neg , \wedge , \lor \}$ and $\mathbb{B}_\maj = \{ \maj \} \cup \mathbb{B}$.

Let $F: \{0,1\}^* \rightarrow \{0,1\}$ be a symmetric Boolean function. Since
$F$ is symmetric we note that for a fixed size input the output of $F$ is
entirely determined by the number of 1's in its input. Then let
$c_{F}:\mathbb{N} \rightarrow 2^{\mathbb{N}}$ define a function where $c_{F}(n)$
is the set of all $m \leq n$ such that for all $\vec{x} \in \{ 0,1 \}^n$ with
$m$ 1's we have $F (\vec{x}) = 1$. Clearly any symmetric Boolean function $F$ is
entirely determined by $c_{F}$.
 
\begin{prop}
  \label{prop:fuctions-maj}
  There is a polynomial $p(k)$ such that for any symmetric function $F$ and a
  given $k \in \mathbb{N}$ there is a circuit $C_k$ on $k$ inputs over the basis
  $\mathbb{B}_\maj$ which is symmetric, constant depth and with width bounded by
  $p(k)$.
\end{prop}

\begin{proof}
  We define the circuit $C_k$ for inputs $\vec{x} = ( x_1, \ldots, x_k )$.

  For $a \in \mathbb{N}$ we define a get $\countgate_a$ by
  \begin{align*}
    &\countgate_a = \maj (x_1, \ldots, x_k, \underbrace {0, \ldots, 0}_{2a -
      k}) \land \neg \maj (x_1, \ldots , x_k, \underbrace{0, \ldots,  0}_{2a - k + 2})\text{ if $a \geq \frac{k}{2}$,} \\
    &\countgate_a = \maj (x_1, \ldots, x_k, \underbrace {1, \ldots, 1}_{x -
      2a}) \land \neg \maj (x_1, \ldots , x_k, \underbrace{1, \ldots,
      1}_{k - 2a -2}) \text{ if $a < \frac{k}{2}$. }
  \end{align*}

  Then let $g = \bigvee_{a \in c_{F_i}(k)}\countgate_a$ and let $C_{k}$ be the
  circuit with input gates labeled by $\vec{x}$ and output gate $g$.

  t is easy to see that $C_k$ is constant depth and it's width is a polynomial
  in $k$. We have that in each layer of $C_{g'}$ each gate is connected to all
  gates in the previous layer, and as such the circuit is symmetric.
\end{proof}

The above proposition has a straight forward application to circuit
characterisations.

\begin{thm}
  Let $F = \{F_i : i \in I \}$ be a family of symmetric Boolean functions where
  $F_i: \{0,1\}^* \rightarrow \{ 0,1 \}$.

  Let $(C_n)_{n \in \mathbb{N}}$ be family of symmetric circuits over the
  Boolean basis $\mathbb{B} \cup F$, where $C_n$ is a circuit on structures of
  size $n$, and the size of each circuit in the family is bounded by some
  function $f(n)$. Then there exists a polynomial $q(n)$ and a family of
  symmetric circuits $(C_n')_{n \in \mathbb{N}}$ over $\mathbb{B}_\maj$, where
  $C_n'$ is a circuit on structures of size $n$ and $\vert C_n' \vert \leq
  q(f(n))$.
\end{thm}

\begin{proof}
  From $C_n$ we construct $C_n'$ in the obvious way. For each gate $g \in C_n$
  of type $F_i$ we have a symmetric circuit $C_g$ from Proposition
  \ref{prop:function-maj} that computes the same function as $g$. Then let
  $C_n'$ be $C_n$ but with each gate $g$ replaced by $C_g$. It is easy to see
  that $C_n'$ is symmetric. We also have that each gate $g$ must have at most
  $f(n)$ inputs, and the size of $C_g$ is bounded by $p(f(n))$. Thus the size of
  $C_n'$ is bounded by $f(n)p(f(n))$.
\end{proof}

\subsection{G-Symmetric Functions}
In order to extend the notion of a circuit by extending the basis we need to
consider functions that are not symmetric in the full sense of the term but have
some weakened notion of symmetry.

\begin{definition}
  Let $f: \{0,1\}^X \rightarrow \{0,1\}$ be a function. Let $G \subseteq \sym_X$
  be a subgroup. We say that $f$ is \emph{$G$-symmetric} if for all $\sigma \in
  G$, $x \in X$, $f(\sigma x) = f(x)$.
\end{definition}

We can also introduce similar notions in the case of a two dimensional Boolean
string.

\begin{definition}
  Let $f: \{0,1\}^{A \times B} \rightarrow \{0,1\}$ be a function. Let $G
  \subseteq \sym_{A \times B}$ be a subgroup. We say that $f$ is
  \emph{matrix-symmetric} if it is $\sym_A \times \sym_B$-symmetric and we say
  that $f$ is \emph{graph-symmetric} if $A = B$ and it is $\sym_A$-symmetric
  (with $\sym_A$ acting on both $A$ and $B$).
\end{definition}

So if we think of $f: \{0,1\}^{A \times B} \rightarrow \{0,1\}$ as representing
a matrix, then matrix-symmetry corresponds to the property of the function being
invariant under row-column permutations. If, instead, we have that $A = B$ and
we then think of $f$ as encoding the adjacency matrix of a graph, then
graph-symmetry corresponds to the function being constant on isomorphism
classes.

Importantly, many natural functions of interest are matrix symmetric. For
example, the function that computes the rank of the matrix over $\mathbb{F}_2$.
or a thresholded rank function, for example the rank of the matrix over
$\mathbb{F}_p$ being larger then $r$, for some particular $(p, r) \in
\mathbb{N}$.


In this section we develop the corresponding notion of a symmetric circuit, ones
which allow for the inclusion of matrix-symmetric gates. In particular, these
circuits are defined using two sets of Boolean functions. The first, denoted by
$\mathbb{B}_\sym$, is taken to be the symmetric Boolean basis (either
$\mathbb{B}_\std$ or $\mathbb{B}_\maj$). The second, denoted by
$\mathbb{B}_\matsym$ is taken to be the matrix-symmetric basis, consisting of a
collection of matrix-symmetric functions.

\begin{definition}[Circuits on Structures]
  For $\mathbb{B}_{\sym}$ a basis of Boolean symmetric functions,
  $\mathbb{B}_{\matsym}$ a basis of Boolean matrix-symmetric functions and
  $\tau$ a set of relation symbols, we define a $(\mathbb{B}_\sym,
  \mathbb{B}_\sym, \tau)$-circuit $C_n$ computing a $q$-ary query $Q$ is a
  structure $\langle G, W, \Omega, \Sigma, \Lambda, L\rangle$.
  \begin{itemize}
    \setlength\itemsep{0mm}
  \item $G$ is called the set of gates of $C_n$ and $\vert C_n \vert := \vert G
    \vert$.
  \item $W \subseteq G \times G$, where $W$ is called the wires of the circuit.
    $(G,W)$ must be a directed acyclic graph. For $g \in G$ we $H_g := \{ h \in
    C_n : W(h,g)\}$ be the set of children of $g$.
  \item $\Omega$ is an injective function from $[n]^q$ to $G$. The gates in the
    image of $\Omega$ are called the output gates. When $q = 0$, $\Omega$ is a
    constant function mapping to a single output gate.
  \item $\Sigma$ is a function from $G$ to $\mathbb{B}_\sym \uplus
    \mathbb{B}_\matstab \uplus \tau \uplus \{0,1\} $ which maps input gates to
    $\tau \uplus \{0,1\}$ and where $\Sigma^{-1} (0) \leq 1$ $\Sigma^{-1} (1)
    \leq 1$ and the internal gates get mapped into $\mathbb{B}_\sym \uplus
    \mathbb{B}_\matstab$. Gates mapped to $\tau$ are called relational gates and
    gates mapped to 1 or 0 are called constant gates.
  \item $\Lambda$ is a sequence of injective functions $(\Lambda_R)_{R \in
      \tau}$ where for each $R \in \tau$, $\Lambda_R$ maps each relational gate
    $g$ with $R = \Sigma (g)$ to the tuple $\Lambda_R (g) \in [n]^r$, where $r$
    is the arity of the symbol $R$. When no ambiguity arises we write $\Lambda
    (g)$ for $\Lambda_R (g)$.
  \item $L$ assigns labels to the gates (and their inputs) in $C_n$. Let $g \in
    G$ be an internal gate and $H$ be the gates input to $g$. Then $L(g) : A
    \times B \rightarrow H$, where $L(g)$ must be surjective and $A, B \subseteq
    [n]$. We call this the \emph{matrix labelling} of $g$.
  \end{itemize}
\end{definition}

Note the additional requirement that each gate have an associated labelling in
order to facilitate the evaluation of the gate.

Given some finite $\tau$-structure $\mathcal{A}$ of size $n$ and some bijection
$\gamma: U \rightarrow [n]$, the evaluation of some $(\mathbb{B}_\sym,
\mathbb{B}_\sym, \tau)$-circuit) circuit $C$ proceeds by recursively evaluating
gates. Here the evaluation of the gate $g$ is denoted by $C[\gamma
\mathcal{A}](g)$. If $g$ is anything but a matrix-symmetric gate, then the
evaluation is as per the symmetric circuits of Anderson and Dawar \cite{}. If
$g$ is matrix-symmetric then the evaluation is given by applying the Boolean
operation $\Sigma(g)$ to the matrix $L^{\gamma}:\dom(L(g)) \rightarrow \{0,1\}$
defined by $L^{\gamma}(i,j) = C[\gamma \mathcal{A}](L(g)(i,j))$.


\begin{definition}[Invariant Circuit\cite{AndersonD17}]
  Let $C_n$ be a $(\mathbb{B}_\sym, \mathbb{B}_\sym, \tau)$-circuit, computing
  some $q$-ary query. We say $C_n$ is \emph{invariant} if for every
  $\tau$-structure $\mathcal{A}$ of size $n$, $a \in \mathcal{A}^q$, and
  bijections $\gamma_1, \gamma_2: U \rightarrow [n]$ we have that $C[\gamma_1
  \mathcal{A}](\Omega (\gamma_1 a)) = C[\gamma_2 \mathcal{A}](\Omega (\gamma_2
  a))$.
\end{definition}


\begin{definition}[Automorphism]
  let $C = \langle G, W, \Omega, \Sigma, \Lambda, L\rangle$ be a
  $(\mathbb{B}_\sym, \mathbb{B}_\matsym, \tau)$-circuit computing at $q$-ary
  query on structures of size $n$. Let $\sigma \in \sym_n$ and $\pi: G
  \rightarrow G$ be a bijection such that
  \begin{itemize}
    \setlength\itemsep{0mm}
  \item for all gates $g, h \in G$, $W(g,h)$ iff $W(\pi g, \pi h)$,
  \item for all output tuples $x \in [n]^q$, $\pi \Omega (x) = \Omega (\sigma
    x)$,
  \item for all gates $g \in G$, $\Sigma (g) = \Sigma (\pi g)$,
  \item for each relational gate $g \in G$, $\sigma \Lambda (g) = \Lambda (\pi
    g)$, and
  \item for each internal gate $g$ if $\Sigma (g) \in \mathbb(B)_\matsym$ then
    we have that $L(\pi g) \sim \pi \cdot L(g)$.
  \end{itemize}

  We call $\pi$ an \emph{automorphism} of $C$, and we say that $\sigma$
  \emph{induces the automorphism} $\pi$. The group of automorphisms of $C$ is
  called $\aut_n (C)$.
\end{definition}

\begin{definition}[Symmetric\cite{AndersonD17}]
  A circuit $C$ on structures of size $n$ is called \emph{symmetric} if every
  $\sigma \in \sym_n$ induces an automorphism on $C$.
\end{definition}

It follows for any symmetric circuit $C_n$ there is a mapping from $\sym_n$ to
$\aut_n(C)$, from $ \sigma$ to the induced automorphism. This homomorphism is
injective so long as a single element of $[n]$ appears in the in the label of
some input gate of $C$ (as then all elements appear by
symmetry)\cite{AndersonD17}. In this paper we always assume that there is always
one such element as otherwise all inputs are constant, and so the circuit just
computes a constant function. In order to assure this homomorphism is surjective
Anderson and Dawar \cite{AndersonD17} introduce the notion of a \emph{rigid}
circuit.

\begin{definition}[Rigidity]
  Let C be a $(\mathbb{B}_\sym, \mathbb{B}_\sym, \tau)$-circuit, where $C =
  \langle G, W, \Omega, \Sigma, \Lambda, L\rangle$. Say that $C$ is rigid if
  there are no internal gates $g, g' \in G$ such that $\Sigma(g) = \Sigma (g')$,
  $\Omega^{-1}(g) = \Omega^{-1}(g')$, and for every $g'' \in G$, $W(g'', g')$
  and $W(g,g'')$ iff $W(g', g'')$.
\end{definition}

Another property which simplifies our analysis is the property of having
\emph{bijective labels}.

\begin{definition}
  We say that a circuit $C$ has bijective labels if for each gate $g$ in $C$,
  $L(g)$ is a bijection.
\end{definition}

We prove in a later section that a circuit may be transformed in polynomial time
into an equivalent circuit that is both rigid and has bijective labelings. Hence
we may assume these two properties without a loss of generality.

With the assumption of rigidity in place, we abuse notation for permutations and
let $\sigma \in \sym_n$ also denote the induced automorphism.

We finally want to define symmetric circuits with rank gates.

\begin{definition}
  Let $\mathbb{B}_{\matsym}$ consist of functions of the form $f_{p,r}:
  \{0,1\}^{A \times B} \rightarrow \{0,1\}$ for $A, B$ initial segments of
  $\mathbb{N}$ and $f, p \in \mathbb{N}$, with $p$ prime, and such that for
  $f_{p,r}(M) = 1$ iff the matrix $M: A \times B \rightarrow \{0,1\}$ has rank
  at least $r$ over the field $\mathbb{F}_p$. A \emph{symmetric circuit with
    rank gates} is a $(\mathbb{B}_\maj, \mathbb{B}_\matsym, \tau)$-circuit, for some
  vocabulary $\tau$.
\end{definition}

The natural restriction to consider on families of circuits is uniformity.

\begin{definition}
  Let $(C_n)_{n \in \mathbb{N}}$ be a family of Boolean circuits. We say that
  $(C_n)_{n \in \mathbb{N}}$ is \emph{$P$-uniform} if the mapping $n \mapsto C_n$ is
  computable in polynomial time.
\end{definition}

We are now ready to state what will be the main theorem of this paper.

\begin{thm}[Main Theorem]
 A graph property is decidable by a $P$-uniform family of symmetric circuits
 with rank gates if, and only if, it is definable by an FPR sentence.
\end{thm}

% \section{Circuits to FPR}

% In this section I discuss how to determine the rank a matrix labelling at some
% gate $g$. This is a rough copy written for the purpose of discussion. In this
% section we fix some circuit $C_n$ and some rank gate $g$ in $C_n$ with a set
% of children $H$ and matrix labelling $L: [a] \times [b] \rightarrow H$ (which
% we assume WLOG is a bijection).
\end{document}

