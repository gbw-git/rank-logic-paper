\documentclass[../paper.tex]{subfiles}
\begin{document}

The main result of this paper establishes an equivalence between the expressive
power of $\FPR$ and the computational power of $\PT$-uniform families of
transparent symmetric rank-circuits. This work generalises an analogous
characterisation of the expressive power of $\FPC$ using $\PT$-uniform families
of symmetric circuits (with symmetric gates) presented by Anderson and
Dawar~\cite{AndersonD17}. Our initial intention in developing this
characterisation was to provide an alternative approach for studying the
expressive power of $\FPR$. Since $\FPR$ is one of the two candidate logics for
capturing $\PT$, understanding its expressive power is a central question in the
field. Our result establishes a correspondence between inexpressibly results for
$\FPR$ and circuit lower bounds (for the specific circuit model discussed), and
so enables the application of many of the well-developed tools for finding
circuit lower bounds to the problem understanding the expressive power of
$\FPR$.

However, while this application is certainly promising, it is perhaps the
developments required for the proof of the main result, rather than the result
itself, that deserve the most attention. When we started to work on this problem
we believed that a circuit characterisation for $\FPR$ would require little more
than adding rank gates to Anderson and Dawar's symmetric circuits, and then
modifying the support theorem and few other key results. However the definition
of a circuit they use, and the definitions used in almost all of the literature,
is incoherent if the basis over which the circuit is defined is allowed to
include non-symmetric functions. Moreover, almost all of the key definitions
introduced by Anderson and Dawar, as well as the proofs of all their major
results, make implicit use this symmetry condition, and often in ways that make
direct generalisation impossible.

As such, in order to develop our characterisation we needed to begin at the
bottom, generalising symmetric functions and a bases, defining a new kind of
circuit that could be defined over such a basis, generalising key notions for
this model, and then proving key results for the characterisation -- often from
scratch. We introduced isomorphism-invariant structured functions as a
generalisation of symmetric functions. While symmetric functions take in a
string of bits and are invariant under permutations on this string,
isomorphism-invariant structured functions take in (an encoding of) a structure
and are invariant under the action of a permutation on the structure (i.e.\
constant on isomorphism classes of structures). We developed a circuit model
that can be defined over a basis of isomorphism-invariant structured functions,
and generalised the necessary concepts (e.g.\ circuit automorphisms, symmetric,
etc.) to these circuits. We used this frame work to define rank circuits, and so
symmetric rank-circuits.

It would be natural to take symmetric rank-circuits as the appropriate
generalisation of Anderson and Dawar's symmetric circuits, and then hope to
prove an equivalence between the expressive power of $\PT$-uniform families of
such circuits and $\FPR$. However, this circuits pose a problem for our
analysis. As we showed in Section~\ref{sec:transparent}, most of the properties
of these rank-circuits are at least as hard to decide as the graph isomorphism
problem. This poses a problem as the approach we take (and the approach Anderson
and Dawar take) to proving this characterisation make repeated use of the
polynomial-time decidability of these properties. Indeed, the two key results
for translating $\PT$-uniform circuit families into equivalent formulas, the
support theorem and the inductive evaluation of a gate (see
Section~\ref{sec:circuits-to-formulas}), make use of polynomial-time algorithms
for transforming a circuit into an equivalent circuit with unique labels, as
well as for deciding if a circuit is symmetric, computing orbits, and deciding
syntactic-equivalence.

As such, we prove our characterisation in terms of $\PT$-uniform families of
\emph{transparent} symmetric rank-circuits. Since all circuits with symmetric
gates are transparent, transparent circuits provide us with an alternative,
possibly more restrictive, generalisation of the circuits studied by Anderson
and Dawar. Crucially, this generalisation preserves the polynomial-time
decidability of the relevant circuit properties, and so allows us to define a
translation from $\PT$-uniform families of transparent rank-circuits to
equivalent formulas of $\FPR$. While the restriction to transparent circuits
suffice for proving one direction of the proof, i.e.\ translating circuits into
formulas, it complicates proving the other direction. It is not hard to see that
the conventional approach for translating formulas of $\FO$ (or $\FPC$, or other
extensions of $\FO$) into $\PT$-uniform families of circuits does not in general
produce a family of transparent circuits. In order to address this we defined a
novel translation from formulas of $\FPR$ to equivalent $\PT$-uniform families
of transparent symmetric rank-circuits. This translation is defined in two
steps, first we define a translation from $\FPR$ to $\PT$-uniform families of
bounded-width formulas of $\FOrk$ and then we define a translation from these
families of formulas into $\PT$-uniform families of transparent symmetric
rank-circuits. This translation, along with the main result of the paper, gives
us an equivalence between the expressive power of these three formalisms, which
we summarise as follows:

\begin{center}
  \begin{tikzcd}
    \FPR \ar[r, equal] \ar[dr, equal] & \text{$\PT$-uniform families of
      bounded-width $\FOrk$-formulas}  \ar[d, equal]\\
    & \text{$\PT$-uniform families of transparent symmetric rank-circuits}
  \end{tikzcd}
\end{center}

We already have a very similar result from from Anderson and
Dawar~\cite{AndersonD17} and Otto~\cite{}:

\begin{center}
  \begin{tikzcd}
    \FPC \ar[r, equal] \ar[dr, equal] & \text{$\PT$-uniform families of
      bounded-width $\FOc$-formulas}  \ar[d, equal]\\
    & \text{$\PT$-uniform families of symmetric circuits with symmetric gates}
  \end{tikzcd}
\end{center}

Notice that each of the formalisms used to characterise the expressive power of
$\FPC$ can be extended by some mechanism for computing rank. In particular, we
can extend $\FOc$ by rank quantifiers in a very natural way and define $\FOrk$,
we can add rank gates to symmetric circuits with symmetric gates and define
transparent symmetric rank-circuits, and we can extend $\FPC$ by a rank operator
and define $\FPR$. Since each of these extensions is defined quite differently,
we might expect these three extensions of $\FPC$ to have different expressive
power. But from the above sets of equivalences we have that all of these models
have equivalent expressive power. In particular, $\FPR$, an extension of $\FPC$,
has the same expressive power as other natural extensions by rank of other
models with the same expressive power as $\FPC$. This suggests that $\FPR$ is
the natural extension of $\FPC$ by rank operators. In contrast the rank logic
introduced by Dawar et al.~\cite{} was shown by Gr\"{a}del and Pakus~\cite{} to
be strictly less expressive than $\FPR$, and hence less expressive than
$\PT$-uniform families of bounded-width $\FOrk$ and $\PT$-uniform families of
transparent symmetric rank-circuits. The results of this paper thus suggest that
$\FPR$, rather than the rank logic of Dawar et al., should be considered the
appropriate extension of $\FPC$ by rank.

\subsection{Future Work}
We have characterised $\FPR$ using $\PT$-uniform families of transparent
symmetric rank-circuits. It is easy to see that $\PT$-uniform families of
symmetric rank-circuits must be at least as powerful as $\PT$-uniform families
of transparent symmetric rank-circuits, but we do not know if they have the same
expressive power or are strictly more expressive. If the general circuit for a
strictly more powerful model then the main result gives us that that $\FPR$ does
not capture $\PT$. We have not proven it here, but we believe it is possible to
show that if the graph isomorphism problem is in $\PT$ then these classes are
equivalent. If this is shown, separating these circuit classes would imply graph
isomorphism is not in $\PT$.

There are also natural questions regarding generalisation. In this paper we have
characterised the expressive power of $\FPR$, an extension of $\FP$ by a rank
operators, in terms of $\PT$-uniform families of symmetric circuits over the
rank basis, where the rank basis is the extension of the standard basis by the a
set of functions, each of which computed the rank operator for some fixed
threshold and characteristic. It would be natural to ask if there is a much
broader class of logics, perhaps some class of extensions of $\FP$, such that we
can define for each logic in this class a natural characterisation in terms of
$\PT$-uniform transparent symmetric circuits over some appropriate basis. We
should note that the support theorem and most of the tools developed in this
paper have application to circuits defined over any basis. In fact, the first
time we explicitly use the properties of the rank function is in the proof of
Lemma~\ref{}, right towards the end of the paper.

It would be natural to ask if there is a class of circuits, perhaps symmetric
circuits defined over an extension of the majority basis, that would suffice to
give a circuit characterisation of $\CPT$ or $\CPTC$ ($\CPT$ with counting). We
recall that $\CPTC$ is one of the two candidate logics for capturing
polynomial-time, and as such there is significant interest in understanding its
expressive power. At present, very little is known about $\CPTC$, and there are
no known promising techniques for proving lower bounds. Since circuits families
seem easier to analysis then $\CPTC$ programs, a circuit characterisation of
$\CPTC$ would provide us with a very promising for studying the expressive power
of $\CPTC$, perhaps allowing us to apply techniques for finding circuit lower
bounds to the problem of deriving inexpressibly results for $\CPTC$. As a
starting point we might ask if there are any interesting extensions of the
majority basis by non-symmetric functions such that $\PT$-uniform symmetric
circuits over such a basis is comparable to $\CPT$ -- i.e.\ known to be at least
or at most as expressive as $\CPT$. We know that any fragment of $\CPTC$ given
by placing a constant bound on the rank of sets used in computation is at least
as expressive as $\FPC$ (see~\cite{}), and hence at least as expressive as
$\PT$-uniform symmetric circuits with symmetric gates. We may ask if for each
fragment of $\CPTC$ with a bound $r$ on the rank of the sets used, there is a
non-trivial class of circuits $\mathcal{C}_r$ (perhaps a family of symmetric
circuits over some basis) such that the fragment of $\CPT$ with bound $r$ is at
most as expressive as the $\PT$-uniform families of circuits from
$\mathcal{C}_r$. It is worth noting that, since we now have a circuit
characterisation for $\FPR$, finding a circuit characterisation for $\CPT$ or
$\CPTC$ (or showing none exists) may help clarify the relationship between
$\CPTC$ and $\FPR$.

We have proposed a new circuit model, presented a characterisation of $\FPR$ in
terms of $\PT$-uniform families of a particular class of circuits, and suggested
how we might generalise these results in order to develop circuit-based
characterisations of other logics -- hopefully even $\CPT$ or $\CPTC$. The
natural next step would be to develop techniques for proving lower bounds for
$\PT$-uniform circuit families. A general approach for finding lower bounds for
families of symmetric circuits over some basis may be very powerful,
particularly if it turns out to be possible to characterise a broad class of
extensions of $\FP$ using $\PT$-uniform families of transparent symmetric
circuits. The usual approach for establishing inexpressibly results for logics
involves first defining a pebble game for the logic. Dawar~\cite{} has shown
that the pebble game of Hella~\cite{} can be directly linked with symmetric
circuits with symmetric gates via a circuit-based argument that makes no
reference to logic. Dawar and Holm~\cite{} have defined a pebble game for
$\FPR$, and it would be interesting to know if a similar link can be established
between this pebble game and transparent symmetric rank-circuits. It may be
possible to generalise these arguments and establish for each basis $\BB$ a
corresponding pebble game for transparent symmetric circuits defined over $\BB$.

The circuit model we have developed is novel, and as such there are many basic
questions of interest. We do not know, for example, whether for each $a \in
\nats$ there is a symmetric circuit with symmetric gates where each gate has
fan-in less than $a$ that computes $\NAND[a+1]$. For a basis $\BB$ let $Q_\BB$
be the set of all queries computed by $\PT$-uniform families of transparent
symmetric circuits defined over $\BB$. We have that $Q_{\BM} \subsetneq
Q_{\RB}$. It follows from Theorem~\ref{} that if $\BB$ is a symmetric basis then
$Q_\BB \subseteq Q_{\BM}$. It follows that $Q_{\BM}$ is a basis including structured
functions defined over a vocabulary with two sorts that defines a circuit model
strictly more powerful than any defined by basis of that only include functions
with one sort.s


defines a strictly more powerful class of circuits than

It is not hard to show that $Q_\BB \subseteq Q_{\BM}$ for any matrix-symmetric
basis $Q_\BB$ consisting of polynomial-time computable functions if, and only
if, $\FPR$ captures $\PT$.


We know from Theorem~\ref{} that the majority basis is `complete' for symmetric
bases in the sense that any query computed by a $\PT$-uniform family of
symmetric circuits defined over a symmetric basis there is a $\PT$-uniform
family of symmetric circuits over the majority basis that computes the same
query. It is easy to see that if


Similarly, we do not know if for $r, a, p \in \nats$, with $p$ prime, the
function $\RANK^r_p [a][a]$ is computable by a symmetric circuit with symmetric
gates.


\end{document}
