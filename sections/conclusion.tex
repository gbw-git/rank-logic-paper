\documentclass[../paper.tex]{subfiles}
\begin{document}
$\FPR$ is one of the most expressive logics we know that is still
contained in $\PT$ and understanding its expressive power is an
important question.  The main result of this paper establishes an equivalence between the
expressive power of $\FPR$ and the computational power of uniform families
of transparent symmetric rank-circuits.  Not only does this result establish an
interesting characterisation of an important
logic, but it also deepens our understanding of the connection between
logic and circuit complexity in general and casts new light on
foundational aspects of the circuit model.

The circuit characterisation also helps emphasise certain important
aspects of the logic.  Given that $\PT$-uniform families of invariant
circuits (without the restriction to symmetry) express all properties
on $\PT$, we can understand  the inability of $\FPC$ (and,
conjecturally, $\FPR$) to express all such properties as essentially
down to symmetry.  As with other (machine) models of computation, the
translation to circuits exposes the inherent combinatorial structure
of an algorithm.  In the case of logics, we find that a key property
of this structure is its symmetry and the translation to circuits
provides us with the tools to study it.

% This alternative view of the logic also helps emphasise certain characteristics
% of the logic that may be much less obvious when working with the conventional
% definition. To choose one notable example, the circuit characterisations (both
% of $\FPR$ and $\FPC$) help bring to the fore the importance of the symmetry
% properties inherent in the the syntactic structure of the logic. Indeed, we
% recall that $\PT$-uniform families of transparent invariant (rather than
% symmetric) rank-circuits and $\PT$-uniform families of invariant circuits with
% symmetric gates decide exactly those queries that are in $\PT$. In this way the
% restriction from families of invariant circuits to families of symmetric
% circuits corresponds to the restriction from $\PT$ to $\FPR$ or $\FPC$. In this
% sense, we might say that it is this underlying symmetry property of the logic,
% brought into focus by the circuit characterisation, that separates $\FPC$ from
% $\PT$, and which determines the relationship between $\FPR$ and $\PT$. In fact,
% we have that $\FPR$ captures $\PT$ if, and only if, $\PT$-uniform families of
% symmetric rank-circuits have the same computational power as $\PT$-uniform
% families of invariant rank-circuits. In this way the circuit-characterisation
% brings out this underlying symmetry property of the logic, demonstrates how
% crucial this symmetry property for understanding the logic, and then provides us
% with a framework for studying symmetry.

% This illustrates just one important way in which this circuit characterisation
% aids in our understanding of $\FPR$, and perhaps in our understanding of
% logics more generally.

% In this way the circuit-characterisation allows us to under would seem then
% that understanding these symmetry properties of the circuit (and logic) is
% central to understanding the relationship between $\FPR$ (or perhaps any
% logic) and $\PT$. As such, we might thus rephrase the question of whether
% $\FPR$ captures $\PT$ in a slightly informal way as, `Does symmetry matter for
% rank-circuits?'.

Still, the most significant contribution of this paper is not in the main result
but in the techniques that are developed to establish it, and we
highlight some of these now.  The
conclusion of~\cite{AndersonD17} says that the support theorem is
``largely agnostic to the particular [\ldots] basis'', suggesting that
it could be easily adapted to include other gates.  This turns out to
have been a misjudgment.  Attempting to prove the support theorem for
a basis that includes rank threshold gates showed us the extent to
which both the proof of the theorem and, more broadly, the definitions
of circuit classes, rest heavily on the assumption that all functions
computed by gates are symmetric.  Thus, in order to define what the
``symmetry'' condition might mean for circuits that include rank
threshold gates, we radically generalise the circuit framework to allow
for gates that take structured inputs (rather than sets of $0$s
and $1$s) and are invariant under isomorphisms.  This leads to a
refined notion of circuit automorphism, which allows us to formulate a
notion of symmetry and prove a version of the support theorem.  Again,
in that proof, substantial new methods are required.

% itself, but rather in the entirely novel framework for circuits and the numerous
% new results that we needed to develop in order to prove the main result of this
% paper. As such, we shall briefly review some of these contributions, placing an
% emphasis both on the nature of these contributions and on the necessity of their
% introduction. Indeed, it would seem at first glance that generalising Anderson
% and Dawar's circuit characterisation of $\FPC$ to $\FPR$ should be a simple
% matter of extending the model so as to allow circuits to be defined over bases
% that includes rank-threshold functions -- and perhaps extending a few key
% results and definitions as needed. We showed in
% Theorem~\ref{thm:symmetric-circuits-bound} that, in fact, we cannot increase the
% expressive power of the model by adding symmetric functions to the basis, and so
% the addition of non-symmetric functions to the basis seems almost unavoidable.
% However, while the inclusion of rank-threshold functions in the basis may seem
% benign, it in fact violates a near-ubiquitous assumption in circuit complexity,
% namely that the basis over which the circuit is defined must consist of
% symmetric functions. While this assumption is usually not stated, the fact that
% the circuit is taken to be a directed acyclic graph means that there is no order
% (or any structure at all) on the inputs of each gate, an assumption which only
% makes sense if it is assumed that the evaluation of each gate is invariant under
% any ordering of its inputs -- which is to say the basis must be symmetric.
% However, we found that in Anderson and Dawar's case this assumption pervades not
% just their definition of a circuit, but also the central ideas of their paper,
% including their definition of a circuit automorphism and of a symmetric circuit,
% as well as almost all of the key results and techniques they use to establish
% their characterisation.

% As such, in order to include rank-threshold functions in the basis we needed to
% construct a new framework for circuits as well as develop novel approaches and
% techniques in order to work with these circuits. We began by generalising the
% notion of a symmetric function, introducing isomorphism-invariant structured
% functions. These functions take in structures encoded and strings and are
% invariant under isomorphism, rather than under arbitrary permutations of the
% string. It is easy to see that rank-threshold functions are
% isomorphism-invariant. We defined a basis to be a set of isomorphism-invariant
% functions, and extended the circuit model, adding to each gate in the circuit a
% structure on its input gates in accord with the isomorphism-invariant structured
% function labelling that gate. Rather than a gate being invariant under arbitrary
% permutations of its input gates, a gate in this circuit should be invariant
% under permutations of its input gate that form a (labelled) isomorphism. We
% extended the notion of an automorphism to reflect this and defined what it means
% for a circuit to be symmetric.


While the new framework allows us to reproduce some results such as
the support theorem, something is lost in the generalisation.  When
our basis contains only symmetric functions, important properties of
a circuit, including the symmetry of the circuit itself, are easily
decidable.  In the new framework, checking circuit automorphism
requires checking isomorphism of structured inputs to gates and codes
hard problems.  For this reason, we imposed a further restriction to
the circuits in the form of \emph{transparency}.  But, it should be
clear that this is a restriction only in the broader framework.  All
symmetric circuits with only symmetric gates are transparent.  Thus,
transparent, symmetric circuits are still a generalisation of those
considered in~\cite{AndersonD17}.  

The condition of transparency makes the translation of uniform circuit
families into formulas of logic (which is the difficult direction of
our characterisation) possible, but it complicates the other direction.
Indeed, the natural translation of formulas of $\FPR$ into uniform
circuit families yields circuits which are symmetric, but not
transparent.  This problem is addressed by introducing gadgets in the
translation---which for ease of exposition, we did in formulas of
$\FOrk$ which are then translated into circuits in the natural way.
Thus, the restriction to transparent circuits is sufficient to get
both directions of the characterisation.  Moreover, we argue that
something like it is necessary in order to avoid having to solve hard
isomorphism problems in the translation.

% While we have generalised a few results for this framework, including the
% support theorem, it is important to note that a number of crucial counting
% techniques and polynomial-time algorithms for circuits are either not easy to
% generalise or otherwise resist generalisation entirely. This follows from the
% fact that a circuit automorphism needs to preserve more structure when acting on
% non-symmetric gates than when acting on symmetric gates. As a result, while many
% symmetry-related circuit properties are polynomial-time decidable for circuits
% with symmetric gates, we have shown that these same properties are at least as
% hard as the graph-isomorphism for circuits with non-symmetric gates. This is a
% problem as the polynomial-time decidability of these properties is essential for
% our translation from circuits into formulas.


% As such, instead of working with general rank-circuits we have restricted our
% attention to \emph{transparent} circuits, and proved our characterisation in
% terms of $\PT$-uniform families of transparent symmetric rank-circuits. As we
% have shown in Section~\ref{sec:transparent}, all of the relevant properties are
% polynomial-time decidable for transparent circuits. Since all circuits with
% symmetric gates are transparent, we might think of transparent circuits as an
% alternative, more `algorithmically accessible', generalisation of the circuits
% studied by Anderson and Dawar. However, while this restriction to transparent
% circuits aids in our translation from circuits to formulas, it complicates our
% translation in the other direction. Indeed, it is not hard to see that the
% conventional approach for translating formulas of $\FO$ (or $\FPC$, or other
% extensions of $\FO$) into $\PT$-uniform families of circuits does not, in
% general, produce a family of transparent circuits. In order to address this we
% defined a novel translation from formulas of $\FPR$ into equivalent
% $\PT$-uniform families of transparent symmetric rank-circuits. This translation
% is defined in two steps. First, we defined a translation from $\FPR$ to
% $\PT$-uniform families of bounded-width formulas of $\FOrk$ and, second, we
% defined a translation from these families of formulas into $\PT$-uniform
% families of transparent symmetric rank-circuits. This translation, along with
% the main result of the paper, gives us an equivalence between the expressive
% power of three formalisms, which we summarise as follows:

% \begin{center}
%   \begin{tikzcd}
%     \FPR \ar[r, equal] \ar[dr, equal] & \text{$\PT$-uniform families of
%     bounded-width $\FOrk$-formulas}  \ar[d, equal]\\
%     & \text{$\PT$-uniform families of transparent symmetric rank-circuits}
%   \end{tikzcd}
% \end{center}

In short, we can represent the proof of our characterisation through
the three equivalences in this triangle.

\begin{center}
  \begin{tikzcd}
    \FPR \ar[r, equal] \ar[dr, equal] & \parbox{0.35\textwidth}{Uniform
      families of
      bounded-width $\FOrk$ formulas}  \ar[d, equal]\\
    & \parbox{0.38\textwidth}{Uniform families of transparent symmetric
      rank-circuits}
  \end{tikzcd}
\end{center}

This highlights another interesting aspect of our result.  The first
translation, of $\FPR$ to uniform families of $\FOrk$ formulas was
given in~\cite{Dawar09logicswith} and used there to establish arity
lower bounds.  However, this was for a weaker version of the rank
logic rather than the strictly more expressive one defined by
Gr\"{a}del and Pakusa~\cite{GradelP15a}.  The fact that we can
complete the cycle of equivalences with the more powerful logic
demonstrates that the definition of Gr\"{a}del and Pakusa is the
``right'' formulation of $\FPR$.

% We already have a very similar result from from Anderson and
% Dawar~\cite{AndersonD17} and Otto~\cite{Otto1996}:

% \begin{center}
%   \begin{tikzcd}
%     \FPC \ar[r, equal] \ar[dr, equal] & \parbox{0.35\textwidth}{$\PT$-uniform
%       families of
%       bounded-width $\FOc$-formulas}  \ar[d, equal]\\
%     & \parbox{0.35\textwidth}{$\PT$-uniform families of symmetric circuits with
%       symmetric gates}
%   \end{tikzcd}
% \end{center}

% These two sets of equivalences each give us three distinct formalisms equivalent
% to $\FPR$ and $\FPC$, respectively. Importantly, each of these three formalisms
% for $\FPC$ may be extended by a rank in quite distinct ways. We can extend
% $\FOc$ by rank quantifiers and define $\FOrk$, we can add rank gates to
% symmetric circuits with symmetric gates and define transparent symmetric
% rank-circuits, and we can extend $\FPC$ by a rank operator and define $\FPR$. We
% might expect these three quite different extensions to result in three
% formalisms with quite different expressive power. However, the main result of
% this paper along with the translation we discussed above, gives us that these
% three extensions are indeed all equivalent. This provides evidence in favour of
% the proposition that $\FPR$ is indeed the canonical, and most natural, extension
% of $\FPC$. This is in contrast with the other rank logic in the literature, the
% one introduced by Dawar et al.~\cite{Dawar09logicswith}, which was shown by
% Gr\"{a}del and Pakusa~\cite{GradelP15a} to be strictly less expressive than
% $\FPR$, and hence less expressive than $\PT$-uniform families of bounded-width
% $\FOrk$ and $\PT$-uniform families of transparent symmetric rank-circuits.

% the cononical nature of the extension from $\FPC$ to $\FPR$. This suggests
% that the definition of $\FPR$ is indeed natural.


% 'This has quite interesting implications for the robustness of the definition
% of $\FPR$. We recall that there are really two rank logics in the literature,
% $\FPR$ and another, strictly less expressive rank logic introduced by Dawar et
% al. However, Gr\"{a}del and Pakusa~\cite{} to be strictly less expressive than
% $\FPR$


% But from these equivalences we have that these three seemingly distinct
% extensions each result in a formalism equivalent to $\FPR$. In contrast if we
% were to extend $\FPC$ by the family of rank non-uniform operators introduced
% by Dawar et al.~\cite{}, rather than the non-uniform rank operator we use
% instead, as was shown by Gr\"{a}del and Pakusa~\cite{} this logic would be
% strictly less expressive than $\FPR$, and hence less expressive than
% $\PT$-uniform families of bounded-width $\FOrk$ and $\PT$-uniform families of
% transparent symmetric rank-circuits. What is perhaps most striking is that
% extending each of these three quite different formalisms equivalent to $\FPC$
% using quite different mechanisms for rank results in a formalism equivalent to
% $\FPR$. of these formalisms by rank Since each of these extensions is defined
% quite differently, we might expect these three extensions of $\FPC$ to have
% different expressive power.

% But from the above sets of equivalences we have that all of these models have
% equivalent expressive power. In particular, $\FPR$, an extension of $\FPC$,
% has the same expressive power as other natural extensions by rank of other
% models with the same expressive power as $\FPC$. This suggests that $\FPR$ is
% the natural extension of $\FPC$ by rank operators. In contrast the rank logic
% introduced by Dawar et al.~\cite{} was shown by Gr\"{a}del and Pakusa~\cite{}
% to be strictly less expressive than $\FPR$, and hence less expressive than
% $\PT$-uniform families of bounded-width $\FOrk$ and $\PT$-uniform families of
% transparent symmetric rank-circuits. The results of this paper thus suggest
% that $\FPR$, rather than the rank logic of Dawar et al., should be considered
% the appropriate extension of $\FPC$ by rank.


% Since each of these extensions is defined quite differently, we might expect
% these three extensions of $\FPC$ to have different expressive power. But from
% the above sets of equivalences we have that all of these models have
% equivalent expressive power. In particular, $\FPR$, an extension of $\FPC$,
% has the same expressive power as other natural extensions by rank of other
% models with the same expressive power as $\FPC$. This suggests that $\FPR$ is
% the natural extension of $\FPC$ by rank operators. In contrast the rank logic
% introduced by Dawar et al.~\cite{} was shown by Gr\"{a}del and Pakusa~\cite{}
% to be strictly less expressive than $\FPR$, and hence less expressive than
% $\PT$-uniform families of bounded-width $\FOrk$ and $\PT$-uniform families of
% transparent symmetric rank-circuits. The results of this paper thus suggest
% that $\FPR$, rather than the rank logic of Dawar et al., should be considered
% the appropriate extension of $\FPC$ by rank.

\subsection*{Future Work}
There are many directions of work suggested by the methods and results
developed in this paper.  First of all, there is the question of
transparency.  We introduce it as a technical device that enables our
characterisation to go through.  Could it be dispensed with?  Or are $\PT$-uniform
families of transparent symmetric rank-circuits strictly weaker than
families without the restriction of transparency?  Of course a
positive answer to the latter question would require separating $\FPR$
from $\PT$.

The framework we have developed for  working with circuits with
structured inputs is very general and not specific to rank gates.
Indeed, the first use we make of linear algebra is in the proof of
Lemma~\ref{lem:rank-triple-equivilence}.  It would be interesting to
apply this framework to other logics.  It appears to be as general a way of
extending the power of circuits as Lindstr\"om quantifiers are in the
context of logic.  We would like to develop this link
further, perhaps for specific quantifiers such as $\FP$ extended by an operator
that expresses the solubility of systems of equations over rings as in~\cite{DGHKP}

% In this paper we have developed a framework for working with circuits over
% non-symmetric bases. In fact, the first time we explicitly use make any
% reference to linear algebra is in the proof of
% Lemma~\ref{lem:rank-triple-equivilence}, right towards the end of the paper. It
% would be natural to ask if our circuit characterisation of $\FPR$ can be
% generalised in order to develop similar characterisations of other logics. It
% might be good idea to start with some other extension of $\FP$ by some operator
% and try to develop a similar circuit characterisation in terms $\PT$-uniform
% transparent symmetric circuits over an appropriate basis. There are numerous
% logics that are similar in definition to $\FPR$ but which are not known to have
% the same expressive power as $\FPR$, for example $\FP$ extended by an operator
% that expresses the solubility of systems of equations over rings (rather than
% fields). A first step might be to develop a circuit characterisation of one of
% these logics.

% If the general circuit for a strictly more powerful model then the main result
% gives us that that $\FPR$ does not capture $\PT$. We have not proven it here,
% but we believe it is possible to show that if the graph isomorphism problem is
% in $\PT$ then these classes are equivalent. If this is shown, separating these
% circuit classes would imply graph isomorphism is not in $\PT$.

% There are also natural questions regarding generalisation. In this paper we
% have characterised the expressive power of $\FPR$, an extension of $\FP$ by a
% rank operators, in terms of $\PT$-uniform families of symmetric circuits over
% the rank basis, where the rank basis is the extension of the standard basis by
% the a set of functions, each of which computed the rank operator for some
% fixed threshold and characteristic. It would be natural to ask if there is a
% much broader class of logics, perhaps some class of extensions of $\FP$, such
% that we can define for each logic in this class a natural characterisation in
% terms of $\PT$-uniform transparent symmetric circuits over some appropriate
% basis. We should note that the support theorem and most of the tools developed
% in this paper have application to circuits defined over any basis. In fact,
% the first time we explicitly use the properties of the rank function is in the
% proof of Lemma~\ref{}, right towards the end of the paper.

At the moment, we have little by way of methods for proving
inexpressibility results for $\FPR$, whether we look at it as a logic
or in the circuit model.  The logical formulation lays emphasis
on some parameters (the number of variables, the arity of the
operators, etc.) which we can treat as resources against which to
prove lower bounds.  On the other hand, the circuit model brings to
the fore other, more combinatorial, parameters.  One such is the
fan-in of gates and a promising and novel approach is to try and prove lower bounds for symmetric circuits with gates with bounded fan-in.
We might ask if it is possible to compute $\AND[3]$ using a symmetric circuit
with gates that have fan-in two.  Perhaps we could also combine the
circuit view with lower-bound methods from logic, such as pebble games.
 Dawar~\cite{Dawar2016} has shownn how the bijection games of
 Hella~\cite{Hella19961} can be used directly to prove lower
 bounds for symmetric without reference to the logic.  We also have pebble
 games for $\FPR$~\cite{DawarH2012}, and it would be interesting to
 know if we can use these on circuits and how the combinatorial
 parameters of the circuit interact with the game.

Finally, we note that some of the interesting directions on the
interplay between logic and symmetric circuits raised
in~\cite{AndersonD17} remain relevant.  Can we relax the symmetry
condition to something in between requiring invariance of the circuit
under the full symmetric group (the case of symmetric circuits) and
requiring no invariance condition at all?  Can such restricted
symmetries give rise to interesting logics in between $\FPR$ and
$\PT$?  It also remains a challenge to find a circuit characterisation
of  $\CPTC$.  Could the general framework for non-symmetric gates we
have developed here help in this respect?

% We have shown that the $\PT$-uniform families of transparent symmetric
% rank-circuits (resp. circuits with symmetric gates) have the same expressive
% power as $\FPR$ (resp. $\FPC$), but we recall that if we remove the symmetry
% restriction these circuit models express exactly those queries in $\PT$. A
% natural next step, also suggested by Anderson and Dawar, would be try and weaken
% the symmetry condition. One way this might be done would be to require that only
% permutations from a particular subgroup of the symmetric group extend to
% automorphisms of the circuit. It is easy to see that if we take this subgroup to
% be the trivial subgroup then this nullifies the the symmetry requirement, and
% the circuit model has the same expressive power as $\PT$, while if we take this
% subgroup to be the entire symmetric group then we get back the full symmetry
% requirement and hence our model has the same expressive power as the logic. As
% such, this approach can be used to interpolate between the logic and $\PT$.

% There is a rather easy way to develop a circuit characterisation for any logic
% or complexity class with complete problems under first-order reductions. All
% that is required is to extend the basis with the family of functions that decide
% a complete problem and consider $\PT$-uniform families of transparent symmetric
% circuits over this basis. We did not take this approach in our characterisation
% of $\FPR$, preferring to use a more `natural' basis that more closely resembles
% the actual definition of the logic. We should like to ask if there is a similar
% circuit characterisation of $\CPTC$. One might start with bounded-rank fragments
% of $\CPTC$ and see if these programs can be translated into circuits. We should
% note that it does seem as though the symmetry inherent in $\CPTC$ programs is in
% some sense different from the symmetry of symmetric circuits. It might be more
% natural then to ask if $\CPTC$ can be characterised using families of circuits
% with a weakened symmetry requirement.


% The most obvious next step would be to try and develop a (natural) circuit
% characterisation of $\CPTC$. There are a number of trivial approaches that can
% be used to define a circuit characterisation for $\CPTC$. For example, if $Q$
% is a decision problem in $\CPTC$ that is complete under first-order
% reductions, and introduce define a circuit over a basis that includes
% functions computing that decide that decide this problem. However, if a more
% nau

% It would be natural to ask if there is a class of circuits, perhaps symmetric
% circuits defined over an extension of the majority basis, that would suffice
% to give a circuit characterisation of $\CPT$ or $\CPTC$ ($\CPT$ with
% counting). We recall that $\CPTC$ is one of the two candidate logics for
% capturing polynomial-time, and as such there is significant interest in
% understanding its expressive power. At present, very little is known about
% $\CPTC$, and there are no known promising techniques for proving lower bounds.
% Since circuits families seem easier to analysis then $\CPTC$ programs, a
% circuit characterisation of $\CPTC$ would provide us with a very promising for
% studying the expressive power of $\CPTC$, perhaps allowing us to apply
% techniques for finding circuit lower bounds to the problem of deriving
% inexpressibly results for $\CPTC$. As a starting point we might ask if there
% are any interesting extensions of the majority basis by non-symmetric
% functions such that $\PT$-uniform symmetric circuits over such a basis is
% comparable to $\CPT$ -- i.e.\ known to be at least or at most as expressive as
% $\CPT$. We know that any fragment of $\CPTC$ given by placing a constant bound
% on the rank of sets used in computation is at least as expressive as $\FPC$
% (see~\cite{}), and hence at least as expressive as $\PT$-uniform symmetric
% circuits with symmetric gates. We may ask if for each fragment of $\CPTC$ with
% a bound $r$ on the rank of the sets used, there is a non-trivial class of
% circuits $\mathcal{C}_r$ (perhaps a family of symmetric circuits over some
% basis) such that the fragment of $\CPT$ with bound $r$ is at most as
% expressive as the $\PT$-uniform families of circuits from $\mathcal{C}_r$. It
% is worth noting that, since we now have a circuit characterisation for $\FPR$,
% finding a circuit characterisation for $\CPT$ or $\CPTC$ (or showing none
% exists) may help clarify the relationship between $\CPTC$ and $\FPR$.

% We have proposed a new circuit model, presented a characterisation of $\FPR$
% in terms of $\PT$-uniform families of a particular class of circuits, and
% suggested how we might generalise these results in order to develop
% circuit-based characterisations of other logics -- hopefully even $\CPT$ or
% $\CPTC$. The natural next step would be to develop techniques for proving
% lower bounds for $\PT$-uniform circuit families. A general approach for
% finding lower bounds for families of symmetric circuits over some basis may be
% very powerful, particularly if it turns out to be possible to characterise a
% broad class of extensions of $\FP$ using $\PT$-uniform families of transparent
% symmetric circuits. The usual approach for establishing inexpressibly results
% for logics involves first defining a pebble game for the logic. Dawar~\cite{}
% has shown that the pebble game of Hella~\cite{} can be directly linked with
% symmetric circuits with symmetric gates via a circuit-based argument that
% makes no reference to logic. Dawar and Holm~\cite{} have defined a pebble game
% for $\FPR$, and it would be interesting to know if a similar link can be
% established between this pebble game and transparent symmetric rank-circuits.
% It may be possible to generalise these arguments and establish for each basis
% $\BB$ a corresponding pebble game for transparent symmetric circuits defined
% over $\BB$.

% The circuit model we have developed is novel, and as such there are many basic
% questions of interest. We do not know, for example, whether for each $a \in
% \nats$ there is a symmetric circuit with symmetric gates where each gate has
% fan-in less than $a$ that computes $\NAND[a+1]$. For a basis $\BB$ let $Q_\BB$
% be the set of all queries computed by $\PT$-uniform families of transparent
% symmetric circuits defined over $\BB$. We have that $Q_{\BM} \subsetneq
% Q_{\RB}$. It follows from Theorem~\ref{} that if $\BB$ is a symmetric basis
% then $Q_\BB \subseteq Q_{\BM}$. It follows that $Q_{\BM}$ is a basis including
% structured functions defined over a vocabulary with two sorts that defines a
% circuit model strictly more powerful than any defined by basis of that only
% include functions with one sort.s


% defines a strictly more powerful class of circuits than

% It is not hard to show that $Q_\BB \subseteq Q_{\BM}$ for any matrix-symmetric
% basis $Q_\BB$ consisting of polynomial-time computable functions if, and only
% if, $\FPR$ captures $\PT$.


% We know from Theorem~\ref{} that the majority basis is `complete' for
% symmetric bases in the sense that any query computed by a $\PT$-uniform family
% of symmetric circuits defined over a symmetric basis there is a $\PT$-uniform
% family of symmetric circuits over the majority basis that computes the same
% query. It is easy to see that if


% Similarly, we do not know if for $r, a, p \in \nats$, with $p$ prime, the
% function $\RANK^r_p [a][a]$ is computable by a symmetric circuit with
% symmetric gates.


\end{document}
 